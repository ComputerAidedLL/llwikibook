% \section{A formal account of nets}\label{a-formal-account-of-nets}

The aim of this page is to provide a common framework for describing
linear logic proof nets, interaction nets, multiport interaction nets,
and the likes, while factoring out most of the tedious, uninteresting
details (clearly not the fanciest page of LLWiki).

\section{Preliminaries}\label{preliminaries}

\subsection{The short story}\label{the-short-story}

\begin{itemize}
\tightlist
\item
  the general flavor is that of multiport interaction nets;
\item
  the top/down or passive/active orientation of cells is related with
  the distinction between premisses and conclusions of rules, (and in
  that sense, a cut is not a logical rule, but the focus of interaction
  between two rules);
\item
  cuts are thus wires rather than cells/links: this fits with the
  intuition of GoI, but not with the most common presentations of proof
  nets;
\item
  because the notion of subnet is not trivial in multiport interaction
  nets, and to avoid the use of geometric conditions (boxes must not
  overlap but can be nested), we introduce boxes as particular cells;
\item
  when representing proof nets, we introduce axioms explicitly as cells,
  so that axiom-cuts do not vanish.
\end{itemize}

\section{Nets}\label{nets}

\subsection{Wires}\label{wires}

A \emph{wiring} is the data of a finite set \(P\) of ports and of a
partition \({W}\) of \(P\) by pairs (the \emph{wires}): if
\(\{p,q\}\in{W}\), we write \({W}(p)=q\) and \({W}(q)=p\). Hence a
wiring is equivalently given by an involutive permutation \({W}\) of
finite domain \(P\), without fixpoints (forall \(p\), \({W}(p)\not=p\)):
the wires are then the orbits. Another equivalent presentation is to
consider \({W}\) as a (simple, loopless, undirected) graph, with
vertices in P, all of degree 1.

We say two wirings are disjoint when their sets of ports are. A
\emph{connection} between two disjoint wirings \(W\) and \(W'\) is a
partial injection \((I,I',f):P\pinj P'\): \(I\subseteq P\),
\(I'\subseteq P'\) and \(f\) is a bijection \(I\cong I'\). We then write
\(W\bowtie_f{{W}'}\) for the wiring obtained by identifying the ports
pairwise mapped by \(f\), and then ``straightening`` the paths thus
obtained to recover wires: notice this might also introduce loops and we
write \(\Inner{W}{W'}_f\) for the number of loops thus appeared.

We describe these operations a bit more formally. Write
\(P = P_0\uplus I\) and \(P' = P_0'\uplus I'\). Then consider the graph
\(W\dblcolon_f{{W}'}\) with vertices in \(P\cup P'\), and such that
there is an edge between \(p\) and \(q\) iff \(q={W}(p)\) or
\(q={W'}(p)\) or \(q=f(p)\) or \(p=f(q)\): in other words,
\(W\dblcolon_f{{W}'}=W\cup W'\cup f\cup f^{-1}\). Vertices in
\(P_0\cup P'_0\) are of degree 1, and the others are of degree 2. Hence
maximal paths in \(W\dblcolon_f{{W}'}\) are of two kinds:

\begin{itemize}
\tightlist
\item
  straight paths, with both ends in \(P_0\cup P_0'\);
\item
  cycles, with vertices all in \(I\cup I'\).
\end{itemize}

Then the wires in \(W\bowtie_f{{W}'}\) are the pairs \(\{p,p'\}\) such
that \(p\) and \(p'\) are the ends of a path in \(W\dblcolon_f{{W}'}\).
And \(\Inner{W}{W'}_{f}\) is the number of cycles in
\(W\dblcolon_f{{W}'}\), or more precisely the number of support sets of
cycles (i.e. we forget about the starting vertice of cycles).

\&\&\textbackslash{}ar{[}dl{]}\^{}\{A\textbackslash{}otimes\textbackslash{}lambda\_B\}A\textbackslash{}otimes(I\textbackslash{}otimes
B)\textbackslash{}\textbackslash{}

\texttt{~~~\&A\textbackslash{}otimes~B\&}

\} commutes. \}\}

\&(B\textbackslash{}otimes C)\textbackslash{}otimes
A\textbackslash{}ar{[}dr{]}\^{}\{\textbackslash{}alpha\_\{B,C,A\}\}\textbackslash{}\textbackslash{}
(A\textbackslash{}otimes B)\textbackslash{}otimes
C\textbackslash{}ar{[}ur{]}\^{}\{\textbackslash{}alpha\_\{A,B,C\}\}\textbackslash{}ar{[}dr{]}\_\{\textbackslash{}gamma\_\{A,B\}\textbackslash{}otimes
C\}\&\&\&B\textbackslash{}otimes (C\textbackslash{}otimes
A)\textbackslash{}\textbackslash{} \&(B\textbackslash{}otimes
A)\textbackslash{}otimes
C\textbackslash{}ar{[}r{]}\_\{\textbackslash{}alpha\_\{B,A,C\}\}\&B\textbackslash{}otimes(A\textbackslash{}otimes
C)\textbackslash{}ar{[}ur{]}\_\{B\textbackslash{}otimes\textbackslash{}gamma\_\{A,C\}\}\textbackslash{}\textbackslash{}
\} and

\[\xymatrix{
&amp;(A\otimes B)\otimes C\ar[r]^{\gamma_{A\otimes B,C}}&amp;C\otimes (A\otimes B)\ar[dr]^{\alpha^{-1}_{C,A,B}}&amp;\\
A\otimes (B\otimes C)\ar[ur]^{\alpha^{-1}_{A,B,C}}\ar[dr]_{A\otimes\gamma_{B,C}}&amp;&amp;&amp;(C\otimes A)\otimes B\\
&amp;A\otimes(C\otimes B)\ar[r]_{\alpha^{-1}_{A,C,B}}&amp;(A\otimes C)\otimes B\ar[ur]_{\gamma_{A,C}\otimes B}&amp;\\
}\] commute for every objects \(A\), \(B\) and \(C\).

A \emph{symmetric} monoidal category is a braided monoidal category in
which the braiding satisfies

\[\gamma_{B,A}\circ\gamma_{A,B}=A\otimes B\] for every objects \(A\) and
\(B\). \}\}

\&B \} commute.

Dually, the monoidal category \(\mathcal{C}\) is \emph{right closed}
when the functor \(B\mapsto B\otimes A\) admits a right adjoint. The
notion of \emph{right closed structure} can be defined similarly. \}\}

In a symmetric monoidal category, a left closed structure induces a
right closed structure and conversely, allowing us to simply speak of a
\emph{closed symmetric monoidal category}.

\section{Modeling the additives}\label{modeling-the-additives}

A category has \emph{finite products} when it has products and a
terminal object.

\textbackslash{}ar{[}r{]}\^{}-\{\textbackslash{}mu\textbackslash{}tens
M\}\&M\textbackslash{}tens
M\textbackslash{}ar{[}dd{]}\^{}\{\textbackslash{}mu\}\textbackslash{}\textbackslash{}
M\textbackslash{}tens(M\textbackslash{}tens
M)\textbackslash{}ar{[}d{]}\_\{M\textbackslash{}tens\textbackslash{}mu\}\&\&\textbackslash{}\textbackslash{}
M\textbackslash{}tens
M\textbackslash{}ar{[}rr{]}\_\{\textbackslash{}mu\}\&\&M\textbackslash{}\textbackslash{}
\} and

\[\xymatrix{
I\tens M\ar[r]^{\eta\tens M}\ar[dr]_{\lambda_M}&amp;M\tens M\ar[d]_\mu&amp;\ar[l]_{M\tens\eta}\ar[dl]^{\rho_M}M\tens I\\
&amp;M&amp;
}\] commute. \}\}

\section{\texorpdfstring{Modeling \hyperref[intuitionistic-linear-logic]{ILL}}{Modeling ILL}}\label{modeling-ill}

Introduced in\footnote{}.

\[\oc=L\circ M\]

This section is devoted to defining the concepts necessary to define
these adjunctions.

\&FA\textbackslash{}bullet(FB\textbackslash{}bullet
FC)\textbackslash{}ar{[}dr{]}\^{}\{FA\textbackslash{}bullet\textbackslash{}phi\_\{B,C\}\}\textbackslash{}\textbackslash{}

\texttt{~~~F(A\textbackslash{}otimes~B)\textbackslash{}bullet~FC\textbackslash{}ar{[}dr{]}\_\{\textbackslash{}phi\_\{A\textbackslash{}otimes~B,C\}\}\&\&FA\textbackslash{}bullet~F(B\textbackslash{}otimes~C)\textbackslash{}ar{[}d{]}\^{}\{\textbackslash{}phi\_\{A,B\textbackslash{}otimes~C\}\}\textbackslash{}\textbackslash{}}\\
\texttt{~~~\&F((A\textbackslash{}otimes~B)\textbackslash{}otimes~C)\textbackslash{}ar{[}r{]}\_\{F\textbackslash{}alpha\_\{A,B,C\}\}\&F(A\textbackslash{}otimes(B\textbackslash{}otimes~C))}

\} and

\[\xymatrix{
    FA\bullet J\ar[d]_{\rho_{FA}}\ar[r]^{FA\bullet\phi}&amp;FA\bullet FI\ar[d]^{\phi_{A,I}}\\
    FA&amp;\ar[l]^{F\rho_A}F(A\otimes I)
}\] and \(\xymatrix{
    J\bullet FB\ar[d]_{\lambda_{FB}}\ar[r]^{\phi\bullet FB}&amp;FI\bullet FB\ar[d]^{\phi_{I,B}}\\
    FB&amp;\ar[l]^{F\lambda_B}F(I\otimes B)
}\) commute for every objects \(A\), \(B\) and \(C\) of \(\mathcal{C}\).
The morphisms \(f_{A,B}\) and \(f\) are called \emph{coherence maps}.

A lax monoidal functor is \emph{strong} when the coherence maps are
invertible and \emph{strict} when they are identities. \}\}

\textbackslash{}ar{[}r{]}\^{}\{\textbackslash{}theta\_A\textbackslash{}bullet\textbackslash{}theta\_B\}\&\textbackslash{}ar{[}d{]}\^{}\{g\_\{A,B\}\}GA\textbackslash{}bullet
GB\textbackslash{}\textbackslash{}

\texttt{~~~F(A\textbackslash{}tens~B)\textbackslash{}ar{[}r{]}\_\{\textbackslash{}theta\_\{A\textbackslash{}tens~B\}\}\&G(A\textbackslash{}tens~B)}

\} and \(\xymatrix{
  &amp;\ar[dl]_{f}J\ar[dr]^{g}&amp;\\
  FI\ar[rr]_{\theta_I}&amp;&amp;GI
}\) commute for every objects \(A\) and \(B\) of \(\mathcal{D}\). \}\}

\section{Modeling negation}\label{modeling-negation}

\subsection{*-autonomous categories}\label{autonomous-categories}

\subsection{Compact closed categories}\label{compact-closed-categories}

\&(A\textbackslash{}tens B)\textbackslash{}tens
A\textbackslash{}ar{[}dr{]}\^{}\{\textbackslash{}varepsilon\textbackslash{}tens
A\}\textbackslash{}\textbackslash{} A\textbackslash{}tens
I\textbackslash{}ar{[}ur{]}\^{}\{A\textbackslash{}tens\textbackslash{}eta\}\&\&\&I\textbackslash{}tens
A\textbackslash{}ar{[}d{]}\^{}\{\textbackslash{}lambda\_A\}\textbackslash{}\textbackslash{}
A\textbackslash{}ar{[}u{]}\^{}\{\textbackslash{}rho\_A\^{}\{-1\}\}\textbackslash{}ar@\{=\}{[}rrr{]}\&\&\&A\textbackslash{}\textbackslash{}
\} and

\[\xymatrix{
&amp;(B\tens A)\tens B\ar[r]^{\alpha_{B,A,B}}&amp;B\tens(A\tens B)\ar[dr]^{B\tens\varepsilon}\\
I\tens B\ar[ur]^{\eta\tens B}&amp;&amp;&amp;B\tens I\ar[d]^{\rho_B}\\
B\ar[u]^{\lambda_B^{-1}}\ar@{=}[rrr]&amp;&amp;&amp;B\\
}\] commute. The object \(A\) is called a left dual of \(B\) (and
conversely \(B\) is a right dual of \(A\)). \}\}

\&I\textbackslash{}tens
B\textbackslash{}ar{[}r{]}\^{}-\{\textbackslash{}eta\_A\textbackslash{}tens
B\}\&(A\^{}*\textbackslash{}tens A)\textbackslash{}tens
B\textbackslash{}ar{[}r{]}\^{}-\{\textbackslash{}alpha\_\{A\^{}*,A,B\}\}\&A\^{}*\textbackslash{}tens(A\textbackslash{}tens
B)\textbackslash{}ar{[}r{]}\^{}-\{A\^{}*\textbackslash{}tens
f\}\&A\textbackslash{}tens C\textbackslash{}\textbackslash{} \} and to
every morphism \(g:B\to A^*\tens C\), we associate a morphism
\(\llcorner g\lrcorner:A\tens B\to C\) defined as

\[\xymatrix{
A\tens B\ar[r]^-{A\tens g}&amp;A\tens(A^*\tens C)\ar[r]^-{\alpha_{A,A^*,C}^{-1}}&amp;(A\tens A^*)\tens C\ar[r]^-{\varepsilon_A\tens C}&amp;I\tens C\ar[r]^-{\lambda_C}&amp;C
}\] It is easy to show that
\(\llcorner \ulcorner f\urcorner\lrcorner=f\) and
\(\ulcorner\llcorner g\lrcorner\urcorner=g\) from which we deduce the
required bijection. \}\}

\section{Other categorical models}\label{other-categorical-models}

\subsection{Lafont categories}\label{lafont-categories}

\subsection{Seely categories}\label{seely-categories}

\subsection{Linear categories}\label{linear-categories}

\section{Properties of categorical models}\label{properties-of-categorical-models}

\subsection{The Kleisli category}\label{the-kleisli-category}


%%% Local Variables:
%%% mode: latex
%%% TeX-master: "main"
%%% End:
