\section{Mix}\label{mix}

The usual notion of \(\rulename{Mix}\) is the binary version of the rule
but a nullary version also exists.

\subsection{\texorpdfstring{Binary \(\rulename{Mix}\) rule}{Binary \textbackslash{}rulename\{Mix\} rule}}\label{binary-rulenamemix-rule}

\begin{prooftree}
\AxRule{\vdash\Gamma}
\AxRule{\vdash\Delta}
\LabelRule{Mix_2}
\BinRule{\vdash\Gamma,\Delta}
\end{prooftree}

The \(\rulename{Mix_2}\) rule is equivalent to \(\bot\vdash\one\):
\begin{equation*}
\LabelRule{\one}
\NulRule{\vdash\one}
\LabelRule{\one}
\NulRule{\vdash\one}
\LabelRule{Mix_2}
\BinRule{\vdash\one,\one}
\DisplayProof
\qquad
\AxRule{\vdash\Gamma}
\LabelRule{\bot}
\UnaRule{\vdash\Gamma,\bot}
\AxRule{\vdash\one,\one}
\LabelRule{\rulename{cut}}
\BinRule{\vdash\Gamma,\one}
\AxRule{\vdash\Delta}
\LabelRule{\bot}
\UnaRule{\vdash\Delta,\bot}
\LabelRule{\rulename{cut}}
\BinRule{\vdash\Gamma,\Delta}
\DisplayProof
\end{equation*}

They are also equivalent to the principle \(A\tens B \vdash A\parr B\):
\begin{equation*}
\LabelRule{\one}
\NulRule{\vdash\one}
\LabelRule{\one}
\NulRule{\vdash\one}
\LabelRule{\tens}
\BinRule{\vdash\one\tens\one}
\AxRule{\vdash\bot\parr\bot,\one\parr\one}
\LabelRule{\rulename{cut}}
\BinRule{\vdash\one\parr\one}
\LabelRule{\rulename{ax}}
\NulRule{\vdash\bot,\one}
\LabelRule{\rulename{ax}}
\NulRule{\vdash\bot,\one}
\LabelRule{\tens}
\BinRule{\vdash\bot\tens\bot,\one,\one}
\LabelRule{\rulename{cut}}
\BinRule{\vdash\one,\one}
\DisplayProof
\qquad
\LabelRule{\rulename{ax}}
\NulRule{\vdash A\orth,A}
\LabelRule{\rulename{ax}}
\NulRule{\vdash B\orth,B}
\LabelRule{Mix_2}
\BinRule{\vdash A\orth,A,B\orth,B}
\LabelRule{\parr}
\UnaRule{\vdash A\orth,B\orth,A\parr B}
\LabelRule{\parr}
\UnaRule{\vdash A\orth\parr B\orth,A\parr B}
\DisplayProof
\end{equation*}

\subsection{\texorpdfstring{Nullary \(\rulename{Mix}\) rule}{Nullary \textbackslash{}rulename\{Mix\} rule}}\label{nullary-rulenamemix-rule}

\begin{prooftree}
\LabelRule{Mix_0}
\NulRule{\vdash}
\end{prooftree}

The \(\rulename{Mix_0}\) rule is equivalent to \(\one\vdash\bot\):
\begin{equation*}
\LabelRule{Mix_0}
\NulRule{\vdash}
\LabelRule{\bot}
\UnaRule{\vdash\bot}
\LabelRule{\bot}
\UnaRule{\vdash\bot,\bot}
\DisplayProof
\qquad
\LabelRule{\one}
\NulRule{\vdash\one}
\AxRule{\vdash\bot,\bot}
\LabelRule{\rulename{cut}}
\BinRule{\vdash\bot}
\LabelRule{\one}
\NulRule{\vdash\one}
\LabelRule{\rulename{cut}}
\BinRule{\vdash}
\DisplayProof
\end{equation*}

The nullary \(\rulename{Mix}\) acts as a unit for the binary one:
\begin{prooftree}
\AxRule{\vdash\Gamma}
\LabelRule{Mix_0}
\NulRule{\vdash}
\LabelRule{Mix_2}
\BinRule{\vdash\Gamma}
\end{prooftree}

If \(\pi\) is a proof which uses no \(\bot\) rule and no weakening rule,
then (up to the simplification of the pattern
\(\rulename{Mix_0}/\rulename{Mix_2}\) above into nothing) \(\pi\) is
either reduced to a \(\rulename{Mix_0}\) rule or does not contain any
\(\rulename{Mix_0}\) rule.


%%% Local Variables:
%%% mode: latex
%%% TeX-master: "main"
%%% End:
