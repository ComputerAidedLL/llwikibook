\section{Lattice of exponential modalities}\label{lattice-of-exponential-modalities}

\(\xymatrix{
 &amp; &amp; {\wn} \\
 &amp; &amp; &amp; &amp; {\wn\oc\wn}\ar[ull] \\
\varepsilon\ar[uurr] &amp; &amp; &amp; {\oc\wn} \ar[ur] &amp; &amp; {\wn\oc} \ar[ul] \\
 &amp; &amp; &amp; &amp; {\oc\wn\oc} \ar[ul]\ar[ur] \\
 &amp; &amp; {\oc} \ar[uull]\ar[urr]
}\)

An \emph{exponential modality} is an arbitrary (possibly empty) sequence
of the two exponential connectives \(\oc\) and \(\wn\). It can be
considered itself as a unary connective. This leads to the notation
\(\mu A\) for applying an exponential modality \(\mu\) to a formula
\(A\).

There is a preorder relation on exponential modalities defined by
\(\mu\lesssim\nu\) if and only if for any formula \(A\) we have
\(\mu A\vdash \nu A\). It induces an
\href{List_of_equivalences}{equivalence} relation on exponential
modalities by \(\mu \sim \nu\) if and only if \(\mu\lesssim\nu\) and
\(\nu\lesssim\mu\).

 and \(\wn{\wn{A}}\vdash\wn{A}\). \}\}

\} and \(\wn{\oc{\wn{A}}}\vdash\wn{A}\). \}\}

This allows to prove that any exponential modality is equivalent to one
of the following seven modalities: \(\varepsilon\) (the empty modality),
\(\oc\), \(\wn\), \(\oc\wn\), \(\wn\oc\), \(\oc\wn\oc\) or
\(\wn\oc\wn\). Indeed any sequence of consecutive \(\oc\) or \(\wn\) in
a modality can be simplified into only one occurrence, and then any
alternating sequence of length at least four can be simplified into a
smaller one.

\textbackslash{}vdash\textbackslash{}oc\{A\} by functoriality from
\(\oc{A}\vdash A\) (and similarly for \(\wn{A}\vdash\wn{\wn{A}}\)). From
\(\oc{A}\vdash \oc{\wn{\oc{A}}}\), we deduce
\(\wn{\oc{A}}\vdash \wn{\oc{\wn{\oc{A}}}}\) by functoriality and
\(\oc{\wn{B}}\vdash \oc{\wn{\oc{\wn{B}}}}\) (with \(A=\wn{B}\)). In a
similar way we have \(\oc{\wn{\oc{\wn{A}}}}\vdash \oc{\wn{A}}\) and
\(\wn{\oc{\wn{\oc{A}}}}\vdash \wn{\oc{A}}\). \}\}

The order relation induced on equivalence classes of exponential
modalities with respect to \(\sim\) can be proved to be the one
represented on the picture in the top of this page. All the represented
relations are valid.

\}. By functoriality we deduce \(\oc{\wn{\oc{A}}}\vdash \oc{\wn{A}}\)
and by \(A=\wn{\oc{B}}\) we deduce
\(\oc{\wn{\oc{B}}}\vdash \wn{\oc{B}}\).

The others are obtained from these ones by duality: \(A\vdash B\)
entails \(B\orth\vdash A\orth\). \}\}

\}. \}\}

We can prove that no other relation between classes is true (by relying
on the previous lemma).

\}.

Then \(\wn\) cannot be smaller than any other of the seven modalities
(since they are all smaller than \(\varepsilon\) or \(\wn\oc\wn\)). For
the same reason, \(\varepsilon\) cannot be smaller than \(\oc\),
\(\oc\wn\), \(\wn\oc\) or \(\oc\wn\oc\). This entails that \(\wn\oc\wn\)
is only smaller than \(\wn\) since it is not smaller than
\(\varepsilon\) (by duality from \(\varepsilon\) not smaller than
\(\oc\wn\oc\)).

From these, \(\wn{\oc{\alpha}}\not\vdash\oc{\wn{\alpha}}\) and
\(\oc{\wn{\alpha}}\not\vdash\wn{\oc{\alpha}}\), we deduce that no other
relation is possible. \}\}

The order relation on equivalence classes of exponential modalities is a
lattice.

