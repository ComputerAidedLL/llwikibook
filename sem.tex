\chapter{Semantics}\label{semantics}

Linear Logic has numerous semantics some of which are described in
details in the next sections.

\begin{itemize}
\item
  \hyperref[coherent-semantics]{Coherent semantics}
\item
  \hyperref[phase-semantics]{Phase semantics}
\item
  \hyperref[categorical-semantics]{Categorical semantics}
\item
  \hyperref[relational-semantics]{Relational semantics}
\item
  \hyperref[finiteness-semantics]{Finiteness semantics}
\item
  \hyperref[geometry-of-interaction]{Geometry of interaction}
\item
  \hyperref[game-semantics]{Game semantics}
\end{itemize}

\hyperref[provable-formulas]{Common properties} may be found in most of
these models. We will denote by \(A\longrightarrow B\) the fact that
there is a canonical morphism from \(A\) to \(B\) and by \(A\cong B\)
the fact that there is a canonical \hyperref[isomorphism]{isomorphism} between \(A\) and
\(B\). By "canonical" we mean that these (iso)morphisms are natural
transformations.


%%% Local Variables:
%%% mode: latex
%%% TeX-master: "main"
%%% End:
