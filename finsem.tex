\chapter{Finiteness semantics}\label{finiteness-semantics}

The category \(\mathbf{Fin}\) of finiteness spaces and finitary
relations was introduced by Ehrhard, refining the
\href{relational_semantics}{purely relational model of linear logic}. A
finiteness space is a set equipped with a finiteness structure, i.e. a
particular set of subsets which are said to be finitary; and the model
is such that the usual relational denotation of a proof in linear logic
is always a finitary subset of its conclusion. By the usual co-Kleisli
construction, this also provides a model of the simply typed
lambda-calculus: the cartesian closed category \(\mathbf{Fin}_\oc\).

The main property of finiteness spaces is that the intersection of two
finitary subsets of dual types is always finite. This feature allows to
reformulate Girard's quantitative semantics in a standard algebraic
setting, where morphisms interpreting typed \(\lambda\)-terms are
analytic functions between the topological vector spaces generated by
vectors with finitary supports. This provided the semantical foundations
of Ehrhard-Regnier's differential \(\lambda\)-calculus and motivated the
general study of a differential extension of linear logic.

It is worth noticing that finiteness spaces can accomodate typed
\(\lambda\)-calculi only: for instance, the relational semantics of
fixpoint combinators is never finitary. The whole point of the
finiteness construction is actually to reject infinite computations.
Indeed, from a logical point of view, computation is cut elimination:
the finiteness structure ensures the intermediate sets involved in the
relational interpretation of a cut are all finite. In that sense, the
finitary semantics is intrinsically typed.

\section{Finiteness spaces}\label{finiteness-spaces}

The construction of finiteness spaces follows a well known pattern. It
is given by the following notion of \hyperref[orthogonality-relation]{orthogonality}: \(a\mathrel \bot a'\)
iff \(a\cap a'\) is finite. Then one unrolls
\href{Orthogonality_relation}{familiar definitions}, as we do in the
following paragraphs.

Let \(A\) be a set. Denote by \(\powerset A\) the powerset of \(A\) and
by \(\finpowerset A\) the set of all finite subsets of \(A\). Let
\({\mathfrak F} \subseteq \powerset A\) any set of subsets of \(A\). We
define the pre-dual of \({\mathfrak F}\) in \(A\) as
\({\mathfrak F}^{\bot_{A}}=\left\{a'\subseteq A;\ \forall a\in{\mathfrak F},\ a\cap a'\in\finpowerset A\right\}\).
In general we will omit the subscript in the pre-dual notation and just
write \({\mathfrak F}\orth\). For all
\({\mathfrak F}\subseteq\powerset A\), we have the following immediate
properties: \(\finpowerset A\subseteq {\mathfrak F}\orth\);
\({\mathfrak F}\subseteq {\mathfrak F}\biorth\); if
\({\mathfrak G}\subseteq{\mathfrak F}\),
\({\mathfrak F}\orth\subseteq {\mathfrak G}\orth\). By the last two, we
get \({\mathfrak F}\orth = {\mathfrak F}\triorth\). A finiteness
structure on \(A\) is then a set \({\mathfrak F}\) of subsets of \(A\)
such that \({\mathfrak F}\biorth = {\mathfrak F}\).

A finiteness space is a dependant pair
\({\mathcal A}=\left(\web{\mathcal A},\mathfrak F\left(\mathcal A\right)\right)\)
where \(\web {\mathcal A}\) is the underlying set (the web of
\({\mathcal A}\)) and \(\mathfrak F\left(\mathcal A\right)\) is a
finiteness structure on \(\web {\mathcal A}\). We then write
\({\mathcal A}\orth\) for the dual finiteness space:
\(\web {{\mathcal A}\orth} = \web {\mathcal A}\) and
\(\mathfrak F\left({\mathcal A}\orth\right)=\mathfrak F\left({\mathcal A}\right)^{\bot}\).
The elements of \(\mathfrak F\left(\mathcal A\right)\) are called the
finitary subsets of \({\mathcal A}\).

\subparagraph{Example.}\label{example.}

\texttt{~~~For~all~set~}\(A\)\texttt{,~}\((A,\finpowerset A)\)\texttt{~is~a~finiteness~space~and~}\((A,\finpowerset A)\orth = (A,\powerset A)\)\texttt{.~~In~particular,~each~finite~set~}\(A\)\texttt{~is~the~web~of~exactly~one~finiteness~space:~}\((A,\finpowerset A)=(A,\powerset A)\)\texttt{.~We~introduce~the~following~two:~}\(\zero = \zero\orth = \left(\emptyset, \{\emptyset\}\right)\)\texttt{~and~}\(\one = \one\orth = \left(\{\emptyset\}, \{\emptyset, \{\emptyset\}\}\right)\)\texttt{.~~We~also~introduce~the~finiteness~space~of~natural~numbers~}\({\mathcal N}\)\texttt{~by:~~~~}\(|{\mathcal N}|={\mathbf N}\)\texttt{~and~}\(a\in\mathfrak F\left(\mathcal N\right)\)\texttt{~iff~}\(a\)\texttt{~is~finite.~~We~write~}\(\mathcal O=\{0\}\in\mathfrak F\left({\mathcal N}\right)\)\texttt{.}

Notice that \({\mathfrak F}\) is a finiteness structure iff it is of the
form \({\mathfrak G}\orth\). It follows that any finiteness structure
\({\mathfrak F}\) is downwards closed for inclusion, and closed under
finite unions and arbitrary intersections. Notice however that
\({\mathfrak F}\) is not closed under directed unions in general: for
all \(k\in{\mathbf N}\), write
\(k{\downarrow}=\left\{j;\  j\le k\right\}\in\mathfrak F\left({\mathcal N}\right)\);
then \(k{\downarrow}\subseteq k'{\downarrow}\) as soon as \(k\le k'\),
but
\(\bigcup_{k\ge0} k{\downarrow}={\mathbf N}\not\in\mathfrak F\left({\mathcal N}\right)\).

\subsubsection{Multiplicatives}\label{multiplicatives}

For all finiteness spaces \({\mathcal A}\) and \({\mathcal B}\), we
define \({\mathcal A} \tens {\mathcal B}\) by
\(\web {{\mathcal A} \tens {\mathcal B}} = \web{\mathcal A} \times \web{\mathcal B}\)
and
\(\mathfrak F\left({\mathcal A} \tens {\mathcal B}\right) = \left\{a\times b;\ a\in \mathfrak F\left(\mathcal A\right),\ b\in\mathfrak F\left(\mathcal B\right)\right\}\biorth\).
It can be shown that
\(\mathfrak F\left({\mathcal A} \tens {\mathcal B}\right) = \left\{ c \subseteq \web{\mathcal A}\times\web{\mathcal B};\  \left.c\right|_l\in \mathfrak F\left(\mathcal A\right),\ \left.c\right|_r\in\mathfrak F\left(\mathcal B\right)\right\}\),
where \(\left.c\right|_l\) and \(\left.c\right|_r\) are the obvious
projections.

Let \(f\subseteq A \times B\) be a relation from \(A\) to \(B\), we
write \(f\orth=\left\{(\beta,\alpha);\  (\alpha,\beta)\in f\right\}\).
For all \(a\subseteq A\), we set
\(f\cdot a = \left\{\beta\in B;\  \exists \alpha\in a,\ (\alpha,\beta)\in f\right\}\).
If moreover \(g\subseteq B \times C\), we define
\(g \bullet f = \left\{(\alpha,\gamma)\in A\times C;\  \exists \beta\in B,\ (\alpha,\beta)\in f\wedge(\beta,\gamma)\in g\right\}\).
Then, setting
\({\mathcal A}\limp{\mathcal B} = \left({\mathcal A}\otimes {\mathcal B}\orth\right)\orth\),
\(\mathfrak F\left({\mathcal A}\limp{\mathcal B}\right)\subseteq {\web{\mathcal A}\times\web{\mathcal B}}\)
is characterized as follows:

\begin{align}
        f\in \mathfrak F\left({\mathcal A}\limp{\mathcal B}\right) &amp;\iff \forall a\in \mathfrak F\left({\mathcal A}\right), f\cdot a \in\mathfrak F\left({\mathcal B}\right) \text{ and } \forall b\in \mathfrak F\left({\mathcal B}\orth\right), f\orth\cdot b \in\mathfrak F\left({\mathcal A}\orth\right)
        \\
        &amp;\iff \forall a\in \mathfrak F\left({\mathcal A}\right), f\cdot a \in\mathfrak F\left({\mathcal B}\right) \text{ and } \forall \beta\in \web{{\mathcal B}}, f\orth\cdot \left\{\beta\right\} \in\mathfrak F\left({\mathcal A}\orth\right)
        \\
        &amp;\iff \forall \alpha\in \web{{\mathcal A}}, f\cdot \left\{\alpha\right\} \in\mathfrak F\left({\mathcal B}\right) \text{ and } \forall b\in \mathfrak F\left({\mathcal B}\orth\right), f\orth\cdot b \in\mathfrak F\left({\mathcal A}\orth\right)
\end{align}

The elements of
\(\mathfrak F\left({\mathcal A}\limp{\mathcal B}\right)\) are called
finitary relations from \({\mathcal A}\) to \({\mathcal B}\). By the
previous characterization, the identity relation
\(\mathsf{id}_{{\mathcal A}} = \left\{(\alpha,\alpha);\  \alpha\in\web{{\mathcal A}}\right\}\)
is finitary, and the composition of two finitary relations is also
finitary. One can thus define the category \(\mathbf{Fin}\) of
finiteness spaces and finitary relations: the objects of
\(\mathbf{Fin}\) are all finiteness spaces, and
\(\mathbf{Fin}({\mathcal A},{\mathcal B})=\mathfrak F\left({\mathcal A}\limp{\mathcal B}\right)\).
Equipped with the tensor product \(\tens\), \(\mathbf{Fin}\) is
symmetric monoidal, with unit \(\one\); it is monoidal closed by the
definition of \(\limp\); it is \(*\)-autonomous by the obvious
isomorphism between \({\mathcal A}\orth\) and \({\mathcal A}\limp\one\).

\subparagraph{Example.}\label{example.-1}

\texttt{~~~Setting~}\(\mathcal{S}=\left\{(k,k+1);\  k\in{\mathbf N}\right\}\)\texttt{~and~}\(\mathcal{P}=\left\{(k+1,k);\  k\in{\mathbf N}\right\}\)\texttt{,~we~have~}\(\mathcal{S},\mathcal{P}\in\mathbf{Fin}({\mathcal N},{\mathcal N})\)\texttt{~and~}\(\mathcal{P}\bullet\mathcal{S}=\mathsf{id}_{{\mathcal N}}\)\texttt{.}

\subsubsection{Additives}\label{additives}

We now introduce the cartesian structure of \(\mathbf{Fin}\). We define
\({\mathcal A} \oplus {\mathcal B}\) by
\(\web {{\mathcal A} \oplus {\mathcal B}} = \web{\mathcal A} \uplus \web{\mathcal B}\)
and
\(\mathfrak F\left({\mathcal A} \oplus {\mathcal B}\right) = \left\{ a\uplus b;\  a\in \mathfrak F\left(\mathcal A\right),\ b\in\mathfrak F\left(\mathcal B\right)\right\}\)
where \(\uplus\) denotes the disjoint union of sets:
\(x\uplus y=(\{1\}\times x)\cup(\{2\}\times y)\). We have
\(\left({\mathcal A}\oplus {\mathcal B}\right)\orth = {\mathcal A}\orth\oplus{\mathcal B}\orth\).\footnote{The
  fact that the additive connectors are identified, i.e. that we obtain
  a biproduct, is to be related with the enrichment of \(\mathbf{Fin}\)
  over the monoid structure of set union: see This identification can
  also be shown to be a \url{isomorphism} of LL with sums of proofs.}
The category \(\mathbf{Fin}\) is both cartesian and co-cartesian, with
\(\oplus\) being the product and co-product, and \(\zero\) the initial
and terminal object. Projections are given by:

\begin{align}
\lambda_{{\mathcal A},{\mathcal B}}&amp;=\left\{\left((1,\alpha),\alpha\right);\ \alpha\in\web{\mathcal A}\right\}
\in\mathbf{Fin}({\mathcal A}\oplus{\mathcal B},{\mathcal A}) \\
\rho_{{\mathcal A},{\mathcal B}}&amp;=\left\{\left((2,\beta),\beta\right);\ \beta\in\web{\mathcal B}\right\}
\in\mathbf{Fin}({\mathcal A}\oplus{\mathcal B},{\mathcal B}) 
\end{align}

and if \(f\in\mathbf{Fin}({\mathcal C},{\mathcal A})\) and
\(g\in\mathbf{Fin}({\mathcal C},{\mathcal B})\), pairing is given by:

\(\left\langle f,g\right\rangle = \left\{\left(\gamma,(1,\alpha)\right);\ (\gamma,\alpha)\in f\right\} \cup \left\{\left(\gamma,(2,\beta)\right);\ (\gamma,\beta)\in g\right\} \in\mathbf{Fin}({\mathcal C},{\mathcal A}\oplus{\mathcal B}).\)

The unique morphism from \({\mathcal A}\) to \(\zero\) is the empty
relation. The co-cartesian structure is obtained symmetrically.

\subparagraph{Example.}\label{example.-2}

\texttt{~~~Write~}\({\mathcal O}\orth=\left\{(0,\emptyset)\right\}\in\mathbf{Fin}({\mathcal N},\one)\)\texttt{.~~Then~}\(\left\langle{{\mathcal O}\orth},{\mathcal{P}}\right\rangle =\{ (0,(1,\emptyset)) \}\cup \{ (k+1,(2,k)) ;\  k\in{\mathbf N} \} \in\mathbf{Fin}\left({\mathcal N},\one\oplus{\mathcal N}\right)\)\\
\texttt{~~~is~an~isomorphism.}\\
\texttt{~~~}\\
\texttt{~~~\ }

\subsubsection{Exponentials}\label{exponentials-1}

If \(A\) is a set, we denote by \(\finmulset A\) the set of all finite
multisets of elements of \(A\), and if \(a\subseteq A\), we write
\(a^{\oc}=\finmulset a\subseteq\finmulset A\). If
\(\overline\alpha\in\finmulset A\), we denote its support by
\(\mathrm{Support}\left(\overline \alpha\right)\in\finpowerset A\). For
all finiteness space \({\mathcal A}\), we define \(\oc {\mathcal A}\)
by: \(\web{\oc {\mathcal A}}= \finmulset{\web{{\mathcal A}}}\) and
\(\mathfrak F\left(\oc{\mathcal A}\right)=\left\{a^{\oc};\  a\in\mathfrak F\left({\mathcal A}\right)\right\}\biorth\).
It can be shown that
\(\mathfrak F\left(\oc{\mathcal A}\right) = \left\{\overline a\subseteq\finmulset{\web{{\mathcal A}}};\ \bigcup_{\overline\alpha\in \overline a}\mathrm{Support}\left(\overline \alpha\right)\in\mathfrak F\left(\mathcal A\right)\right\}\).
Then, for all \(f\in\mathbf{Fin}({\mathcal A},{\mathcal B})\), we set

\(\oc f =\left\{\left(\left[\alpha_1,\ldots,\alpha_n\right],\left[\beta_1,\ldots,\beta_n\right]\right);\  \forall i,\ (\alpha_i,\beta_i)\in f\right\} \in \mathbf{Fin}(\oc {\mathcal A}, \oc {\mathcal B}),\)

which defines a functor. Natural transformations
\(\mathsf{der}_{{\mathcal A}}=\left\{([\alpha],\alpha);\  \alpha\in \web{{\mathcal A}}\right\}\in\mathbf{Fin}(\oc{\mathcal A},{\mathcal A})\)
and
\(\mathsf{digg}_{{\mathcal A}}=\left\{\left(\sum_{i=1}^n\overline\alpha_i,\left[\overline\alpha_1,\ldots,\overline\alpha_n\right]\right);\ \forall i,\ \overline\alpha_i\in\web{\oc {\mathcal A}}\right\}\)
make this functor a comonad.

\subparagraph{Example.}\label{example.-3}

We have isomorphisms
\(\left\{([],\emptyset)\right\}\in\mathbf{Fin}(\oc\zero,\one)\) and

\(\left\{ \left(\overline\alpha_l+\overline\beta_r,\left(\overline\alpha,\overline\beta\right)\right);\ (\overline\alpha_l,\overline\alpha)\in\oc\lambda_{{\mathcal A},{\mathcal B}}\wedge(\overline\beta_r,\overline\beta)\in\oc\rho_{{\mathcal A},{\mathcal B}}\right\} \in\mathbf{Fin}(\oc({\mathcal A}\oplus{\mathcal B}),\oc{\mathcal A}\tens\oc{\mathcal B}).\)

More generally, we have
\(\oc\left({\mathcal A}_1\oplus\cdots\oplus{\mathcal A}_n\right)\cong\oc{\mathcal A}_1\tens\cdots\tens\oc{\mathcal A}_n\).


%%% Local Variables:
%%% mode: latex
%%% TeX-master: "main"
%%% End:
