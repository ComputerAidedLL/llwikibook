\chapter{Geometry of interaction}\label{geometry-of-interaction}

\section{Introduction}

The \emph{geometry of interaction}, GoI in short, was defined in the
early nineties by Girard as an interpretation of linear logic into
operators algebra: formulae were interpreted by Hilbert spaces and
proofs by partial isometries.

This was a striking novelty as it was the first time that a mathematical
model of logic (lambda-calculus) didn't interpret a proof of
\(A\limp B\) as a morphism \emph{from} \(A\) \emph{to} \(B\) and proof
composition (cut rule) as the composition of morphisms. Rather the proof
was interpreted as an operator acting \emph{on} \(A\limp B\), that is a
morphism from \(A\limp B\) to \(A\limp B\). For proof composition the
problem was then, given an operator on \(A\limp B\) and another one on
\(B\limp C\) to construct a new operator on \(A\limp C\). This problem
was solved by the \emph{execution formula} that bares some formal
analogies with Kleene's formula for recursive functions. For this reason
GoI was claimed to be an \emph{operational semantics}, as opposed to
traditionnal \href{Semantics}{denotational semantics}.

The first instance of the GoI was restricted to the \(MELL\) fragment of
linear logic (Multiplicative and Exponential fragment) which is enough
to encode lambda-calculus. Since then Girard proposed several
improvements: firstly the extension to the additive connectives known as
\emph{Geometry of Interaction 3} and more recently a complete
reformulation using Von Neumann algebras that allows to deal with some
aspects of \href{Light_linear_logics}{implicit complexity}

The GoI has been a source of inspiration for various authors. Danos and
Regnier have reformulated the original model exhibiting its
combinatorial nature using a theory of reduction of paths in proof-nets
and showing the link with abstract machines; the execution formula
appears as the composition of two automata interacting through a common
interface. Also the execution formula has rapidly been understood as
expressing the composition of strategies in game semantics. It has been
used in the theory of sharing reduction for lambda-calculus in the
Abadi-Gonthier-Lévy reformulation and simplification of Lamping's
representation of sharing. Finally the original GoI for the \(MELL\)
fragment has been reformulated in the framework of traced monoidal
categories following an idea originally proposed by Joyal.


\subsection{The Geometry of Interaction as operators}\label{the-geometry-of-interaction-as-operators}

The original construction of GoI by Girard follows a general pattern
already mentionned in the section on \href{coherent_semantics}{coherent
semantics} under the name \emph{symmetric reducibility} and that was
first put to use in \href{phase_semantics}{phase semantics}. First set a
general space \(P\) called the \emph{proof space} because this is where
the interpretations of proofs will live. Make sure that \(P\) is a (not
necessarily commutative) monoid. In the case of GoI, the proof space is
a subset of the space of bounded operators on \(\ell^2\).

Second define a particular subset of \(P\) that will be denoted by
\(\bot\); then derive a duality on \(P\): for \(u,v\in P\), \(u\) and
\(v\) are dual\footnote{In modern terms one says that \(u\) and \(v\)
  are \emph{polar}.}iff \(uv\in\bot\).

Such a duality defines an \href{orthogonality_relation}{orthogonality
relation}, with the usual derived definitions and properties.

For the GoI, two dualities have proved to work; we will consider the
first one: nilpotency, \emph{ie}, \(\bot\) is the set of nilpotent
operators in \(P\). Let us explicit this: two operators \(u\) and \(v\)
are dual if there is a nonnegative integer \(n\) such that
\((uv)^n = 0\). This duality is symmetric: if \(uv\) is nilpotent then
\(vu\) is nilpotent also.

Last define a \emph{type} as a subset \(T\) of the proof space that is
equal to its bidual: \(T = T\biorth\). This means that \(u\in T\) iff
for all operator \(v\in T\orth\), that is such that \(u'v\in\bot\) for
all \(u'\in T\), we have \(uv\in\bot\).

The real work\footnote{The difficulty is to find the right duality that
  will make logical operations interpretable. General conditions that
  allows to achieve this have been formulated by Hyland and Schalk
  thanks to their theory of \emph{\href{double_glueing}{double
  glueing}}.}is now to interpret logical operations, that is to
associate a type to each formula, an object to each proof and show the
\emph{adequacy lemma}: if \(u\) is the interpretation of a proof of the
formula \(A\) then \(u\) belongs to the type associated to \(A\).

\subsubsection{\texorpdfstring{\href{GoI_for_MELL:_partial_isometries}{Partial isometries}}{Partial isometries}}\label{partial-isometries}

The first step is to build the proof space. This is constructed as a
special set of partial isometries on a separable Hilbert space \(H\)
which turns out to be generated by partial permutations on the canonical
basis of \(H\).

These so-called \emph{\(p\)-isometries} enjoy some nice properties, the
most important one being that a \(p\)-isometry is a sum of
\(p\)-isometries iff all the terms of the sum have disjoint domains and
disjoint codomains. As a consequence we get that a sum of
\(p\)-isometries is null iff each term of the sum is null.

A second important property is that operators on \(H\) can be
\emph{externalized} using \(p\)-isometries into operators acting on
\(H\oplus H\), and conversely operators on \(H\oplus H\) may be
\emph{internalized} into operators on \(H\). This is widely used in the
sequel.

\subsubsection{\texorpdfstring{\href{GoI_for_MELL:_the_*-autonomous_structure}{The *-autonomous structure}}{The *-autonomous structure}}\label{the--autonomous-structure}

The second step is to interpret the linear logic multiplicative
operations, most importantly the cut rule.

Internalization/externalization is the key for this: typically the type
\(A\tens B\) is interpreted by a set of \(p\)-isometries which are
internalizations of operators acting on \(H\oplus H\).

The (interpretation of) the cut-rule is defined in two steps: firstly we
use nilpotency to define an operation corresponding to lambda-calculus
application which given two \(p\)-isometries in respectively
\(A\limp B\) and \(A\) produces an operator in \(B\). From this we
deduce the composition and finally obtain a structure of *-autonomous
category, that is a model of multiplicative linear logic.

\subsubsection{\texorpdfstring{\href{GoI_for_MELL:_exponentials}{The exponentials}}{The exponentials}}\label{the-exponentials}

Finally we turn to define exponentials, that is connectives managing
duplication. To do this we introduce an isomorphism (induced by a
\(p\)-isometry) between \(H\) and \(H\tens H\): the first component of
the tensor is intended to hold the address of the the copy whereas the
second component contains the content of the copy.

We eventually get a quasi-model of full MELL; quasi in the sense that if
we can construct \(p\)-isometries for usual structural operations in
MELL (contraction, dereliction, digging), the interpretation of linear
logic proofs is not invariant w.r.t. cut elimination in general. It is
however invariant in some good cases, which are enough to get a
correction theorem for the interpretation.


\subsection{The Geometry of Interaction as an abstract machine}\label{the-geometry-of-interaction-as-an-abstract-machine}


%%% Local Variables:
%%% mode: latex
%%% TeX-master: "main"
%%% End:
