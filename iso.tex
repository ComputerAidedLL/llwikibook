\section{Isomorphism}\label{isomorphism}

\subsection{Definition}

Two formulas \(A\) and \(B\) are isomorphic (denoted \(A\cong B\)), when
there are two proofs \(\pi\) of \(A \vdash B\) and \(\rho\) of
\(B \vdash A\) such that eliminating the cut on \(A\) in

\(\AxRule{}\VdotsRule{\pi}{A \vdash B}\AxRule{}\VdotsRule{\rho}{B \vdash A}\LabelRule{\rulename{cut}}\BinRule{B\vdash B}\DisplayProof\)

leads to an
\hyperref[expansion-of-identities]{\(\eta\)-expansion} of

\(\LabelRule{\rulename{ax}}\NulRule{B\vdash B}\DisplayProof\),

and eliminating the cut on \(B\) in

\(\AxRule{}\VdotsRule{\pi}{A \vdash B}\AxRule{}\VdotsRule{\rho}{B \vdash A}\LabelRule{\rulename{cut}}\BinRule{A\vdash A}\DisplayProof\)

leads to an
\hyperref[expansion-of-identities]{\(\eta\)-expansion} of

\(\LabelRule{\rulename{ax}}\NulRule{A\vdash A}\DisplayProof\).

Linear logic admits \hyperref[list-of-isomorphisms]{many isomorphisms}, but
it is not known wether all of them have been discovered or not.


%%% Local Variables:
%%% mode: latex
%%% TeX-master: "main"
%%% End:
