\section{Categorical semantics}\label{categorical-semantics}

Constructing denotational models of linear can be a tedious work.
Categorical semantics are useful to identify the fundamental structure
of these models, and thus simplify and make more abstract the
elaboration of those models.

\texttt{TODO:~why~categories?~how~to~extract~categorical~models?~etc.}

See \footnote{}for a more detailed introduction to category theory. See
\footnote{}for a detailed treatment of categorical semantics of linear
logic.

\subsection{Basic category theory recalled}\label{basic-category-theory-recalled}

\subsection{Overview}\label{overview}

In order to interpret the various \href{fragment}{fragments} of linear
logic, we define incrementally what structure we need in a categorical
setting.

\begin{itemize}
\tightlist
\item
  The most basic underlying structure are \textbf{symmetric monoidal
  categories} which model the symmetric tensor \(\otimes\) and its unit
  \(1\).
\item
  The \(\otimes, \multimap\) fragment (\url{IMLL}) is captured by
  so-called \textbf{symmetric monoidal closed categories}.
\item
  Upgrading to \url{ILL}, that is, adding the exponential \(\oc\)
  modality to IMLL requires modelling it categorically. There are
  various ways to do so: using rich enough \textbf{adjunctions}, or with
  an ad-hoc definition of a well-behaved comonad which leads to
  \textbf{linear categories} and close relatives.
\item
  Dealing with the additives \(\with, \oplus\) is quite easy, as they
  are plain \textbf{cartesian product} and \textbf{coproduct}, usually
  defined through universal properties in category theory.
\item
  Retrieving \(\parr\), \(\bot\) and \(\wn\) is just a matter of
  dualizing \(\otimes\), \(1\) and \(\oc\), thus requiring the model to
  be a \textbf{*-autonomous category} for that purpose.
\end{itemize}

\subsection{\texorpdfstring{Modeling \url{IMLL}}{Modeling IMLL}}\label{modeling-imll}

A model of \url{IMLL} is a \emph{closed symmetric monoidal category}. We
recall the definition of these categories below.

\textbackslash{}ar{[}r{]}\^{}\{\textbackslash{}alpha\_\{A,B,C\}\textbackslash{}otimes
D\}\&(A\textbackslash{}otimes(B\textbackslash{}otimes
C))\textbackslash{}otimes
D\textbackslash{}ar{[}r{]}\^{}\{\textbackslash{}alpha\_\{A,B\textbackslash{}otimes
C,D\}\}\&A\textbackslash{}otimes((B\textbackslash{}otimes
C)\textbackslash{}otimes
D)\textbackslash{}ar{[}d{]}\^{}\{A\textbackslash{}otimes\textbackslash{}alpha\_\{B,C,D\}\}\textbackslash{}\textbackslash{}

\texttt{~~~(A\textbackslash{}otimes~B)\textbackslash{}otimes(C\textbackslash{}otimes~D)\textbackslash{}ar{[}rr{]}\_\{\textbackslash{}alpha\_\{A,B,C\textbackslash{}otimes~D\}\}\&\&A\textbackslash{}otimes(B\textbackslash{}otimes~(C\textbackslash{}otimes~D))}

\} commutes,

\begin{itemize}
\tightlist
\item
  for every objects \(A\) and \(B\) in \(\mathcal{C}\), the diagram
\end{itemize}

\[\xymatrix{
    (A\otimes I)\otimes B\ar[dr]_{\rho_A\otimes B}\ar[rr]^{\alpha_{A,I,B}}&amp;&amp;\ar[dl]^{A\otimes\lambda_B}A\otimes(I\otimes B)\\
    &amp;A\otimes B&amp;
}\] commutes. \}\}

\&(B\textbackslash{}otimes C)\textbackslash{}otimes
A\textbackslash{}ar{[}dr{]}\^{}\{\textbackslash{}alpha\_\{B,C,A\}\}\textbackslash{}\textbackslash{}
(A\textbackslash{}otimes B)\textbackslash{}otimes
C\textbackslash{}ar{[}ur{]}\^{}\{\textbackslash{}alpha\_\{A,B,C\}\}\textbackslash{}ar{[}dr{]}\_\{\textbackslash{}gamma\_\{A,B\}\textbackslash{}otimes
C\}\&\&\&B\textbackslash{}otimes (C\textbackslash{}otimes
A)\textbackslash{}\textbackslash{} \&(B\textbackslash{}otimes
A)\textbackslash{}otimes
C\textbackslash{}ar{[}r{]}\_\{\textbackslash{}alpha\_\{B,A,C\}\}\&B\textbackslash{}otimes(A\textbackslash{}otimes
C)\textbackslash{}ar{[}ur{]}\_\{B\textbackslash{}otimes\textbackslash{}gamma\_\{A,C\}\}\textbackslash{}\textbackslash{}
\} and

\[\xymatrix{
&amp;(A\otimes B)\otimes C\ar[r]^{\gamma_{A\otimes B,C}}&amp;C\otimes (A\otimes B)\ar[dr]^{\alpha^{-1}_{C,A,B}}&amp;\\
A\otimes (B\otimes C)\ar[ur]^{\alpha^{-1}_{A,B,C}}\ar[dr]_{A\otimes\gamma_{B,C}}&amp;&amp;&amp;(C\otimes A)\otimes B\\
&amp;A\otimes(C\otimes B)\ar[r]_{\alpha^{-1}_{A,C,B}}&amp;(A\otimes C)\otimes B\ar[ur]_{\gamma_{A,C}\otimes B}&amp;\\
}\] commute for every objects \(A\), \(B\) and \(C\).

A \emph{symmetric} monoidal category is a braided monoidal category in
which the braiding satisfies

\[\gamma_{B,A}\circ\gamma_{A,B}=A\otimes B\] for every objects \(A\) and
\(B\). \}\}

\&B \} commute.

Dually, the monoidal category \(\mathcal{C}\) is \emph{right closed}
when the functor \(B\mapsto B\otimes A\) admits a right adjoint. The
notion of \emph{right closed structure} can be defined similarly. \}\}

In a symmetric monoidal category, a left closed structure induces a
right closed structure and conversely, allowing us to simply speak of a
\emph{closed symmetric monoidal category}.

\subsection{Modeling the additives}\label{modeling-the-additives}

A category has \emph{finite products} when it has products and a
terminal object.

\textbackslash{}ar{[}r{]}\^{}-\{\textbackslash{}mu\textbackslash{}tens
M\}\&M\textbackslash{}tens
M\textbackslash{}ar{[}dd{]}\^{}\{\textbackslash{}mu\}\textbackslash{}\textbackslash{}
M\textbackslash{}tens(M\textbackslash{}tens
M)\textbackslash{}ar{[}d{]}\_\{M\textbackslash{}tens\textbackslash{}mu\}\&\&\textbackslash{}\textbackslash{}
M\textbackslash{}tens
M\textbackslash{}ar{[}rr{]}\_\{\textbackslash{}mu\}\&\&M\textbackslash{}\textbackslash{}
\} and

\[\xymatrix{
I\tens M\ar[r]^{\eta\tens M}\ar[dr]_{\lambda_M}&amp;M\tens M\ar[d]_\mu&amp;\ar[l]_{M\tens\eta}\ar[dl]^{\rho_M}M\tens I\\
&amp;M&amp;
}\] commute. \}\}

\subsection{\texorpdfstring{Modeling \url{ILL}}{Modeling ILL}}\label{modeling-ill}

Introduced in\footnote{}.

\todo[citation]

\[\oc=L\circ M\]

This section is devoted to defining the concepts necessary to define
these adjunctions.

\&FA\textbackslash{}bullet(FB\textbackslash{}bullet
FC)\textbackslash{}ar{[}dr{]}\^{}\{FA\textbackslash{}bullet\textbackslash{}phi\_\{B,C\}\}\textbackslash{}\textbackslash{}

\texttt{~~~F(A\textbackslash{}otimes~B)\textbackslash{}bullet~FC\textbackslash{}ar{[}dr{]}\_\{\textbackslash{}phi\_\{A\textbackslash{}otimes~B,C\}\}\&\&FA\textbackslash{}bullet~F(B\textbackslash{}otimes~C)\textbackslash{}ar{[}d{]}\^{}\{\textbackslash{}phi\_\{A,B\textbackslash{}otimes~C\}\}\textbackslash{}\textbackslash{}}\\
\texttt{~~~\&F((A\textbackslash{}otimes~B)\textbackslash{}otimes~C)\textbackslash{}ar{[}r{]}\_\{F\textbackslash{}alpha\_\{A,B,C\}\}\&F(A\textbackslash{}otimes(B\textbackslash{}otimes~C))}

\} and

\[\xymatrix{
    FA\bullet J\ar[d]_{\rho_{FA}}\ar[r]^{FA\bullet\phi}&amp;FA\bullet FI\ar[d]^{\phi_{A,I}}\\
    FA&amp;\ar[l]^{F\rho_A}F(A\otimes I)
}\] and \(\xymatrix{
    J\bullet FB\ar[d]_{\lambda_{FB}}\ar[r]^{\phi\bullet FB}&amp;FI\bullet FB\ar[d]^{\phi_{I,B}}\\
    FB&amp;\ar[l]^{F\lambda_B}F(I\otimes B)
}\) commute for every objects \(A\), \(B\) and \(C\) of \(\mathcal{C}\).
The morphisms \(f_{A,B}\) and \(f\) are called \emph{coherence maps}.

A lax monoidal functor is \emph{strong} when the coherence maps are
invertible and \emph{strict} when they are identities. \}\}

\textbackslash{}ar{[}r{]}\^{}\{\textbackslash{}theta\_A\textbackslash{}bullet\textbackslash{}theta\_B\}\&\textbackslash{}ar{[}d{]}\^{}\{g\_\{A,B\}\}GA\textbackslash{}bullet
GB\textbackslash{}\textbackslash{}

\texttt{~~~F(A\textbackslash{}tens~B)\textbackslash{}ar{[}r{]}\_\{\textbackslash{}theta\_\{A\textbackslash{}tens~B\}\}\&G(A\textbackslash{}tens~B)}

\} and \(\xymatrix{
  &amp;\ar[dl]_{f}J\ar[dr]^{g}&amp;\\
  FI\ar[rr]_{\theta_I}&amp;&amp;GI
}\) commute for every objects \(A\) and \(B\) of \(\mathcal{D}\). \}\}

\subsection{Modeling negation}\label{modeling-negation}

\subsubsection{*-autonomous categories}\label{autonomous-categories}

\subsubsection{Compact closed categories}\label{compact-closed-categories}

\&(A\textbackslash{}tens B)\textbackslash{}tens
A\textbackslash{}ar{[}dr{]}\^{}\{\textbackslash{}varepsilon\textbackslash{}tens
A\}\textbackslash{}\textbackslash{} A\textbackslash{}tens
I\textbackslash{}ar{[}ur{]}\^{}\{A\textbackslash{}tens\textbackslash{}eta\}\&\&\&I\textbackslash{}tens
A\textbackslash{}ar{[}d{]}\^{}\{\textbackslash{}lambda\_A\}\textbackslash{}\textbackslash{}
A\textbackslash{}ar{[}u{]}\^{}\{\textbackslash{}rho\_A\^{}\{-1\}\}\textbackslash{}ar@\{=\}{[}rrr{]}\&\&\&A\textbackslash{}\textbackslash{}
\} and

\[\xymatrix{
&amp;(B\tens A)\tens B\ar[r]^{\alpha_{B,A,B}}&amp;B\tens(A\tens B)\ar[dr]^{B\tens\varepsilon}\\
I\tens B\ar[ur]^{\eta\tens B}&amp;&amp;&amp;B\tens I\ar[d]^{\rho_B}\\
B\ar[u]^{\lambda_B^{-1}}\ar@{=}[rrr]&amp;&amp;&amp;B\\
}\] commute. The object \(A\) is called a left dual of \(B\) (and
conversely \(B\) is a right dual of \(A\)). \}\}

\&I\textbackslash{}tens
B\textbackslash{}ar{[}r{]}\^{}-\{\textbackslash{}eta\_A\textbackslash{}tens
B\}\&(A\^{}*\textbackslash{}tens A)\textbackslash{}tens
B\textbackslash{}ar{[}r{]}\^{}-\{\textbackslash{}alpha\_\{A\^{}*,A,B\}\}\&A\^{}*\textbackslash{}tens(A\textbackslash{}tens
B)\textbackslash{}ar{[}r{]}\^{}-\{A\^{}*\textbackslash{}tens
f\}\&A\textbackslash{}tens C\textbackslash{}\textbackslash{} \} and to
every morphism \(g:B\to A^*\tens C\), we associate a morphism
\(\llcorner g\lrcorner:A\tens B\to C\) defined as

\[\xymatrix{
A\tens B\ar[r]^-{A\tens g}&amp;A\tens(A^*\tens C)\ar[r]^-{\alpha_{A,A^*,C}^{-1}}&amp;(A\tens A^*)\tens C\ar[r]^-{\varepsilon_A\tens C}&amp;I\tens C\ar[r]^-{\lambda_C}&amp;C
}\] It is easy to show that
\(\llcorner \ulcorner f\urcorner\lrcorner=f\) and
\(\ulcorner\llcorner g\lrcorner\urcorner=g\) from which we deduce the
required bijection. \}\}

\subsection{Other categorical models}\label{other-categorical-models}

\subsubsection{Lafont categories}\label{lafont-categories}

\subsubsection{Seely categories}\label{seely-categories}

\subsubsection{Linear categories}\label{linear-categories}

\subsection{Properties of categorical models}\label{properties-of-categorical-models}

\subsubsection{The Kleisli category}\label{the-kleisli-category}
