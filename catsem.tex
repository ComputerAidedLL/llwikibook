\chapter{Categorical semantics}\label{categorical-semantics}

Constructing denotational models of linear can be a tedious work.
Categorical semantics are useful to identify the fundamental structure
of these models, and thus simplify and make more abstract the
elaboration of those models.

\begin{quotation}
\texttt{TODO: why categories? how to extract categorical models? etc.}
\end{quotation}

See~\cite{categoriesworkmath} for a more detailed introduction to category theory. See~\cite{catsemll} for a detailed treatment of categorical semantics of linear logic.

\section{Basic category theory recalled}\label{basic-category-theory-recalled}

\begin{definition}[Category]
\end{definition}

\begin{definition}[Functor]
\end{definition}

\begin{definition}[Natural transformation]
\end{definition}

\begin{definition}[Adjunction]
\end{definition}


\begin{definition}[Monad]
\end{definition}



\section{Overview}\label{overview}

In order to interpret the various \hyperref[fragment]{fragments} of linear
logic, we define incrementally what structure we need in a categorical
setting.

\begin{itemize}
\item The most basic underlying structure are \textbf{symmetric monoidal
  categories} which model the symmetric tensor \(\otimes\) and its unit
  \(1\).
\item The \(\otimes, \multimap\) fragment (\wantedpage{IMLL}) is captured by
  so-called \textbf{symmetric monoidal closed categories}.
\item Upgrading to \hyperref[intuitionistic-linear-logic]{ILL}, that is, adding the exponential \(\oc\)
  modality to IMLL requires modelling it categorically. There are
  various ways to do so: using rich enough \textbf{adjunctions}, or with
  an ad-hoc definition of a well-behaved comonad which leads to
  \textbf{linear categories} and close relatives.
\item Dealing with the additives \(\with, \oplus\) is quite easy, as they
  are plain \textbf{cartesian product} and \textbf{coproduct}, usually
  defined through universal properties in category theory.
\item Retrieving \(\parr\), \(\bot\) and \(\wn\) is just a matter of
  dualizing \(\otimes\), \(1\) and \(\oc\), thus requiring the model to
  be a \textbf{*-autonomous category} for that purpose.
\end{itemize}

\section{\texorpdfstring{Modeling \wantedpage{IMLL}}{Modeling IMLL}}\label{modeling-imll}

A model of \wantedpage{IMLL} is a \emph{closed symmetric monoidal category}. We
recall the definition of these categories below.

\begin{definition}[Monoidal category]
A \emph{monoidal category} $(\mathcal{C},\otimes,I,\alpha,\lambda,\rho)$ is a category $\mathcal{C}$ equipped with
\begin{itemize}
\item a functor $\otimes:\mathcal{C}\times\mathcal{C}\to\mathcal{C}$ called \emph{tensor product},
\item an object $I$ called \emph{unit object},
\item three natural isomorphisms $\alpha$, $\lambda$ and $\rho$, called respectively \emph{associator}, \emph{left unitor} and \emph{right unitor}, whose components are
\begin{equation*}
\alpha_{A,B,C}:(A\otimes B)\otimes C\to A\otimes (B\otimes C)
\qquad
\lambda_A:I\otimes A\to A
\qquad
\rho_A:A\otimes I\to A
\end{equation*}
\end{itemize}
such that
\begin{itemize}
\item for every objects $A,B,C,D$ in $\mathcal{C}$, the diagram
\begin{equation*}
\xymatrix{
    ((A\otimes B)\otimes C)\otimes D\ar[d]_{\alpha_{A\otimes B,C,D}}\ar[r]^{\alpha_{A,B,C}\otimes D}& (A\otimes(B\otimes C))\otimes D\ar[r]^{\alpha_{A,B\otimes C,D}}& A\otimes((B\otimes C)\otimes D)\ar[d]^{A\otimes\alpha_{B,C,D}}\\
    (A\otimes B)\otimes(C\otimes D)\ar[rr]_{\alpha_{A,B,C\otimes D}}& & A\otimes(B\otimes (C\otimes D))
}
\end{equation*}
commutes,
\item for every objects $A$ and $B$ in $\mathcal{C}$, the diagram
\begin{equation*}
\xymatrix{
    (A\otimes I)\otimes B\ar[dr]_{\rho_A\otimes B}\ar[rr]^{\alpha_{A,I,B}}& & \ar[dl]^{A\otimes\lambda_B}A\otimes(I\otimes B)\\
    & A\otimes B& 
}
\end{equation*}
commutes.
\end{itemize}
\end{definition}

\begin{definition}[Braided, symmetric monoidal category]
A \emph{braided} monoidal category is a category together with a natural isomorphism of components
\begin{equation*}
\gamma_{A,B}:A\otimes B\to B\otimes A
\end{equation*}
called \emph{braiding}, such that the two diagrams
\begin{equation*}
\xymatrix{
& A\otimes(B\otimes C)\ar[r]^{\gamma_{A,B\otimes C}}& (B\otimes C)\otimes A\ar[dr]^{\alpha_{B,C,A}}\\
(A\otimes B)\otimes C\ar[ur]^{\alpha_{A,B,C}}\ar[dr]_{\gamma_{A,B}\otimes C}& & & B\otimes (C\otimes A)\\
& (B\otimes A)\otimes C\ar[r]_{\alpha_{B,A,C}}& B\otimes(A\otimes C)\ar[ur]_{B\otimes\gamma_{A,C}}\\
}
\end{equation*}
and
\begin{equation*}
\xymatrix{
& (A\otimes B)\otimes C\ar[r]^{\gamma_{A\otimes B,C}}& C\otimes (A\otimes B)\ar[dr]^{\alpha^{-1}_{C,A,B}}& \\
A\otimes (B\otimes C)\ar[ur]^{\alpha^{-1}_{A,B,C}}\ar[dr]_{A\otimes\gamma_{B,C}}& & & (C\otimes A)\otimes B\\
& A\otimes(C\otimes B)\ar[r]_{\alpha^{-1}_{A,C,B}}& (A\otimes C)\otimes B\ar[ur]_{\gamma_{A,C}\otimes B}& \\
}
\end{equation*}
commute for every objects $A$, $B$ and $C$.

A \emph{symmetric} monoidal category is a braided monoidal category in which the braiding satisfies
\begin{equation*}
\gamma_{B,A}\circ\gamma_{A,B}=A\otimes B
\end{equation*}
for every objects $A$ and $B$.
\end{definition}

\begin{definition}[Closed monoidal category]
A monoidal category $(\mathcal{C},\tens,I)$ is \emph{left closed} when for every object $A$, the functor $B\mapsto A\otimes B$ has a right adjoint, written $B\mapsto(A\limp B)$.
This means that there exists a bijection
\begin{equation*}
\mathcal{C}(A\tens B, C) \cong \mathcal{C}(B,A\limp C)
\end{equation*}
which is natural in $B$ and $C$.
Equivalently, a monoidal category is left closed when it is equipped with a \emph{left closed structure}, which consists of
\begin{itemize}
\item an object $A\limp B$,
\item a morphism $\mathrm{eval}_{A,B}:A\tens (A\limp B)\to B$, called \emph{left evaluation},
\end{itemize}
for every objects $A$ and $B$, such that for every morphism $f:A\otimes X\to B$ there exists a unique morphism $h:X\to A\limp B$ making the diagram
\begin{equation*}
\xymatrix{
A\tens X\ar@{.>}[d]_{A\tens h}\ar[dr]^{f}\\
A\tens(A\limp B)\ar[r]_-{\mathrm{eval}_{A,B}}& B
}
\end{equation*}
commute.

Dually, the monoidal category $\mathcal{C}$ is \emph{right closed} when the functor $B\mapsto B\otimes A$ admits a right adjoint. The notion of \emph{right closed structure} can be defined similarly.
\end{definition}

In a symmetric monoidal category, a left closed structure induces a
right closed structure and conversely, allowing us to simply speak of a
\emph{closed symmetric monoidal category}.

\section{Modeling the additives}\label{modeling-the-additives}

\begin{definition}[Product]
A \emph{product} $(X,\pi_1,\pi_2)$ of two coinitial morphisms $f:A\to B$ and $g:A\to C$ in a category $\mathcal{C}$ is an object $X$ of $\mathcal{C}$ together with two morphisms $\pi_1:X\to A$ and $\pi_2:X\to B$ such that there exists a unique morphism $h:A\to X$ making the diagram
\begin{equation*}
\xymatrix{
& \ar[ddl]_fA\ar@{.>}[d]_h\ar[ddr]^g& \\
& \ar[dl]^{\pi_1}X\ar[dr]_{\pi_2}& \\
B& & C
}
\end{equation*}
commute.
\end{definition}

A category has \emph{finite products} when it has products and a terminal object.

\begin{definition}[Monoid]
A \emph{monoid} $(M,\mu,\eta)$ in a monoidal category $(\mathcal{C},\tens,I)$ is an object $M$ together with two morphisms $\mu:M\tens M \to M$ and $\eta:I\to M$
such that the diagrams
\begin{equation*}
\xymatrix{
& (M\tens M)\tens M\ar[dl]_{\alpha_{M,M,M}}\ar[r]^-{\mu\tens M}& M\tens M\ar[dd]^{\mu}\\
M\tens(M\tens M)\ar[d]_{M\tens\mu}& & \\
M\tens M\ar[rr]_{\mu}& & M\\
}
\end{equation*}
and
\begin{equation*}
\xymatrix{
I\tens M\ar[r]^{\eta\tens M}\ar[dr]_{\lambda_M}& M\tens M\ar[d]_\mu& \ar[l]_{M\tens\eta}\ar[dl]^{\rho_M}M\tens I\\
& M& 
}
\end{equation*}
commute.
\end{definition}

\begin{property}
Categories with products vs monoidal categories.
\end{property}

\todo[missing in LLWiki]


\section{\texorpdfstring{Modeling \hyperref[intuitionistic-linear-logic]{ILL}}{Modeling ILL}}\label{modeling-ill}

Introduced in~\cite{mixedlnll}.

\begin{definition}[Linear-non linear (LNL) adjunction]
A \emph{linear-non linear adjunction} is a symmetric monoidal adjunction between lax monoidal functors
\begin{equation*}
\xymatrix{
(\mathcal{M},\times,\top)\ar@/^/[rr]^{(L,l)}& \bot& \ar@/^/[ll]^{(M,m)}(\mathcal{L},\otimes,I)
}
\end{equation*}
in which the category $\mathcal{M}$ has finite products.
\end{definition}

\[\oc=L\circ M\]

This section is devoted to defining the concepts necessary to define
these adjunctions.

\begin{definition}[Monoidal functor]
A \emph{lax monoidal functor} $(F,f)$ between two monoidal categories $(\mathcal{C},\tens,I)$ and $(\mathcal{D},\bullet,J)$ consists of
\begin{itemize}
\item a functor $F:\mathcal{C}\to\mathcal{D}$ between the underlying categories,
\item a natural transformation $f$ of components $f_{A,B}:FA\bullet FB\to F(A\tens B)$,
\item a morphism $f:J\to FI$
\end{itemize}
such that the diagrams
\begin{equation*}
\xymatrix{
    (FA\bullet FB)\bullet FC\ar[d]_{\phi_{A,B}\bullet FC}\ar[r]^{\alpha_{FA,FB,FC}}& FA\bullet(FB\bullet FC)\ar[dr]^{FA\bullet\phi_{B,C}}\\
    F(A\otimes B)\bullet FC\ar[dr]_{\phi_{A\otimes B,C}}& & FA\bullet F(B\otimes C)\ar[d]^{\phi_{A,B\otimes C}}\\
    & F((A\otimes B)\otimes C)\ar[r]_{F\alpha_{A,B,C}}& F(A\otimes(B\otimes C))
}
\end{equation*}
and
\begin{equation*}
\vcenter{\vbox{\xymatrix{
    FA\bullet J\ar[d]_{\rho_{FA}}\ar[r]^{FA\bullet\phi}& FA\bullet FI\ar[d]^{\phi_{A,I}}\\
    FA& \ar[l]^{F\rho_A}F(A\otimes I)
}}}
\qquad\text{and}\qquad
\vcenter{\vbox{\xymatrix{
    J\bullet FB\ar[d]_{\lambda_{FB}}\ar[r]^{\phi\bullet FB}& FI\bullet FB\ar[d]^{\phi_{I,B}}\\
    FB& \ar[l]^{F\lambda_B}F(I\otimes B)
}}}
\end{equation*}
commute for every objects $A$, $B$ and $C$ of $\mathcal{C}$. The morphisms $f_{A,B}$ and $f$ are called \emph{coherence maps}.

A lax monoidal functor is \emph{strong} when the coherence maps are invertible and \emph{strict} when they are identities.
\end{definition}

\begin{definition}[Monoidal natural transformation]
Suppose that $(\mathcal{C},\tens,I)$ and $(\mathcal{D},\bullet,J)$ are two monoidal categories and
$(F,f):(\mathcal{C},\tens,I)\Rightarrow(\mathcal{D},\bullet,J)$
and
$(G,g):(\mathcal{C},\tens,I)\Rightarrow(\mathcal{D},\bullet,J)$
are two monoidal functors between these categories. A \emph{monoidal natural transformation} $\theta:(F,f)\to (G,g)$ between these monoidal functors is a natural transformation $\theta:F\Rightarrow G$ between the underlying functors such that the diagrams
\begin{equation*}
\vcenter{\vbox{\xymatrix{
    FA\bullet FB\ar[d]_{f_{A,B}}\ar[r]^{\theta_A\bullet\theta_B}& \ar[d]^{g_{A,B}}GA\bullet GB\\
    F(A\tens B)\ar[r]_{\theta_{A\tens B}}& G(A\tens B)
}}}
\qquad\text{and}\qquad
\vcenter{\vbox{\xymatrix{
  & \ar[dl]_{f}J\ar[dr]^{g}& \\
  FI\ar[rr]_{\theta_I}& & GI
}}}
\end{equation*}
commute for every objects $A$ and $B$ of $\mathcal{D}$.
\end{definition}

\begin{definition}[Monoidal adjunction]
A \emph{monoidal adjunction} between two monoidal functors
$(F,f):(\mathcal{C},\tens,I)\Rightarrow(\mathcal{D},\bullet,J)$
and
$(G,g):(\mathcal{D},\bullet,J)\Rightarrow(\mathcal{C},\tens,I)$
is an adjunction between the underlying functors $F$ and $G$ such that the unit and the counit
$\eta:\mathcal{C}\Rightarrow G\circ F$ and $\varepsilon:F\circ G\Rightarrow\mathcal{D}$
induce monoidal natural transformations between the corresponding monoidal functors.
\end{definition}


\section{Modeling negation}\label{modeling-negation}

\subsection{*-autonomous categories}\label{autonomous-categories}

\begin{definition}[*-autonomous category]
Suppose that we are given a symmetric monoidal closed category $(\mathcal{C},\tens,I)$ and an object $R$ of $\mathcal{C}$. For every object $A$, we define a morphism $\partial_{A}:A\to(A\limp R)\limp R$ as follows. By applying the bijection of the adjunction defining (left) closed monoidal categories to the identity morphism $\mathrm{id}_{A\limp R}:A\limp R \to A\limp R$, we get a morphism $A\tens (A\limp R)\to R$, and thus a morphism $(A\limp R)\tens A\to R$ by precomposing with the symmetry $\gamma_{A\limp R,A}$. The morphism $\partial_A$ is finally obtained by applying the bijection of the adjunction defining (left) closed monoidal categories to this morphism. The object $R$ is called \emph{dualizing} when the morphism $\partial_A$ is a bijection for every object $A$ of $\mathcal{C}$. A symmetric monoidal closed category is \emph{*-autonomous} when it admits such a dualizing object.
\end{definition}

\subsection{Compact closed categories}\label{compact-closed-categories}

\begin{definition}[Dual objects]
A \emph{dual object} structure $(A,B,\eta,\varepsilon)$ in a monoidal category $(\mathcal{C},\tens,I)$ is a pair of objects $A$ and $B$ together with two morphisms
$\eta:I\to B\otimes A$ and $\varepsilon:A\otimes B\to I$
such that the diagrams
\begin{equation*}
\xymatrix{
& A\tens(B\tens A)\ar[r]^{\alpha_{A,B,A}^{-1}}& (A\tens B)\tens A\ar[dr]^{\varepsilon\tens A}\\
A\tens I\ar[ur]^{A\tens\eta}& & & I\tens A\ar[d]^{\lambda_A}\\
A\ar[u]^{\rho_A^{-1}}\ar@{=}[rrr]& & & A\\
}
\end{equation*}
and
\begin{equation*}
\xymatrix{
& (B\tens A)\tens B\ar[r]^{\alpha_{B,A,B}}& B\tens(A\tens B)\ar[dr]^{B\tens\varepsilon}\\
I\tens B\ar[ur]^{\eta\tens B}& & & B\tens I\ar[d]^{\rho_B}\\
B\ar[u]^{\lambda_B^{-1}}\ar@{=}[rrr]& & & B\\
}
\end{equation*}
commute. The object $A$ is called a left dual of $B$ (and conversely $B$ is a right dual of $A$).
\end{definition}

\begin{lemma}
Two left (resp.\ right) duals of a same object $B$ are necessarily isomorphic.
\end{lemma}

\begin{definition}[Compact closed category]
A symmetric monoidal category $(\mathcal{C},\tens,I)$ is \emph{compact closed} when every object $A$ has a right dual $A^*$. We write
$\eta_A:I\to A^*\tens A$ and $\varepsilon_A:A\tens A^*\to I$
for the corresponding duality morphisms.
\end{definition}

\begin{lemma}
In a compact closed category the left and right duals of an object $A$ are isomorphic.
\end{lemma}

\begin{property}
A compact closed category $\mathcal{C}$ is monoidal closed, with closure defined by
$\mathcal{C}(A\tens B,C)\cong\mathcal{C}(B,A^*\tens C)$.
\end{property}

\begin{proof}
To every morphism $f:A\tens B\to C$, we associate a morphism $\ulcorner f\urcorner:B\to A^*\tens C$ defined as
\begin{equation*}
\xymatrix{
B\ar[r]^-{\lambda_B^{-1}}& I\tens B\ar[r]^-{\eta_A\tens B}& (A^*\tens A)\tens B\ar[r]^-{\alpha_{A^*,A,B}}& A^*\tens(A\tens B)\ar[r]^-{A^*\tens f}& A\tens C\\
}
\end{equation*}
and to every morphism $g:B\to A^*\tens C$, we associate a morphism $\llcorner g\lrcorner:A\tens B\to C$ defined as
\begin{equation*}
\xymatrix{
A\tens B\ar[r]^-{A\tens g}& A\tens(A^*\tens C)\ar[r]^-{\alpha_{A,A^*,C}^{-1}}& (A\tens A^*)\tens C\ar[r]^-{\varepsilon_A\tens C}& I\tens C\ar[r]^-{\lambda_C}& C
}
\end{equation*}
It is easy to show that $\llcorner \ulcorner f\urcorner\lrcorner=f$ and $\ulcorner\llcorner g\lrcorner\urcorner=g$ from which we deduce the required bijection.
\end{proof}

\begin{property}
A compact closed category is a (degenerated) *-autonomous category, with the obvious duality structure. In particular, $(A \otimes B)^* \cong A^*\otimes B^*$.
\end{property}

\begin{remark}
The above isomorphism does not hold in *-autonomous categories in general. This means that models which are compact closed categories identify $\otimes$ and $\parr$ as well as $1$ and $\bot$.
\end{remark}

\begin{proof}
The dualizing object $R$ is simply $I^*$.

For any $A$, the reverse isomorphism $\delta_A : (A \multimap R)\multimap R \rightarrow A$ is constructed as follows:
\begin{equation*}
\mathcal{C}((A \multimap R)\multimap R, A) := \mathcal{C}((A \otimes I^{**})\otimes I^{**}, A) \cong \mathcal{C}((A \otimes I)\otimes I, A) \cong \mathcal{C}(A, A)
\end{equation*}

Identity on $A$ is taken as the canonical morphism required.
\end{proof}


\section{Other categorical models}\label{other-categorical-models}

\subsection{Lafont categories}\label{lafont-categories}

\subsection{Seely categories}\label{seely-categories}

\subsection{Linear categories}\label{linear-categories}

\section{Properties of categorical models}\label{properties-of-categorical-models}

\subsection{The Kleisli category}\label{the-kleisli-category}

%%% Local Variables:
%%% mode: latex
%%% TeX-master: "main"
%%% End:
