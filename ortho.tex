\section{Orthogonality relation}\label{orthogonality-relation}

\emph{Orthogonality relations} are used pervasively throughout linear
logic models, being often used to define somehow the duality operator
\((-)\orth\).

\begin{definition}[Orthogonality relation]
Let $A$ and $B$ be two sets. An \emph{orthogonality relation} on $A$ and $B$ is a binary relation $\mathcal{R}\subseteq A\times B$. We say that $a\in A$ and $b\in B$ are \emph{orthogonal}, and we note $a\perp b$, whenever $(a, b)\in\mathcal{R}$.
\end{definition}

Let us now assume an orthogonality relation over \(A\) and \(B\).

\begin{definition}[Orthogonal sets]
Let $\alpha\subseteq A$. We define its orthogonal set $\alpha\orth$ as $\alpha\orth:=\{b\in B \mid \forall a\in \alpha, a\perp b\}$.

Symmetrically, for any $\beta\subseteq B$, we define $\beta\orth:=\{a\in A \mid \forall b\in \beta, a\perp b\}$.
\end{definition}

Orthogonal sets define Galois connections and share many common
properties.

\begin{proposition}
For any sets $\alpha, \alpha'\subseteq A$:
\begin{itemize}
\item $\alpha\subseteq \alpha\biorth$
\item If $\alpha\subseteq\alpha'$, then ${\alpha'}\orth\subseteq\alpha\orth$
\item $\alpha\triorth = \alpha\orth$
\end{itemize}
\end{proposition}

%%% Local Variables:
%%% mode: latex
%%% TeX-master: "main"
%%% End:
