\section{Reversibility and focusing}\label{reversibility-and-focusing}

\subsection{Reversibility}\label{reversibility}

Remark that this result is proved using only commutation rules, except
when the formula is introduced by an axiom rule. Furthermore, if axioms
are applied only on atoms, this particular case disappears.

A consequence of this fact is that, when searching for a proof of some
sequent \(\vdash\Gamma\), one can always start by decomposing negative
connectives in \(\Gamma\) without losing provability. Applying this
result to successive connectives, we can get generalized formulations
for more complex formulas. For instance:

\begin{itemize}
\tightlist
\item
  \(\vdash\Gamma,(A\parr B)\parr(B\with C)\) is provable
\item
  iff \(\vdash\Gamma,A\parr B,B\with C\) is provable
\item
  iff \(\vdash\Gamma,A\parr B,B\) and \(\vdash\Gamma,A\parr B,C\) are
  provable
\item
  iff \(\vdash\Gamma,A,B,B\) and \(\vdash\Gamma,A,B,C\) are provable
\end{itemize}

So without loss of generality, we can assume that any proof of
\(\vdash\Gamma,(A\parr B)\parr(B\with C)\) ends like

\(\AxRule{ \vdash \Gamma, A, B, B }
  \UnaRule{ \vdash \Gamma, A\parr B, B }
  \AxRule{ \vdash \Gamma, A, B, C }
  \UnaRule{ \vdash \Gamma, A\parr B, C }
  \BinRule{ \vdash \Gamma, A\parr B, B\with C }
  \UnaRule{ \vdash \Gamma, (A\parr B)\parr(B\with C) }
  \DisplayProof\)

In order to define a general statement for compound formulas, as well as
an analogous result for positive connectives, we need to give a proper
status to clusters of connectives of the same polarity.

\subsection{Generalized connectives and
rules}\label{generalized-connectives-and-rules}

It is clear that dualization of generalized connectives is involutive
and exchanges polarities. We do not include quantifiers in this
definition, mainly for simplicity. Extending the notion to quantifiers
would only require taking proper care of the scope of variables.

Sequent calculus provides introduction rules for each connective.
Negative connectives have one rule, positive ones may have any number of
rules, namely 2 for \(\plus\) and 0 for \(\zero\). We can derive
introduction rules for the generalized connectives by combining the
different possible introduction rules for each of their components.

Considering the previous example
\(N[X_1,X_2,X_3]=(X_1\parr X_2)\parr(X_2\with X_3)\), we can derive an
introduction rule for \(N\) as

\(\AxRule{ \vdash \Gamma, X_1, X_2, X_2 }
  \UnaRule{ \vdash \Gamma, X_1\parr X_2, X_2 }
  \AxRule{ \vdash \Gamma, X_1, X_2, X_3 }
  \UnaRule{ \vdash \Gamma, X_1\parr X_2, X_3 }
  \BinRule{ \vdash \Gamma, X_1\parr X_2, X_2\with X_3 }
  \UnaRule{ \vdash \Gamma, (X_1\parr X_2)\parr(X_2\with X_3) }
  \DisplayProof
\quad\text{or}\quad
  \AxRule{ \vdash \Gamma, X_1, X_2, X_2 }
  \AxRule{ \vdash \Gamma, X_1, X_2, X_3 }
  \BinRule{ \vdash \Gamma, X_1, X_2, X_2\with X_3 }
  \UnaRule{ \vdash \Gamma, X_1\parr X_2, X_2\with X_3 }
  \UnaRule{ \vdash \Gamma, (X_1\parr X_2)\parr(X_2\with X_3) }
  \DisplayProof\)

but these rules only differ by the commutation of independent rules. In
particular, their premisses are the same. The dual of \(N\) is
\(P[X_1,X_2,X_3]=(X_1\tens X_2)\tens(X_2\plus X_3)\), for which we have
two possible derivations:

\(\AxRule{ \vdash \Gamma_1, X_1 }
  \AxRule{ \vdash \Gamma_2, X_2 }
  \BinRule{ \vdash \Gamma_1, \Gamma_2, X_1\tens X_2 }
  \AxRule{ \vdash \Gamma_3, X_2 }
  \UnaRule{ \vdash \Gamma_3, X_2\plus X_3 }
  \BinRule{ \vdash \Gamma_1, \Gamma_2, \Gamma_3, (X_1\tens X_2)\tens(X_2\plus X_3) }
  \DisplayProof
\qquad
  \AxRule{ \vdash \Gamma_1, X_1 }
  \AxRule{ \vdash \Gamma_2, X_2 }
  \BinRule{ \vdash \Gamma_1, \Gamma_2, X_1\tens X_2 }
  \AxRule{ \vdash \Gamma_3, X_3 }
  \UnaRule{ \vdash \Gamma_3, X_2\plus X_3 }
  \BinRule{ \vdash \Gamma_1, \Gamma_2, \Gamma_3, (X_1\tens X_2)\tens(X_2\plus X_3) }
  \DisplayProof\)

These are actually different, in particular their premisses differ. Each
possible derivation corresponds to the choice of one side of the
\(\plus\) connective.

We can remark that the branches of the negative inference precisely
correspond to the possible positive inferences:

\begin{itemize}
\tightlist
\item
  the first branch of the negative inference has a premiss
  \(X_1,X_2,X_2\) and the first positive inference has three premisses,
  holding \(X_1\), \(X_2\) and \(X_2\) respectively.
\item
  the second branch of the negative inference has a premiss
  \(X_1,X_2,X_3\) and the second positive inference has three premisses,
  holding \(X_1\), \(X_2\) and \(X_3\) respectively.
\end{itemize}

This phenomenon extends to all generalized connectives.

In the example above, the branching will be \([[1,2,2],[1,2,3]]\), which
corresponds to the granches of the negative inference and to the cases
of positive inference.

, \textbackslash{}ldots, A\_\{i\_\{1,\textbackslash{}ell\_1\}\} \}

\texttt{~~~\textbackslash{}AxRule\{~\textbackslash{}cdots~\}}\\
\texttt{~~~\textbackslash{}AxRule\{~\textbackslash{}vdash~\textbackslash{}Gamma,~A\_\{i\_\{k,1\}\},~\textbackslash{}ldots,~A\_\{i\_\{k,\textbackslash{}ell\_k\}\}~\}}\\
\texttt{~~~\textbackslash{}LabelRule\{N\}}\\
\texttt{~~~\textbackslash{}TriRule\{~\textbackslash{}vdash~\textbackslash{}Gamma,~N{[}A\_1,\textbackslash{}ldots,A\_n{]}~\}}\\
\texttt{~~~\textbackslash{}DisplayProof}\\
\texttt{~}

For each branch \(I=[i_1,\ldots,i_\ell]\) of a positive generalized
connective \(P\), the derived rule for branch \(I\) of \(P\) is

\(\AxRule{ \vdash \Gamma_1, A_{i_1} }
    \AxRule{ \cdots }
    \AxRule{ \vdash \Gamma_\ell, A_{i_\ell} }
    \LabelRule{P_I}
    \TriRule{ \vdash \Gamma_1, \ldots, \Gamma_\ell, P[A_1,\ldots,A_n] }
    \DisplayProof\) \}\}

The reversibility property of negative connectives can be rephrased in a
generalized way as

The corresponding property for positive connectives is the focusing
property, defined in the next section.

\subsection{Focusing}\label{focusing}

If we extended the theory to include quantifiers in generalized
connectives, then the definition of positive and negative formulas would
be extended to include them too.

\texttt{~~~\textbackslash{}TriRule\{~\textbackslash{}vdash~~\textbackslash{}Gamma\_1,~\textbackslash{}ldots,~\textbackslash{}Gamma\_\textbackslash{}ell,~P{[}A\_1,\textbackslash{}ldots,A\_n{]}~\}}\\
\texttt{~~~\textbackslash{}DisplayProof}\\
\texttt{~}

where \(P\) is a positive generalized connective, the \(A_i\) ar
non-positive and \(A=P[A_1,\ldots,A_n]\). The formula \(A\) is called
the \emph{focus} of the proof \(\pi\). \}\}

In other words, a proof is positively focused on a conclusion \(A\) if
its last rules build \(A\) from some of its non-positive subformulas in
one cluster of inferences. Note that this notion only makes sense for a
sequent made only of positive formulas, since by this definition a proof
is obviously positively focused on any of its non-positive conclusions,
using the degenerate generalized connective \(P[X]=X\).

As in the reversibility theorem, this proof only makes use of
commutation of independent rules.

These results say nothing about exponential modalities, because they
respect neither reversibility nor focusing. However, if we consider
the fragment of LL which consists only of multiplicative and additive
connectives, we can restrict the proof rules to enforce focusing
without loss of expressiveness.


%%% Local Variables:
%%% mode: latex
%%% TeX-master: "main"
%%% End:
