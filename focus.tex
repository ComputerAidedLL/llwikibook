\section{Reversibility and focusing}\label{reversibility-and-focusing}

\subsection{Reversibility}\label{reversibility}

\begin{theorem}
Negative connectives are reversible:
\begin{itemize}
\item A sequent $\vdash\Gamma,A\parr B$ is provable if and only if $\vdash\Gamma,A,B$ is provable.
\item A sequent $\vdash\Gamma,A\with B$ is provable if and only if $\vdash\Gamma,A$ and $\vdash\Gamma,B$ are provable.
\item A sequent $\vdash\Gamma,\bot$ is provable if and only if $\vdash\Gamma$ is provable.
\item A sequent $\vdash\Gamma,\forall\xi A$ is provable if and only if $\vdash\Gamma,A$ is provable, for some fresh variable $\xi$.
\end{itemize}
\end{theorem}

\begin{proof}
We start with the case of the $\parr$ connective.
If $\vdash\Gamma,A,B$ is provable, then by the introduction rule for $\parr$
we know that $\vdash\Gamma,A\parr B$ is provable.
For the reverse implication we proceed by induction on a proof $\pi$ of
$\vdash\Gamma,A\parr B$.
\begin{itemize}
\item If the last rule of $\pi$ is the introduction of the $\parr$ in $A\parr B$, then the premiss is exacty $\vdash\Gamma,A,B$ so we can conclude.
\item The other case where the last rule introduces $A\parr B$ is when $\pi$ is an axiom rule, hence $\Gamma=A\orth\tens B\orth$. Then we can conclude with the proof
\item Otherwise $A\parr B$ is in the context of the last rule. If the last rule is a tensor, then $\pi$ has the shape
\todo[missing in LLWiki]
or the same with $A\parr B$ in the conclusion of $\pi_2$ instead. By induction hypothesis on $\pi_1$ we get a proof $\pi'_1$ of $\vdash\Gamma_1,A,B,C$, then we can conclude with the proof
\todo[missing in LLWiki]
\item The case of the cut rule has the same structure as the tensor rule.
\item In the case of the $\with$ rule, we have $A\parr B$ in both premisses and we conclude similarly, using the induction hypothesis on both $\pi_1$ and $\pi_2$.
\item If $A\parr B$ is in the context of a rules for $\parr$, $\plus$, $\bot$ or quantifiers, or in the context of a dereliction, weakening or contraction, the situation is similar as for $\tens$ except that we have only one premiss.
\item If $A\parr B$ is in the context of $\top$ rules, we can freely change the context of the rule to get the expected one.
\item The two remaining cases are if the last rule is the rule for $1$ or a promotion. By the constraints these rules impose on the contexts, these cases cannot happen.
\end{itemize}
The $\with$ connective is treated in the same way.
In cases where $A\with B$ is in the context of a rule with two
premisses, the premiss where this formula is not present will be duplicated,
with one copy in the premiss for $A$ and one in the premiss for $B$.

The $\forall$ connective is also treated similarly.
Its peculiarity is that introducing $\forall\xi$ requires that $\xi$ does
not appear free in the context.
For all rules with one premiss except the quantifier rules, the set of fresh
variables is the same in the premiss and the conclusion, so everything works
well.
Other rules might change the set of free variables, but problems are avoided
by choosing for $\xi$ a variable that is fresh for the whole proof we are considering.
\end{proof}

Remark that this result is proved using only commutation of rules, except
when the formula is introduced by an axiom rule. Furthermore, if axioms
are applied only on atoms, this particular case disappears.

A consequence of this fact is that, when searching for a proof of some
sequent \(\vdash\Gamma\), one can always start by decomposing negative
connectives in \(\Gamma\) without losing provability. Applying this
result to successive connectives, we can get generalized formulations
for more complex formulas. For instance:
\begin{itemize}
\item
  \(\vdash\Gamma,(A\parr B)\parr(B\with C)\) is provable
\item
  iff \(\vdash\Gamma,A\parr B,B\with C\) is provable
\item
  iff \(\vdash\Gamma,A\parr B,B\) and \(\vdash\Gamma,A\parr B,C\) are
  provable
\item
  iff \(\vdash\Gamma,A,B,B\) and \(\vdash\Gamma,A,B,C\) are provable
\end{itemize}

So without loss of generality, we can assume that any proof of
\(\vdash\Gamma,(A\parr B)\parr(B\with C)\) ends like
\begin{prooftree}
  \AxRule{ \vdash \Gamma, A, B, B }
  \UnaRule{ \vdash \Gamma, A\parr B, B }
  \AxRule{ \vdash \Gamma, A, B, C }
  \UnaRule{ \vdash \Gamma, A\parr B, C }
  \BinRule{ \vdash \Gamma, A\parr B, B\with C }
  \UnaRule{ \vdash \Gamma, (A\parr B)\parr(B\with C) }
\end{prooftree}

In order to define a general statement for compound formulas, as well as
an analogous result for positive connectives, we need to give a proper
status to clusters of connectives of the same polarity.

\subsection{Generalized connectives and
rules}\label{generalized-connectives-and-rules}

\begin{definition}
A \emph{positive generalized connective} is a parametrized formula
$P[X_1,\ldots,X_n]$ made from the variables $X_i$ using the connectives
$\tens$, $\plus$, $\one$, $\zero$.

A \emph{negative generalized connective} is a parametrized formula
$N[X_1,\ldots,X_n]$ made from the variables $X_i$ using the connectives
$\parr$, $\with$, $\bot$, $\top$.

If $C[X_1,\ldots,X_n]$ is a generalized connectives (of any polarity), the
\emph{dual} of $C$ is the connective $C^*$ such that
$C^*[X_1\orth,\ldots,X_n\orth]=C[X_1,\ldots,X_n]\orth$.
\end{definition}

It is clear that dualization of generalized connectives is involutive
and exchanges polarities. We do not include quantifiers in this
definition, mainly for simplicity. Extending the notion to quantifiers
would only require taking proper care of the scope of variables.

Sequent calculus provides introduction rules for each connective.
Negative connectives have one rule, positive ones may have any number of
rules, namely 2 for \(\plus\) and 0 for \(\zero\). We can derive
introduction rules for the generalized connectives by combining the
different possible introduction rules for each of their components.

Considering the previous example
\(N[X_1,X_2,X_3]=(X_1\parr X_2)\parr(X_2\with X_3)\), we can derive an
introduction rule for \(N\) as
\begin{equation*}
\AxRule{ \vdash \Gamma, X_1, X_2, X_2 }
  \UnaRule{ \vdash \Gamma, X_1\parr X_2, X_2 }
  \AxRule{ \vdash \Gamma, X_1, X_2, X_3 }
  \UnaRule{ \vdash \Gamma, X_1\parr X_2, X_3 }
  \BinRule{ \vdash \Gamma, X_1\parr X_2, X_2\with X_3 }
  \UnaRule{ \vdash \Gamma, (X_1\parr X_2)\parr(X_2\with X_3) }
  \DisplayProof
\quad\quad\text{or}\quad\quad
  \AxRule{ \vdash \Gamma, X_1, X_2, X_2 }
  \AxRule{ \vdash \Gamma, X_1, X_2, X_3 }
  \BinRule{ \vdash \Gamma, X_1, X_2, X_2\with X_3 }
  \UnaRule{ \vdash \Gamma, X_1\parr X_2, X_2\with X_3 }
  \UnaRule{ \vdash \Gamma, (X_1\parr X_2)\parr(X_2\with X_3) }
  \DisplayProof
\end{equation*}
but these rules only differ by the commutation of independent rules. In
particular, their premisses are the same. The dual of \(N\) is
\(P[X_1,X_2,X_3]=(X_1\tens X_2)\tens(X_2\plus X_3)\), for which we have
two possible derivations:

\(\AxRule{ \vdash \Gamma_1, X_1 }
  \AxRule{ \vdash \Gamma_2, X_2 }
  \BinRule{ \vdash \Gamma_1, \Gamma_2, X_1\tens X_2 }
  \AxRule{ \vdash \Gamma_3, X_2 }
  \UnaRule{ \vdash \Gamma_3, X_2\plus X_3 }
  \BinRule{ \vdash \Gamma_1, \Gamma_2, \Gamma_3, (X_1\tens X_2)\tens(X_2\plus X_3) }
  \DisplayProof
\qquad
  \AxRule{ \vdash \Gamma_1, X_1 }
  \AxRule{ \vdash \Gamma_2, X_2 }
  \BinRule{ \vdash \Gamma_1, \Gamma_2, X_1\tens X_2 }
  \AxRule{ \vdash \Gamma_3, X_3 }
  \UnaRule{ \vdash \Gamma_3, X_2\plus X_3 }
  \BinRule{ \vdash \Gamma_1, \Gamma_2, \Gamma_3, (X_1\tens X_2)\tens(X_2\plus X_3) }
  \DisplayProof\)

These are actually different, in particular their premisses differ. Each
possible derivation corresponds to the choice of one side of the
\(\plus\) connective.

We can remark that the branches of the negative inference precisely
correspond to the possible positive inferences:
\begin{itemize}
\item
  the first branch of the negative inference has a premiss
  \(X_1,X_2,X_2\) and the first positive inference has three premisses,
  holding \(X_1\), \(X_2\) and \(X_2\) respectively.
\item
  the second branch of the negative inference has a premiss
  \(X_1,X_2,X_3\) and the second positive inference has three premisses,
  holding \(X_1\), \(X_2\) and \(X_3\) respectively.
\end{itemize}

This phenomenon extends to all generalized connectives.

\begin{definition}
The \emph{branching} of a generalized connective $P[X_1,\ldots,X_n]$ is the
multiset $\mathcal{I}_P$ of multisets over $\{1,\ldots,n\}$ defined
inductively as

$ \mathcal{I}_{P\tens Q} := [ I+J \mid I\in\mathcal{I}_P, J\in\mathcal{I}_Q ] $,
$ \mathcal{I}_{P\plus Q} := \mathcal{I}_P + \mathcal{I}_Q $,
$ \mathcal{I}_\one := [[]] $,
$ \mathcal{I}_\zero := [] $,
$ \mathcal{I}_{X_i} := [[i]] $.

The branching of a negative generalized connective is the branching of its
dual. Elements of a branching are called branches.
\end{definition}

In the example above, the branching will be \([[1,2,2],[1,2,3]]\), which
corresponds to the branches of the negative inference and to the cases
of positive inference.

\begin{definition}
Let $\mathcal{I}$ be a branching.
Write $\mathcal{I}$ as $[I_1,\ldots,I_k]$ and write each $I_j$ as
$[i_{j,1},\ldots,i_{j,\ell_j}]$.
The derived rule for a negative generalized connective $N$ with
branching $\mathcal{I}$ is
\begin{prooftree}
    \AxRule{ \vdash \Gamma, A_{i_{1,1}}, \ldots, A_{i_{1,\ell_1}} }
    \AxRule{ \cdots }
    \AxRule{ \vdash \Gamma, A_{i_{k,1}}, \ldots, A_{i_{k,\ell_k}} }
    \LabelRule{N}
    \TriRule{ \vdash \Gamma, N[A_1,\ldots,A_n] }
\end{prooftree}
  
For each branch $I=[i_1,\ldots,i_\ell]$ of a positive generalized connective
$P$, the derived rule for branch $I$ of $P$ is
\begin{prooftree}
    \AxRule{ \vdash \Gamma_1, A_{i_1} }
    \AxRule{ \cdots }
    \AxRule{ \vdash \Gamma_\ell, A_{i_\ell} }
    \LabelRule{P_I}
    \TriRule{ \vdash \Gamma_1, \ldots, \Gamma_\ell, P[A_1,\ldots,A_n] }
\end{prooftree}
\end{definition}

The reversibility property of negative connectives can be rephrased in a
generalized way as

\begin{theorem}
Let $N$ be a negative generalized connective. A sequent
$\vdash\Gamma,N[A_1,\ldots,A_n]$ is provable if and only if, for each
$[i_1,\ldots,i_k]\in\mathcal{I}_N$, the sequent
$\vdash\Gamma,A_{i_1},\ldots,A_{i_k}$ is provable.
\end{theorem}

The corresponding property for positive connectives is the focusing
property, defined in the next section.

\subsection{Focusing}\label{focusing}

\begin{definition}
A formula is \emph{positive} if it has a main connective among
$\tens$, $\plus$, $\one$, $\zero$.
It is called \emph{negative} if it has a main connective among
$\parr$, $\with$, $\bot$, $\top$.
It is called \emph{neutral} if it is neither positive nor negative.
\end{definition}

If we extended the theory to include quantifiers in generalized
connectives, then the definition of positive and negative formulas would
be extended to include them too.

\begin{definition}
A proof $\pi\vdash\Gamma,A$ is said to be \emph{positively focused on $A$} if it has the shape
\begin{prooftree}
    \AxRule{ \pi_1 \vdash \Gamma_1, A_{i_1} }
    \AxRule{ \cdots }
    \AxRule{ \pi_\ell \vdash \Gamma_\ell, A_{i_\ell} }
    \LabelRule{P_{[i_1,\ldots,i_\ell]}}
    \TriRule{ \vdash  \Gamma_1, \ldots, \Gamma_\ell, P[A_1,\ldots,A_n] }
\end{prooftree}  
where $P$ is a positive generalized connective, the $A_i$ ar non-positive
and $A=P[A_1,\ldots,A_n]$. The formula $A$ is called the \emph{focus} of the proof $\pi$.
\end{definition}

In other words, a proof is positively focused on a conclusion \(A\) if
its last rules build \(A\) from some of its non-positive subformulas in
one cluster of inferences. Note that this notion only makes sense for a
sequent made only of positive formulas, since by this definition a proof
is obviously positively focused on any of its non-positive conclusions,
using the degenerate generalized connective \(P[X]=X\).

\begin{theorem}
A sequent $\vdash\Gamma$ is cut-free provable if and only if it is provable by a cut-free proof that is positively focused.
\end{theorem}

\begin{proof}
We reason by induction on a proof $\pi$ of $\Gamma$.
As noted above, the result  trivially holds if $\Gamma$ has a non-positive
formula.
We can thus assume that $\Gamma$ contains only positive formulas and reason
on the nature of the last rule, which is necessarily the introduction of a
positive connective (it cannot be an axiom rule because an axiom  always has
at least one non-positive conclusion).

Suppose that the last rule of $\pi$ introduces a tensor, so that $\pi$ is
\begin{prooftree}
    \AxRule{ \rho \vdash \Gamma, A }
    \AxRule{ \theta \vdash \Delta, B }
    \BinRule{ \vdash \Gamma, \Delta, A\tens B }
\end{prooftree}
  
By induction hypothesis, there are positively focused proofs $\rho'\vdash\Gamma,A$
and $\theta'\vdash\Delta,B$.
If $A$ is the focus of $\rho'$ and $B$ is the focus of $\theta'$, then the
proof
\begin{prooftree}
    \AxRule{ \rho' \vdash \Gamma, A }
    \AxRule{ \theta' \vdash \Delta, B }
    \BinRule{ \vdash \Gamma, \Delta, A\tens B }
\end{prooftree}  
is positively focused on $A\tens B$, so we can conclude.
Otherwise, one of the two proofs is positively focused on another conclusion.
Without loss of generality, suppose that $\rho'$ is not positively focused on $A$.
Then it decomposes as
\begin{prooftree}
    \AxRule{ \rho_1 \vdash \Gamma_1, C_{i_1} }
    \AxRule{ \cdots }
    \AxRule{ \rho_\ell \vdash \Gamma_\ell, C_{i_\ell} }
    \TriRule{ \vdash  \Gamma_1, \ldots, \Gamma_\ell, P[C_1,\ldots,C_n] }
\end{prooftree}  
where the $C_i$ are not positive and $A$ belongs to some context $\Gamma_j$
that we will write $\Gamma'_j,A$.
Then we can conclude with the proof
\begin{prooftree}
    \AxRule{ \rho_1 \vdash \Gamma_1, C_{i_1} \quad\cdots }
    \AxRule{ \rho_j \vdash \Gamma_j, A, C_{i_j} }
    \AxRule{ \theta \vdash \Delta, B }
    \BinRule{ \vdash \Gamma_j, \Delta, A\tens B, C_{i_j} }
    \AxRule{ \cdots\quad \rho_\ell \vdash \Gamma_\ell, C_{i_\ell} }
    \TriRule{ \vdash \Gamma_1, \ldots, \Gamma_\ell, \Delta, A\tens B, P[C_1,\ldots,C_n] }
\end{prooftree}
which is positively focused on $P[C_1,\ldots,C_n]$.

If the last rule of $\pi$ introduces a $\plus$, we proceed the same way
except that there is only one premiss.
If the last rule of $\pi$ introduces a $\one$, then it is the only rule of
$\pi$, which is thus positively focused on this $\one$.
\end{proof}

As in the reversibility theorem, this proof only makes use of
commutation of independent rules.

These results say nothing about exponential modalities, because they
respect neither reversibility nor focusing. However, if we consider
the fragment of LL which consists only of multiplicative and additive
connectives, we can restrict the proof rules to enforce focusing
without loss of expressiveness.


%%% Local Variables:
%%% mode: latex
%%% TeX-master: "main"
%%% End:
