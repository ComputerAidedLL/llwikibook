\section{Positive formula}\label{positive-formula}

A \emph{positive formula} is a formula \(P\) such that \(P\limp\oc P\)
(thus a \href{https://en.wikipedia.org/wiki/F-coalgebra}{coalgebra} for the
\href{https://en.wikipedia.org/wiki/Comonad}{comonad} \(\oc\)). As a consequence \(P\) and
\(\oc P\) are \hyperref[equivalences]{equivalent}.

A formula \(P\) is positive if and only if \(P\orth\) is
\hyperref[negative-formula]{negative}.

\subsection{Positive connectives}\label{positive-connectives}

A connective \(c\) of arity \(n\) is \emph{positive} if for any positive
formulas \(P_1\),...,\(P_n\), \(c(P_1,\dots,P_n)\) is positive.

\begin{proposition}[Positive connectives]
$\tens$, $\one$, $\plus$, $\zero$, $\oc$ and $\exists$ are positive connectives.
\end{proposition}

\begin{proof}\
\begin{prooftree}
\AxRule{P_2\vdash\oc{P_2}}
\AxRule{P_1\vdash\oc{P_1}}
\LabelRule{\rulename{ax}}
\NulRule{P_1\vdash P_1}
\LabelRule{\rulename{ax}}
\NulRule{P_2\vdash P_2}
\LabelRule{\tens R}
\BinRule{P_1,P_2\vdash P_1\tens P_2}
\LabelRule{\oc d L}
\UnaRule{\oc{P_1},P_2\vdash P_1\tens P_2}
\LabelRule{\oc d L}
\UnaRule{\oc{P_1},\oc{P_2}\vdash P_1\tens P_2}
\LabelRule{\oc R}
\UnaRule{\oc{P_1},\oc{P_2}\vdash\oc{(P_1\tens P_2)}}
\LabelRule{\rulename{cut}}
\BinRule{P_1,\oc{P_2}\vdash\oc{(P_1\tens P_2)}}
\LabelRule{\rulename{cut}}
\BinRule{P_1,P_2\vdash\oc{(P_1\tens P_2)}}
\LabelRule{\tens L}
\UnaRule{P_1\tens P_2\vdash\oc{(P_1\tens P_2)}}
\end{prooftree}

\begin{prooftree}
\LabelRule{\one R}
\NulRule{\vdash\one}
\LabelRule{\oc R}
\UnaRule{\vdash\oc{\one}}
\LabelRule{\one L}
\UnaRule{\one\vdash\oc{\one}}
\end{prooftree}

\begin{prooftree}
\AxRule{P_1\vdash\oc{P_1}}
\LabelRule{\rulename{ax}}
\NulRule{P_1\vdash P_1}
\LabelRule{\plus_1 R}
\UnaRule{P_1\vdash P_1\plus P_2}
\LabelRule{\oc d L}
\UnaRule{\oc{P_1}\vdash P_1\plus P_2}
\LabelRule{\oc R}
\UnaRule{\oc{P_1}\vdash\oc{(P_1\plus P_2)}}
\LabelRule{\rulename{cut}}
\BinRule{P_1\vdash\oc{(P_1\plus P_2)}}
\AxRule{P_2\vdash\oc{P_2}}
\LabelRule{\rulename{ax}}
\NulRule{P_2\vdash P_2}
\LabelRule{\plus_2 R}
\UnaRule{P_2\vdash P_1\plus P_2}
\LabelRule{\oc d L}
\UnaRule{\oc{P_2}\vdash P_1\plus P_2}
\LabelRule{\oc R}
\UnaRule{\oc{P_2}\vdash\oc{(P_1\plus P_2)}}
\LabelRule{\rulename{cut}}
\BinRule{P_2\vdash\oc{(P_1\plus P_2)}}
\LabelRule{\plus L}
\BinRule{P_1\plus P_2\vdash\oc{(P_1\plus P_2)}}
\end{prooftree}

\begin{prooftree}
\LabelRule{\zero L}
\NulRule{\zero\vdash\oc{\zero}}
\end{prooftree}

\begin{prooftree}
\LabelRule{\rulename{ax}}
\NulRule{\oc{P}\vdash\oc{P}}
\LabelRule{\oc R}
\UnaRule{\oc{P}\vdash\oc{\oc{P}}}
\end{prooftree}

\begin{prooftree}
\AxRule{P\vdash\oc{P}}
\LabelRule{\rulename{ax}}
\NulRule{P\vdash P}
\LabelRule{\exists R}
\UnaRule{P\vdash \exists\xi P}
\LabelRule{\oc d L}
\UnaRule{\oc{P}\vdash \exists\xi P}
\LabelRule{\oc R}
\UnaRule{\oc{P}\vdash\oc{\exists\xi P}}
\LabelRule{\rulename{cut}}
\BinRule{P\vdash\oc{\exists\xi P}}
\LabelRule{\exists L}
\UnaRule{\exists\xi P\vdash\oc{\exists\xi P}}
\end{prooftree}
\end{proof}

More generally, \(\oc A\) is positive for any formula \(A\).

The notion of positive connective is related with but different from the
notion of \hyperref[synchrony]{synchronous connective}.

\subsection{Generalized structural rules}\label{generalized-structural-rules-pos}

Positive formulas admit generalized left structural rules corresponding
to a structure of \href{https://en.wikipedia.org/wiki/Comonoid}{\(\tens\)-comonoid}:
\(P\limp P\tens P\) and \(P\limp\one\). The following rules are derivable:
\begin{equation*}
\AxRule{\Gamma,P,P\vdash\Delta}
\LabelRule{+ c L}
\UnaRule{\Gamma,P\vdash\Delta}
\DisplayProof
\qquad\qquad
\AxRule{\Gamma\vdash\Delta}
\LabelRule{+ w L}
\UnaRule{\Gamma,P\vdash\Delta}
\DisplayProof
\end{equation*}

\begin{proof}
\begin{equation*}
\AxRule{P\vdash\oc{P}}
\AxRule{\Gamma,P,P\vdash\Delta}
\LabelRule{\oc L}
\UnaRule{\Gamma,P,\oc P\vdash\Delta}
\LabelRule{\oc L}
\UnaRule{\Gamma,\oc P,\oc P\vdash\Delta}
\LabelRule{\oc c L}
\UnaRule{\Gamma,\oc P\vdash\Delta}
\LabelRule{\rulename{cut}}
\BinRule{\Gamma,P\vdash\Delta}
\DisplayProof
\qquad\qquad
\AxRule{P\vdash\oc{P}}
\AxRule{\Gamma\vdash\Delta}
\LabelRule{\oc w L}
\UnaRule{\Gamma,\oc P\vdash\Delta}
\LabelRule{\rulename{cut}}
\BinRule{\Gamma,P\vdash\Delta}
\DisplayProof
\end{equation*}
\end{proof}

Positive formulas are also acceptable in the left-hand side context of
the promotion rule. The following rule is derivable:
\begin{prooftree}
\AxRule{\oc\Gamma,P_1,\dots,P_n\vdash A,\wn\Delta}
\LabelRule{+ \oc R}
\UnaRule{\oc\Gamma,P_1,\dots,P_n\vdash \oc{A},\wn\Delta}
\end{prooftree}

\begin{proof}\
\begin{prooftree}
\AxRule{P_1\vdash\oc{P_1}}
\AxRule{P_n\vdash\oc{P_n}}
\AxRule{\oc\Gamma,P_1,\dots,P_n\vdash A,\wn\Delta}
\LabelRule{\oc L}
\UnaRule{\oc\Gamma,P_1,\dots,P_{n-1},\oc{P_n}\vdash A,\wn\Delta}
\VdotsRule{}{\oc\Gamma,P_1,\oc{P_2},\dots,\oc{P_n}\vdash A,\wn\Delta}
\LabelRule{\oc L}
\UnaRule{\oc\Gamma,\oc{P_1},\dots,\oc{P_n}\vdash A,\wn\Delta}
\LabelRule{\oc R}
\UnaRule{\oc\Gamma,\oc{P_1},\dots,\oc{P_n}\vdash \oc{A},\wn\Delta}
\LabelRule{\rulename{cut}}
\BinRule{\oc\Gamma,\oc{P_1},\dots,\oc{P_{n-1}},P_n\vdash \oc{A},\wn\Delta}
\VdotsRule{}{\oc\Gamma,\oc{P_1},P_2,\dots,P_n\vdash \oc{A},\wn\Delta}
\LabelRule{\rulename{cut}}
\BinRule{\oc\Gamma,P_1,\dots,P_n\vdash \oc{A},\wn\Delta}
\end{prooftree}
\end{proof}


%%% Local Variables:
%%% mode: latex
%%% TeX-master: "main"
%%% End:
