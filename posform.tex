\section{Positive formula}\label{positive-formula}

A \emph{positive formula} is a formula \(P\) such that \(P\limp\oc P\)
(thus a \href{Wikipedia:F-coalgebra}{coalgebra} for the
\href{Wikipedia:Comonad}{comonad} \(\oc\)). As a consequence \(P\) and
\(\oc P\) are \href{Sequent_calculus\#Equivalences}{equivalent}.

A formula \(P\) is positive if and only if \(P\orth\) is
\href{Negative_formula}{negative}.

\subsection{Positive connectives}\label{positive-connectives}

A connective \(c\) of arity \(n\) is \emph{positive} if for any positive
formulas \(P_1\),...,\(P_n\), \(c(P_1,\dots,P_n)\) is positive.

\textbackslash{}AxRule\{P\_1\textbackslash{}vdash\textbackslash{}oc\{P\_1\}\}
\textbackslash{}LabelRule\{\textbackslash{}rulename\{ax\}\}
\textbackslash{}NulRule\{P\_1\textbackslash{}vdash P\_1\}
\textbackslash{}LabelRule\{\textbackslash{}rulename\{ax\}\}
\textbackslash{}NulRule\{P\_2\textbackslash{}vdash P\_2\}
\textbackslash{}LabelRule\{\textbackslash{}tens R\}
\textbackslash{}BinRule\{P\_1,P\_2\textbackslash{}vdash
P\_1\textbackslash{}tens P\_2\}
\textbackslash{}LabelRule\{\textbackslash{}oc d L\}
\textbackslash{}UnaRule\{\textbackslash{}oc\{P\_1\},P\_2\textbackslash{}vdash
P\_1\textbackslash{}tens P\_2\}
\textbackslash{}LabelRule\{\textbackslash{}oc d L\}
\textbackslash{}UnaRule\{\textbackslash{}oc\{P\_1\},\textbackslash{}oc\{P\_2\}\textbackslash{}vdash
P\_1\textbackslash{}tens P\_2\}
\textbackslash{}LabelRule\{\textbackslash{}oc R\}
\textbackslash{}UnaRule\{\textbackslash{}oc\{P\_1\},\textbackslash{}oc\{P\_2\}\textbackslash{}vdash\textbackslash{}oc\{(P\_1\textbackslash{}tens
P\_2)\}\} \textbackslash{}LabelRule\{\textbackslash{}rulename\{cut\}\}
\textbackslash{}BinRule\{P\_1,\textbackslash{}oc\{P\_2\}\textbackslash{}vdash\textbackslash{}oc\{(P\_1\textbackslash{}tens
P\_2)\}\} \textbackslash{}LabelRule\{\textbackslash{}rulename\{cut\}\}
\textbackslash{}BinRule\{P\_1,P\_2\textbackslash{}vdash\textbackslash{}oc\{(P\_1\textbackslash{}tens
P\_2)\}\} \textbackslash{}LabelRule\{\textbackslash{}tens L\}
\textbackslash{}UnaRule\{P\_1\textbackslash{}tens
P\_2\textbackslash{}vdash\textbackslash{}oc\{(P\_1\textbackslash{}tens
P\_2)\}\} \textbackslash{}DisplayProof

\(\LabelRule{\one R}
\NulRule{\vdash\one}
\LabelRule{\oc R}
\UnaRule{\vdash\oc{\one}}
\LabelRule{\one L}
\UnaRule{\one\vdash\oc{\one}}
\DisplayProof\)

\(\AxRule{P_1\vdash\oc{P_1}}
\LabelRule{\rulename{ax}}
\NulRule{P_1\vdash P_1}
\LabelRule{\plus_1 R}
\UnaRule{P_1\vdash P_1\plus P_2}
\LabelRule{\oc d L}
\UnaRule{\oc{P_1}\vdash P_1\plus P_2}
\LabelRule{\oc R}
\UnaRule{\oc{P_1}\vdash\oc{(P_1\plus P_2)}}
\LabelRule{\rulename{cut}}
\BinRule{P_1\vdash\oc{(P_1\plus P_2)}}
\AxRule{P_2\vdash\oc{P_2}}
\LabelRule{\rulename{ax}}
\NulRule{P_2\vdash P_2}
\LabelRule{\plus_2 R}
\UnaRule{P_2\vdash P_1\plus P_2}
\LabelRule{\oc d L}
\UnaRule{\oc{P_2}\vdash P_1\plus P_2}
\LabelRule{\oc R}
\UnaRule{\oc{P_2}\vdash\oc{(P_1\plus P_2)}}
\LabelRule{\rulename{cut}}
\BinRule{P_2\vdash\oc{(P_1\plus P_2)}}
\LabelRule{\plus L}
\BinRule{P_1\plus P_2\vdash\oc{(P_1\plus P_2)}}
\DisplayProof\)

\(\LabelRule{\zero L}
\NulRule{\zero\vdash\oc{\zero}}
\DisplayProof\)

\(\LabelRule{\rulename{ax}}
\NulRule{\oc{P}\vdash\oc{P}}
\LabelRule{\oc R}
\UnaRule{\oc{P}\vdash\oc{\oc{P}}}
\DisplayProof\)

\(\AxRule{P\vdash\oc{P}}
\LabelRule{\rulename{ax}}
\NulRule{P\vdash P}
\LabelRule{\exists R}
\UnaRule{P\vdash \exists\xi P}
\LabelRule{\oc d L}
\UnaRule{\oc{P}\vdash \exists\xi P}
\LabelRule{\oc R}
\UnaRule{\oc{P}\vdash\oc{\exists\xi P}}
\LabelRule{\rulename{cut}}
\BinRule{P\vdash\oc{\exists\xi P}}
\LabelRule{\exists L}
\UnaRule{\exists\xi P\vdash\oc{\exists\xi P}}
\DisplayProof\) \}\}

More generally, \(\oc A\) is positive for any formula \(A\).

The notion of positive connective is related with but different from the
notion of \href{asynchronous_connective}{asynchronous connective}.

\subsection{Generalized structural
rules}\label{generalized-structural-rules-1}

Positive formulas admit generalized left structural rules corresponding
to a structure of \href{Wikipedia:Comonoid}{\(\tens\)-comonoid}:
\(P\limp P\tens P\) and \(P\limp\one\). The following rule is derivable:

\(\AxRule{\Gamma,P,P\vdash\Delta}
\LabelRule{+ c L}
\UnaRule{\Gamma,P\vdash\Delta}
\DisplayProof
\qquad
\AxRule{\Gamma\vdash\Delta}
\LabelRule{+ w L}
\UnaRule{\Gamma,P\vdash\Delta}
\DisplayProof\)

\textbackslash{}AxRule\{\textbackslash{}Gamma,P,P\textbackslash{}vdash\textbackslash{}Delta\}
\textbackslash{}LabelRule\{\textbackslash{}oc L\}
\textbackslash{}UnaRule\{\textbackslash{}Gamma,P,\textbackslash{}oc
P\textbackslash{}vdash\textbackslash{}Delta\}
\textbackslash{}LabelRule\{\textbackslash{}oc L\}
\textbackslash{}UnaRule\{\textbackslash{}Gamma,\textbackslash{}oc
P,\textbackslash{}oc P\textbackslash{}vdash\textbackslash{}Delta\}
\textbackslash{}LabelRule\{\textbackslash{}oc c L\}
\textbackslash{}UnaRule\{\textbackslash{}Gamma,\textbackslash{}oc
P\textbackslash{}vdash\textbackslash{}Delta\}
\textbackslash{}LabelRule\{\textbackslash{}rulename\{cut\}\}
\textbackslash{}BinRule\{\textbackslash{}Gamma,P\textbackslash{}vdash\textbackslash{}Delta\}
\textbackslash{}DisplayProof

\(\AxRule{P\vdash\oc{P}}
\AxRule{\Gamma\vdash\Delta}
\LabelRule{\oc w L}
\UnaRule{\Gamma,\oc P\vdash\Delta}
\LabelRule{\rulename{cut}}
\BinRule{\Gamma,P\vdash\Delta}
\DisplayProof\) \}\}

Positive formulas are also acceptable in the left-hand side context of
the promotion rule. The following rule is derivable:

\(\AxRule{\oc\Gamma,P_1,\dots,P_n\vdash A,\wn\Delta}
\LabelRule{+ \oc R}
\UnaRule{\oc\Gamma,P_1,\dots,P_n\vdash \oc{A},\wn\Delta}
\DisplayProof\)

\textbackslash{}AxRule\{P\_n\textbackslash{}vdash\textbackslash{}oc\{P\_n\}\}
\textbackslash{}AxRule\{\textbackslash{}oc\textbackslash{}Gamma,P\_1,\textbackslash{}dots,P\_n\textbackslash{}vdash
A,\textbackslash{}wn\textbackslash{}Delta\}
\textbackslash{}LabelRule\{\textbackslash{}oc L\}
\textbackslash{}UnaRule\{\textbackslash{}oc\textbackslash{}Gamma,P\_1,\textbackslash{}dots,P\_\{n-1\},\textbackslash{}oc\{P\_n\}\textbackslash{}vdash
A,\textbackslash{}wn\textbackslash{}Delta\}
\textbackslash{}VdotsRule\{\}\{\textbackslash{}oc\textbackslash{}Gamma,P\_1,\textbackslash{}oc\{P\_2\},\textbackslash{}dots,\textbackslash{}oc\{P\_n\}\textbackslash{}vdash
A,\textbackslash{}wn\textbackslash{}Delta\}
\textbackslash{}LabelRule\{\textbackslash{}oc L\}
\textbackslash{}UnaRule\{\textbackslash{}oc\textbackslash{}Gamma,\textbackslash{}oc\{P\_1\},\textbackslash{}dots,\textbackslash{}oc\{P\_n\}\textbackslash{}vdash
A,\textbackslash{}wn\textbackslash{}Delta\}
\textbackslash{}LabelRule\{\textbackslash{}oc R\}
\textbackslash{}UnaRule\{\textbackslash{}oc\textbackslash{}Gamma,\textbackslash{}oc\{P\_1\},\textbackslash{}dots,\textbackslash{}oc\{P\_n\}\textbackslash{}vdash
\textbackslash{}oc\{A\},\textbackslash{}wn\textbackslash{}Delta\}
\textbackslash{}LabelRule\{\textbackslash{}rulename\{cut\}\}
\textbackslash{}BinRule\{\textbackslash{}oc\textbackslash{}Gamma,\textbackslash{}oc\{P\_1\},\textbackslash{}dots,\textbackslash{}oc\{P\_\{n-1\}\},P\_n\textbackslash{}vdash
\textbackslash{}oc\{A\},\textbackslash{}wn\textbackslash{}Delta\}
\textbackslash{}VdotsRule\{\}\{\textbackslash{}oc\textbackslash{}Gamma,\textbackslash{}oc\{P\_1\},P\_2,\textbackslash{}dots,P\_n\textbackslash{}vdash
\textbackslash{}oc\{A\},\textbackslash{}wn\textbackslash{}Delta\}
\textbackslash{}LabelRule\{\textbackslash{}rulename\{cut\}\}
\textbackslash{}BinRule\{\textbackslash{}oc\textbackslash{}Gamma,P\_1,\textbackslash{}dots,P\_n\textbackslash{}vdash
\textbackslash{}oc\{A\},\textbackslash{}wn\textbackslash{}Delta\}
\textbackslash{}DisplayProof \}\}

