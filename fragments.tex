\section{Fragment}\label{fragment}

In general, a \textbf{fragment} of a logical system \emph{S} is a
logical system obtained by restricting the language of \emph{S}, and by
restricting the rules of \emph{S} accordingly. In linear logic, the most
well known fragments are obtained by combining/removing in different
ways the classes of connectives present in the
\href{Sequent_calculus}{language of linear logic} itself:

\begin{itemize}
\tightlist
\item
  \textbf{Multiplicative connectives:} the conjunction \(\tens\)
  (\emph{tensor}) and the disjunction \(\parr\) (\emph{par}), with their
  respective units \(\one\) (\emph{one}) and \(\bot\) (\emph{bottom});
  these connectives are the combinatorial base of linear logic
  (permutations, circuits, etc.).
\item
  \textbf{Additive connectives:} the conjunction \(\with\) (\emph{with})
  and the disjunction \(\plus\) (\emph{plus}), with their respective
  units \(\top\) (\emph{top}) and \(\zero\) (\emph{zero}); the
  computational content of these connectives, which behave more closely
  to their intuitionistic counterparts (\emph{e.g.}, \(A\with B\limp A\)
  and \(A\with B\limp B\) are provable), is strongly related to choice
  (\emph{if...then...else}, product and sum types, etc.).
\item
  \textbf{Exponential connectives:} the modalities \(\oc\) (\emph{of
  course}) and \(\wn\) (\emph{why not}) handle the structural rules in
  linear logic, and are necessary to recover the expressive power of
  intuitionistic or classical logic.
\item
  \textbf{Quantifiers:} just as in classical logic, quantifiers may be
  added to propositional linear logic, at any order. The most frequently
  considered are the second order ones (in analogy with System F).
\end{itemize}

The additive and exponential connectives, if taken alone, yield
fragments of limited interest, so one usually considers only fragments
containing at least the multiplicative connectives (perhaps without
units). It is important to observe that the
\href{Sequent_calculus\#Cut_elimination_and_consequences}{cut
elimination rules} of linear logic do not introduce connectives
belonging to a different class than that of the pair of dual formulas
whose cut is being reduced. Hence, any fragment defined by combining the
above classes will enjoy cut elimination. Since cut elimination implies
the subformula property, all of the
\href{Sequent_calculus\#Equivalences}{fundamental equivalences} provable
in full linear logic remain valid within such fragments, as soon as the
formulas concerned belong to the fragment itself.

Conventionally, if \(LL\) denotes full linear logic, its fragments are
denoted by prefixing \(LL\) with letters corresponding to the classes of
connectives being considered: \(M\) for multiplicative connectives,
\(A\) for additive connectives, and \(E\) for exponential connectives.
Additional subscripts may specify whether units and/or quantifiers are
present or not, and, for quantifiers, of what order (see the article on
\url{notations}).

\subsection{Motivations}\label{motivations}

The main interest of studying fragments of linear logic is that these
are usually simpler than the whole system, so that certain properties
may be first analyzed on fragments, and then extended or adapted to
increasingly larger fragments. It may also be interesting to see, given
a property that does not hold for full linear logic, whether it holds
for a fragment, and where the "breaking point" is situated. Examples of
such questions include:

\begin{itemize}
\tightlist
\item
  \textbf{logical complexity:} proving cut elimination for full linear
  logic with second order quantification is equivalent to proving the
  consistency of second order Peano arithmetic (Girard, via
  \href{Translations_of_intuitionistic_logic}{translations of System F}
  in linear logic). One may expect that smaller fragments have lower
  logical complexity.
\item
  \textbf{provability:} the \emph{provability problem} for a logical
  system \emph{S} is defined as follows: given a formula \(A\) in the
  language of \emph{S}, is \(A\) provable in \emph{S}? This problem is
  undecidable in full linear logic with quantifiers, of whatever order
  (again, because \href{Translations_of_classical_logic}{classical logic
  can be translated} in linear logic). On what fragments does it become
  decidable? And if it does, what is its computational complexity?
\item
  \textbf{computational complexity of cut elimination:} the \emph{cut
  elimination problem} (Mairson-Terui) for a logical system \emph{S} is
  defined as follows: given two proofs of \(A\) in \emph{S}, do they
  reduce to the same cut-free proof? Although decidable (thanks to
  strong normalization), this problem is not elementary recursive in
  full propositional linear logic (Statman, again via the
  above-mentioned translations). Does the problem fall into any
  interesting complexity class when applied to fragments?
\item
  \textbf{proof nets:} the definition of proof nets, and in particular
  the formulation of correctness criteria and the study of their
  complexity, is a good example of how a methodology can be applied to a
  small fragment of linear logic and later adapted (more or less
  successfully) to wider fragments.
\item
  \textbf{denotational semantics:} several problems related to
  denotational semantics (formulation of
  \href{Categorical_semantics}{categorical models}, full abstraction,
  full completeness, injectivity, etc.) may be first attacked in the
  simpler case of fragments, and then extended to wider subsystems.
\end{itemize}

\subsection{Multiplicative fragments}\label{multiplicative-fragments}

Multiplicative linear logic (\(MLL\)) is the simplest of the well known
fragments of linear logic. Its formulas are obtained by combining
propositional atoms with the connectives \emph{tensor} and \emph{par}
only. As a consequence, the
\href{Sequent_calculus\#Sequents_and_proofs}{sequent calculus} of
\(MLL\) is limited to the rules \(\rulename{axiom}\),
\(\rulename{cut}\), \(\tens\), and \(\parr\). These rules actually
determine the multiplicative connectives: if a dual pair of connectives
\(\tens'\) and \(\parr'\) is introduced, with the same rules as
\(\tens\) and \(\parr\), respectively, then one can show \(A\tens' B\)
to be provably equivalent to \(A\tens B\) (and, dually, \(A\parr'B\) to
be provably equivalent to \(A\parr B\)).

The cut elimination problem for \(MLL\) is \(\mathbf P\)-complete
(Mairson-Terui), even though there exists a deterministic algorithm
solving the problem in logarithmic space if one considers only
\href{Sequent_calculus\#Expansion_of_identities}{eta-expanded} proofs
(Mairson-Terui). On the other hand, provability for \(MLL\) is
\(\mathbf{NP}\)-complete, and it remains so even in presence of first
order quantifiers.

Another multiplicative fragment, less considered in the literature, can
be defined by using the units \(\one\) and \(\bot\) instead of the
propositional atoms. In this fragment, denoted by \(MLL_u\), one can
also eliminate the \(\rulename{axiom}\) rule from sequent calculus,
since it is redundant. \(MLL_u\) is even simpler than \(MLL\): its
provability problem is in \(\mathbf P\), and, since all proofs are
eta-expandend, its cut elimination problem is in \(\mathbf L\).

The union of \(MLL\) and \(MLL_u\) is the full propositional
multiplicative fragment of linear logic, and is denoted by \(MLL_0\). It
has the same properties as \(MLL\), which shows that the
presence/absence of propositional atoms (and of the \(\rulename{axiom}\)
rule) has a non-trivial effect on the complexity of provability and cut
elimination, \emph{i.e.}, the complexity is not altered iff
\(\mathbf P\subsetneq\mathbf{NP}\) and \(\mathbf L\subsetneq\mathbf P\),
respectively.

If we add second order quantifiers to \(MLL\) (resp. \(MLL_u\)), we
obtain a system denoted by \(MLL_2\) (resp. \(MLL_{02}\)). In
\(MLL_{02}\) one can show that \(\one\) and \(\bot\) are provably
equivalent to \(\forall X.(X\orth\parr X)\) and
\(\exists X.(X\orth\tens X)\), respectively. Hence, \(MLL_2\) is as
expressive as \(MLL_{02}\). In these second order fragments, provability
is undecidable, while cut elimination is still \(\mathbf P\)-complete.

\subsection{Additive fragments}\label{additive-fragments}

The most studied additive fragments of linear logic are defined by
taking \(MLL\) or \(MLL_0\) and by enriching their language with the
additive connectives, with or without units. The same can be done in
presence of quantifiers. We thus obtain:

\begin{itemize}
\tightlist
\item
  \(MALL\): formulas built from propositional atoms using
  \(\tens,\parr,\with,\plus\);
\item
  \(MALL_0\): formulas built from propositional atoms and
  \(\one,\bot,\top,\zero\), using \(\tens,\parr,\with,\plus\);
\item
  \(MALL_n\): \(MALL\) with quantifiers of order \(n\);
\item
  \(MALL_{0n}\): \(MALL_0\) with quantifiers of order \(n\).
\end{itemize}

The \href{Additive_linear_logic}{purely additive framents} are less
common:

\begin{itemize}
\tightlist
\item
  \(ALL\): formulas built from propositional atoms using
  \(\with,\plus\);
\item
  \(ALL_0\): formulas built from propositional atoms and \(\top,\zero\),
  using \(\with,\plus\);
\item
  \(ALL_n\): \(ALL\) with quantifiers of order \(n\);
\item
  \(ALL_{0n}\): \(ALL_0\) with quantifiers of order \(n\).
\end{itemize}

As for the multiplicative connectives, the additive connectives are also
defined by their rules: adding a pair of dual connectives
\(\with',\plus'\) to \(MALL\), and giving them the same rules as
\(\with,\plus\), makes the new connectives provably equivalent to the
old ones.

In \(MALL_{02}\), the additive units \(\top\) and \(\zero\) are provably
equivalent to \(\exists X.X\orth\) and \(\forall X.X\), respectively.
Since multiplicative units are also definable in terms of second order
quantification, we obtain that \(MALL_2\) is as expressive as
\(MALL_{02}\).

The cut elimination problem is \(\mathbf{coNP}\)-complete for all of the
fragments defined above (Mairson-Terui).

Provability is undecidable in any additive fragment as soon as second
order quantification is considered. It is decidable, although quite
complex, in the propositional and first order case: it is
\(\mathbf{PSPACE}\)-complete in \(MALL_0\), and
\(\mathbf{NEXP}\)-complete in \(MALL_{01}\). This latter result is
indicative of the fact that the undecidability of predicate calculus is
not ascribable to existential quantification alone, but rather to the
simultaneous presence of existential quantification and contraction.

\subsection{Exponential fragments}\label{exponential-fragments}

The most common proper fragments of linear logic containing the
exponential connectives are defined as in the case of the additive
fragments, \emph{i.e.}, by adding the modalities on top of \(MLL\) and
its variants:

\begin{itemize}
\tightlist
\item
  \(MELL\): formulas built from propositional atoms using
  \(\tens,\parr,\oc,\wn\);
\item
  \(MELL_0\): formulas built from propositional atoms and \(\one,\bot\),
  using \(\tens,\parr,\oc,\wn\);
\item
  \(MELL_n\): \(MELL\) with quantifiers of order \(n\);
\item
  \(MELL_{0n}\): \(MELL_0\) with quantifiers of order \(n\).
\end{itemize}

If, instead of taking \(MLL\), we add the modalities to \(MALL\), we
obtain of course various versions of full linear logic:

\begin{itemize}
\tightlist
\item
  \(LL\): full linear logic, without units;
\item
  \(LL_0\): full linear logic, with units;
\item
  \(LL_n\): \(LL\) with quantifiers of order \(n\);
\item
  \(LL_{0n}\): \(LL_0\) with quantifiers of order \(n\).
\end{itemize}

In \(LL_{02}\) the formulas \(A\with B\) and \(A\plus B\) are provably
equivalent to
\(\exists X.(\oc{(X\orth\parr A)}\tens\oc{(X\orth\parr B)}\tens X)\) and
\(\forall X.(\wn{(X\orth\tens A)}\parr\wn{(X\orth \tens B)}\parr X)\),
respectively, for all \(A,B\). Thanks to the second-order definability
of units discussed above, we obtain that \(MELL_2\) is as expressive as
\(LL_{02}\), \emph{i.e.}, full propositional second order linear logic
embeds in its second order multiplicative exponential fragment without
units.

Girard showed how cut elimination for \(LL_{02}\) \emph{without the
contraction rule} can be proved by a simple induction up to \(\omega\),
\emph{i.e.}, in first order Peano arithmetic. This gives a huge gap
between the logical complexity of full linear logic and its
contraction-free subsystem: in fact, still by Girard's results, we know
that cut elimination in \(MELL_2\) is equivalent to the consistency of
second order Peano arithmetic, for which no ordinal analysis is known.
There are nevertheless subsystems of \(MELL_2\), the so-called
\href{Light_linear_logics}{light subsystems} of linear logic, in which
the exponential connectives are weakened, whose cut elimination can be
proved in seconder order Peano arithmetic even in presence of
contraction.

The cut elimination problem is never elementary recursive in presence of
exponential connectives: the simply typed \(\lambda\)-calculus with
arrow types only can be encoded in \(MELL\), and this is enough for
Statman's lower bound to apply. However, it becomes elementary recursive
in the above mentioned \href{Light_linear_logics}{light logics}.

Albeit perhaps surprisingly, provability in \(LL\) is already
undecidable. This result, obtained by coding Minsky machines with linear
logic formulas, contrasts with the situation in classical logic, whose
propositional fragment is notoriously decidable. It is indicative of the
fact that modalities are themselves a form of quantification, although
this claim is far from being clear: as a matter of fact, the
decidability of propositional provability in the absence of additives,
\emph{i.e.}, in \(MELL\) alone, is still an open problem. It is known
that adding first order quantification to \(MELL\) makes it undecidable.

\subsubsection{About exponential rules}\label{about-exponential-rules}

In this section, provability is assumed to be in \(LL_{02}\),
\emph{i.e.}, full propositional second order linear logic.

In contrast with multiplicative and additive connectives, the modalities
of linear logic are not defined by their rules: one may introduce a pair
of dual modalities \(\oc',\wn'\), each with the same rules as
\(\oc,\wn\), without \(\oc'A\) (resp. \(\wn'A\)) being in general
provably equivalent to \(\oc A\) (resp. \(\wn A\)).

The \href{Sequent_calculus\#Sequents_and_proofs}{promotion rule} is
derivable from the following two rules, called \emph{functorial
promotion} and \emph{digging}, respectively:

\(\AxRule{\vdash\Gamma,A}
\LabelRule{f\oc}
\UnaRule{\vdash\wn\Gamma,\oc A}
\DisplayProof
\qquad\qquad\qquad
\AxRule{\vdash\Gamma,\wn{\wn A}}
\LabelRule{dig}
\UnaRule{\vdash\Gamma,\wn A}
\DisplayProof\)

Functorial promotion is itself derivable from dereliction and promotion;
the digging rule is also derivable, but only using the
\(\rulename{cut}\) rule (in fact, digging does not enjoy the subformula
property). It may be convenient to consider this pair of rules instead
of the standard promotion rule in the context of
\href{categorical_semantics}{categorical semantics} of linear logic.

In presence of the digging rule, dereliction, weakening, and contraction
can be derived from the following rule, called \emph{multiplexing}, in
which \(A^{(n)}\) stands for the sequence \(A,\ldots,A\) containing
\(n\) occurrences of \(A\):

\(\AxRule{\vdash\Gamma,A^{(n)}}
\LabelRule{mux}
\UnaRule{\vdash\Gamma,\wn A}
\DisplayProof\)

Of course, multiplexing is itself derivable from dereliction, weakening,
and contraction. Hence, there are several alternative but equivalent
presentations of the exponential fragment of linear logic, such as

\begin{enumerate}
\tightlist
\item
  remove promotion, and replace it with functorial promotion and
  digging;
\item
  remove promotion, dereliction, weakening, and contraction, and replace
  them with functorial promotion, digging, and multiplexing.
\end{enumerate}

Apart from their usefulness in \href{categorical_semantics}{categorical
semantics}, these alternative formulations are of interest in the
context of the so-called \href{light_linear_logics}{light linear logics}
mentioned above. For example, \emph{elementary linear logic} is obtained
by removing dereliction and digging from formulation 1, and \emph{soft
linear logic} is obtained by removing digging from formulation 2.

Multiplexing is invertible in certain circumstances. A sequent
\(\vdash\Gamma,\wn A\) containing no occurrence of \(\with\), \(\oc\),
or second order \(\exists\) is provable iff \(\vdash\Gamma,A^{(n)}\) is
provable for some \(n\) (this is easily checked by induction on cut-free
proofs). To see that this does not hold in general, take for instance
\(A=X\orth\) and \(\Gamma=X\with\one\), or \(\Gamma=\oc X\). The
restriction on the presence of additive conjunction can be removed by
slightly changing the statement: a sequent \(\vdash\Gamma,\wn A\)
containing no occurrence of \(\oc\) or second order \(\exists\) is
provable iff \(\vdash\Gamma,(A\plus\bot)^{(n)}\) is provable for some
\(n\).

The latter result can be generalized as follows. If \(A\) is a formula,
\(\oc_nA\) stands for the formula
\((A\with\one)\tens\cdots\tens(A\with\one)\) (\(n\) times) and
\(\wn_nA\) for the formula \((A\plus\bot)\parr\cdots\parr(A\plus\bot)\)
(\(n\) times). Then, we have

A \emph{structural formula} is a formula \(C\) such that
\(C\limp C\tens C\) and \(C\limp\one\) are provable. Obviously, any
formula of the form \(\wn B\) is structural. However, the promotion rule
cannot be extended to arbitrary structural formulas, \emph{i.e.}, the
following rule is \emph{not} admissible:

\(\AxRule{C\vdash A}
\AxRule{C\vdash C\tens C}
\AxRule{C\vdash\one}
\TriRule{C\vdash\oc A}
\DisplayProof\)

For instance, if
\(A=C=\alpha\tens\oc{(\alpha\limp\alpha\tens\alpha)}\tens\oc{(\alpha\limp\one)}\),
the three premises are provable but not the conclusion.

The following rule, called \emph{absorption}, is derivable in the
standard sequent calculus:

\(\AxRule{\vdash\Gamma,\wn A,A}
\UnaRule{\vdash\Gamma,\wn A}
\DisplayProof\)

The absorption rule is useful in the context of proof search in linear
logic.

\subsection{The provability problem}\label{the-provability-problem}

It is well known that the decidability of the provability problem is
connected to the \href{Phase_semantics}{finite model property}: if a
fragment of a logic with a truth semantics enjoys the finite model
property, then the provability in that fragment is decidable. Note of
course that the converse may fail.

In this section, we summarize the known results about the validity of
the final model property and the decidability of provability, with its
complexity, for the various fragments of linear logic introduced above.
Question marks in the tables below denote open problems. For brevity,
all fragments are assumed to have units and propositional atoms,
\emph{e.g.}, \(MLL\) actually denotes what we called \(MLL_0\) above.

\subsubsection{The finite model
property}\label{the-finite-model-property}

\begin{longtable}[]{@{}llll@{}}
\toprule
\(MLL\) & \(MALL\) & \(MELL\) & \(LL\)\tabularnewline
yes & yes & no & no\tabularnewline
\bottomrule
\end{longtable}

\subsubsection{Provability}\label{provability}

\begin{longtable}[]{@{}lllll@{}}
\toprule
& \(MLL\) & \(MALL\) & \(MELL\) & \(LL\)\tabularnewline
propositional case & \(\mathbf{NP}\)-complete &
\(\mathbf{PSPACE}\)-complete & ? & undecidable\tabularnewline
first order case & \(\mathbf{NP}\)-complete & \(\mathbf{NEXP}\)-complete
& undecidable & undecidable\tabularnewline
second order case & undecidable & undecidable & undecidable &
undecidable\tabularnewline
\bottomrule
\end{longtable}

\subsection{The cut elimination
problem}\label{the-cut-elimination-problem}

In this section, we summarize the known results about the complexity of
the cut elimination problem for the various fragments of linear logic
introduced above, plus some \href{light_linear_logics}{light linear
logics}. All fragments are assumed to be propositional; the results do
not change in presence of quantification of any order.

\begin{longtable}[]{@{}llllll@{}}
\toprule
\(MLL_u\) & \(MLL\) & \(MALL\) & \(MSLL\) & \(MLLL\) &
\(MELL\)\tabularnewline
\(\mathbf L\) & \(\mathbf{P}\)-complete & \(\mathbf{coNP}\)-complete &
\(\mathbf{EXP}\)-complete & \(\mathbf{2EXP}\)-complete & not elementary
recursive\tabularnewline
\bottomrule
\end{longtable}

Notations used in the above table:

\begin{itemize}
\tightlist
\item
  \(MSLL\): multiplicative soft linear logic;
\item
  \(MLLL\): multiplicative light linear logic.
\end{itemize}

