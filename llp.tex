\chapter{Polarized linear logic}\label{polarized-linear-logic}

\textbf{Polarized linear logic} (LLP) is a logic close to plain linear
logic in which structural rules, usually restricted to \(\wn\)-formulas,
have been \hyperref[generalized-structural-rules-pos]{extended} to the
whole class of so called \emph{negative} formulas.

\section{Polarization}\label{polarization}

LLP relies on the notion of \emph{polarization}, that is, it
discriminates between two types of formulas, \emph{negative} (noted
\(M, N...\)) vs. \emph{\hyperref[positive-formula]{positive}} (\(P, Q...\)).
They are mutually defined as follows:
\begin{align*}
M, N &::= X \mid M \parr N \mid \bot \mid M \with N \mid \top \mid \wn{P} \\
P, Q &::= X\orth \mid P \otimes Q \mid 1 \mid P \oplus Q \mid 0 \mid \oc{N}
\end{align*}

The dual operation \((-)\orth\) extended to propositions exchanges the
roles of connectives and reverses the polarity of formulas.

\section{Deduction rules}\label{deduction-rules}

There are several design choices for the structure of sequents. In
particular, LLP proofs are \emph{focused}, i.e. they contain at most
one positive formula. We choose to represent this explicitly using
sequents of the form \(\vdash\Gamma\mid\Delta\), where \(\Gamma\) is a
multiset of negative formulas, and \(\Delta\) is a \emph{stoup} that
contains at most one positive formula (though it may be empty).

\(\LabelRule{\rulename{ax}}
\NulRule{\vdash P\orth \mid P}
\DisplayProof
\qquad
\AxRule{\vdash \Gamma_1, N \mid \Delta}
\AxRule{\vdash \Gamma_2 \mid N\orth}
\LabelRule{\rulename{cut}}
\BinRule{\vdash\Gamma_1, \Gamma_2 \mid \Delta}
\DisplayProof\)

\(\AxRule{\vdash\Gamma, N\mid\cdot}
\LabelRule{p}
\UnaRule{\vdash\Gamma\mid \oc{N}}
\DisplayProof
\qquad
\AxRule{\vdash\Gamma\mid P}
\LabelRule{d}
\UnaRule{\vdash\Gamma,\wn{P}\mid \cdot}
\DisplayProof
\qquad
\AxRule{\vdash\Gamma,N,N\mid \Delta}
\LabelRule{c}
\UnaRule{\vdash\Gamma, N\mid\Delta}
\DisplayProof
\qquad
\AxRule{\vdash\Gamma\mid \Delta}
\LabelRule{w}
\UnaRule{\vdash\Gamma,N\mid\Delta}
\DisplayProof\)

\(\AxRule{\vdash\Gamma_1\mid P}
\AxRule{\vdash\Gamma_2\mid Q}
\LabelRule{\tens}
\BinRule{\vdash\Gamma_1,\Gamma_2\mid P\otimes Q}
\DisplayProof
\qquad
\LabelRule{\one}
\NulRule{\vdash\cdot\mid\one}
\DisplayProof
\qquad
\AxRule{\vdash\Gamma, M, N\mid \Delta}
\LabelRule{\parr}
\UnaRule{\vdash\Gamma, M\parr N\mid \Delta}
\DisplayProof
\qquad
\AxRule{\vdash\Gamma\mid \Delta}
\LabelRule{\bot}
\UnaRule{\vdash\Gamma, \bot\mid\Delta}
\DisplayProof\)

\(\AxRule{\vdash\Gamma\mid P}
\LabelRule{\plus_1}
\UnaRule{\vdash\Gamma\mid P\plus Q}
\DisplayProof
\qquad
\AxRule{\vdash\Gamma\mid Q}
\LabelRule{\plus_2}
\UnaRule{\vdash\Gamma\mid P\plus Q}
\DisplayProof
\qquad
\AxRule{\vdash\Gamma,M\mid \Delta}
\AxRule{\vdash\Gamma,N\mid \Delta}
\LabelRule{\with}
\BinRule{\vdash\Gamma,M\with N\mid \Delta}
\DisplayProof
\qquad
\LabelRule{\top}
\NulRule{\vdash\Gamma,\top\mid \Delta}
\DisplayProof\)


%%% Local Variables:
%%% mode: latex
%%% TeX-master: "main"
%%% End:
