\chapter{Relational semantics}\label{relational-semantics}

This is the simplest denotational semantics of linear logic. It consists
in interpreting a formula \(A\) as a set \(A^*\) and a proof \(\pi\) of
\(A\) as a subset \(\pi^*\) of \(A^*\).

\section{The category of sets and relations}\label{the-category-of-sets-and-relations}

It is the category \(\mathbf{Rel}\) whose objects are sets, and such
that \(\mathbf{Rel}(X,Y)=\powerset{X\times Y}\). Composition is the
ordinary composition of relations: given \(s\in\mathbf{Rel}(X,Y)\) and
\(t\in\mathbf{Rel}(Y,Z)\), one sets
\(t\circ s=\set{(a,c)\in X\times Z}{\exists b\in Y\ (a,b)\in s\ \text{and}\ (b,c)\in t}\)
and the identity morphism is the diagonal relation
\(\mathsf{Id}_X=\set{(a,a)}{a\in X}\).

An isomorphism in the category \(\mathbf{Rel}\) is a relation which is a
bijection, as easily checked.

\subsection{Monoidal structure}\label{monoidal-structure}

The tensor product is the usual cartesian product of sets
\(X\tens Y=X\times Y\) (which is not a cartesian product in the category
\(\mathbf{Rel}\) in the categorical sense). It is a bifunctor: given
\(s_i\in\mathbf{Rel}(X_i,Y_i)\) (for \(i=1,2\)), one sets
\(s_1\tens s_2=\set{((a_1,a_2),(b_1,b_2))}{(a_i,b_i)\in s_i\ \text{for}\ i=1,2}\).
The unit of this tensor product is \(\one=\{*\}\) where \(*\) is an
arbitrary element.

For defining a monoidal category, it is not sufficient to provide the
definition of the tensor product functor \(\tens\) and its unit
\(\one\), one has also to provide natural isomorphisms
\(\lambda_X\in\mathbf{Rel}(\one\tens X,X)\),
\(\rho_X\in\mathbf{Rel}(X\tens\one,X)\) (left and right neutrality of
\(\one\) for \(\tens\)) and
\(\alpha_{X,Y,Z}\in\mathbf{Rel}((X\tens Y)\tens Z,X\tens(Y\tens Z))\)
(associativity of \(\tens\)). All these isomorphisms have to satisfy a
number of commutations. In the present case, they are defined in the
obvious way.

This monoidal category \((\mathbf{Rel},\tens,\one,\lambda,\rho)\) is
symmetric, meaning that it is endowed with an additional natural
isomorphism \(\sigma_{X,Y}\in\mathbf{Rel}(X\tens Y,Y\tens X)\), also
subject to some commutations. Here, again, this isomorphism is defined
in the obvious way (symmetry of the cartesian product). So, to be
precise, the SMCC (symmetric monoidal closed category) \(\mathbf{Rel}\)
is the tuple \((\mathbf{Rel},\tens,\one,\lambda,\rho,\alpha,\sigma)\),
but we shall simply denote it as \(\mathbf{Rel}\).

The SMCC \(\mathbf{Rel}\) is closed. This means that, given any object
\(X\) of \(\mathbf{Rel}\) (a set), the functor \(Z\mapsto Z\tens X\)
(from \(\mathbf{Rel}\) to \(\mathbf{Rel}\)) admits a right adjoint
\(Y\mapsto (X\limp Y)\) (from \(\mathbf{Rel}\) to \(\mathbf{Rel}\)). In
other words, for any objects \(X\) and \(Y\), we are given an object
\(X\limp Y\) and a morphism
\(\mathsf{ev}_{X,Y}\in\mathbf{Rel}((X\limp Y)\tens X,Y)\) with the
following universal property: for any morphism
\(s\in\mathbf{Rel}(Z\tens X,Y)\), there is a unique morphism
\(\mathsf{fun}(s)\in\mathbf{Rel}(Z,X\limp Y)\) such that
\(\mathsf{ev}_{X,Y}\circ(\mathsf{fun}(s)\tens\mathsf{Id}_X)=s\).

The definition of all these data is quite simple in \(\mathbf{Rel}\):
\(X\limp Y=X\times Y\),
\(\mathsf{ev}_{X,Y}=\set{(((a,b),a),b)}{(a,b)\in X\times Y}\) and
\(\mathsf{fun}(s)=\set{(c,(a,b))}{((c,a),b)\in s}\).

Let \(\bot=\one=\{*\}\). Then we have
\(\mathsf{ev}\circ\sigma:\mathbf{Rel}(X\tens(X\limp\bot),\bot)\) and
hence
\(\eta_X=\mathsf{fun}(\mathsf{ev}\circ\sigma)\in\mathbf{Rel}(X,(X\limp\bot)\limp\bot)\).
It is clear that \(\eta=\set{(a,((a,*),*))}{a\in X}\) and hence \(\eta\)
is a natural isomorphism: one says that the SMCC \(\mathbf{Rel}\) is a
*-autonomous category, with \(\bot\) as dualizing object.

\subsection{Additives}\label{additives-2}

Given a family \((X_i)_{i\in I}\), let
\(\with_{i\in I}X_i=\cup_{i\in I}\{i\}\times X_i\). Let
\(\pi_j\in\mathbf{Rel}(\with_{i\in I}X_i,X_j)\) given by
\(\pi_j=\set{((j,a),a)}{a\in X_j}\). Then
\((\with_{i\in I}X_i,(\pi_i)_{i\in I})\) is the cartesian product of the
\(X_i\)s in the category \(\mathbf{Rel}\).

\subsection{Exponentials}\label{exponentials-4}

One defines \(\oc X\) as the set of all finite multisets of elements of
\(X\). Given \(s\in\mathbf{Rel}(X,Y)\), one defines
\(\oc s=\set{([a_1,\dots,a_n],[b_1,\dots,b_n])}{n\in\mathbb N\ \text{and}\ \forall i\ (a_i,b_i)\in s}\)
where \([a_1,\dots,a_n]\) is the multiset containing \(a_1,\dots,a_n\),
taking multiplicities into account. This defines a functor
\(\mathbf{Rel}\to\mathbf{Rel}\), that we endow with a comonad structure
as follows:

\begin{itemize}
\tightlist
\item
  the counit, called dereliction, is the natural transformation
  \(\mathsf{der}_X\in\mathbf{Rel}(\oc X,X)\), given by
  \(\mathsf{der}_X=\set{([a],a)}{a\in X}\)
\item
  the comultiplication, called digging, is the natural transformation
  \(\mathsf{digg}_X\in\mathbf{Rel}(\oc X,\oc\oc X)\), given by
  \(\mathsf{digg}_X=\set{(m_1+\cdots+m_n,[m_1,\dots,m_n])}{n\in\mathbb N\ \text{and}\ m_1,\dots,m_n\in\oc X}\)
\end{itemize}

\section{\texorpdfstring{Interpretation of propositional linear logic (\(LL_0\))}{Interpretation of propositional linear logic (LL\_0)}}\label{interpretation-of-propositional-linear-logic-ll_0}

The structure described above gives rise to the following interpretation
of formulas and proofs of linear logic.

For all propositional variable \(X\), fix a set \(\web X\). Then with
each formula \(A\), we associate a set \(\web A\) as follows:

\begin{itemize}
\tightlist
\item
  \(\web{A\orth}=\web A\);
\item
  \(\web{A\tens B}=\web{A\parr B}=\web A\times\web B\);
\item
  \(\web{A\with B}=\web{A\plus B}=(\{1\}\times\web A)\cup(\{2\}\times\web B)\);
\item
  \(\web{\oc A}=\web{\wn A}=\finmulset{\web A}\).
\end{itemize}

We then interpret the proofs of \(LL_0\) as follows: with each proof
\(\pi\) of sequent \(\vdash A_1,\ldots,A_n\), we associate a subset
\(\sem\pi\subseteq\web{A_1}\times\cdots\times\web{A_n}\).

\begin{itemize}
\tightlist
\item
  Identity group:

  \begin{description}
  \tightlist
  \item[]
  \textbackslash{}sem\{
  \end{description}
\end{itemize}

\textbackslash{}LabelRule\{ \textbackslash{}rulename\{axiom\} \}
\textbackslash{}NulRule\{ \textbackslash{}vdash A\textbackslash{}orth, A
\}
\textbackslash{}DisplayProof\}=\textbackslash{}set\{(a,a)\}\{a\textbackslash{}in\textbackslash{}web
A\}

*: \(\sem{
\AxRule{}
\VdotsRule{ \pi }{ \vdash \Gamma, A }
\AxRule{}
\VdotsRule{ \rho }{ \vdash \Delta, A\orth }
\LabelRule{ \rulename{cut} }
\BinRule{ \vdash \Gamma, \Delta }
\DisplayProof} = \set{(\gamma,\delta)}{\exists a\in\web A,\ (\gamma,a)\in\sem\pi\land(\delta,a)\in\sem\rho}\)

\begin{itemize}
\tightlist
\item
  Multiplicative group:

  \begin{description}
  \tightlist
  \item[]
  \end{description}
\end{itemize}

\textbackslash{}sem\{ \textbackslash{}AxRule\{\}
\textbackslash{}VdotsRule\{ \textbackslash{}pi \}\{
\textbackslash{}vdash \textbackslash{}Gamma, A \}
\textbackslash{}AxRule\{\} \textbackslash{}VdotsRule\{
\textbackslash{}rho \}\{ \textbackslash{}vdash \textbackslash{}Delta, B
\} \textbackslash{}LabelRule\{ \textbackslash{}tens \}
\textbackslash{}BinRule\{ \textbackslash{}vdash \textbackslash{}Gamma,
\textbackslash{}Delta, A \textbackslash{}tens B \}
\textbackslash{}DisplayProof\} =
\textbackslash{}set\{(\textbackslash{}gamma,\textbackslash{}delta,a,b)\}\{(\textbackslash{}gamma,a)\textbackslash{}in\textbackslash{}sem\textbackslash{}pi\textbackslash{}land(\textbackslash{}delta,b)\textbackslash{}in\textbackslash{}sem\textbackslash{}rho\}

*: \(\sem{
\AxRule{ }
\VdotsRule{ \pi }{ \vdash \Gamma, A, B }
\LabelRule{ \parr }
\UnaRule{ \vdash \Gamma, A \parr B }
\DisplayProof} =  \set{(\gamma,(a,b))}{(\gamma,a,b)\in\sem\pi}\)

*: \(\sem{
\LabelRule{ \one }
\NulRule{ \vdash \one }
\DisplayProof} = \{ * \}\)

*: \(\sem{
\AxRule{}
\VdotsRule{ \pi }{ \vdash \Gamma }
\LabelRule{ \bot }
\UnaRule{ \vdash \Gamma, \bot }
\DisplayProof} = \set{(\gamma,*)}{\gamma\in\sem\pi}\)

\begin{itemize}
\tightlist
\item
  Additive group:

  \begin{description}
  \tightlist
  \item[]
  \end{description}
\end{itemize}

\textbackslash{}sem\{ \textbackslash{}AxRule\{\}
\textbackslash{}VdotsRule\{ \textbackslash{}pi \}\{
\textbackslash{}vdash \textbackslash{}Gamma, A \}
\textbackslash{}LabelRule\{ \textbackslash{}plus\_1 \}
\textbackslash{}UnaRule\{ \textbackslash{}vdash \textbackslash{}Gamma, A
\textbackslash{}plus B \} \textbackslash{}DisplayProof\} =
\textbackslash{}set\{(\textbackslash{}gamma,(1,a))\}\{(\textbackslash{}gamma,a)\textbackslash{}in\textbackslash{}sem\textbackslash{}pi\}

*: \(\sem{
\AxRule{}
\VdotsRule{ \pi }{ \vdash \Gamma, B }
\LabelRule{ \plus_2 }
\UnaRule{ \vdash \Gamma, A \plus B }
\DisplayProof} = \set{(\gamma,(2,b))}{(\gamma,b)\in\sem\pi}\)

*: \(\sem{
\AxRule{}
\VdotsRule{ \pi }{ \vdash \Gamma, A }
\AxRule{}
\VdotsRule{ \rho }{ \vdash \Gamma, B }
\LabelRule{ \with }
\BinRule{ \vdash \Gamma, A \with B }
\DisplayProof} = \set{(\gamma,(1,a))}{(\gamma,a)\in\sem\pi} \cup \set{(\gamma,(2,b))}{(\gamma,b)\in\sem\rho}\)

*: \(\sem{
\LabelRule{ \top }
\NulRule{ \vdash \Gamma, \top }
\DisplayProof} = \emptyset\)

\begin{itemize}
\tightlist
\item
  Exponential group:

  \begin{description}
  \tightlist
  \item[]
  \end{description}
\end{itemize}

\textbackslash{}sem\{ \textbackslash{}AxRule\{\}
\textbackslash{}VdotsRule\{ \textbackslash{}pi \}\{
\textbackslash{}vdash \textbackslash{}Gamma, A \}
\textbackslash{}LabelRule\{ d \} \textbackslash{}UnaRule\{
\textbackslash{}vdash \textbackslash{}Gamma, \textbackslash{}wn A \}
\textbackslash{}DisplayProof\} =
\textbackslash{}set\{(\textbackslash{}gamma,{[}a{]})\}\{(\textbackslash{}gamma,a)\textbackslash{}in\textbackslash{}sem\textbackslash{}pi\}

*: \(\sem{
\AxRule{}
\VdotsRule{ \pi }{ \vdash \Gamma }
\LabelRule{ w }
\UnaRule{ \vdash \Gamma, \wn A }
\DisplayProof} =  \set{(\gamma,[])}{\gamma\in\sem\pi}\)

*: \(\sem{
\AxRule{}
\VdotsRule{ \pi }{ \vdash \Gamma, \wn A, \wn A }
\LabelRule{ c }
\UnaRule{ \vdash \Gamma, \wn A }
\DisplayProof} =  \set{(\gamma,m+n)}{(\gamma,m,n)\in\sem\pi}\)

*: \(\sem{
\AxRule{}
\VdotsRule{ \pi }{ \vdash \wn A_1,\ldots,\wn A_n,B }
\LabelRule{ \oc }
\UnaRule{ \vdash \wn A_1,\ldots,\wn A_n,\oc B }
\DisplayProof} = \set{
\left(\sum_{i=1}^k m_1^i,\ldots,\sum_{i=1}^k m_n^i,[b_1,\ldots,b_k]\right)}
{k \in \mathbb{N} \ \text{and} \ \forall 1 \leq i \leq k,\ (m_1^i,\ldots,m_n^i,b_i)\in\sem\pi}\)


%%% Local Variables:
%%% mode: latex
%%% TeX-master: "main"
%%% End:
