\section{GoI for MELL: exponentials}\label{goi-for-mell-exponentials}

\subsection{The tensor product of Hilbert spaces}\label{the-tensor-product-of-hilbert-spaces}

Recall that we work in the Hilbert space \(H=\ell^2(\mathbb{N})\)
endowed with its canonical hilbertian basis denoted by
\((e_k)_{k\in\mathbb{N}}\).

The space \(H\tens H\) is the collection of sequences
\((x_{np})_{n,p\in\mathbb{N}}\) of complex numbers such that
\(\sum_{n,p}|x_{np}|^2\) converges. The scalar product is defined just
as before:
\begin{equation*}
\langle (x_{np}), (y_{np})\rangle = \sum_{n,p} x_{np}\bar y_{np}.
\end{equation*}

If \(x = (x_n)_{n\in\mathbb{N}}\) and \(y = (y_p)_{p\in\mathbb{N}}\) are
vectors in \(H\) then their tensor is the sequence:
\begin{equation*}
x\tens y = (x_ny_p)_{n,p\in\mathbb{N}}.
\end{equation*}

We define: \(e_{np} = e_n\tens e_p\) so that \(e_{np}\) is the sequence
\((e_{npij})_{i,j\in\mathbb{N}}\) of complex numbers given by
\(e_{npij} = \delta_{ni}\delta_{pj}\). By bilinearity of tensor we have:
\begin{equation*}
x\tens y = \left(\sum_n x_n e_n\right)\tens\left(\sum_p y_p e_p\right) =
\sum_{n,p}x_n y_p\, e_n\tens e_p = \sum_{n,p} x_n y_p\, e_{np}
\end{equation*}

Furthermore the system of vectors \((e_{np})\) is a hilbertian basis of
\(H\tens H\): the sequence \(x=(x_{np})_{n,p\in\mathbb{N}}\) may be
written:
\begin{align*}
x &= \sum_{n,p\in\mathbb{N}}x_{np}\,e_{np} \\
  &= \sum_{n,p\in\mathbb{N}}x_{np}\,e_n\tens e_p
.
\end{align*}

\subsubsection{An algebra isomorphism}\label{an-algebra-isomorphism}

Being both separable Hilbert spaces, \(H\) and \(H\tens H\) are
isomorphic. We will now define explicitely an iso based on partial
permutations.

We fix, once for all, a bijection from couples of natural numbers to
natural numbers that we will denote by
\((n,p)\mapsto\langle n,p\rangle\). For example set
\(\langle n,p\rangle = 2^n(2p+1) - 1\). Conversely, given
\(n\in\mathbb{N}\) we denote by \(n_{(1)}\) and \(n_{(2)}\) the unique
integers such that \(\langle n_{(1)},
n_{(2)}\rangle = n\).

\begin{remark}
Just as it was convenient but actually not necessary to choose $p$ and $q$ so that $pp^* + qq^* = 1$ it is actually not necessary to have a \emph{bijection}, a one-to-one mapping from $\mathbb{N}^2$ \emph{into} $\mathbb{N}$ would be sufficient for our purpose.
\end{remark}

This bijection can be extended into a Hilbert space isomorphism
\(\Phi:H\tens H\rightarrow H\) by defining:
\begin{equation*}
e_n\tens e_p = e_{np} \mapsto e_{\langle n,p\rangle}.
\end{equation*}

Now given an operator \(u\) on \(H\) we define the operator \(!u\) on
\(H\) by:
\begin{equation*}
!u(e_{\langle n,p\rangle}) = \Phi(e_n\tens u(e_p)).
\end{equation*}

\begin{remark}
The operator $!u$ is defined by:
\begin{equation*}
!u = \Phi\circ (1\tens u)\circ \Phi^{-1}
\end{equation*}
where $1\tens u$ denotes the operator on $H\tens H$ defined by $(1\tens u)(x\tens y) = x\tens u(y)$ for any $x,y$ in $H$. However this notation must not be confused with the \hyperref[the-tensor-rule]{tensor of operators} that was defined in the previous section in order to interpret the tensor rule of linear logic; we therefore will not use it.
\end{remark}

One can check that given two operators \(u\) and \(v\) we have:
\begin{itemize}
\item \(!u!v = {!(uv)}\);
\item \(!(u^*) = (!u)^*\).
\end{itemize}

Due to the fact that \(\Phi\) is an isomorphism \emph{onto} we also have
\(!1=1\); this however will not be used.

We therefore have that \(!\) is a morphism on \(\mathcal{B}(H)\); it is
easily seen to be an iso (not \emph{onto} though). As this is the
crucial ingredient for interpreting the structural rules of linear
logic, we will call it the \emph{copying iso}.

\subsubsection{Interpretation of exponentials}\label{interpretation-of-exponentials}

If we suppose that \(u = u_\varphi\) is a \(p\)-isometry generated by
the partial permutation \(\varphi\) then we have:
\begin{equation*}
!u(e_{\langle n,p\rangle}) = \Phi(e_n\tens u(e_p)) = \Phi(e_n\tens e_{\varphi(p)}) = e_{\langle n,\varphi(p)\rangle}.
\end{equation*}

Thus \(!u_\varphi\) is itself a \(p\)-isometry generated by the partial
permutation
\(!\varphi:n\mapsto \langle n_{(1)}, \varphi(n_{(2)})\rangle\), which
shows that the proof space is stable under the copying iso.

Given a type \(A\) we define the type \(!A\) by:
\begin{equation*}
!A = \{!u, u\in A\}\biorth
\end{equation*}


%%% Local Variables:
%%% mode: latex
%%% TeX-master: "main"
%%% End:
