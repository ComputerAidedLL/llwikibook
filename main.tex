\documentclass[a4paper,11pt,oneside]{book}
\usepackage{fullpage}
\usepackage{llwiki}
\usepackage{amssymb,amsmath,amsthm}

%\usepackage{lmodern}
\usepackage{ifxetex,ifluatex}
\usepackage{fixltx2e} % provides \textsubscript
\ifnum 0\ifxetex 1\fi\ifluatex 1\fi=0 % if pdftex
  \usepackage[T1]{fontenc}
  \usepackage[utf8]{inputenc}
\else % if luatex or xelatex
  \ifxetex
    \usepackage{mathspec}
  \else
    \usepackage{fontspec}
  \fi
  \defaultfontfeatures{Ligatures=TeX,Scale=MatchLowercase}
\fi
% use upquote if available, for straight quotes in verbatim environments
\IfFileExists{upquote.sty}{\usepackage{upquote}}{}
% use microtype if available
\IfFileExists{microtype.sty}{%
\usepackage[]{microtype}
\UseMicrotypeSet[protrusion]{basicmath} % disable protrusion for tt fonts
}{}
\PassOptionsToPackage{hyphens}{url} % url is loaded by hyperref
\usepackage[unicode=true]{hyperref}
\hypersetup{pdfborder={0 0 0},colorlinks=true,linkcolor=blue,
            breaklinks=true}
\urlstyle{same}  % don't use monospace font for urls
\usepackage{longtable,booktabs}
% Fix footnotes in tables (requires footnote package)
\IfFileExists{footnote.sty}{\usepackage{footnote}\makesavenoteenv{long table}}{}



\usepackage{cleveref}


\newcommand*{\wantedpage}[1]{\textcolor{red}{#1}}

\newtheorem{theorem}{Theorem}
\newtheorem{proposition}{Proposition}
\newtheorem{lemma}{Lemma}
\newtheorem{corollary}{Corollary}
\newtheorem{property}{Property}
\newtheorem{definition}{Definition}
\newtheorem{remark}{Remark}
\newtheorem{example}{Example}

\setcounter{secnumdepth}{3}
\setcounter{tocdepth}{3}     % TODO reduce to 2?

\newcommand*{\ie}{\emph{i.e.}}
\newcommand*{\todo}[1][]{\fbox{\textbf{TODO}}\ \textit{#1}}

\title{LLWiki book}

\begin{document}
\maketitle

\chapter*{Informations}

\textit{This document has not been built as a linear \LaTeX{} document but from an extraction of \href{http://llwiki.ens-lyon.fr/}{LLWiki}.
This explains why the structure of the document may look strange at various points.}

\vspace{1.5cm}


The \href{http://llwiki.ens-lyon.fr/}{LLWiki} is a collective work created under the authority of the editorial board:
\begin{itemize}
\item Emmanuel Beffara (IML)
\item Thomas Ehrhard (PPS)
\item Jean-Yves Girard (IML)
\item Olivier Laurent (LIP)
\item Damiano Mazza (LIPN)
\item Laurent Regnier (IML)
\item Lorenzo Tortora de Falco (Roma 3)
\end{itemize}

The editorial board is granted a monopoly of rights on the content of this collective work.\footnote{The contributors are the individuals owning an account on the \href{http://llwiki.ens-lyon.fr/}{LLWiki}, which allows them to edit the content.

By applying for the creation of their account, the candidates to the status of contributor agree to grant the monopoly of rights on the LLWiki as a collective work to the editorial board; they moreover agree that their contributions are distributed under the terms of the \href{http://creativecommons.org/licenses/by-nc-sa/2.0/fr/deed.en License}{Creative Commons Attribution-Noncommercial-Share Alike 2.0 France}, the editorial board being cited as the licensor.

The contributors however keep their rights on their respective contributions and are free to exploit them for their own sake.}


\vspace{1.5cm}


The content of the LLWiki is licensed under a \href{http://creativecommons.org/licenses/by-nc-sa/2.0/fr/deed.en License}{Creative Commons Attribution-Noncommercial-Share Alike 2.0 France}.
This means you can share and adapt this content, under the following conditions:
\begin{itemize}
\item you must cite the \emph{``editorial board of the LLWiki''} as the licensor;
\item you may not use this content for commercial purposes;
\item if you distribute a possibly modified version of this content, you must do so under the same or similar license.
\end{itemize}



\tableofcontents

\part{Syntax}

\chapter{Sequent calculus}\label{sequent-calculus}

This article presents the language and sequent calculus of second-order
linear logic and the basic properties of this sequent calculus. The core
of the article uses the two-sided system with negation as a proper
connective; the \hyperref[one-sided-sequent-calculus]{one-sided
system}, often used as the definition of linear logic, is presented at
the end of the page.

\section{Formulas}\label{formulas}

Atomic formulas, written \(\alpha,\beta,\gamma\), are predicates of the
form \(p(t_1,\ldots,t_n)\), where the \(t_i\) are terms from some
first-order language. The predicate symbol \(p\) may be either a
predicate constant or a second-order variable. By convention we will
write first-order variables as \(x,y,z\), second-order variables as
\(X,Y,Z\), and \(\xi\) for a variable of arbitrary order (see~\cref{notations}).

Formulas, represented by capital letters \(A\), \(B\), \(C\), are built
using the following connectives:

\begin{longtable}[]{@{}llll@{}}
\toprule
\(\alpha\) & atom & \(A\orth\) & negation\tabularnewline
\(A \tens B\) & tensor & \(A \parr B\) & par\tabularnewline
\(\one\) & one & \(\bot\) & bottom\tabularnewline
\(A \plus B\) & plus & \(A \with B\) & with\tabularnewline
\(\zero\) & zero & \(\top\) & top\tabularnewline
\(\oc A\) & of course & \(\wn A\) & why not\tabularnewline
\(\exists \xi.A\) & there exists & \(\forall \xi.A\) & for
all\tabularnewline
\bottomrule
\end{longtable}

Each line (except the first one) corresponds to a particular class of
connectives, and each class consists in a pair of connectives. Those in
the left column are called \hyperref[positive-formula]{positive} and those
in the right column are called \hyperref[negative-formula]{negative}. The
\emph{tensor} and \emph{with} connectives are conjunctions while
\emph{par} and \emph{plus} are disjunctions. The exponential connectives
are called \emph{modalities}, and traditionally read \emph{of course
\(A\)} for \(\oc A\) and \emph{why not \(A\)} for \(\wn A\). Quantifiers
may apply to first- or second-order variables.

There is no connective for implication in the syntax of standard linear
logic. Instead, a \emph{linear implication} is defined similarly to the
decomposition \(A\imp B=\neg A\vee B\) in classical logic, as
\(A\limp B:=A\orth\parr B\).

Free and bound variables and first-order substitution are defined in the
standard way. Formulas are always considered up to renaming of bound
names. If \(A\) is a formula, \(X\) is a second-order variable and
\(B[x_1,\ldots,x_n]\) is a formula with variables \(x_i\), then the
formula \(A[B/X]\) is \(A\) where every atom \(X(t_1,\ldots,t_n)\) is
replaced by \(B[t_1,\ldots,t_n]\).

\section{Sequents and proofs}\label{sequents-and-proofs}

A sequent is an expression \(\Gamma\vdash\Delta\) where \(\Gamma\) and
\(\Delta\) are finite multisets of formulas. For a multiset
\(\Gamma=A_1,\ldots,A_n\), the notation \(\wn\Gamma\) represents the
multiset \(\wn A_1,\ldots,\wn A_n\). Proofs are labelled trees of
sequents, built using the following inference rules:

\begin{itemize}
\item
  Identity group: 
  \begin{equation*}
  \LabelRule{\rulename{axiom}}
  \NulRule{ A \vdash A }
  \DisplayProof
  \qquad\qquad
  \AxRule{ \Gamma \vdash A, \Delta }
  \AxRule{ \Gamma', A \vdash \Delta' }
  \LabelRule{\rulename{cut}}
  \BinRule{ \Gamma, \Gamma' \vdash \Delta, \Delta' }
  \DisplayProof  
  \end{equation*}
\item
  Negation: 
  \begin{equation*}
    \AxRule{ \Gamma \vdash A,\Delta }
    \LabelRule{n_L}
    \UnaRule{\Gamma, A\orth \vdash \Delta }
    \DisplayProof
    \qquad\qquad
    \AxRule{ \Gamma, A \vdash \Delta }
    \LabelRule{n_R}
    \UnaRule{ \Gamma \vdash A\orth, \Delta }
    \DisplayProof
  \end{equation*}
\item
  Multiplicative group:
  \begin{itemize}
  \item
    tensor: 
    \begin{equation*}
      \AxRule{ \Gamma, A, B \vdash \Delta }
      \LabelRule{ \tens_L }
      \UnaRule{ \Gamma, A \tens B \vdash \Delta }
      \DisplayProof
      \qquad\qquad
      \AxRule{ \Gamma \vdash A, \Delta }
      \AxRule{ \Gamma' \vdash B, \Delta' }
      \LabelRule{ \tens_R }
      \BinRule{ \Gamma, \Gamma' \vdash A \tens B, \Delta, \Delta' }
      \DisplayProof      
    \end{equation*}
  \item
    par: 
    \begin{equation*}
      \AxRule{ \Gamma, A \vdash \Delta }
      \AxRule{ \Gamma', B \vdash \Delta' }
      \LabelRule{ \parr_L }
      \BinRule{ \Gamma, \Gamma', A \parr B \vdash \Delta, \Delta' }
      \DisplayProof
      \qquad\qquad
      \AxRule{ \Gamma \vdash A, B, \Delta }
      \LabelRule{ \parr_R }
      \UnaRule{ \Gamma \vdash A \parr B, \Delta }
      \DisplayProof    
    \end{equation*}
  \item
    one:
    \begin{equation*}
       \AxRule{ \Gamma \vdash \Delta }
       \LabelRule{\one_L }
       \UnaRule{ \Gamma, \one \vdash \Delta }
       \DisplayProof
       \qquad\qquad
       \LabelRule{ \one_R }
       \NulRule{ \vdash \one }
       \DisplayProof   
    \end{equation*}
  \item
    bottom: 
    \begin{equation*}
       \LabelRule{ \bot_L }
       \NulRule{ \bot \vdash }
       \DisplayProof
       \qquad\qquad
       \AxRule{ \Gamma \vdash \Delta }
       \LabelRule{ \bot_R }
       \UnaRule{ \Gamma \vdash \bot, \Delta }
       \DisplayProof      
    \end{equation*}
  \end{itemize}
\item
  Additive group:
  \begin{itemize}
  \item
    plus: 
    \begin{equation*}
      \AxRule{ \Gamma, A \vdash \Delta }
      \AxRule{ \Gamma, B \vdash \Delta }
      \LabelRule{ \plus_L }
      \BinRule{ \Gamma, A \plus B \vdash \Delta }
      \DisplayProof
      \qquad\qquad
      \AxRule{ \Gamma \vdash A, \Delta }
      \LabelRule{ \plus_{R1} }
      \UnaRule{ \Gamma \vdash A \plus B, \Delta }
      \DisplayProof
      \qquad\qquad
      \AxRule{ \Gamma \vdash B, \Delta }
      \LabelRule{ \plus_{R2} }
      \UnaRule{ \Gamma \vdash A \plus B, \Delta }
      \DisplayProof
    \end{equation*}
  \item
    with: 
    \begin{equation*}
      \AxRule{ \Gamma, A \vdash \Delta }
      \LabelRule{ \with_{L1} }
      \UnaRule{ \Gamma, A \with B \vdash \Delta }
      \DisplayProof
      \qquad\qquad
      \AxRule{ \Gamma, B \vdash \Delta }
      \LabelRule{ \with_{L2} }
      \UnaRule{ \Gamma, A \with B \vdash \Delta }
      \DisplayProof
      \qquad\qquad
      \AxRule{ \Gamma \vdash A, \Delta }
      \AxRule{ \Gamma \vdash B, \Delta }
      \LabelRule{ \with_R }
      \BinRule{ \Gamma \vdash A \with B, \Delta }
      \DisplayProof      
    \end{equation*}
  \item
    zero: 
    \begin{equation*}
       \LabelRule{ \zero_L }
       \NulRule{ \Gamma, \zero \vdash \Delta }
       \DisplayProof
    \end{equation*}
  \item
    top: 
    \begin{equation*}
       \LabelRule{ \top_R }
       \NulRule{ \Gamma \vdash \top, \Delta }
       \DisplayProof    
    \end{equation*}
  \end{itemize}
\item
  Exponential group:
  \begin{itemize}
  \item
    of course: 
    \begin{gather*}
       \AxRule{ \Gamma, A \vdash \Delta }
       \LabelRule{ d_L }
       \UnaRule{ \Gamma, \oc A \vdash \Delta }
       \DisplayProof
       \qquad\qquad
       \AxRule{ \Gamma \vdash \Delta }
       \LabelRule{ w_L }
       \UnaRule{ \Gamma, \oc A \vdash \Delta }
       \DisplayProof
       \qquad\qquad
       \AxRule{ \Gamma, \oc A, \oc A \vdash \Delta }
       \LabelRule{ c_L }
       \UnaRule{ \Gamma, \oc A \vdash \Delta }
       \DisplayProof
       \\[2ex]
       \AxRule{ \oc A_1, \ldots, \oc A_n \vdash B ,\wn B_1, \ldots, \wn B_m }
       \LabelRule{ \oc_R }
       \UnaRule{ \oc A_1, \ldots, \oc A_n \vdash \oc B ,\wn B_1, \ldots, \wn B_m }
       \DisplayProof      
    \end{gather*}
  \item
    why not:
    \begin{gather*}
      \AxRule{ \Gamma \vdash A, \Delta }
      \LabelRule{ d_R }
      \UnaRule{ \Gamma \vdash \wn A, \Delta }
      \DisplayProof
      \qquad\qquad
      \AxRule{ \Gamma \vdash \Delta }
      \LabelRule{ w_R }
      \UnaRule{ \Gamma \vdash \wn A, \Delta }
      \DisplayProof
      \qquad\qquad
      \AxRule{ \Gamma \vdash \wn A, \wn A, \Delta }
      \LabelRule{ c_R }
      \UnaRule{ \Gamma \vdash \wn A, \Delta }
      \DisplayProof
      \\[2ex]
      \AxRule{ \oc A_1, \ldots, \oc A_n, A \vdash \wn B_1, \ldots, \wn B_m }
      \LabelRule{ \wn_L }
      \UnaRule{ \oc A_1, \ldots, \oc A_n, \wn A \vdash \wn B_1, \ldots, \wn B_m }
      \DisplayProof      
    \end{gather*}
  \end{itemize}
\item
  Quantifier group (in the \(\exists_L\) and \(\forall_R\) rules,
  \(\xi\) must not occur free in \(\Gamma\) or \(\Delta\)):
  \begin{itemize}
  \item
    there exists: 
    \begin{equation*}
      \AxRule{ \Gamma , A \vdash \Delta }
      \LabelRule{ \exists_L }
      \UnaRule{\Gamma, \exists\xi.A \vdash \Delta }
      \DisplayProof
      \qquad\qquad
      \AxRule{ \Gamma \vdash \Delta, A[t/x] }
      \LabelRule{ \exists^1_R }
      \UnaRule{ \Gamma \vdash \Delta, \exists x.A }
      \DisplayProof
      \qquad\qquad
      \AxRule{ \Gamma \vdash \Delta, A[B/X] }
      \LabelRule{ \exists^2_R }
      \UnaRule{ \Gamma \vdash \Delta, \exists X.A }
      \DisplayProof
    \end{equation*}
  \item
    for all: 
    \begin{equation*}
      \AxRule{ \Gamma, A[t/x] \vdash \Delta }
      \LabelRule{ \forall^1_L }
      \UnaRule{ \Gamma, \forall x.A \vdash \Delta }
      \DisplayProof
      \qquad\qquad
      \AxRule{ \Gamma, A[B/X] \vdash \Delta }
      \LabelRule{ \forall^2_L }
      \UnaRule{ \Gamma, \forall X.A \vdash \Delta }
      \DisplayProof
      \qquad\qquad
      \AxRule{ \Gamma \vdash \Delta, A }
      \LabelRule{ \forall_R }
      \UnaRule{ \Gamma \vdash \Delta, \forall\xi.A }
      \DisplayProof
    \end{equation*}
  \end{itemize}
\end{itemize}


The left rules for \emph{of course} and right rules for \emph{why not}
are called \emph{dereliction}, \emph{weakening} and \emph{contraction}
rules. The right rule for \emph{of course} and the left rule for
\emph{why not} are called \emph{promotion} rules. Note the fundamental
fact that there are no contraction and weakening rules for arbitrary
formulas, but only for the formulas starting with the \(\wn\) modality.
This is what distinguishes linear logic from classical logic: if
weakening and contraction were allowed for arbitrary formulas, then
\(\tens\) and \(\with\) would be identified, as well as \(\plus\) and
\(\parr\), \(\one\) and \(\top\), \(\zero\) and \(\bot\). By
\emph{identified}, we mean here that replacing a \(\tens\) with a
\(\with\) or vice versa would preserve provability.

Sequents are considered as multisets, in other words as sequences up to
permutation. An alternative presentation would be to define a sequent as
a finite sequence of formulas and to add the exchange rules:

\begin{equation*}
  \AxRule{ \Gamma\_1, A, B, \Gamma\_2 \vdash \Delta }
  \LabelRule{\rulename{exchange}_L}
  \UnaRule{ \Gamma\_1, B, A, \Gamma\_2 \vdash \Delta }
  \DisplayProof
  \qquad\qquad
  \AxRule{ \Gamma \vdash \Delta_1, A, B, \Delta_2 }
  \LabelRule{\rulename{exchange}_R}
  \UnaRule{ \Gamma \vdash \Delta_1, B, A, \Delta_2 }
  \DisplayProof
\end{equation*}


\section{Equivalences}\label{equivalences}

Two formulas \(A\) and \(B\) are (linearly) equivalent, written
\(A\linequiv B\), if both implications \(A\limp B\) and \(B\limp A\) are
provable. Equivalently, \(A\linequiv B\) if both \(A\vdash B\) and
\(B\vdash A\) are provable. Another formulation of \(A\linequiv B\) is
that, for all \(\Gamma\) and \(\Delta\), \(\Gamma\vdash\Delta,A\) is
provable if and only if \(\Gamma\vdash\Delta,B\) is provable.

Two related notions are \hyperref[isomorphism]{isomorphism} (stronger than equivalence)
and \hyperref[equiprovability]{equiprovability} (weaker than equivalence).

\subsection{De Morgan laws}\label{de-morgan-laws}

Negation is involutive:
\qquad
\(A\linequiv A\biorth\)

Duality between connectives:
\begin{longtable}[]{@{}lllll@{}}
\toprule
\(( A \tens B )\orth\) & \(\linequiv A\orth \parr B\orth\) & &
\(( A \parr B )\orth\) &
\(\linequiv A\orth \tens B\orth\)\tabularnewline
\(\one\orth\) & \(\linequiv \bot\) & & \(\bot\orth\) &
\(\linequiv \one\)\tabularnewline
\(( A \plus B )\orth\) & \(\linequiv A\orth \with B\orth\) & &
\(( A \with B )\orth\) &
\(\linequiv A\orth \plus B\orth\)\tabularnewline
\(\zero\orth\) & \(\linequiv \top\) & & \(\top\orth\) &
\(\linequiv \zero\)\tabularnewline
\(( \oc A )\orth\) & \(\linequiv \wn ( A\orth )\) & & \(( \wn A )\orth\)
& \(\linequiv \oc ( A\orth )\)\tabularnewline
\(( \exists \xi.A )\orth\) & \(\linequiv \forall \xi.( A\orth )\) & &
\(( \forall \xi.A )\orth\) &
\(\linequiv \exists \xi.( A\orth )\)\tabularnewline
\bottomrule
\end{longtable}

\subsection{Fundamental equivalences}\label{fundamental-equivalences}

\begin{itemize}
\item
  Associativity, commutativity, neutrality:
  \begin{equation*}
    \begin{array}{ccc}
    A \tens (B \tens C) \linequiv (A \tens B) \tens C & A \tens B \linequiv B \tens A & A \tens \one \linequiv A \\
    A \parr (B \parr C) \linequiv (A \parr B) \parr C & A \parr B \linequiv B \parr A & A \parr \bot \linequiv A \\
    A \plus (B \plus C) \linequiv (A \plus B) \plus C & A \plus B \linequiv B \plus A & A \plus \zero \linequiv A \\
A \with (B \with C) \linequiv (A \with B) \with C & A \with B \linequiv B \with A & A \with \top \linequiv A
    \end{array}
  \end{equation*}
\item
  Idempotence of additives:
  \begin{equation*}
    \begin{array}{cc}
      A \plus A \linequiv A & A \with A \linequiv A
    \end{array}
  \end{equation*}
\item
  Distributivity of multiplicatives over additives:
  \begin{equation*}
    \begin{array}{cc}
       A \tens (B \plus C) \linequiv (A \tens B) \plus (A \tens C) & A \tens \zero \linequiv \zero \\
       A \parr (B \with C) \linequiv (A \parr B) \with (A \parr C) & A \parr \top \linequiv \top
    \end{array}
  \end{equation*}
\item
  Defining property of exponentials:
  \begin{equation*}
    \begin{array}{cc}
       \oc(A \with B) \linequiv \oc A \tens \oc B & \oc\top \linequiv \one \\
       \wn(A \plus B) \linequiv \wn A \parr \wn B & \wn\zero \linequiv \bot
    \end{array}
  \end{equation*}
\item
  Monoidal structure of exponentials:
  \begin{equation*}
    \begin{array}{cc}
       \oc A \tens \oc A \linequiv \oc A & \oc \one \linequiv \one \\
       \wn A \parr \wn A \linequiv \wn A & \wn \bot \linequiv \bot
    \end{array}
  \end{equation*}
\item
  Digging:
  \begin{equation*}
    \begin{array}{cc}
       \oc\oc A \linequiv \oc A & \wn\wn A \linequiv \wn A
    \end{array}
  \end{equation*}
\item
  Other properties of exponentials:
  \begin{equation*}
    \begin{array}{cc}
       \oc\wn\oc\wn A \linequiv \oc\wn A & \oc\wn \one \linequiv \one \\
       \wn\oc\wn\oc A \linequiv \wn\oc A & \wn\oc \bot \linequiv \bot
    \end{array}
  \end{equation*}
These properties of exponentials lead to
the \hyperref[lattice-of-exponential-modalities]{lattice of exponential modalities}.
\item
  Commutation of quantifiers (\(\zeta\) does not occur in \(A\)):
  \begin{equation*}
    \begin{array}{cccc}
       \exists \xi. \exists \psi. A \linequiv \exists \psi. \exists \xi. A & \exists \xi.(A \plus B) \linequiv \exists \xi.A \plus \exists \xi.B & \exists \zeta.(A\tens B) \linequiv A\tens\exists \zeta.B & \exists \zeta.A \linequiv A \\
       \forall \xi. \forall \psi. A \linequiv \forall \psi. \forall \xi. A & \forall \xi.(A \with B) \linequiv \forall \xi.A \with \forall \xi.B & \forall \zeta.(A\parr B) \linequiv A\parr\forall \zeta.B & \forall \zeta.A \linequiv A
    \end{array}
  \end{equation*}
\end{itemize}


\subsection{Definability}\label{definability}

The units and the additive connectives can be defined using second-order
quantification and exponentials, indeed the following equivalences hold:

\begin{itemize}
\item \(\zero \linequiv \forall X.X\)
\item \(\one \linequiv \forall X.(X \limp X)\)
\item \(A \plus B \linequiv \forall X.(\oc(A \limp X) \limp \oc(B \limp X) \limp X)\)
\end{itemize}

The constants \(\top\) and \(\bot\) and the connective \(\with\) can be
defined by duality.

Any pair of connectives that has the same rules as \(\tens/\parr\) is
equivalent to it, the same holds for additives, but not for
exponentials.

Other \hyperref[list-of-equivalences]{basic equivalences} exist.

\section{Properties of proofs}\label{properties-of-proofs}

\subsection{Cut elimination and consequences}\label{cut-elimination-and-consequences}

\begin{theorem}[cut elimination]
For every sequent $\Gamma\vdash\Delta$, there is a proof of
$\Gamma\vdash\Delta$ if and only if there is a proof of
$\Gamma\vdash\Delta$ that does not use the cut rule.
\end{theorem}

This property is proved using a set of rewriting rules on proofs, using
appropriate termination arguments (see the specific articles on
\wantedpage{cut elimination} for detailed proofs), it is the
core of the proof/program correspondence.

It has several important consequences:

\begin{definition}[subformula]
The subformulas of a formula $A$ are $A$ and, inductively, the subformulas of its immediate subformulas:
\begin{itemize}
\item the immediate subformulas of $A\tens B$, $A\parr B$, $A\plus B$, $A\with B$ are $A$ and $B$,
\item the only immediate subformula of $\oc A$ and $\wn A$ is $A$,
\item $\one$, $\bot$, $\zero$, $\top$ and atomic formulas have no immediate subformula,
\item the immediate subformulas of $\exists x.A$ and $\forall x.A$ are all the $A[t/x]$ for all first-order terms $t$,
\item the immediate subformulas of $\exists X.A$ and $\forall X.A$ are all the $A[B/X]$ for all formulas $B$ (with the appropriate number of parameters).
\end{itemize}
\end{definition}

\begin{theorem}[subformula property]
A sequent $\Gamma\vdash\Delta$ is provable if and only if it is the conclusion of a proof in which each intermediate conclusion is made of subformulas of the
formulas of $\Gamma$ and $\Delta$.
\end{theorem}

\begin{proof}
By the cut elimination theorem, if a sequent is provable, then it is provable by a cut-free proof.
In each rule except the cut rule, all formulas of the premisses are either
formulas of the conclusion, or immediate subformulas of it, therefore
cut-free proofs have the subformula property.
\end{proof}

The subformula property means essentially nothing in the second-order
system, since any formula is a subformula of a quantified formula where
the quantified variable occurs. However, the property is very meaningful
if the sequent \(\Gamma\) does not use second-order quantification, as
it puts a strong restriction on the set of potential proofs of a given
sequent. In particular, it implies that the first-order fragment without
quantifiers is decidable.

\begin{theorem}[consistency]
The empty sequent $\vdash$ is not provable.
Subsequently, it is impossible to prove both a formula $A$ and its
negation $A\orth$; it is impossible to prove $\zero$ or
$\bot$.
\end{theorem}

\begin{proof}
If a sequent is provable, then it is the conclusion of a cut-free proof.
In each rule except the cut rule, there is at least one formula in conclusion.
Therefore $\vdash$ cannot be the conclusion of a proof.
The other properties are immediate consequences: if $\vdash A\orth$
and $\vdash A$ are provable, then by the left negation rule
$A\orth\vdash$ is provable, and by the cut rule one gets empty
conclusion, which is not possible.
As particular cases, since $\one$ and $\top$ are
provable, $\bot$ and $\zero$ are not, since they are
equivalent to $\one\orth$ and $\top\orth$
respectively.
\end{proof}

\subsection{Expansion of identities}\label{expansion-of-identities}

Let us write \(\pi:\Gamma\vdash\Delta\) to signify that \(\pi\) is a
proof with conclusion \(\Gamma\vdash\Delta\).

\begin{proposition}[$\eta$-expansion]
For every proof $\pi$, there is a proof $\pi'$ with the
same conclusion as $\pi$ in which the axiom rule is only used with
atomic formulas.
If $\pi$ is cut-free, then there is a cut-free $\pi'$.
\end{proposition}

\begin{proof}
It suffices to prove that for every formula $A$, the sequent
$A\vdash A$ has a cut-free proof in which the axiom rule is used
only for atomic formulas.
We prove this by induction on $A$.
\begin{itemize}
\item If $A$ is atomic, then $A\vdash A$ is an instance of the atomic axiom rule.
\item If $A=A_1\tens A_2$ then we have:
\begin{prooftree}
\AxRule{ \pi_1 : A_1 \vdash A_1 }
\AxRule{ \pi_2 : A_2 \vdash A_2 }
\LabelRule{ \tens_R }
\BinRule{ A_1, A_2 \vdash A_1 \tens A_2 }
\LabelRule{ \tens_L }
\UnaRule{ A_1 \tens A_2 \vdash A_1 \tens A_2 }
\end{prooftree}
where $\pi_1$ and $\pi_2$ exist by induction hypothesis.
\item If $A=A_1\parr A_2$ then we have:
\begin{prooftree}
\AxRule{ \pi_1 : A_1 \vdash A_1 }
\AxRule{ \pi_2 : A_2 \vdash A_2 }
\LabelRule{ \parr_L }
\BinRule{ A_1 \parr A_2 \vdash A_1, A_2 }
\LabelRule{ \parr_R }
\UnaRule{ A_1 \parr A_2 \vdash A_1 \parr A_2 }
\end{prooftree}
where $\pi_1$ and $\pi_2$ exist by induction hypothesis.
\item All other connectives follow the same pattern.
\end{itemize}
\end{proof}

The interesting thing with \(\eta\)-expansion is that, we can always
assume that each connective is explicitly introduced by its associated
rule (except in the case where there is an occurrence of the \(\top\)
rule).

\subsection{Reversibility}\label{reversibility-1}

\begin{definition}[reversibility]
A connective $c$ is called \emph{reversible} if
\begin{itemize}
\item for every proof $\pi:\Gamma\vdash\Delta,c(A_1,\ldots,A_n)$, there is a proof $\pi'$ with the same conclusion in which $c(A_1,\ldots,A_n)$ is introduced by the last rule,
\item if $\pi$ is cut-free then there is a cut-free $\pi'$.
\end{itemize}
\end{definition}

\begin{proposition}
The connectives $\parr$, $\bot$, $\with$, $\top$ and $\forall$ are reversible.
\end{proposition}

\begin{proof}
Using the $\eta$-expansion property, we assume that the axiom rule is only applied to atomic formulas.
Then each top-level connective is introduced either by its associated (left or
right) rule or in an instance of the $\zero_L$ or
$\top_R$ rule.

For $\parr$, consider a proof $\pi\Gamma\vdash\Delta,A\parr
B$.
If $A\parr B$ is introduced by a $\parr_R$ rule (not
necessarily the last rule in $\pi$), then if we remove this rule
we get a proof of $\vdash\Gamma,A,B$ (this can be proved by a
straightforward induction on $\pi$).
If it is introduced in the context of a $\zero_L$ or
$\top_R$ rule, then this rule can be changed so that
$A\parr B$ is replaced by $A,B$.
In either case, we can apply a final $\parr$ rule to get the
expected proof.

For $\bot$, the same technique applies: if it is introduced by a
$\bot_R$ rule, then remove this rule to get a proof of
$\vdash\Gamma$, if it is introduced by a $\zero_L$ or
$\top_R$ rule, remove the $\bot$ from this rule, then
apply the $\bot$ rule at the end of the new proof.

For $\with$, consider a proof
$\pi:\Gamma\vdash\Delta,A\with B$.
If the connective is introduced by a $\with$ rule then this rule is
applied in a context like
\begin{prooftree}
\AxRule{ \pi_1: \Gamma' \vdash \Delta', A }
\AxRule{ \pi_2: \Gamma' \vdash \Delta', B }
\LabelRule{ \with }
\BinRule{ \Gamma' \vdash \Delta', A \with B }
\end{prooftree}

Since the formula $A\with B$ is not involved in other rules (except
as context), if we replace this step by $\pi_1$ in $\pi$
we finally get a proof $\pi'_1:\Gamma\vdash\Delta,A$.
If we replace this step by $\pi_2$ we get a proof
$\pi'_2:\Gamma\vdash\Delta,B$.
Combining $\pi_1$ and $\pi_2$ with a final
$\with$ rule we finally get the expected proof.
The case when the $\with$ was introduced in a $\top$
rule is solved as before.

For $\top$ the result is trivial: just choose $\pi'$ as
an instance of the $\top$ rule with the appropriate conclusion.

For $\forall$, consider a proof
$\pi:\Gamma\vdash\Delta,\forall\xi.A$.
Up to renaming, we can assume that $\xi$ occurs free only above the
rule that introduces the quantifier.
If the quantifier is introduced by a $\forall$ rule, then if we
remove this rule, we can check that we get a proof of
$\Gamma\vdash\Delta,A$ on which we can finally apply the
$\forall$ rule.
The case when the $\forall$ was introduced in a $\top$
rule is solved as before.

Note that, in each case, if the proof we start from is cut-free, our
transformations do not introduce a cut rule.
However, if the original proof has cuts, then the final proof may have more
cuts, since in the case of $\with$ we duplicated a part of the
original proof.
\end{proof}

A corresponding property for positive connectives is
\hyperref[reversibility-and-focusing]{focusing}, which states that
clusters of positive formulas can be treated in one step, under certain
circumstances.

\section{One-sided sequent calculus}\label{one-sided-sequent-calculus}

The sequent calculus presented above is very symmetric: for every left
introduction rule, there is a right introduction rule for the dual
connective that has the exact same structure. Moreover, because of the
involutivity of negation, a sequent \(\Gamma,A\vdash\Delta\) is provable
if and only if the sequent \(\Gamma\vdash A\orth,\Delta\) is provable.
From these remarks, we can define an equivalent one-sided sequent
calculus:

\begin{itemize}
\item Formulas are considered up to De Morgan duality. Equivalently, one can
  consider that negation is not a connective but a syntactically defined
  operation on formulas. In this case, negated atoms \(\alpha\orth\)
  must be considered as another kind of atomic formulas.
\item Sequents have the form \(\vdash\Gamma\).
\end{itemize}

The inference rules are essentially the same except that the left hand
side of sequents is kept empty:

\begin{itemize}
\item
  Identity group:
\begin{equation*}
  \LabelRule{\rulename{axiom}}
  \NulRule{ \vdash A\orth, A}
  \DisplayProof
  \qquad\qquad
  \AxRule{ \vdash \Gamma, A }
  \AxRule{ \vdash \Delta, A\orth }
  \LabelRule{\rulename{cut}}
  \BinRule{ \vdash \Gamma, \Delta }
  \DisplayProof
\end{equation*}
\item
  Multiplicative group:
\begin{equation*}
\AxRule{ \vdash \Gamma, A
} \AxRule{ \vdash \Delta,
B } \LabelRule{ \tens }
\BinRule{ \vdash \Gamma,
\Delta, A \tens B }
\DisplayProof
\qquad\qquad
\AxRule{ \vdash \Gamma, A, B }
\LabelRule{ \parr }
\UnaRule{ \vdash \Gamma, A \parr B }
\DisplayProof
\qquad\qquad
\LabelRule{ \one }
\NulRule{ \vdash \one }
\DisplayProof
\qquad\qquad
\AxRule{ \vdash \Gamma }
\LabelRule{ \bot }
\UnaRule{ \vdash \Gamma, \bot }
\DisplayProof
\end{equation*}
\item
  Additive group:
\begin{equation*}
\AxRule{ \vdash \Gamma, A} \LabelRule{ \plus\_1 }
\UnaRule{ \vdash \Gamma, A \plus B }
\DisplayProof
\qquad\qquad
\AxRule{ \vdash \Gamma, B }
\LabelRule{ \plus_2 }
\UnaRule{ \vdash \Gamma, A \plus B }
\DisplayProof
\qquad\qquad
\AxRule{ \vdash \Gamma, A }
\AxRule{ \vdash \Gamma, B }
\LabelRule{ \with }
\BinRule{ \vdash, \Gamma, A \with B }
\DisplayProof
\qquad\qquad
\LabelRule{ \top }
\NulRule{ \vdash \Gamma, \top }
\DisplayProof
\end{equation*}
\item
  Exponential group:
\begin{equation*}
\AxRule{ \vdash \Gamma, A } \LabelRule{ d }
\UnaRule{ \vdash \Gamma, \wn A }
\DisplayProof
\qquad\qquad
\AxRule{ \vdash \Gamma }
\LabelRule{ w }
\UnaRule{ \vdash \Gamma, \wn A }
\DisplayProof
\qquad\qquad
\AxRule{ \vdash \Gamma, \wn A, \wn A }
\LabelRule{ c }
\UnaRule{ \vdash \Gamma, \wn A }
\DisplayProof
\qquad\qquad
\AxRule{ \vdash \wn\Gamma, B }
\LabelRule{ \oc }
\UnaRule{ \vdash \wn\Gamma, \oc B }
\DisplayProof
\end{equation*}
\item
  Quantifier group (in the \(\forall\) rule, \(\xi\) must not occur free in \(\Gamma\)):
\begin{equation*}
\AxRule{ \vdash \Gamma,
A[t/x] } \LabelRule{ \exists^1 }
\UnaRule{ \vdash \Gamma, \exists x.A }
\DisplayProof
\qquad\qquad
\AxRule{ \vdash \Gamma, A[B/X] }
\LabelRule{ \exists^2 }
\UnaRule{ \vdash \Gamma, \exists X.A }
\DisplayProof
\qquad\qquad
\AxRule{ \vdash \Gamma, A }
\LabelRule{ \forall }
\UnaRule{ \vdash \Gamma, \forall \xi.A }
\DisplayProof
\end{equation*}
\end{itemize}

\begin{theorem}
A two-sided sequent $\Gamma\vdash\Delta$ is provable if
and only if the sequent $\vdash\Gamma\orth,\Delta$ is provable in
the one-sided system.
\end{theorem}

The one-sided system enjoys the same properties as the two-sided one,
including cut elimination, the subformula property, etc. This
formulation is often used when studying proofs because it is much
lighter than the two-sided form while keeping the same expressiveness.
In particular, \hyperref[proof-nets]{proof-nets} can be seen as a quotient of one-sided
sequent calculus proofs under commutation of rules.

\section{Variations}\label{variations}

\subsection{Exponential rules}\label{exponential-rules}

\begin{itemize}
\item
  The promotion rule, on the right-hand side for example,
\(\AxRule{ \oc A_1, \ldots, \oc A_n \vdash B, \wn B_1, \ldots, \wn B_m }
\LabelRule{ \oc_R }
\UnaRule{ \oc A_1, \ldots, \oc A_n \vdash \oc B, \wn B_1, \ldots, \wn B_m }
\DisplayProof\) can be replaced by a \emph{multi-functorial} promotion
rule \(\AxRule{ A_1, \ldots, A_n \vdash B, B_1, \ldots, B_m }
\LabelRule{ \oc_R \rulename{mf}}
\UnaRule{ \oc A_1, \ldots, \oc A_n \vdash \oc B, \wn B_1, \ldots, \wn B_m }
\DisplayProof\) and a \emph{digging} rule
\(\AxRule{ \Gamma \vdash \wn\wn A, \Delta }
\LabelRule{ \wn\wn}
\UnaRule{ \Gamma \vdash \wn A, \Delta }
\DisplayProof\), without modifying the provability.

Note that digging violates the subformula property.
\item In presence of the digging rule 
\AxRule{ \Gamma \vdash
\wn\wn A, \Delta }
\LabelRule{ \wn\wn}
\UnaRule{ \Gamma \vdash
\wn A, \Delta }
\DisplayProof , the multiplexing rule
\(\AxRule{\Gamma\vdash A^{(n)},\Delta}
\LabelRule{\rulename{mplex}}
\UnaRule{\Gamma\vdash \wn A,\Delta}
\DisplayProof\) (where \(A^{(n)}\) stands for n occurrences of formula
\(A\)) is equivalent (for provability) to the triple of rules:
contraction, weakening, dereliction.
\end{itemize}

\subsection{Non-symmetric sequents}\label{non-symmetric-sequents}

The same remarks that lead to the definition of the one-sided calculus
can lead the definition of other simplified systems:

\begin{itemize}
\item A one-sided variant with sequents of the form \(\Gamma\vdash\) could
  be defined.
\item When considering formulas up to De Morgan duality, an equivalent
  system is obtained by considering only the left and right rules for
  positive connectives (or the ones for negative connectives only,
  obviously).
\item \hyperref[intuitionistic-linear-logic]{Intuitionistic linear logic} is the
  two-sided system where the right-hand side is constrained to always
  contain exactly one formula (with a few associated restrictions).
\item Similar restrictions are used in various \hyperref[semantics]{semantics} and \wantedpage{proof search} formalisms.
\end{itemize}

\subsection{Mix rules}\label{mix-rules}

It is quite common to consider \hyperref[mix]{mix rules}:
\(\LabelRule{\rulename{Mix}_0}
\NulRule{\vdash}
\DisplayProof
\qquad
\AxRule{\Gamma \vdash \Delta}
\AxRule{\Gamma' \vdash \Delta'}
\LabelRule{\rulename{Mix}_2}
\BinRule{\Gamma,\Gamma' \vdash \Delta,\Delta'}
\DisplayProof\)


%%% Local Variables:
%%% mode: latex
%%% TeX-master: "main"
%%% End:

\section{Equiprovability}\label{equiprovability}

Two formulas \(A\) and \(B\) are equiprovable, when \(\vdash A\) is
provable if and only if \(\vdash B\) is provable.

\begin{itemize}
\tightlist
\item
  for any \(A\) and \(B\), \(A\tens B\) and \(A\with B\) are
  equiprovable.
\item
  for any \(A\), \(A\), \(\oc A\) and \(\forall\xi A\) are equiprovable.
\end{itemize}


\section{Isomorphism}\label{isomorphism}

\subsection{Definition}

Two formulas \(A\) and \(B\) are isomorphic (denoted \(A\cong B\)), when
there are two proofs \(\pi\) of \(A \vdash B\) and \(\rho\) of
\(B \vdash A\) such that eliminating the cut on \(A\) in

\(\AxRule{}\VdotsRule{\pi}{A \vdash B}\AxRule{}\VdotsRule{\rho}{B \vdash A}\LabelRule{\rulename{cut}}\BinRule{B\vdash B}\DisplayProof\)

leads to an
\hyperref[expansion-of-identities]{\(\eta\)-expansion} of

\(\LabelRule{\rulename{ax}}\NulRule{B\vdash B}\DisplayProof\),

and eliminating the cut on \(B\) in

\(\AxRule{}\VdotsRule{\pi}{A \vdash B}\AxRule{}\VdotsRule{\rho}{B \vdash A}\LabelRule{\rulename{cut}}\BinRule{A\vdash A}\DisplayProof\)

leads to an
\hyperref[expansion-of-identities]{\(\eta\)-expansion} of

\(\LabelRule{\rulename{ax}}\NulRule{A\vdash A}\DisplayProof\).

Linear logic admits \hyperref[list-of-isomorphisms]{many isomorphisms}, but
it is not known wether all of them have been discovered or not.


%%% Local Variables:
%%% mode: latex
%%% TeX-master: "main"
%%% End:

\subsection{List of isomorphisms}\label{list-of-isomorphisms}

\subsubsection{Linear negation}\label{linear-negation-1}

\(\begin{array}{rclcrcl}
  A\biorth & \cong&  A\\
  (A\tens B)\orth & \cong&  A\orth\parr B\orth & \quad&  \one\orth  & \cong&  \bot\\
  (A\parr B)\orth & \cong&  A\orth\tens B\orth & \quad&  \bot\orth  & \cong&  \one\\
  (A\with B)\orth & \cong&  A\orth\plus B\orth & \quad&  \top\orth  & \cong&  \zero\\
  (A\plus B)\orth & \cong&  A\orth\with B\orth & \quad&  \zero\orth & \cong&  \top\\
  (\oc A)\orth & \cong&  \wn A\orth\\
  (\wn A)\orth & \cong&  \oc A\orth\\
\end{array}\)

\subsubsection{Neutrals}\label{neutrals}

\(\begin{array}{rclcl}
  A\tens\one  & \cong&  \one\tens A&\cong &A\\
  A\parr\bot  & \cong&  \bot\parr A&\cong &A\\
  A\with\top  & \cong&  \top\with A&\cong &A\\
  A\plus\zero & \cong& \zero\plus A&\cong &A\\
\end{array}\)

\subsubsection{Commutativity}\label{commutativity-1}

\(\begin{array}{rcl}
  A\tens B & \cong&  B\tens A\\
  A\parr B & \cong&  B\parr A\\
  A\with B & \cong&  B\with A\\
  A\plus B & \cong&  B\plus A\\
\end{array}\)

\subsubsection{Associativity}\label{associativity}

\(\begin{array}{rcl}
  (A\tens B)\tens C & \cong&  A\tens(B\tens C)\\
  (A\parr B)\parr C & \cong&  A\parr(B\parr C)\\
  (A\with B)\with C & \cong&  A\with(B\with C)\\
  (A\plus B)\plus C & \cong&  A\plus(B\plus C)\\
\end{array}\)

\subsubsection{Multiplicative-additive distributivity}\label{multiplicative-additive-distributivity}

\(\begin{array}{rclcrcl}
  A\tens(B\plus C) & \cong&  (A\tens B)\plus(A\tens C) & \quad& 
  A\tens\zero & \cong&  \zero\\
  A\parr(B\with C) & \cong&  (A\parr B)\with(A\parr C) & \quad& 
  A\parr\top & \cong&  \top\\
\end{array}\)

\subsubsection{Linear implication}\label{linear-implication}

\(\begin{array}{rclcrcl}
  A\limp B & \cong&  A\orth\parr B\\
  A\limp B & \cong&  B\orth\limp A\orth\\
  A\tens B \limp C & \cong&  A\limp B \limp C\\
\end{array}\)

\subsubsection{The exponential isomorphisms}\label{the-exponential-isomorphisms}

\(\begin{array}{rclcrcl}
  \oc(A\with B) & \cong&  \oc A\tens\oc B & \quad&  \oc\top & \cong&  \one\\
  \wn(A\plus B) & \cong&  \wn A\parr\wn B & \quad&  \wn\zero & \cong&  \bot\\
\end{array}\)

\subsubsection{Quantifiers}\label{quantifiers-1}

\(\begin{array}{rclcrcl}
  \forall \xi_1. \forall\xi_2. A & \cong&  \forall\xi_2. \forall\xi_1. A\\
  \exists \xi_1. \exists\xi_2.A & \cong&  \exists\xi_2.\exists\xi_1.A\\
\\
  \forall \xi . (A \parr B) & \cong&  A \parr \forall \xi.B \quad (\xi\notin A) \\
  \exists \xi . (A \tens B) & \cong&  A \tens \exists \xi.B \quad (\xi\notin A) \\
\\
  \forall \xi . (A \with B) & \cong&  (\forall \xi . A) \with (\forall \xi . B) &  &  \forall \xi . \top & \cong&  \top \\
  \exists \xi . (A \plus B) & \cong&  (\exists \xi . A) \plus (\exists \xi . B) &  &  \exists \xi . \zero & \cong&  \zero
\end{array}\)


%%% Local Variables:
%%% mode: latex
%%% TeX-master: "main"
%%% End:

\section{List of equivalences}\label{list-of-equivalences}

Each \href{List_of_isomorphisms}{isomorphism} gives an equivalence of
formulas. The following equivalences are \emph{not} isomorphisms.

\subsection{Multiplicatives}\label{multiplicatives-1}

\(\begin{array}{rcccl}
A &amp;\linequiv&amp; A \tens (A\orth\parr A) &amp;\linequiv&amp; (A\tens A\orth)\parr A \\
&amp; &amp; A\parr A\orth &amp;\linequiv&amp; (A\parr A\orth)\tens(A\parr A\orth)
\end{array}\)

\subsection{Additives}\label{additives-1}

\(\begin{array}{rclcrcl}
A \with A &amp;\linequiv&amp; A \\
A \plus A &amp;\linequiv&amp; A \\
\\
  A \with (A \plus B) &amp;\linequiv&amp; A &amp;\quad&amp; A \plus \top &amp;\linequiv&amp; \top \\
  A \plus (A \with B) &amp;\linequiv&amp; A &amp;\quad&amp; A \with \zero &amp;\linequiv&amp; \zero
\end{array}\)

\subsection{Quantifiers}\label{quantifiers}

\(\begin{array}{rcll}
  \forall X.A &amp;\linequiv&amp; A &amp;\quad (X\notin A) \\
  \exists X.A &amp;\linequiv&amp; A &amp;\quad (X\notin A)
\end{array}\)

\subsection{Exponentials}\label{exponentials-2}

\(\begin{array}{rclcrcl}
  \oc A &amp;\linequiv&amp; \oc A\tens\oc A &amp;\quad&amp; 
  \wn A &amp;\linequiv&amp; \wn A\parr\wn A\\
  \oc A &amp;\linequiv&amp; \oc\oc A &amp;\quad&amp; \wn A &amp;\linequiv&amp; \wn\wn A\\
  \oc\wn A &amp;\linequiv&amp; \oc\wn\oc\wn A &amp;\quad&amp; \wn\oc A &amp;\linequiv&amp; \wn\oc\wn\oc A\\
\end{array}\)

Some of these equivalences are related with the
\href{lattice_of_exponential_modalities}{lattice of exponential
modalities}.

\subsection{Polarities}\label{polarities}

\begin{longtable}[]{@{}ll@{}}
\toprule
\(\wn N \linequiv N\) & \&nbsp;\&nbsp;(N
\href{Negative_formula}{negative})\tabularnewline
\(\oc P \linequiv P\) & \&nbsp;\&nbsp;(P
\href{Positive_formula}{positive})\tabularnewline
\(\wn\oc R \linequiv R\) & \&nbsp;\&nbsp;(R
\href{Regular_formula}{regular})\tabularnewline
\(\oc\wn L \linequiv L\) & \&nbsp;\&nbsp;(L
\href{Co-regular_formula}{co-regular})\tabularnewline
\bottomrule
\end{longtable}

\subsection{Second order encodings}\label{second-order-encodings}

\(\begin{array}{rclcrcl}
  A &amp;\linequiv &amp;\forall X . (A \tens X\orth) \parr X \\
  A &amp;\linequiv &amp;\exists X . (A \parr X\orth) \tens X \\
\\
  A \with B &amp;\linequiv&amp; \exists X . \oc{(A \parr X\orth)} \tens \oc{(B \parr X\orth)} \tens X &amp;\quad&amp; \top &amp;\linequiv&amp; \exists X . X \\
  A \plus B &amp;\linequiv&amp; \forall X . \wn{(A \tens X\orth)} \parr \wn{(B \tens X\orth)} \parr X &amp;\quad&amp; \zero &amp;\linequiv&amp; \forall X . X \\
\\
 \bot &amp;\linequiv&amp; \exists X . X\tens X\orth \\
 \one &amp;\linequiv&amp; \forall X . X\orth\parr X \\
\\
  \forall \xi . A &amp;\linequiv&amp; \exists X . (\forall \xi . (A \parr X\orth)) \tens X \\
  \exists \xi . A &amp;\linequiv&amp; \forall X . (\exists \xi . (A \tens X\orth)) \parr X
\end{array}\)

\subsection{Miscellaneous}\label{miscellaneous}

\(\begin{array}{rcl}
  \one &amp;\linequiv&amp; \oc{(A\orth\parr A)} \\
  \bot &amp;\linequiv&amp; \wn{(A\orth\tens A)} \\
\\
  \oc{\wn{(\oc{A}\with\oc{B})}} &amp;\linequiv&amp; \oc{(\wn{\oc{A}}\with\wn{\oc{B}})} \\
  \wn{\oc{(\wn{A}\plus\wn{B})}} &amp;\linequiv&amp; \wn{(\oc{\wn{A}}\plus\oc{\wn{B}})}
\end{array}\)


\section{Lattice of exponential modalities}\label{lattice-of-exponential-modalities}

\(\xymatrix{
 &amp; &amp; {\wn} \\
 &amp; &amp; &amp; &amp; {\wn\oc\wn}\ar[ull] \\
\varepsilon\ar[uurr] &amp; &amp; &amp; {\oc\wn} \ar[ur] &amp; &amp; {\wn\oc} \ar[ul] \\
 &amp; &amp; &amp; &amp; {\oc\wn\oc} \ar[ul]\ar[ur] \\
 &amp; &amp; {\oc} \ar[uull]\ar[urr]
}\)

An \emph{exponential modality} is an arbitrary (possibly empty) sequence
of the two exponential connectives \(\oc\) and \(\wn\). It can be
considered itself as a unary connective. This leads to the notation
\(\mu A\) for applying an exponential modality \(\mu\) to a formula
\(A\).

There is a preorder relation on exponential modalities defined by
\(\mu\lesssim\nu\) if and only if for any formula \(A\) we have
\(\mu A\vdash \nu A\). It induces an
\href{List_of_equivalences}{equivalence} relation on exponential
modalities by \(\mu \sim \nu\) if and only if \(\mu\lesssim\nu\) and
\(\nu\lesssim\mu\).

 and \(\wn{\wn{A}}\vdash\wn{A}\). \}\}

\} and \(\wn{\oc{\wn{A}}}\vdash\wn{A}\). \}\}

This allows to prove that any exponential modality is equivalent to one
of the following seven modalities: \(\varepsilon\) (the empty modality),
\(\oc\), \(\wn\), \(\oc\wn\), \(\wn\oc\), \(\oc\wn\oc\) or
\(\wn\oc\wn\). Indeed any sequence of consecutive \(\oc\) or \(\wn\) in
a modality can be simplified into only one occurrence, and then any
alternating sequence of length at least four can be simplified into a
smaller one.

\textbackslash{}vdash\textbackslash{}oc\{A\} by functoriality from
\(\oc{A}\vdash A\) (and similarly for \(\wn{A}\vdash\wn{\wn{A}}\)). From
\(\oc{A}\vdash \oc{\wn{\oc{A}}}\), we deduce
\(\wn{\oc{A}}\vdash \wn{\oc{\wn{\oc{A}}}}\) by functoriality and
\(\oc{\wn{B}}\vdash \oc{\wn{\oc{\wn{B}}}}\) (with \(A=\wn{B}\)). In a
similar way we have \(\oc{\wn{\oc{\wn{A}}}}\vdash \oc{\wn{A}}\) and
\(\wn{\oc{\wn{\oc{A}}}}\vdash \wn{\oc{A}}\). \}\}

The order relation induced on equivalence classes of exponential
modalities with respect to \(\sim\) can be proved to be the one
represented on the picture in the top of this page. All the represented
relations are valid.

\}. By functoriality we deduce \(\oc{\wn{\oc{A}}}\vdash \oc{\wn{A}}\)
and by \(A=\wn{\oc{B}}\) we deduce
\(\oc{\wn{\oc{B}}}\vdash \wn{\oc{B}}\).

The others are obtained from these ones by duality: \(A\vdash B\)
entails \(B\orth\vdash A\orth\). \}\}

\}. \}\}

We can prove that no other relation between classes is true (by relying
on the previous lemma).

\}.

Then \(\wn\) cannot be smaller than any other of the seven modalities
(since they are all smaller than \(\varepsilon\) or \(\wn\oc\wn\)). For
the same reason, \(\varepsilon\) cannot be smaller than \(\oc\),
\(\oc\wn\), \(\wn\oc\) or \(\oc\wn\oc\). This entails that \(\wn\oc\wn\)
is only smaller than \(\wn\) since it is not smaller than
\(\varepsilon\) (by duality from \(\varepsilon\) not smaller than
\(\oc\wn\oc\)).

From these, \(\wn{\oc{\alpha}}\not\vdash\oc{\wn{\alpha}}\) and
\(\oc{\wn{\alpha}}\not\vdash\wn{\oc{\alpha}}\), we deduce that no other
relation is possible. \}\}

The order relation on equivalence classes of exponential modalities is a
lattice.


%%% Local Variables:
%%% mode: latex
%%% TeX-master: "main"
%%% End:

\section{Provable formulas}\label{provable-formulas}

Important provable formulas are given by
\href{List_of_isomorphisms}{isomorphisms} and by
\href{List_of_equivalences}{equivalences}.

In many of the cases below the \href{Non_provable_formulas}{converse
implication does not hold}.

\subsection{Distributivities}\label{distributivities}

\subsubsection{Standard
distributivities}\label{standard-distributivities}

\(A\plus (B\with C) \limp (A\plus B)\with (A\plus C)\)

\(A\tens (B\with C) \limp (A\tens B)\with (A\tens C)\)

\(\exists \xi . (A \with B) \limp (\exists \xi . A) \with (\exists \xi . B)\)

\subsubsection{Linear distributivities}\label{linear-distributivities}

\(A\tens (B\parr C) \limp (A\tens B)\parr C\)

\(\exists \xi. (A \parr B) \limp A \parr \exists \xi.B  \quad  (\xi\notin A)\)

\(A \tens \forall \xi.B \limp \forall \xi. (A \tens B) \quad  (\xi\notin A)\)

\subsection{Factorizations}\label{factorizations}

\((A\with B)\plus (A\with C) \limp A\with (B\plus C)\)

\((A\parr B)\plus (A\parr C) \limp A\parr (B\plus C)\)

\((\forall \xi . A) \plus (\forall \xi . B) \limp \forall \xi . (A \plus B)\)

\subsection{Identities}\label{identities}

\(\one \limp A\orth\parr A\)

\(A\tens A\orth \limp\bot\)

\subsection{Additive structure}\label{additive-structure}

\(\begin{array}{rclcrclcrcl}
  A\with B &amp;\limp&amp; A &amp;\quad&amp; A\with B &amp;\limp&amp; B &amp;\quad&amp; A &amp;\limp&amp; \top\\
  A &amp;\limp&amp; A\plus B &amp;\quad&amp; B &amp;\limp&amp; A\plus B &amp;\quad&amp; \zero &amp;\limp&amp; A
\end{array}\)

\subsection{Quantifiers}\label{quantifiers-2}

\(\begin{array}{rcll}
  A &amp;\limp&amp; \forall \xi.A  &amp;\quad  (\xi\notin A) \\
  \exists \xi.A &amp;\limp&amp; A  &amp;\quad  (\xi\notin A)
\end{array}\)

\(\begin{array}{rcl}
  \forall \xi_1.\forall \xi_2. A &amp;\limp&amp; \forall \xi. A[^\xi/_{\xi_1},^\xi/_{\xi_2}] \\
  \exists \xi.A[^\xi/_{\xi_1},^\xi/_{\xi_2}] &amp;\limp&amp; \exists \xi_1. \exists \xi_2.A
\end{array}\)

\subsection{Exponential structure}\label{exponential-structure}

Provable formulas involving exponential connectives only provide us with
the \href{lattice_of_exponential_modalities}{lattice of exponential
modalities}.

\(\begin{array}{rclcrcl}
  \oc A &amp;\limp&amp; A &amp;\quad&amp; A&amp;\limp&amp;\wn A\\
  \oc A &amp;\limp&amp; 1 &amp;\quad&amp; \bot &amp;\limp&amp; \wn A
\end{array}\)

\subsection{Monoidality of
exponentials}\label{monoidality-of-exponentials}

\(\begin{array}{rcl}
  \wn(A\parr B) &amp;\limp&amp; \wn A\parr\wn B \\
  \oc A\tens\oc B &amp;\limp&amp; \oc(A\tens B) \\
\\
 \oc{(A \with B)} &amp;\limp&amp; \oc{A} \with \oc{B} \\
 \wn{A} \plus \wn{B} &amp;\limp&amp; \wn{(A \plus B)} \\
\\
 \wn{(A \with B)} &amp;\limp&amp; \wn{A} \with \wn{B} \\
 \oc{A} \plus \oc{B} &amp;\limp&amp; \oc{(A \plus B)}
\end{array}\)

\subsection{Promotion principles}\label{promotion-principles}

\(\begin{array}{rcl}
 \oc{A} \tens \wn{B} &amp;\limp&amp; \wn{(A \tens B)} \\
 \oc{(A \parr B)} &amp;\limp&amp; \wn{A} \parr \oc{B}
\end{array}\)

\subsection{Commutations}\label{commutations}

\(\exists \xi . \wn A \limp \wn{\exists \xi . A}\)

\(\oc{\forall \xi . A} \limp \forall \xi . \oc A\)

\(\wn{\forall \xi . A} \limp \forall \xi . \wn A\)

\(\exists \xi . \oc A \limp \oc{\exists \xi . A}\)


\section{Non provable formulas}\label{non-provable-formulas}

\begin{align*}
A \with (B\plus C) &\not\limp (A\with B)\plus (A\with C) \\
(A\plus B)\with (A\plus C) &\not\limp A\plus (B\with C) \\
(A\tens B)\parr C &\not\limp A\tens (B\parr C) \\
A &\not\limp \oc{A} \\
\oc{\wn{\oc{A}}} &\not\limp A \\
\oc{\wn{\oc{A}}} &\not\limp \oc{A} \\
\oc{\wn{A}} &\not\limp \wn{\oc{A}} \\
\wn{\oc{A}} &\not\limp \oc{\wn{A}}
\end{align*}

%%% Local Variables:
%%% mode: latex
%%% TeX-master: "main"
%%% End:

\section{Mix}\label{mix}

The usual notion of \(\rulename{Mix}\) is the binary version of the rule
but a nullary version also exists.

\subsection{\texorpdfstring{Binary \(\rulename{Mix}\)
rule}{Binary \textbackslash{}rulename\{Mix\} rule}}\label{binary-rulenamemix-rule}

\(\AxRule{\vdash\Gamma}
\AxRule{\vdash\Delta}
\LabelRule{Mix_2}
\BinRule{\vdash\Gamma,\Delta}
\DisplayProof\)

The \(\rulename{Mix_2}\) rule is equivalent to \(\bot\vdash\one\):

\(\LabelRule{\one}
\NulRule{\vdash\one}
\LabelRule{\one}
\NulRule{\vdash\one}
\LabelRule{Mix_2}
\BinRule{\vdash\one,\one}
\DisplayProof
\qquad
\AxRule{\vdash\Gamma}
\LabelRule{\bot}
\UnaRule{\vdash\Gamma,\bot}
\AxRule{\vdash\one,\one}
\LabelRule{\rulename{cut}}
\BinRule{\vdash\Gamma,\one}
\AxRule{\vdash\Delta}
\LabelRule{\bot}
\UnaRule{\vdash\Delta,\bot}
\LabelRule{\rulename{cut}}
\BinRule{\vdash\Gamma,\Delta}
\DisplayProof\)

They are also equivalent to the principle \(A\tens B \vdash A\parr B\):

\(\LabelRule{\one}
\NulRule{\vdash\one}
\LabelRule{\one}
\NulRule{\vdash\one}
\LabelRule{\tens}
\BinRule{\vdash\one\tens\one}
\AxRule{\vdash\bot\parr\bot,\one\parr\one}
\LabelRule{\rulename{cut}}
\BinRule{\vdash\one\parr\one}
\LabelRule{\rulename{ax}}
\NulRule{\vdash\bot,\one}
\LabelRule{\rulename{ax}}
\NulRule{\vdash\bot,\one}
\LabelRule{\tens}
\BinRule{\vdash\bot\tens\bot,\one,\one}
\LabelRule{\rulename{cut}}
\BinRule{\vdash\one,\one}
\DisplayProof
\qquad
\LabelRule{\rulename{ax}}
\NulRule{\vdash A\orth,A}
\LabelRule{\rulename{ax}}
\NulRule{\vdash B\orth,B}
\LabelRule{Mix_2}
\BinRule{\vdash A\orth,A,B\orth,B}
\LabelRule{\parr}
\UnaRule{\vdash A\orth,B\orth,A\parr B}
\LabelRule{\parr}
\UnaRule{\vdash A\orth\parr B\orth,A\parr B}
\DisplayProof\)

\subsection{\texorpdfstring{Nullary \(\rulename{Mix}\)
rule}{Nullary \textbackslash{}rulename\{Mix\} rule}}\label{nullary-rulenamemix-rule}

\(\LabelRule{Mix_0}
\NulRule{\vdash}
\DisplayProof\)

The \(\rulename{Mix_0}\) rule is equivalent to \(\one\vdash\bot\):

\(\LabelRule{Mix_0}
\NulRule{\vdash}
\LabelRule{\bot}
\UnaRule{\vdash\bot}
\LabelRule{\bot}
\UnaRule{\vdash\bot,\bot}
\DisplayProof
\qquad
\LabelRule{\one}
\NulRule{\vdash\one}
\AxRule{\vdash\bot,\bot}
\LabelRule{\rulename{cut}}
\BinRule{\vdash\bot}
\LabelRule{\one}
\NulRule{\vdash\one}
\LabelRule{\rulename{cut}}
\BinRule{\vdash}
\DisplayProof\)

The nullary \(\rulename{Mix}\) acts as a unit for the binary one:

\(\AxRule{\vdash\Gamma}
\LabelRule{Mix_0}
\NulRule{\vdash}
\LabelRule{Mix_2}
\BinRule{\vdash\Gamma}
\DisplayProof\)

If \(\pi\) is a proof which uses no \(\bot\) rule and no weakening rule,
then (up to the simplification of the pattern
\(\rulename{Mix_0}/\rulename{Mix_2}\) above into nothing) \(\pi\) is
either reduced to a \(\rulename{Mix_0}\) rule or does not contain any
\(\rulename{Mix_0}\) rule.


%%% Local Variables:
%%% mode: latex
%%% TeX-master: "main"
%%% End:

\section{Additive cut rule}\label{additive-cut-rule}

The additive cut rule is: \(\AxRule{\Gamma\vdash A,\Delta}
\AxRule{\Gamma,A\vdash\Delta}
\LabelRule{\rulename{cut\;add}}
\BinRule{\Gamma\vdash\Delta}
\DisplayProof\)

In contrary to what happens in classical logic, this rule is
\textbf{not} admissible in linear logic.

The formula \(\alpha\plus\alpha\orth\) is not provable in linear logic,
while it is derivable with the additive cut rule:
\begin{prooftree}
\NulRule{\alpha\vdash\alpha}
\UnaRule{\vdash\alpha,\alpha\orth}
\LabelRule{\plus_{R2}}
\UnaRule{\vdash\alpha,\alpha\plus\alpha\orth}
\NulRule{\alpha\vdash\alpha}
\LabelRule{\plus_{R1}}
\UnaRule{\alpha\vdash\alpha\plus\alpha\orth}
\LabelRule{\rulename{cut\;add}}
\BinRule{\vdash\alpha\plus\alpha\orth}
\end{prooftree}

%%% Local Variables:
%%% mode: latex
%%% TeX-master: "main"
%%% End:

\section{Focusing}\label{reversibility-and-focusing}

\subsection{Asynchrony}\label{asynchrony}

The connectives $\parr$, $\with$, $\bot$, $\top$ and $\forall$ are called \emph{asynchronous}.

\begin{theorem}
Asynchronous connectives are \hyperref[reversibledef]{reversible}:
\begin{itemize}
\item A sequent $\vdash\Gamma,A\parr B$ is provable if and only if $\vdash\Gamma,A,B$ is provable.
\item A sequent $\vdash\Gamma,A\with B$ is provable if and only if $\vdash\Gamma,A$ and $\vdash\Gamma,B$ are provable.
\item A sequent $\vdash\Gamma,\bot$ is provable if and only if $\vdash\Gamma$ is provable.
\item A sequent $\vdash\Gamma,\forall\xi A$ is provable if and only if $\vdash\Gamma,A$ is provable, for some fresh variable $\xi$.
\end{itemize}
\end{theorem}

\begin{proof}
See proof of \cref{reversibilityproperty}.
% We start with the case of the $\parr$ connective.
% If $\vdash\Gamma,A,B$ is provable, then by the introduction rule for $\parr$
% we know that $\vdash\Gamma,A\parr B$ is provable.
% For the reverse implication we proceed by induction on a proof $\pi$ of
% $\vdash\Gamma,A\parr B$.
% \begin{itemize}
% \item If the last rule of $\pi$ is the introduction of the $\parr$ in $A\parr B$, then the premiss is exacty $\vdash\Gamma,A,B$ so we can conclude.
% \item The other case where the last rule introduces $A\parr B$ is when $\pi$ is an axiom rule, hence $\Gamma=A\orth\tens B\orth$. Then we can conclude with the proof
% \begin{prooftree}
% \LabelRule{\rulename{ax}}
% \NulRule{\vdash A, A\orth}
% \LabelRule{\rulename{ax}}
% \NulRule{\vdash B, B\orth}
% \LabelRule{\tens}
% \BinRule{\vdash A, B, A\orth\tens B\orth}
% \end{prooftree}
% \item Otherwise $A\parr B$ is in the context of the last rule. If the last rule is a tensor, then $\pi$ has the shape
% \begin{prooftree}
% \AxRule{\pi_1 : {} \vdash \Gamma_1, A\parr B, C}
% \AxRule{\pi_2 : {} \vdash \Gamma_2, D}
% \LabelRule{\tens}
% \BinRule{\vdash\Gamma_1, \Gamma_2, A\parr B, C\tens D}
% \end{prooftree}
% or the same with $A\parr B$ in the conclusion of $\pi_2$ instead. By induction hypothesis on $\pi_1$ we get a proof $\pi'_1$ of $\vdash\Gamma_1,A,B,C$, then we can conclude with the proof
% \begin{prooftree}
% \AxRule{\pi'_1 : {} \vdash \Gamma_1, A, B, C}
% \AxRule{\pi_2 : {} \vdash \Gamma_2, D}
% \LabelRule{\tens}
% \BinRule{\vdash\Gamma_1, \Gamma_2, A, B, C\tens D}
% \end{prooftree}
% \item The case of the cut rule has the same structure as the tensor rule.
% \item In the case of the $\with$ rule, we have $A\parr B$ in both premisses and we conclude similarly, using the induction hypothesis on both $\pi_1$ and $\pi_2$.
% \item If $A\parr B$ is in the context of a rules for $\parr$, $\plus$, $\bot$ or quantifiers, or in the context of a dereliction, weakening or contraction, the situation is similar as for $\tens$ except that we have only one premiss.
% \item If $A\parr B$ is in the context of $\top$ rules, we can freely change the context of the rule to get the expected one.
% \item The two remaining cases are if the last rule is the rule for $1$ or a promotion. By the constraints these rules impose on the contexts, these cases cannot happen.
% \end{itemize}
% The $\with$ connective is treated in the same way.
% In cases where $A\with B$ is in the context of a rule with two
% premisses, the premiss where this formula is not present will be duplicated,
% with one copy in the premiss for $A$ and one in the premiss for $B$.

% The $\forall$ connective is also treated similarly.
% Its peculiarity is that introducing $\forall\xi$ requires that $\xi$ does
% not appear free in the context.
% For all rules with one premiss except the quantifier rules, the set of fresh
% variables is the same in the premiss and the conclusion, so everything works
% well.
% Other rules might change the set of free variables, but problems are avoided
% by choosing for $\xi$ a variable that is fresh for the whole proof we are considering.
\end{proof}

Remark that this result is proved using only commutation of rules, except
when the formula is introduced by an axiom rule. Furthermore, if axioms
are applied only on atoms, this particular case disappears.

A consequence of this fact is that, when searching for a proof of some
sequent \(\vdash\Gamma\), one can always start by decomposing asynchronous
connectives in \(\Gamma\) without losing provability. Applying this
result to successive connectives, we can get generalized formulations
for more complex formulas. For instance:
\begin{align*}
&\vdash\Gamma,(A\parr B)\parr(B\with C) \text{ is provable} \\
\text{iff } &\vdash\Gamma,A\parr B,B\with C \text{ is provable} \\
\text{iff } &\vdash\Gamma,A\parr B,B\text{ and }\vdash\Gamma,A\parr B,C \text{ are
  provable} \\
\text{iff } &\vdash\Gamma,A,B,B\text{ and }\vdash\Gamma,A,B,C \text{ are provable}
\end{align*}

So without loss of generality, we can assume that any proof of
\(\vdash\Gamma,(A\parr B)\parr(B\with C)\) ends like
\begin{prooftree}
  \AxRule{ \vdash \Gamma, A, B, B }
  \UnaRule{ \vdash \Gamma, A\parr B, B }
  \AxRule{ \vdash \Gamma, A, B, C }
  \UnaRule{ \vdash \Gamma, A\parr B, C }
  \BinRule{ \vdash \Gamma, A\parr B, B\with C }
  \UnaRule{ \vdash \Gamma, (A\parr B)\parr(B\with C) }
\end{prooftree}

In order to define a general statement for compound formulas, as well as
an analogous result for synchronous connectives, we need to give a proper
status to clusters of connectives of the same polarity.

\subsection{Generalized connectives and rules}\label{generalized-connectives-and-rules}

\begin{definition}
A \emph{synchronous generalized connective} is a parametrized formula
$P[X_1,\ldots,X_n]$ made from the variables $X_i$ using the connectives
$\tens$, $\plus$, $\one$, $\zero$.

An \emph{asynchronous generalized connective} is a parametrized formula
$N[X_1,\ldots,X_n]$ made from the variables $X_i$ using the connectives
$\parr$, $\with$, $\bot$, $\top$.

If $C[X_1,\ldots,X_n]$ is a generalized connectives (of any polarity), the
\emph{dual} of $C$ is the connective $C^*$ such that
$C^*[X_1\orth,\ldots,X_n\orth]=C[X_1,\ldots,X_n]\orth$.
\end{definition}

It is clear that dualization of generalized connectives is involutive
and exchanges polarities. We do not include quantifiers in this
definition, mainly for simplicity. Extending the notion to quantifiers
would only require taking proper care of the scope of variables.

Sequent calculus provides introduction rules for each connective.
Asynchronous connectives have one rule, synchronous ones may have any number of
rules, namely 2 for \(\plus\) and 0 for \(\zero\). We can derive
introduction rules for the generalized connectives by combining the
different possible introduction rules for each of their components.

Considering the previous example
\(N[X_1,X_2,X_3]=(X_1\parr X_2)\parr(X_2\with X_3)\), we can derive an
introduction rule for \(N\) as
\begin{equation*}
\AxRule{ \vdash \Gamma, X_1, X_2, X_2 }
  \UnaRule{ \vdash \Gamma, X_1\parr X_2, X_2 }
  \AxRule{ \vdash \Gamma, X_1, X_2, X_3 }
  \UnaRule{ \vdash \Gamma, X_1\parr X_2, X_3 }
  \BinRule{ \vdash \Gamma, X_1\parr X_2, X_2\with X_3 }
  \UnaRule{ \vdash \Gamma, (X_1\parr X_2)\parr(X_2\with X_3) }
  \DisplayProof
\quad\quad\text{or}\quad\quad
  \AxRule{ \vdash \Gamma, X_1, X_2, X_2 }
  \AxRule{ \vdash \Gamma, X_1, X_2, X_3 }
  \BinRule{ \vdash \Gamma, X_1, X_2, X_2\with X_3 }
  \UnaRule{ \vdash \Gamma, X_1\parr X_2, X_2\with X_3 }
  \UnaRule{ \vdash \Gamma, (X_1\parr X_2)\parr(X_2\with X_3) }
  \DisplayProof
\end{equation*}
but these rules only differ by the commutation of independent rules. In
particular, their premisses are the same. The dual of \(N\) is
\(P[X_1,X_2,X_3]=(X_1\tens X_2)\tens(X_2\plus X_3)\), for which we have
two possible derivations:
\begin{equation*}
  \AxRule{ \vdash \Gamma_1, X_1 }
  \AxRule{ \vdash \Gamma_2, X_2 }
  \BinRule{ \vdash \Gamma_1, \Gamma_2, X_1\tens X_2 }
  \AxRule{ \vdash \Gamma_3, X_2 }
  \UnaRule{ \vdash \Gamma_3, X_2\plus X_3 }
  \BinRule{ \vdash \Gamma_1, \Gamma_2, \Gamma_3, (X_1\tens X_2)\tens(X_2\plus X_3) }
  \DisplayProof
\qquad
  \AxRule{ \vdash \Gamma_1, X_1 }
  \AxRule{ \vdash \Gamma_2, X_2 }
  \BinRule{ \vdash \Gamma_1, \Gamma_2, X_1\tens X_2 }
  \AxRule{ \vdash \Gamma_3, X_3 }
  \UnaRule{ \vdash \Gamma_3, X_2\plus X_3 }
  \BinRule{ \vdash \Gamma_1, \Gamma_2, \Gamma_3, (X_1\tens X_2)\tens(X_2\plus X_3) }
  \DisplayProof
\end{equation*}

These are actually different, in particular their premisses differ. Each
possible derivation corresponds to the choice of one side of the
\(\plus\) connective.

We can remark that the branches of the asynchronous inference precisely
correspond to the possible synchronous inferences:
\begin{itemize}
\item
  the first branch of the asynchronous inference has a premiss
  \(X_1,X_2,X_2\) and the first synchronous inference has three premisses,
  holding \(X_1\), \(X_2\) and \(X_2\) respectively.
\item
  the second branch of the asynchronous inference has a premiss
  \(X_1,X_2,X_3\) and the second synchronous inference has three premisses,
  holding \(X_1\), \(X_2\) and \(X_3\) respectively.
\end{itemize}

This phenomenon extends to all generalized connectives.

\begin{definition}
The \emph{branching} of a generalized connective $P[X_1,\ldots,X_n]$ is the
multiset $\mathcal{I}_P$ of multisets over $\{1,\ldots,n\}$ defined
inductively as

$ \mathcal{I}_{P\tens Q} := [ I+J \mid I\in\mathcal{I}_P, J\in\mathcal{I}_Q ] $,
$ \mathcal{I}_{P\plus Q} := \mathcal{I}_P + \mathcal{I}_Q $,
$ \mathcal{I}_\one := [[]] $,
$ \mathcal{I}_\zero := [] $,
$ \mathcal{I}_{X_i} := [[i]] $.

The branching of an asynchronous generalized connective is the branching of its
dual. Elements of a branching are called branches.
\end{definition}

In the example above, the branching will be \([[1,2,2],[1,2,3]]\), which
corresponds to the branches of the asynchronous inference and to the cases
of the synchronous inference.

\begin{definition}
Let $\mathcal{I}$ be a branching.
Write $\mathcal{I}$ as $[I_1,\ldots,I_k]$ and write each $I_j$ as
$[i_{j,1},\ldots,i_{j,\ell_j}]$.
The derived rule for an asynchronous generalized connective $N$ with
branching $\mathcal{I}$ is
\begin{prooftree}
    \AxRule{ \vdash \Gamma, A_{i_{1,1}}, \ldots, A_{i_{1,\ell_1}} }
    \AxRule{ \cdots }
    \AxRule{ \vdash \Gamma, A_{i_{k,1}}, \ldots, A_{i_{k,\ell_k}} }
    \LabelRule{N}
    \TriRule{ \vdash \Gamma, N[A_1,\ldots,A_n] }
\end{prooftree}
  
For each branch $I=[i_1,\ldots,i_\ell]$ of a synchronous generalized connective
$P$, the derived rule for branch $I$ of $P$ is
\begin{prooftree}
    \AxRule{ \vdash \Gamma_1, A_{i_1} }
    \AxRule{ \cdots }
    \AxRule{ \vdash \Gamma_\ell, A_{i_\ell} }
    \LabelRule{P_I}
    \TriRule{ \vdash \Gamma_1, \ldots, \Gamma_\ell, P[A_1,\ldots,A_n] }
\end{prooftree}
\end{definition}

The reversibility property of asynchronous connectives can be rephrased in a
generalized way as

\begin{theorem}
Let $N$ be an asynchronous generalized connective. A sequent
$\vdash\Gamma,N[A_1,\ldots,A_n]$ is provable if and only if, for each
$[i_1,\ldots,i_k]\in\mathcal{I}_N$, the sequent
$\vdash\Gamma,A_{i_1},\ldots,A_{i_k}$ is provable.
\end{theorem}

The corresponding property for synchronous connectives is the focusing
property, defined in the next section.

\subsection{Synchrony}\label{synchrony}

\begin{definition}
A formula is \emph{synchronous} if it has a main connective among
$\tens$, $\plus$, $\one$, $\zero$.
It is called \emph{asynchronous} if it has a main connective among
$\parr$, $\with$, $\bot$, $\top$.
It is called \emph{neutral} if it is neither synchronous nor asynchronous.
\end{definition}

If we extended the theory to include quantifiers in generalized
connectives, then the definition of synchronous and asynchronous formulas would
be extended to include them too.

\begin{definition}
A proof $\pi : {} \vdash\Gamma,A$ is said to be \emph{synchronously focused on $A$} if it has the shape
\begin{prooftree}
    \AxRule{ \pi_1 : {} \vdash \Gamma_1, A_{i_1} }
    \AxRule{ \cdots }
    \AxRule{ \pi_\ell : {} \vdash \Gamma_\ell, A_{i_\ell} }
    \LabelRule{P_{[i_1,\ldots,i_\ell]}}
    \TriRule{ \vdash  \Gamma_1, \ldots, \Gamma_\ell, P[A_1,\ldots,A_n] }
\end{prooftree}  
where $P$ is a synchronous generalized connective, the $A_i$ are non-synchronous
and $A=P[A_1,\ldots,A_n]$. The formula $A$ is called the \emph{focus} of the proof $\pi$.
\end{definition}

In other words, a proof is synchronously focused on a conclusion \(A\) if
its last rules build \(A\) from some of its non-synchronous subformulas in
one cluster of inferences. Note that this notion only makes sense for a
sequent made only of synchronous formulas, since by this definition a proof
is obviously synchronously focused on any of its non-synchronous conclusions,
using the degenerate generalized connective \(P[X]=X\).

\begin{theorem}
A sequent $\vdash\Gamma$ is cut-free provable if and only if it is provable by a cut-free proof that is synchronously focused.
\end{theorem}

\begin{proof}
We reason by induction on a proof $\pi$ of $\Gamma$.
As noted above, the result  trivially holds if $\Gamma$ has a non-synchronous formula.
We can thus assume that $\Gamma$ contains only synchronous formulas and reason
on the nature of the last rule, which is necessarily the introduction of a
synchronous connective (it cannot be an axiom rule because an axiom always has
at least one non-synchronous conclusion).

Suppose that the last rule of $\pi$ introduces a tensor, so that $\pi$ is
\begin{prooftree}
    \AxRule{ \rho : {} \vdash \Gamma, A }
    \AxRule{ \theta : {} \vdash \Delta, B }
    \BinRule{ \vdash \Gamma, \Delta, A\tens B }
\end{prooftree}
  
By induction hypothesis, there are synchronously focused proofs $\rho' : {} \vdash\Gamma,A$ and $\theta' : {} \vdash\Delta,B$.
If $A$ is the focus of $\rho'$ and $B$ is the focus of $\theta'$, then the proof
\begin{prooftree}
    \AxRule{ \rho' : {} \vdash \Gamma, A }
    \AxRule{ \theta' : {} \vdash \Delta, B }
    \BinRule{ \vdash \Gamma, \Delta, A\tens B }
\end{prooftree}  
is synchronously focused on $A\tens B$, so we can conclude.
Otherwise, one of the two proofs is synchronously focused on another conclusion.
Without loss of generality, suppose that $\rho'$ is not synchronously focused on $A$.
Then it decomposes as
\begin{prooftree}
    \AxRule{ \rho_1 : {} \vdash \Gamma_1, C_{i_1} }
    \AxRule{ \cdots }
    \AxRule{ \rho_\ell : {} \vdash \Gamma_\ell, C_{i_\ell} }
    \TriRule{ \vdash  \Gamma_1, \ldots, \Gamma_\ell, P[C_1,\ldots,C_n] }
\end{prooftree}  
where the $C_i$ are not synchronous and $A$ belongs to some context $\Gamma_j$
that we will write $\Gamma'_j,A$.
Then we can conclude with the proof
\begin{prooftree}
    \AxRule{ \rho_1 : {} \vdash \Gamma_1, C_{i_1} \quad\cdots }
    \AxRule{ \rho_j : {} \vdash \Gamma_j, A, C_{i_j} }
    \AxRule{ \theta : {} \vdash \Delta, B }
    \BinRule{ \vdash \Gamma_j, \Delta, A\tens B, C_{i_j} }
    \AxRule{ \cdots\quad \rho_\ell : {} \vdash \Gamma_\ell, C_{i_\ell} }
    \TriRule{ \vdash \Gamma_1, \ldots, \Gamma_\ell, \Delta, A\tens B, P[C_1,\ldots,C_n] }
\end{prooftree}
which is synchronously focused on $P[C_1,\ldots,C_n]$.

If the last rule of $\pi$ introduces a $\plus$, we proceed the same way
except that there is only one premiss.
If the last rule of $\pi$ introduces a $\one$, then it is the only rule of
$\pi$, which is thus synchronously focused on this $\one$.
\end{proof}

As in the reversibility theorem (\cref{reversibilityproperty}), this proof only makes use of commutation of independent rules.

These results say nothing about exponential modalities, because they
respect neither reversibility nor focusing. However, if we consider
the fragment of LL which consists only of multiplicative and additive
connectives, we can restrict the proof rules to enforce focusing
without loss of expressiveness.


%%% Local Variables:
%%% mode: latex
%%% TeX-master: "main"
%%% End:

\section{Positive formula}\label{positive-formula}

A \emph{positive formula} is a formula \(P\) such that \(P\limp\oc P\)
(thus a \href{https://en.wikipedia.org/wiki/F-coalgebra}{coalgebra} for the
\href{https://en.wikipedia.org/wiki/Comonad}{comonad} \(\oc\)). As a consequence \(P\) and
\(\oc P\) are \hyperref[equivalences]{equivalent}.

A formula \(P\) is positive if and only if \(P\orth\) is
\hyperref[negative-formula]{negative}.

\subsection{Positive connectives}\label{positive-connectives}

A connective \(c\) of arity \(n\) is \emph{positive} if for any positive
formulas \(P_1\),...,\(P_n\), \(c(P_1,\dots,P_n)\) is positive.

\begin{proposition}[Positive connectives]
$\tens$, $\one$, $\plus$, $\zero$, $\oc$ and $\exists$ are positive connectives.
\end{proposition}

\begin{proof}\
\begin{prooftree}
\AxRule{P_2\vdash\oc{P_2}}
\AxRule{P_1\vdash\oc{P_1}}
\LabelRule{\rulename{ax}}
\NulRule{P_1\vdash P_1}
\LabelRule{\rulename{ax}}
\NulRule{P_2\vdash P_2}
\LabelRule{\tens R}
\BinRule{P_1,P_2\vdash P_1\tens P_2}
\LabelRule{\oc d L}
\UnaRule{\oc{P_1},P_2\vdash P_1\tens P_2}
\LabelRule{\oc d L}
\UnaRule{\oc{P_1},\oc{P_2}\vdash P_1\tens P_2}
\LabelRule{\oc R}
\UnaRule{\oc{P_1},\oc{P_2}\vdash\oc{(P_1\tens P_2)}}
\LabelRule{\rulename{cut}}
\BinRule{P_1,\oc{P_2}\vdash\oc{(P_1\tens P_2)}}
\LabelRule{\rulename{cut}}
\BinRule{P_1,P_2\vdash\oc{(P_1\tens P_2)}}
\LabelRule{\tens L}
\UnaRule{P_1\tens P_2\vdash\oc{(P_1\tens P_2)}}
\end{prooftree}

\begin{prooftree}
\LabelRule{\one R}
\NulRule{\vdash\one}
\LabelRule{\oc R}
\UnaRule{\vdash\oc{\one}}
\LabelRule{\one L}
\UnaRule{\one\vdash\oc{\one}}
\end{prooftree}

\begin{prooftree}
\AxRule{P_1\vdash\oc{P_1}}
\LabelRule{\rulename{ax}}
\NulRule{P_1\vdash P_1}
\LabelRule{\plus_1 R}
\UnaRule{P_1\vdash P_1\plus P_2}
\LabelRule{\oc d L}
\UnaRule{\oc{P_1}\vdash P_1\plus P_2}
\LabelRule{\oc R}
\UnaRule{\oc{P_1}\vdash\oc{(P_1\plus P_2)}}
\LabelRule{\rulename{cut}}
\BinRule{P_1\vdash\oc{(P_1\plus P_2)}}
\AxRule{P_2\vdash\oc{P_2}}
\LabelRule{\rulename{ax}}
\NulRule{P_2\vdash P_2}
\LabelRule{\plus_2 R}
\UnaRule{P_2\vdash P_1\plus P_2}
\LabelRule{\oc d L}
\UnaRule{\oc{P_2}\vdash P_1\plus P_2}
\LabelRule{\oc R}
\UnaRule{\oc{P_2}\vdash\oc{(P_1\plus P_2)}}
\LabelRule{\rulename{cut}}
\BinRule{P_2\vdash\oc{(P_1\plus P_2)}}
\LabelRule{\plus L}
\BinRule{P_1\plus P_2\vdash\oc{(P_1\plus P_2)}}
\end{prooftree}

\begin{prooftree}
\LabelRule{\zero L}
\NulRule{\zero\vdash\oc{\zero}}
\end{prooftree}

\begin{prooftree}
\LabelRule{\rulename{ax}}
\NulRule{\oc{P}\vdash\oc{P}}
\LabelRule{\oc R}
\UnaRule{\oc{P}\vdash\oc{\oc{P}}}
\end{prooftree}

\begin{prooftree}
\AxRule{P\vdash\oc{P}}
\LabelRule{\rulename{ax}}
\NulRule{P\vdash P}
\LabelRule{\exists R}
\UnaRule{P\vdash \exists\xi P}
\LabelRule{\oc d L}
\UnaRule{\oc{P}\vdash \exists\xi P}
\LabelRule{\oc R}
\UnaRule{\oc{P}\vdash\oc{\exists\xi P}}
\LabelRule{\rulename{cut}}
\BinRule{P\vdash\oc{\exists\xi P}}
\LabelRule{\exists L}
\UnaRule{\exists\xi P\vdash\oc{\exists\xi P}}
\end{prooftree}
\end{proof}

More generally, \(\oc A\) is positive for any formula \(A\).

The notion of positive connective is related with but different from the
notion of \wantedpage{synchronous connective}.

\subsection{Generalized structural rules}\label{generalized-structural-rules-pos}

Positive formulas admit generalized left structural rules corresponding
to a structure of \href{https://en.wikipedia.org/wiki/Comonoid}{\(\tens\)-comonoid}:
\(P\limp P\tens P\) and \(P\limp\one\). The following rules are derivable:
\begin{equation*}
\AxRule{\Gamma,P,P\vdash\Delta}
\LabelRule{+ c L}
\UnaRule{\Gamma,P\vdash\Delta}
\DisplayProof
\qquad\qquad
\AxRule{\Gamma\vdash\Delta}
\LabelRule{+ w L}
\UnaRule{\Gamma,P\vdash\Delta}
\DisplayProof
\end{equation*}

\begin{proof}
\begin{equation*}
\AxRule{P\vdash\oc{P}}
\AxRule{\Gamma,P,P\vdash\Delta}
\LabelRule{\oc L}
\UnaRule{\Gamma,P,\oc P\vdash\Delta}
\LabelRule{\oc L}
\UnaRule{\Gamma,\oc P,\oc P\vdash\Delta}
\LabelRule{\oc c L}
\UnaRule{\Gamma,\oc P\vdash\Delta}
\LabelRule{\rulename{cut}}
\BinRule{\Gamma,P\vdash\Delta}
\DisplayProof
\qquad\qquad
\AxRule{P\vdash\oc{P}}
\AxRule{\Gamma\vdash\Delta}
\LabelRule{\oc w L}
\UnaRule{\Gamma,\oc P\vdash\Delta}
\LabelRule{\rulename{cut}}
\BinRule{\Gamma,P\vdash\Delta}
\DisplayProof
\end{equation*}
\end{proof}

Positive formulas are also acceptable in the left-hand side context of
the promotion rule. The following rule is derivable:
\begin{prooftree}
\AxRule{\oc\Gamma,P_1,\dots,P_n\vdash A,\wn\Delta}
\LabelRule{+ \oc R}
\UnaRule{\oc\Gamma,P_1,\dots,P_n\vdash \oc{A},\wn\Delta}
\end{prooftree}

\begin{proof}\
\begin{prooftree}
\AxRule{P_1\vdash\oc{P_1}}
\AxRule{P_n\vdash\oc{P_n}}
\AxRule{\oc\Gamma,P_1,\dots,P_n\vdash A,\wn\Delta}
\LabelRule{\oc L}
\UnaRule{\oc\Gamma,P_1,\dots,P_{n-1},\oc{P_n}\vdash A,\wn\Delta}
\VdotsRule{}{\oc\Gamma,P_1,\oc{P_2},\dots,\oc{P_n}\vdash A,\wn\Delta}
\LabelRule{\oc L}
\UnaRule{\oc\Gamma,\oc{P_1},\dots,\oc{P_n}\vdash A,\wn\Delta}
\LabelRule{\oc R}
\UnaRule{\oc\Gamma,\oc{P_1},\dots,\oc{P_n}\vdash \oc{A},\wn\Delta}
\LabelRule{\rulename{cut}}
\BinRule{\oc\Gamma,\oc{P_1},\dots,\oc{P_{n-1}},P_n\vdash \oc{A},\wn\Delta}
\VdotsRule{}{\oc\Gamma,\oc{P_1},P_2,\dots,P_n\vdash \oc{A},\wn\Delta}
\LabelRule{\rulename{cut}}
\BinRule{\oc\Gamma,P_1,\dots,P_n\vdash \oc{A},\wn\Delta}
\end{prooftree}
\end{proof}


%%% Local Variables:
%%% mode: latex
%%% TeX-master: "main"
%%% End:

\section{Negative formula}\label{negative-formula}

A \emph{negative formula} is a formula \(N\) such that \(\wn N\limp N\)
(thus a \href{https://en.wikipedia.org/wiki/F-algebra}{algebra} for the
\href{https://en.wikipedia.org/wiki/Monad_(category_theory)}{monad} \(\wn\)). As a consequence \(N\) and \(\wn N\) are \hyperref[equivalences]{equivalent}.

A formula \(N\) is negative if and only if \(N\orth\) is \hyperref[positive-formula]{positive}.

\subsection{Negative connectives}\label{negative-connectives}

A connective \(c\) of arity \(n\) is \emph{negative} if for any negative
formulas \(N_1\),...,\(N_n\), \(c(N_1,\dots,N_n)\) is negative.

\begin{proposition}[Negative connectives]
$\parr$, $\bot$, $\with$, $\top$, $\wn$ and $\forall$ are negative connectives.
\end{proposition}

\begin{proof}
This is equivalent to the fact that $\tens$, $\one$, $\plus$, $\zero$, $\oc$ and $\exists$ are \hyperref[positive-formula]{positive connectives}.
\end{proof}

More generally, \(\wn A\) is negative for any formula \(A\).

The notion of negative connective is related with but different from the
notion of \wantedpage{asynchronous connective}.

\subsection{Generalized structural rules}\label{generalized-structural-rules}

Negative formulas admit generalized right structural rules corresponding
to a structure of
\href{https://en.wikipedia.org/wiki/Monoid_(category_theory)}{\(\parr\)-monoid}:
\(N\parr N\limp N\) and \(\bot\limp N\).
The following rules are derivable:
\begin{equation*}
\AxRule{\Gamma\vdash N,N,\Delta}
\LabelRule{- c R}
\UnaRule{\Gamma\vdash N,\Delta}
\DisplayProof
\qquad\qquad
\AxRule{\Gamma\vdash\Delta}
\LabelRule{- w R}
\UnaRule{\Gamma\vdash N,\Delta}
\DisplayProof
\end{equation*}

\begin{proof}
This is equivalent to the \hyperref[generalized-structural-rules-pos]{generalized left structural rules} for positive formulas.
\end{proof}

Negative formulas are also acceptable in the context of the promotion
rule. The following rule is derivable:
\begin{prooftree}
\AxRule{\vdash A,N_1,\dots,N_n}
\LabelRule{- \oc R}
\UnaRule{\vdash \oc{A},N_1,\dots,N_n}
\end{prooftree}

\begin{proof}
This is equivalent to the possibility of having positive formulas in the \hyperref[generalized-structural-rules-pos]{left-hand side context of the promotion rule}.
\end{proof}


%%% Local Variables:
%%% mode: latex
%%% TeX-master: "main"
%%% End:

\section{Regular formula}\label{regular-formula}

A \emph{regular formula} is a formula \(R\) such that
\(R\linequiv\wn\oc R\).

A formula \(L\) is \emph{co-regular} if its dual \(L\orth\) is regular,
that is if \(L\linequiv\oc\wn L\).

\subsection{Alternative
characterization}\label{alternative-characterization}

\(R\) is regular if and only if it is
\href{Sequent_calculus\#Equivalences}{equivalent} to a formula of the
shape \(\wn P\) for some \href{positive_formula}{positive formula}
\(P\).

\subsection{Regular connectives}\label{regular-connectives}

A connective \(c\) of arity \(n\) is \emph{regular} if for any regular
formulas \(R_1\),...,\(R_n\), \(c(R_1,\dots,R_n)\) is regular.

More generally, \(\wn\oc A\) is regular for any formula \(A\).


%%% Local Variables:
%%% mode: latex
%%% TeX-master: "main"
%%% End:


\chapter{Intuitionistic linear logic}\label{intuitionistic-linear-logic}

Intuitionistic Linear Logic (\(ILL\)) is the
\wantedpage{intuitionnistic restriction} of
linear logic: the sequent calculus of \(ILL\) is obtained from the
\hyperref[sequents-and-proofs]{two-sided sequent calculus
of linear logic} by constraining sequents to have exactly one formula on
the right-hand side: \(\Gamma\vdash A\).

The connectives \(\parr\), \(\bot\) and \(\wn\) are not available
anymore, but the linear implication \(\limp\) is.

\section{Sequent Calculus}

\(\LabelRule{\rulename{ax}}
\NulRule{A\vdash A}
\DisplayProof
\qquad
\AxRule{\Gamma\vdash A}
\AxRule{\Delta,A\vdash C}
\LabelRule{\rulename{cut}}
\BinRule{\Gamma,\Delta\vdash C}
\DisplayProof\)

\(\AxRule{\Gamma\vdash A}
\AxRule{\Delta\vdash B}
\LabelRule{\tens R}
\BinRule{\Gamma,\Delta\vdash A\tens B}
\DisplayProof
\qquad
\AxRule{\Gamma,A,B\vdash C}
\LabelRule{\tens L}
\UnaRule{\Gamma,A\tens B\vdash C}
\DisplayProof
\qquad
\LabelRule{\one R}
\NulRule{{}\vdash\one}
\DisplayProof
\qquad
\AxRule{\Gamma\vdash C}
\LabelRule{\one L}
\UnaRule{\Gamma,\one\vdash C}
\DisplayProof\)

\(\AxRule{\Gamma,A\vdash B}
\LabelRule{\limp R}
\UnaRule{\Gamma\vdash A\limp B}
\DisplayProof
\qquad
\AxRule{\Gamma\vdash A}
\AxRule{\Delta,B\vdash C}
\LabelRule{\limp L}
\BinRule{\Gamma,\Delta,A\limp B\vdash C}
\DisplayProof\)

\(\AxRule{\Gamma\vdash A}
\AxRule{\Gamma\vdash B}
\LabelRule{\with R}
\BinRule{\Gamma\vdash A\with B}
\DisplayProof
\qquad
\AxRule{\Gamma,A\vdash C}
\LabelRule{\with_1 L}
\UnaRule{\Gamma,A\with B\vdash C}
\DisplayProof
\qquad
\AxRule{\Gamma,B\vdash C}
\LabelRule{\with_2 L}
\UnaRule{\Gamma,A\with B\vdash C}
\DisplayProof
\qquad
\LabelRule{\top R}
\NulRule{\Gamma\vdash\top}
\DisplayProof\)

\(\AxRule{\Gamma\vdash A}
\LabelRule{\plus_1 R}
\UnaRule{\Gamma\vdash A\plus B}
\DisplayProof
\qquad
\AxRule{\Gamma\vdash B}
\LabelRule{\plus_2 R}
\UnaRule{\Gamma\vdash A\plus B}
\DisplayProof
\qquad
\AxRule{\Gamma,A\vdash C}
\AxRule{\Gamma,B\vdash C}
\LabelRule{\plus L}
\BinRule{\Gamma,A\plus B\vdash C}
\DisplayProof
\qquad
\LabelRule{\zero L}
\NulRule{\Gamma,\zero\vdash C}
\DisplayProof\)

\(\AxRule{\oc{\Gamma}\vdash A}
\LabelRule{\oc R}
\UnaRule{\oc{\Gamma}\vdash\oc{A}}
\DisplayProof
\qquad
\AxRule{\Gamma,A\vdash C}
\LabelRule{\oc d L}
\UnaRule{\Gamma,\oc{A}\vdash C}
\DisplayProof
\qquad
\AxRule{\Gamma,\oc{A},\oc{A}\vdash C}
\LabelRule{\oc c L}
\UnaRule{\Gamma,\oc{A}\vdash C}
\DisplayProof
\qquad
\AxRule{\Gamma\vdash C}
\LabelRule{\oc w L}
\UnaRule{\Gamma,\oc{A}\vdash C}
\DisplayProof\)

\(\AxRule{\Gamma\vdash A}
\LabelRule{\forall R}
\UnaRule{\Gamma\vdash \forall\xi A}
\DisplayProof
\qquad
\AxRule{\Gamma,A[\tau/\xi]\vdash C}
\LabelRule{\forall L}
\UnaRule{\Gamma,\forall\xi A\vdash C}
\DisplayProof
\qquad
\AxRule{\Gamma\vdash A[\tau/\xi]}
\LabelRule{\exists R}
\UnaRule{\Gamma\vdash\exists\xi A}
\DisplayProof
\qquad
\AxRule{\Gamma,A\vdash C}
\LabelRule{\exists L}
\UnaRule{\Gamma,\exists\xi A\vdash C}
\DisplayProof\)

with \(\xi\) not free in \(\Gamma,C\) in the rules \(\forall R\) and
\(\exists L\).

\section{The intuitionistic fragment of linear logic}\label{the-intuitionistic-fragment-of-linear-logic}

In order to characterize intuitionistic linear logic inside linear
logic, we define the intuitionistic restriction of linear formulas:

\(J ::= X \mid J\tens J \mid \one \mid J\limp J \mid J\with J \mid \top \mid J\plus J \mid \zero \mid \oc{J} \mid \forall\xi J \mid \exists\xi J\)

\(JLL\) is the \hyperref[fragment]{fragment} of linear logic obtained by restriction to
intuitionistic formulas.

\begin{proposition}[From $ILL$ to $JLL$]
If $\Gamma\vdash A$ is provable in $ILL_{012}$, it is provable in $JLL_{012}$.
\end{proposition}

\begin{proof}
$ILL_{012}$ is included in $JLL_{012}$.
\end{proof}

\begin{theorem}[From $JLL$ to $ILL$]
If $\Gamma\vdash\Delta$ is provable in $JLL_{12}$, it is provable in $ILL_{12}$.
\end{theorem}

\begin{proof}
We only prove the first order case, a proof of the full result is given in the PhD thesis of Harold Schellinx~\cite{phdschellinx}.

Consider a cut-free proof of $\Gamma\vdash\Delta$ in $JLL_{12}$, we can prove by induction on the length of such a proof that it belongs to $ILL_{12}$.
\end{proof}

\begin{corollary}[Unique conclusion in $JLL$]
If $\Gamma\vdash\Delta$ is provable in $JLL_{12}$ then $\Delta$ is a singleton.
\end{corollary}

The theorem is also valid for formulas containing \(\one\) or \(\top\)
but not anymore with \(\zero\).
\({}\vdash((X\limp Y)\limp\zero)\limp(X\tens(\zero\limp Z))\) is
provable in \(JLL_0\):
\begin{prooftree}
\LabelRule{\rulename{ax}}
\NulRule{X\vdash X}
\LabelRule{\zero L}
\NulRule{\zero\vdash Y,Z}
\LabelRule{\limp R}
\UnaRule{{}\vdash Y,\zero\limp Z}
\LabelRule{\tens R}
\BinRule{X\vdash Y,X\tens(\zero\limp Z)}
\LabelRule{\limp R}
\UnaRule{{}\vdash X\limp Y,X\tens(\zero\limp Z)}
\LabelRule{\zero L}
\NulRule{\zero\vdash {}}
\LabelRule{\limp L}
\BinRule{(X\limp Y)\limp\zero\vdash X\tens(\zero\limp Z)}
\LabelRule{\limp R}
\UnaRule{{}\vdash((X\limp Y)\limp\zero)\limp(X\tens(\zero\limp Z))}
\end{prooftree}
but not in \(ILL_0\).

\section{Input / output polarities}\label{input-output-polarities}

In order to go to \(LL\) without \(\limp\), we consider two classes of
formulas: \emph{input formulas} (\(I\)) and \emph{output formulas}
(\(O\)).

\(I ::= X\orth \mid I\parr I \mid \bot \mid I\tens O \mid O\tens I \mid I\plus I \mid \zero \mid I\with I \mid \top \mid \wn{I} \mid \exists\xi I \mid \forall\xi I\)
\(O ::= X \mid O\tens O \mid \one \mid O\parr I \mid I\parr O \mid O\with O \mid \top \mid O\plus O \mid \zero \mid \oc{O} \mid \forall\xi O \mid \exists\xi O\)

By applying the definition of the linear implication
\(A\limp B = A\orth\parr B\), any formula of \(JLL\) is mapped to an
output formula (and the dual of a \(JLL\) formula to an input formula).
Conversely, any output formula is coming from a \(JLL\) formula in this
way (up to commutativity of \(\parr\): \(O\parr I = I\parr O\)).

The \hyperref[fragment]{fragment} of linear logic obtained by restriction to
input/output formulas is thus equivalent to \(JLL\), but the closure of
the set of input/output formulas under orthogonal allows for a one-sided
presentation.

\(\LabelRule{\rulename{ax}}
\NulRule{\vdash O\orth,O}
\DisplayProof
\qquad
\AxRule{{}\vdash \Gamma,O}
\AxRule{{}\vdash\Delta,O\orth}
\LabelRule{\rulename{cut}}
\BinRule{{}\vdash\Gamma,\Delta}
\DisplayProof\)

\(\AxRule{{}\vdash\Gamma,A}
\AxRule{{}\vdash\Delta,B}
\LabelRule{\tens}
\BinRule{{}\vdash\Gamma,\Delta,A\tens B}
\DisplayProof
\qquad
\AxRule{{}\vdash\Gamma,A,B}
\LabelRule{\parr}
\UnaRule{{}\vdash\Gamma,A\parr B}
\DisplayProof
\qquad
\LabelRule{\one}
\NulRule{{}\vdash\one}
\DisplayProof
\qquad
\AxRule{{}\vdash\Gamma}
\LabelRule{\bot}
\UnaRule{{}\vdash\Gamma,\bot}
\DisplayProof\)

\(\AxRule{{}\vdash\Gamma,A}
\AxRule{{}\vdash\Gamma,B}
\LabelRule{\with}
\BinRule{{}\vdash\Gamma,A\with B}
\DisplayProof
\qquad
\AxRule{{}\vdash\Gamma,A}
\LabelRule{\plus_1}
\UnaRule{{}\vdash\Gamma,A\plus B}
\DisplayProof
\qquad
\AxRule{{}\vdash\Gamma,B}
\LabelRule{\plus_2}
\UnaRule{{}\vdash\Gamma,A\plus B}
\DisplayProof
\qquad
\LabelRule{\top}
\NulRule{{}\vdash\Gamma,\top}
\DisplayProof\)

\(\AxRule{{}\vdash\wn{\Gamma},O}
\LabelRule{\oc}
\UnaRule{{}\vdash\wn{\Gamma},\oc{O}}
\DisplayProof
\qquad
\AxRule{{}\vdash\Gamma,I}
\LabelRule{\wn d}
\UnaRule{{}\vdash\Gamma,\wn{I}}
\DisplayProof
\qquad
\AxRule{{}\vdash\Gamma,\wn{I},\wn{I}}
\LabelRule{\wn c}
\UnaRule{{}\vdash\Gamma,\wn{I}}
\DisplayProof
\qquad
\AxRule{{}\vdash\Gamma}
\LabelRule{\wn w}
\UnaRule{{}\vdash\Gamma,\wn{I}}
\DisplayProof\)

\(\AxRule{{}\vdash\Gamma,A}
\LabelRule{\forall}
\UnaRule{{}\vdash\Gamma,\forall\xi A}
\DisplayProof
\qquad
\AxRule{{}\vdash\Gamma,A[\tau/\xi]}
\LabelRule{\exists}
\UnaRule{{}\vdash\Gamma,\exists\xi A}
\DisplayProof\)

with \(A\) and \(B\) arbitrary input or output formulas (under the
condition that the composite formulas containing them are input or
output formulas) and \(\xi\) not free in \(\Gamma\) in the rule
\(\forall\).

\begin{lemma}[Output formula]
If ${}\vdash\Gamma$ is provable in $LL_{12}$ and contains only input and output formulas, then $\Gamma$ contains exactly one output formula.
\end{lemma}

\begin{proof}
Assume $\Gamma_O$ is obtained by turning the output formulas of $\Gamma$ into $JLL$ formulas and $\Gamma_I$ is obtained by turning the dual of the input formulas of $\Gamma$ into $JLL$ formulas, $\Gamma_I\vdash\Gamma_O$ is provable in $LL_{12}$ thus in $JLL_{12}$. By corollary (Unique conclusion in $JLL$), $\Gamma_O$ is a singleton, thus $\Gamma$ contains exactly one output formula.
\end{proof}


%%% Local Variables:
%%% mode: latex
%%% TeX-master: "main"
%%% End:


\chapter{Polarized linear logic}\label{polarized-linear-logic}

\emph{Polarized linear logic} (LLP) is a logic close to plain linear
logic in which structural rules, usually restricted to \(\wn\)-formulas,
have been \hyperref[generalized-structural-rules-pos]{extended} to the
whole class of so called \emph{negative} formulas.

\section{Polarization}\label{polarization}

LLP relies on the notion of \emph{polarization}, that is, it
discriminates between two types of formulas, \emph{negative} (noted
\(M, N...\)) vs. \emph{\hyperref[positive-formula]{positive}} (\(P, Q...\)).
They are mutually defined as follows:
\begin{align*}
N &::= X \mid N \parr N \mid \bot \mid N \with N \mid \top \mid \wn{P} \\
P &::= X\orth \mid P \otimes P \mid 1 \mid P \oplus P \mid 0 \mid \oc{N}
\end{align*}

The dual operation \((-)\orth\) extended to propositions exchanges the
roles of connectives and reverses the polarity of formulas.

\section{Deduction rules}\label{deduction-rules}

There are several design choices for the structure of sequents. In
particular, LLP proofs are \emph{focused}, i.e. they contain at most
one positive formula. We choose to represent this explicitly using
sequents of the form \(\vdash\Gamma\mid\Delta\), where \(\Gamma\) is a
multiset of negative formulas, and \(\Delta\) is a \emph{stoup} that
contains at most one positive formula (though it may be empty).

\(\LabelRule{\rulename{ax}}
\NulRule{\vdash P\orth \mid P}
\DisplayProof
\qquad
\AxRule{\vdash \Gamma_1, N \mid \Delta}
\AxRule{\vdash \Gamma_2 \mid N\orth}
\LabelRule{\rulename{cut}}
\BinRule{\vdash\Gamma_1, \Gamma_2 \mid \Delta}
\DisplayProof\)

\(\AxRule{\vdash\Gamma, N\mid\cdot}
\LabelRule{p}
\UnaRule{\vdash\Gamma\mid \oc{N}}
\DisplayProof
\qquad
\AxRule{\vdash\Gamma\mid P}
\LabelRule{d}
\UnaRule{\vdash\Gamma,\wn{P}\mid \cdot}
\DisplayProof
\qquad
\AxRule{\vdash\Gamma,N,N\mid \Delta}
\LabelRule{c}
\UnaRule{\vdash\Gamma, N\mid\Delta}
\DisplayProof
\qquad
\AxRule{\vdash\Gamma\mid \Delta}
\LabelRule{w}
\UnaRule{\vdash\Gamma,N\mid\Delta}
\DisplayProof\)

\(\AxRule{\vdash\Gamma_1\mid P}
\AxRule{\vdash\Gamma_2\mid Q}
\LabelRule{\tens}
\BinRule{\vdash\Gamma_1,\Gamma_2\mid P\otimes Q}
\DisplayProof
\qquad
\LabelRule{\one}
\NulRule{\vdash\cdot\mid\one}
\DisplayProof
\qquad
\AxRule{\vdash\Gamma, M, N\mid \Delta}
\LabelRule{\parr}
\UnaRule{\vdash\Gamma, M\parr N\mid \Delta}
\DisplayProof
\qquad
\AxRule{\vdash\Gamma\mid \Delta}
\LabelRule{\bot}
\UnaRule{\vdash\Gamma, \bot\mid\Delta}
\DisplayProof\)

\(\AxRule{\vdash\Gamma\mid P}
\LabelRule{\plus_1}
\UnaRule{\vdash\Gamma\mid P\plus Q}
\DisplayProof
\qquad
\AxRule{\vdash\Gamma\mid Q}
\LabelRule{\plus_2}
\UnaRule{\vdash\Gamma\mid P\plus Q}
\DisplayProof
\qquad
\AxRule{\vdash\Gamma,M\mid \Delta}
\AxRule{\vdash\Gamma,N\mid \Delta}
\LabelRule{\with}
\BinRule{\vdash\Gamma,M\with N\mid \Delta}
\DisplayProof
\qquad
\LabelRule{\top}
\NulRule{\vdash\Gamma,\top\mid \Delta}
\DisplayProof\)


%%% Local Variables:
%%% mode: latex
%%% TeX-master: "main"
%%% End:


\chapter{Fragment}\label{fragment}

In general, a \emph{fragment} of a logical system \emph{S} is a
logical system obtained by restricting the language of \emph{S}, and by
restricting the rules of \emph{S} accordingly. In linear logic, the most
well known fragments are obtained by combining/removing in different
ways the classes of connectives present in the
\hyperref[sequent-calculus]{language of linear logic} itself:
\begin{description}
\item[Multiplicative connectives:] the conjunction \(\tens\)
  (\emph{tensor}) and the disjunction \(\parr\) (\emph{par}), with their
  respective units \(\one\) (\emph{one}) and \(\bot\) (\emph{bottom});
  these connectives are the combinatorial base of linear logic
  (permutations, circuits, etc.).
\item[Additive connectives:] the conjunction \(\with\) (\emph{with})
  and the disjunction \(\plus\) (\emph{plus}), with their respective
  units \(\top\) (\emph{top}) and \(\zero\) (\emph{zero}); the
  computational content of these connectives, which behave more closely
  to their intuitionistic counterparts (\emph{e.g.}, \(A\with B\limp A\)
  and \(A\with B\limp B\) are provable), is strongly related to choice
  (\emph{if...then...else}, product and sum types, etc.).
\item[Exponential connectives:] the modalities \(\oc\) (\emph{of
  course}) and \(\wn\) (\emph{why not}) handle the structural rules in
  linear logic, and are necessary to recover the expressive power of
  \hyperref[translations-of-intuitionistic-logic]{intuitionistic} or \hyperref[translations-of-classical-logic]{classical} logic.
\item[Quantifiers:] just as in classical logic, quantifiers may be
  added to propositional linear logic, at any order. The most frequently
  considered are the second order ones (in analogy with System F).
\end{description}

The additive and exponential connectives, if taken alone, yield
fragments of limited interest, so one usually considers only fragments
containing at least the multiplicative connectives (perhaps without
units). It is important to observe that the
\hyperref[cut-elimination-and-consequences]{cut elimination rules} of linear logic do not introduce connectives
belonging to a different class than that of the pair of dual formulas
whose cut is being reduced. Hence, any fragment defined by combining the
above classes will enjoy cut elimination. Since cut elimination implies
the subformula property, all of the
\hyperref[equivalences]{fundamental equivalences} provable
in full linear logic remain valid within such fragments, as soon as the
formulas concerned belong to the fragment itself.

Conventionally, if \(LL\) denotes full linear logic, its fragments are
denoted by prefixing \(LL\) with letters corresponding to the classes of
connectives being considered: \(M\) for multiplicative connectives,
\(A\) for additive connectives, and \(E\) for exponential connectives.
Additional subscripts may specify whether units and/or quantifiers are
present or not, and, for quantifiers, of what order (see~\cref{notations}).

\section{Motivations}\label{motivations}

The main interest of studying fragments of linear logic is that these
are usually simpler than the whole system, so that certain properties
may be first analyzed on fragments, and then extended or adapted to
increasingly larger fragments. It may also be interesting to see, given
a property that does not hold for full linear logic, whether it holds
for a fragment, and where the ``breaking point'' is situated. Examples of
such questions include:
\begin{description}
\item[logical complexity:] proving cut elimination for full linear
  logic with second order quantification is equivalent to proving the
  consistency of second order Peano arithmetic (Girard, via
  \hyperref[translations-of-intuitionistic-logic]{translations of System F}
  in linear logic). One may expect that smaller fragments have lower
  logical complexity.
\item[provability:] the \emph{provability problem} for a logical
  system \emph{S} is defined as follows: given a formula \(A\) in the
  language of \emph{S}, is \(A\) provable in \emph{S}? This problem is
  undecidable in full linear logic with quantifiers, of whatever order
  (again, because \hyperref[translations-of-classical-logic]{classical logic
  can be translated} in linear logic). On what fragments does it become
  decidable? And if it does, what is its computational complexity?
\item[computational complexity of cut elimination:] the \emph{cut
  elimination problem} (Mairson-Terui~\cite{cutelimcomplexity}) for a logical system \emph{S} is
  defined as follows: given two proofs of \(A\) in \emph{S}, do they
  reduce to the same cut-free proof? Although decidable (thanks to
  strong normalization), this problem is not elementary recursive in
  full propositional linear logic (Statman, again via the
  above-mentioned translations). Does the problem fall into any
  interesting complexity class when applied to fragments?
\item[proof nets:] the definition of proof nets, and in particular
  the formulation of correctness criteria and the study of their
  complexity, is a good example of how a methodology can be applied to a
  small fragment of linear logic and later adapted (more or less
  successfully) to wider fragments.
\item[denotational semantics:] several problems related to
  denotational semantics (formulation of
  \hyperref[categorical-semantics]{categorical models}, full abstraction,
  full completeness, injectivity, etc.) may be first attacked in the
  simpler case of fragments, and then extended to wider subsystems.
\end{description}

\section{Multiplicative fragments}\label{multiplicative-fragments}

Multiplicative linear logic (\(MLL\)) is the simplest of the well known
fragments of linear logic. Its formulas are obtained by combining
propositional atoms with the connectives \emph{tensor} and \emph{par}
only. As a consequence, the
\hyperref[sequents-and-proofs]{sequent calculus} of
\(MLL\) is limited to the rules \(\rulename{axiom}\),
\(\rulename{cut}\), \(\tens\), and \(\parr\). These rules actually
determine the multiplicative connectives: if a dual pair of connectives
\(\tens'\) and \(\parr'\) is introduced, with the same rules as
\(\tens\) and \(\parr\), respectively, then one can show \(A\tens' B\)
to be provably equivalent to \(A\tens B\) (and, dually, \(A\parr'B\) to
be provably equivalent to \(A\parr B\)).

The cut elimination problem for \(MLL\) is \(\mathbf P\)-complete
(Mairson-Terui~\cite{cutelimcomplexity}), even though there exists a deterministic algorithm
solving the problem in logarithmic space if one considers only
\hyperref[expansion-of-identities]{eta-expanded} proofs
(Mairson-Terui~\cite{cutelimcomplexity}). On the other hand, provability for \(MLL\) is
\(\mathbf{NP}\)-complete, and it remains so even in presence of first
order quantifiers.

Another multiplicative fragment, less considered in the literature, can
be defined by using the units \(\one\) and \(\bot\) instead of the
propositional atoms. In this fragment, denoted by \(MLL_u\), one can
also eliminate the \(\rulename{axiom}\) rule from sequent calculus,
since it is redundant. \(MLL_u\) is even simpler than \(MLL\): its
provability problem is in \(\mathbf P\), and, since all proofs are
eta-expandend, its cut elimination problem is in \(\mathbf L\).

The union of \(MLL\) and \(MLL_u\) is the full propositional
multiplicative fragment of linear logic, and is denoted by \(MLL_0\). It
has the same properties as \(MLL\), which shows that the
presence/absence of propositional atoms (and of the \(\rulename{axiom}\)
rule) has a non-trivial effect on the complexity of provability and cut
elimination, \emph{i.e.}, the complexity is not altered iff
\(\mathbf P\subsetneq\mathbf{NP}\) and \(\mathbf L\subsetneq\mathbf P\),
respectively.

If we add second order quantifiers to \(MLL\) (resp. \(MLL_u\)), we
obtain a system denoted by \(MLL_2\) (resp. \(MLL_{02}\)). In
\(MLL_{02}\) one can show that \(\one\) and \(\bot\) are provably
equivalent to \(\forall X.(X\orth\parr X)\) and
\(\exists X.(X\orth\tens X)\), respectively. Hence, \(MLL_2\) is as
expressive as \(MLL_{02}\). In these second order fragments, provability
is undecidable, while cut elimination is still \(\mathbf P\)-complete.

\section{Additive fragments}\label{additive-fragments}

The most studied additive fragments of linear logic are defined by
taking \(MLL\) or \(MLL_0\) and by enriching their language with the
additive connectives, with or without units. The same can be done in
presence of quantifiers. We thus obtain:
\begin{itemize}
\item
  \(MALL\): formulas built from propositional atoms using
  \(\tens,\parr,\with,\plus\);
\item
  \(MALL_0\): formulas built from propositional atoms and
  \(\one,\bot,\top,\zero\), using \(\tens,\parr,\with,\plus\);
\item
  \(MALL_n\): \(MALL\) with quantifiers of order \(n\);
\item
  \(MALL_{0n}\): \(MALL_0\) with quantifiers of order \(n\).
\end{itemize}

The \wantedpage{purely additive framents} are less
common:
\begin{itemize}
\item
  \(ALL\): formulas built from propositional atoms using
  \(\with,\plus\);
\item
  \(ALL_0\): formulas built from propositional atoms and \(\top,\zero\),
  using \(\with,\plus\);
\item
  \(ALL_n\): \(ALL\) with quantifiers of order \(n\);
\item
  \(ALL_{0n}\): \(ALL_0\) with quantifiers of order \(n\).
\end{itemize}

As for the multiplicative connectives, the additive connectives are also
defined by their rules: adding a pair of dual connectives
\(\with',\plus'\) to \(MALL\), and giving them the same rules as
\(\with,\plus\), makes the new connectives provably equivalent to the
old ones.

In \(MALL_{02}\), the additive units \(\top\) and \(\zero\) are provably
equivalent to \(\exists X.X\orth\) and \(\forall X.X\), respectively.
Since multiplicative units are also definable in terms of second order
quantification, we obtain that \(MALL_2\) is as expressive as
\(MALL_{02}\).

The cut elimination problem is \(\mathbf{coNP}\)-complete for all of the
fragments defined above (Mairson-Terui~\cite{cutelimcomplexity}).

Provability is undecidable in any additive fragment as soon as second
order quantification is considered. It is decidable, although quite
complex, in the propositional and first order case: it is
\(\mathbf{PSPACE}\)-complete in \(MALL_0\), and
\(\mathbf{NEXP}\)-complete in \(MALL_{01}\). This latter result is
indicative of the fact that the undecidability of predicate calculus is
not ascribable to existential quantification alone, but rather to the
simultaneous presence of existential quantification and contraction.

\section{Exponential fragments}\label{exponential-fragments}

The most common proper fragments of linear logic containing the
exponential connectives are defined as in the case of the additive
fragments, \emph{i.e.}, by adding the modalities on top of \(MLL\) and
its variants:
\begin{itemize}
\item
  \(MELL\): formulas built from propositional atoms using
  \(\tens,\parr,\oc,\wn\);
\item
  \(MELL_0\): formulas built from propositional atoms and \(\one,\bot\),
  using \(\tens,\parr,\oc,\wn\);
\item
  \(MELL_n\): \(MELL\) with quantifiers of order \(n\);
\item
  \(MELL_{0n}\): \(MELL_0\) with quantifiers of order \(n\).
\end{itemize}

If, instead of taking \(MLL\), we add the modalities to \(MALL\), we
obtain of course various versions of full linear logic:
\begin{itemize}
\item
  \(LL\): full linear logic, without units;
\item
  \(LL_0\): full linear logic, with units;
\item
  \(LL_n\): \(LL\) with quantifiers of order \(n\);
\item
  \(LL_{0n}\): \(LL_0\) with quantifiers of order \(n\).
\end{itemize}

In \(LL_{02}\) the formulas \(A\with B\) and \(A\plus B\) are provably
equivalent to
\(\exists X.(\oc{(X\orth\parr A)}\tens\oc{(X\orth\parr B)}\tens X)\) and
\(\forall X.(\wn{(X\orth\tens A)}\parr\wn{(X\orth \tens B)}\parr X)\),
respectively, for all \(A,B\). Thanks to the second-order definability
of units discussed above, we obtain that \(MELL_2\) is as expressive as
\(LL_{02}\), \emph{i.e.}, full propositional second order linear logic
embeds in its second order multiplicative exponential fragment without
units.

Girard showed how cut elimination for \(LL_{02}\) \emph{without the
contraction rule} can be proved by a simple induction up to \(\omega\),
\emph{i.e.}, in first order Peano arithmetic. This gives a huge gap
between the logical complexity of full linear logic and its
contraction-free subsystem: in fact, still by Girard's results, we know
that cut elimination in \(MELL_2\) is equivalent to the consistency of
second order Peano arithmetic, for which no ordinal analysis is known.
There are nevertheless subsystems of \(MELL_2\), the so-called
\hyperref[light-linear-logics]{light subsystems} of linear logic, in which
the exponential connectives are weakened, whose cut elimination can be
proved in seconder order Peano arithmetic even in presence of
contraction.

The cut elimination problem is never elementary recursive in presence of
exponential connectives: the simply typed \(\lambda\)-calculus with
arrow types only can be encoded in \(MELL\), and this is enough for
Statman's lower bound to apply. However, it becomes elementary recursive
in the above mentioned \hyperref[light-linear-logics]{light logics}.

Albeit perhaps surprisingly, provability in \(LL\) is already
undecidable. This result, obtained by coding Minsky machines with linear
logic formulas, contrasts with the situation in classical logic, whose
propositional fragment is notoriously decidable. It is indicative of the
fact that modalities are themselves a form of quantification, although
this claim is far from being clear: as a matter of fact, the
decidability of propositional provability in the absence of additives,
\emph{i.e.}, in \(MELL\) alone, is still an open problem. It is known
that adding first order quantification to \(MELL\) makes it undecidable.

\subsection{About exponential rules}\label{about-exponential-rules}

In this section, provability is assumed to be in \(LL_{02}\),
\emph{i.e.}, full propositional second order linear logic.

In contrast with multiplicative and additive connectives, the modalities
of linear logic are not defined by their rules: one may introduce a pair
of dual modalities \(\oc',\wn'\), each with the same rules as
\(\oc,\wn\), without \(\oc'A\) (resp. \(\wn'A\)) being in general
provably equivalent to \(\oc A\) (resp. \(\wn A\)).

The \hyperref[sequents-and-proofs]{promotion rule} is
derivable from the following two rules, called \emph{(multi)functorial
promotion} and \emph{digging}, respectively:
\begin{equation*}
\AxRule{\vdash\Gamma,A}
\LabelRule{\oc\rulename{mf}}
\UnaRule{\vdash\wn\Gamma,\oc A}
\DisplayProof
\qquad\qquad\qquad
\AxRule{\vdash\Gamma,\wn{\wn A}}
\LabelRule{\wn\wn}
\UnaRule{\vdash\Gamma,\wn A}
\DisplayProof
\end{equation*}

Functorial promotion is itself derivable from dereliction and promotion;
the digging rule is also derivable, but only using the
\(\rulename{cut}\) rule (in fact, digging does not enjoy the subformula
property). It may be convenient to consider this pair of rules instead
of the standard promotion rule in the context of
\hyperref[categorical-semantics]{categorical semantics} of linear logic.

In presence of the digging rule, dereliction, weakening, and contraction
can be derived from the following rule, called \emph{multiplexing}, in
which \(A^{(n)}\) stands for the sequence \(A,\ldots,A\) containing
\(n\) occurrences of \(A\):
\begin{prooftree}
\AxRule{\vdash\Gamma,A^{(n)}}
\LabelRule{\rulename{mux}}
\UnaRule{\vdash\Gamma,\wn A}
\end{prooftree}

Of course, multiplexing is itself derivable from dereliction, weakening,
and contraction. Hence, there are several alternative but equivalent
presentations of the exponential fragment of linear logic, such as
\begin{enumerate}
\item remove promotion, and replace it with functorial promotion and
  digging;
\item remove promotion, dereliction, weakening, and contraction, and replace
  them with functorial promotion, digging, and multiplexing.
\end{enumerate}

Apart from their usefulness in \hyperref[categorical-semantics]{categorical
semantics}, these alternative formulations are of interest in the
context of the so-called \hyperref[light-linear-logics]{light linear logics}
mentioned above. For example, \emph{elementary linear logic} is obtained
by removing dereliction and digging from formulation 1, and \emph{soft
linear logic} is obtained by removing digging from formulation 2.

Multiplexing is invertible in certain circumstances. A sequent
\(\vdash\Gamma,\wn A\) containing no occurrence of \(\with\), \(\oc\),
or second order \(\exists\) is provable iff \(\vdash\Gamma,A^{(n)}\) is
provable for some \(n\) (this is easily checked by induction on cut-free
proofs). To see that this does not hold in general, take for instance
\(A=X\orth\) and \(\Gamma=X\with\one\), or \(\Gamma=\oc X\). The
restriction on the presence of additive conjunction can be removed by
slightly changing the statement: a sequent \(\vdash\Gamma,\wn A\)
containing no occurrence of \(\oc\) or second order \(\exists\) is
provable iff \(\vdash\Gamma,(A\plus\bot)^{(n)}\) is provable for some
\(n\).

The latter result can be generalized as follows. If \(A\) is a formula,
\(\oc_nA\) stands for the formula
\((A\with\one)\tens\cdots\tens(A\with\one)\) (\(n\) times) and
\(\wn_nA\) for the formula \((A\plus\bot)\parr\cdots\parr(A\plus\bot)\)
(\(n\) times). Then, we have

\begin{theorem}[Approximation Theorem]
Let $\vdash\Gamma$ be a provable sequent containing $p$ occurrences of $\oc$, $q$ occurrences of $\wn$, and no occurrence of second order $\exists$. Then, for all $m_1,\dotsc,m_p\in\mathbb N$, there are $n_1,\dotsc,n_q\in\mathbb N$ such that the sequent obtained from $\vdash\Gamma$ by replacing the $p$ occurrences of $\oc$ with $\oc_{m_1},\dotsc,\oc_{m_p}$ and the $q$ occurrences of $\wn$ with $\wn_{n_1},\dotsc,\wn_{n_q}$ is provable.
\end{theorem}

A \emph{structural formula} is a formula \(C\) such that
\(C\limp C\tens C\) and \(C\limp\one\) are provable. Obviously, any
formula of the form \(\wn B\) is structural. However, the promotion rule
cannot be extended to arbitrary structural formulas, \emph{i.e.}, the
following rule is \emph{not} admissible:
\begin{prooftree}
\AxRule{C\vdash A}
\AxRule{C\vdash C\tens C}
\AxRule{C\vdash\one}
\TriRule{C\vdash\oc A}
\end{prooftree}
For instance, if
\(A=C=\alpha\tens\oc{(\alpha\limp\alpha\tens\alpha)}\tens\oc{(\alpha\limp\one)}\),
the three premisses are provable but not the conclusion.

The following rule, called \emph{absorption}, is derivable in the
standard sequent calculus:
\begin{prooftree}
\AxRule{\vdash\Gamma,\wn A,A}
\UnaRule{\vdash\Gamma,\wn A}
\end{prooftree}
The absorption rule is useful in the context of proof search in linear logic.

\section{The provability problem}\label{the-provability-problem}

It is well known that the decidability of the provability problem is
connected to the \hyperref[phase-semantics]{finite model property}: if a
fragment of a logic with a truth semantics enjoys the finite model
property, then the provability in that fragment is decidable. Note of
course that the converse may fail.

In this section, we summarize the known results about the validity of
the finite model property and the decidability of provability, with its
complexity, for the various fragments of linear logic introduced above.
Question marks in the tables below denote open problems. For brevity,
all fragments are assumed to have units and propositional atoms,
\emph{e.g.}, \(MLL\) actually denotes what we called \(MLL_0\) above.

\subsection{The finite model property}\label{the-finite-model-property}

\begin{center}
\begin{tabular}{llll}
\hline
\(MLL\) & \(MALL\) & \(MELL\) & \(LL\)\\
yes & yes & no & no\\
\hline
\end{tabular}
\end{center}

\subsection{Provability}\label{provability}

\begin{center}
\begin{tabular}{lllll}
\hline
& \(MLL\) & \(MALL\) & \(MELL\) & \(LL\)\\
propositional case & \(\mathbf{NP}\)-complete &
\(\mathbf{PSPACE}\)-complete & ? & undecidable\\
first order case & \(\mathbf{NP}\)-complete & \(\mathbf{NEXP}\)-complete
& undecidable & undecidable\\
second order case & undecidable & undecidable & undecidable &
undecidable\\
\hline
\end{tabular}
\end{center}

\section{The cut elimination problem}\label{the-cut-elimination-problem}

In this section, we summarize the known results about the complexity of
the cut elimination problem for the various fragments of linear logic
introduced above, plus some \hyperref[light-linear-logics]{light linear
logics}. All fragments are assumed to be propositional; the results do
not change in presence of quantification of any order.

\begin{center}
\begin{tabular}{llllll}
\hline
\(MLL_u\) & \(MLL\) & \(MALL\) & \(MSLL\) & \(MLLL\) &
\(MELL\)\\
\(\mathbf L\) & \(\mathbf{P}\)-complete & \(\mathbf{coNP}\)-complete &
\(\mathbf{EXP}\)-complete & \(\mathbf{2EXP}\)-complete & not elementary
recursive\\
\hline
\end{tabular}
\end{center}

Notations used in the above table:
\begin{itemize}
\item \(MSLL\): multiplicative soft linear logic;
\item \(MLLL\): multiplicative light linear logic.
\end{itemize}


%%% Local Variables:
%%% mode: latex
%%% TeX-master: "main"
%%% End:


\chapter{Proof-nets}\label{proof-nets}

We provide a formal account of nets,
but it is probably not the best way to learn about proof-nets if you
have never seen them before.


%%% Local Variables:
%%% mode: latex
%%% TeX-master: "main"
%%% End:

\section{A formal account of nets}\label{a-formal-account-of-nets}

The aim of this page is to provide a common framework for describing
linear logic proof nets, interaction nets, multiport interaction nets,
and the likes, while factoring out most of the tedious, uninteresting
details (clearly not the fanciest page of LLWiki).

\subsection{Preliminaries}\label{preliminaries}

\subsubsection{The short story}\label{the-short-story}

\begin{itemize}
\tightlist
\item
  the general flavor is that of multiport interaction nets;
\item
  the top/down or passive/active orientation of cells is related with
  the distinction between premisses and conclusions of rules, (and in
  that sense, a cut is not a logical rule, but the focus of interaction
  between two rules);
\item
  cuts are thus wires rather than cells/links: this fits with the
  intuition of GoI, but not with the most common presentations of proof
  nets;
\item
  because the notion of subnet is not trivial in multiport interaction
  nets, and to avoid the use of geometric conditions (boxes must not
  overlap but can be nested), we introduce boxes as particular cells;
\item
  when representing proof nets, we introduce axioms explicitly as cells,
  so that axiom-cuts do not vanish.
\end{itemize}

\subsection{Nets}\label{nets}

\subsubsection{Wires}\label{wires}

A \emph{wiring} is the data of a finite set \(P\) of ports and of a
partition \({W}\) of \(P\) by pairs (the \emph{wires}): if
\(\{p,q\}\in{W}\), we write \({W}(p)=q\) and \({W}(q)=p\). Hence a
wiring is equivalently given by an involutive permutation \({W}\) of
finite domain \(P\), without fixpoints (forall \(p\), \({W}(p)\not=p\)):
the wires are then the orbits. Another equivalent presentation is to
consider \({W}\) as a (simple, loopless, undirected) graph, with
vertices in P, all of degree 1.

We say two wirings are disjoint when their sets of ports are. A
\emph{connection} between two disjoint wirings \(W\) and \(W'\) is a
partial injection \((I,I',f):P\pinj P'\): \(I\subseteq P\),
\(I'\subseteq P'\) and \(f\) is a bijection \(I\cong I'\). We then write
\(W\bowtie_f{{W}'}\) for the wiring obtained by identifying the ports
pairwise mapped by \(f\), and then ``straightening`` the paths thus
obtained to recover wires: notice this might also introduce loops and we
write \(\Inner{W}{W'}_f\) for the number of loops thus appeared.

We describe these operations a bit more formally. Write
\(P = P_0\uplus I\) and \(P' = P_0'\uplus I'\). Then consider the graph
\(W\dblcolon_f{{W}'}\) with vertices in \(P\cup P'\), and such that
there is an edge between \(p\) and \(q\) iff \(q={W}(p)\) or
\(q={W'}(p)\) or \(q=f(p)\) or \(p=f(q)\): in other words,
\(W\dblcolon_f{{W}'}=W\cup W'\cup f\cup f^{-1}\). Vertices in
\(P_0\cup P'_0\) are of degree 1, and the others are of degree 2. Hence
maximal paths in \(W\dblcolon_f{{W}'}\) are of two kinds:

\begin{itemize}
\tightlist
\item
  straight paths, with both ends in \(P_0\cup P_0'\);
\item
  cycles, with vertices all in \(I\cup I'\).
\end{itemize}

Then the wires in \(W\bowtie_f{{W}'}\) are the pairs \(\{p,p'\}\) such
that \(p\) and \(p'\) are the ends of a path in \(W\dblcolon_f{{W}'}\).
And \(\Inner{W}{W'}_{f}\) is the number of cycles in
\(W\dblcolon_f{{W}'}\), or more precisely the number of support sets of
cycles (i.e. we forget about the starting vertice of cycles).

\&\&\textbackslash{}ar{[}dl{]}\^{}\{A\textbackslash{}otimes\textbackslash{}lambda\_B\}A\textbackslash{}otimes(I\textbackslash{}otimes
B)\textbackslash{}\textbackslash{}

\texttt{~~~\&A\textbackslash{}otimes~B\&}

\} commutes. \}\}

\&(B\textbackslash{}otimes C)\textbackslash{}otimes
A\textbackslash{}ar{[}dr{]}\^{}\{\textbackslash{}alpha\_\{B,C,A\}\}\textbackslash{}\textbackslash{}
(A\textbackslash{}otimes B)\textbackslash{}otimes
C\textbackslash{}ar{[}ur{]}\^{}\{\textbackslash{}alpha\_\{A,B,C\}\}\textbackslash{}ar{[}dr{]}\_\{\textbackslash{}gamma\_\{A,B\}\textbackslash{}otimes
C\}\&\&\&B\textbackslash{}otimes (C\textbackslash{}otimes
A)\textbackslash{}\textbackslash{} \&(B\textbackslash{}otimes
A)\textbackslash{}otimes
C\textbackslash{}ar{[}r{]}\_\{\textbackslash{}alpha\_\{B,A,C\}\}\&B\textbackslash{}otimes(A\textbackslash{}otimes
C)\textbackslash{}ar{[}ur{]}\_\{B\textbackslash{}otimes\textbackslash{}gamma\_\{A,C\}\}\textbackslash{}\textbackslash{}
\} and

\[\xymatrix{
&amp;(A\otimes B)\otimes C\ar[r]^{\gamma_{A\otimes B,C}}&amp;C\otimes (A\otimes B)\ar[dr]^{\alpha^{-1}_{C,A,B}}&amp;\\
A\otimes (B\otimes C)\ar[ur]^{\alpha^{-1}_{A,B,C}}\ar[dr]_{A\otimes\gamma_{B,C}}&amp;&amp;&amp;(C\otimes A)\otimes B\\
&amp;A\otimes(C\otimes B)\ar[r]_{\alpha^{-1}_{A,C,B}}&amp;(A\otimes C)\otimes B\ar[ur]_{\gamma_{A,C}\otimes B}&amp;\\
}\] commute for every objects \(A\), \(B\) and \(C\).

A \emph{symmetric} monoidal category is a braided monoidal category in
which the braiding satisfies

\[\gamma_{B,A}\circ\gamma_{A,B}=A\otimes B\] for every objects \(A\) and
\(B\). \}\}

\&B \} commute.

Dually, the monoidal category \(\mathcal{C}\) is \emph{right closed}
when the functor \(B\mapsto B\otimes A\) admits a right adjoint. The
notion of \emph{right closed structure} can be defined similarly. \}\}

In a symmetric monoidal category, a left closed structure induces a
right closed structure and conversely, allowing us to simply speak of a
\emph{closed symmetric monoidal category}.

\subsection{Modeling the additives}\label{modeling-the-additives}

A category has \emph{finite products} when it has products and a
terminal object.

\textbackslash{}ar{[}r{]}\^{}-\{\textbackslash{}mu\textbackslash{}tens
M\}\&M\textbackslash{}tens
M\textbackslash{}ar{[}dd{]}\^{}\{\textbackslash{}mu\}\textbackslash{}\textbackslash{}
M\textbackslash{}tens(M\textbackslash{}tens
M)\textbackslash{}ar{[}d{]}\_\{M\textbackslash{}tens\textbackslash{}mu\}\&\&\textbackslash{}\textbackslash{}
M\textbackslash{}tens
M\textbackslash{}ar{[}rr{]}\_\{\textbackslash{}mu\}\&\&M\textbackslash{}\textbackslash{}
\} and

\[\xymatrix{
I\tens M\ar[r]^{\eta\tens M}\ar[dr]_{\lambda_M}&amp;M\tens M\ar[d]_\mu&amp;\ar[l]_{M\tens\eta}\ar[dl]^{\rho_M}M\tens I\\
&amp;M&amp;
}\] commute. \}\}

\subsection{\texorpdfstring{Modeling
\url{ILL}}{Modeling ILL}}\label{modeling-ill}

Introduced in\footnote{}.

\[\oc=L\circ M\]

This section is devoted to defining the concepts necessary to define
these adjunctions.

\&FA\textbackslash{}bullet(FB\textbackslash{}bullet
FC)\textbackslash{}ar{[}dr{]}\^{}\{FA\textbackslash{}bullet\textbackslash{}phi\_\{B,C\}\}\textbackslash{}\textbackslash{}

\texttt{~~~F(A\textbackslash{}otimes~B)\textbackslash{}bullet~FC\textbackslash{}ar{[}dr{]}\_\{\textbackslash{}phi\_\{A\textbackslash{}otimes~B,C\}\}\&\&FA\textbackslash{}bullet~F(B\textbackslash{}otimes~C)\textbackslash{}ar{[}d{]}\^{}\{\textbackslash{}phi\_\{A,B\textbackslash{}otimes~C\}\}\textbackslash{}\textbackslash{}}\\
\texttt{~~~\&F((A\textbackslash{}otimes~B)\textbackslash{}otimes~C)\textbackslash{}ar{[}r{]}\_\{F\textbackslash{}alpha\_\{A,B,C\}\}\&F(A\textbackslash{}otimes(B\textbackslash{}otimes~C))}

\} and

\[\xymatrix{
    FA\bullet J\ar[d]_{\rho_{FA}}\ar[r]^{FA\bullet\phi}&amp;FA\bullet FI\ar[d]^{\phi_{A,I}}\\
    FA&amp;\ar[l]^{F\rho_A}F(A\otimes I)
}\] and \(\xymatrix{
    J\bullet FB\ar[d]_{\lambda_{FB}}\ar[r]^{\phi\bullet FB}&amp;FI\bullet FB\ar[d]^{\phi_{I,B}}\\
    FB&amp;\ar[l]^{F\lambda_B}F(I\otimes B)
}\) commute for every objects \(A\), \(B\) and \(C\) of \(\mathcal{C}\).
The morphisms \(f_{A,B}\) and \(f\) are called \emph{coherence maps}.

A lax monoidal functor is \emph{strong} when the coherence maps are
invertible and \emph{strict} when they are identities. \}\}

\textbackslash{}ar{[}r{]}\^{}\{\textbackslash{}theta\_A\textbackslash{}bullet\textbackslash{}theta\_B\}\&\textbackslash{}ar{[}d{]}\^{}\{g\_\{A,B\}\}GA\textbackslash{}bullet
GB\textbackslash{}\textbackslash{}

\texttt{~~~F(A\textbackslash{}tens~B)\textbackslash{}ar{[}r{]}\_\{\textbackslash{}theta\_\{A\textbackslash{}tens~B\}\}\&G(A\textbackslash{}tens~B)}

\} and \(\xymatrix{
  &amp;\ar[dl]_{f}J\ar[dr]^{g}&amp;\\
  FI\ar[rr]_{\theta_I}&amp;&amp;GI
}\) commute for every objects \(A\) and \(B\) of \(\mathcal{D}\). \}\}

\subsection{Modeling negation}\label{modeling-negation}

\subsubsection{*-autonomous categories}\label{autonomous-categories}

\subsubsection{Compact closed
categories}\label{compact-closed-categories}

\&(A\textbackslash{}tens B)\textbackslash{}tens
A\textbackslash{}ar{[}dr{]}\^{}\{\textbackslash{}varepsilon\textbackslash{}tens
A\}\textbackslash{}\textbackslash{} A\textbackslash{}tens
I\textbackslash{}ar{[}ur{]}\^{}\{A\textbackslash{}tens\textbackslash{}eta\}\&\&\&I\textbackslash{}tens
A\textbackslash{}ar{[}d{]}\^{}\{\textbackslash{}lambda\_A\}\textbackslash{}\textbackslash{}
A\textbackslash{}ar{[}u{]}\^{}\{\textbackslash{}rho\_A\^{}\{-1\}\}\textbackslash{}ar@\{=\}{[}rrr{]}\&\&\&A\textbackslash{}\textbackslash{}
\} and

\[\xymatrix{
&amp;(B\tens A)\tens B\ar[r]^{\alpha_{B,A,B}}&amp;B\tens(A\tens B)\ar[dr]^{B\tens\varepsilon}\\
I\tens B\ar[ur]^{\eta\tens B}&amp;&amp;&amp;B\tens I\ar[d]^{\rho_B}\\
B\ar[u]^{\lambda_B^{-1}}\ar@{=}[rrr]&amp;&amp;&amp;B\\
}\] commute. The object \(A\) is called a left dual of \(B\) (and
conversely \(B\) is a right dual of \(A\)). \}\}

\&I\textbackslash{}tens
B\textbackslash{}ar{[}r{]}\^{}-\{\textbackslash{}eta\_A\textbackslash{}tens
B\}\&(A\^{}*\textbackslash{}tens A)\textbackslash{}tens
B\textbackslash{}ar{[}r{]}\^{}-\{\textbackslash{}alpha\_\{A\^{}*,A,B\}\}\&A\^{}*\textbackslash{}tens(A\textbackslash{}tens
B)\textbackslash{}ar{[}r{]}\^{}-\{A\^{}*\textbackslash{}tens
f\}\&A\textbackslash{}tens C\textbackslash{}\textbackslash{} \} and to
every morphism \(g:B\to A^*\tens C\), we associate a morphism
\(\llcorner g\lrcorner:A\tens B\to C\) defined as

\[\xymatrix{
A\tens B\ar[r]^-{A\tens g}&amp;A\tens(A^*\tens C)\ar[r]^-{\alpha_{A,A^*,C}^{-1}}&amp;(A\tens A^*)\tens C\ar[r]^-{\varepsilon_A\tens C}&amp;I\tens C\ar[r]^-{\lambda_C}&amp;C
}\] It is easy to show that
\(\llcorner \ulcorner f\urcorner\lrcorner=f\) and
\(\ulcorner\llcorner g\lrcorner\urcorner=g\) from which we deduce the
required bijection. \}\}

\subsection{Other categorical models}\label{other-categorical-models}

\subsubsection{Lafont categories}\label{lafont-categories}

\subsubsection{Seely categories}\label{seely-categories}

\subsubsection{Linear categories}\label{linear-categories}

\subsection{Properties of categorical
models}\label{properties-of-categorical-models}

\subsubsection{The Kleisli category}\label{the-kleisli-category}



\section{System L}\label{system-l}

\textbf{System L} is a family of syntax for a variety of variants of
linear logic, inspired from classical calculi such as
\(\bar\lambda\mu\tilde\mu\)-calculus. These, in turn, stem from the
study of abstract machines for \(\lambda\)-calculus. In this realm,
\href{Polarized_linear_logic}{polarization} and \url{focalization} are
features that appear naturally. Positives are typically values, and
negatives pattern-matches. Contraction and weakening are implicit.

We present here a system for explicitely polarized and focalized linear
logic. Polarization classifies terms and types between positive and
negative; focalization separates values from non-values.

\subsection{Definitions}\label{definitions}

Positive types:
\(P ::= 1 \mid P_1 \otimes P_2 \mid 0 \mid P_1 \oplus P_2 \mid \shpos N \mid \oc N\)

Negative types:
\(N ::= \bot \mid N_1 \parr N_2 \mid \top \mid N_1 \with N_2 \mid \shneg P \mid \wn P\)

Positive values:
\(v^+ ::= x^+ \mid () \mid (v_1^+, v_2^+) \mid inl(v^+) \mid inr(v^+) \mid \shpos t^- \mid \mu(\wn x^+).c\)

Positive terms: \(t^+ ::= v^+ \mid \mu x^-.c\)

Negative terms:
\(t^- ::= x^- \mid \mu x^+.c \mid \mu().c \mid \mu(x^+, y^+).c \mid \mu [\cdot] \mid \mu[inl(x^+).c_1 \mid inr(y^+).c_2] \mid \mu(\shpos x^-).c \mid \wn v^+\)

Commands: \(c ::= \langle t^+ \mid t^- \rangle\)

\subsection{Typing}\label{typing}

There are as many typing sequents classes as there are terms classes.
Typing of positive values corresponds to focalized sequents, and
commands are cuts.

Positive values: sequents are of the form \(\vdash \Gamma :: v^+ : P\).

\(\LabelRule{\rulename{ax}^+}
\NulRule{\vdash x^+:P\orth :: x^+: P}
\DisplayProof\)

\(\LabelRule{1}
\NulRule{\vdash \ :: () : 1}
\DisplayProof\)

\(\AxRule{\vdash \Gamma_1 :: v_1^+ : P_1}
\AxRule{\vdash \Gamma_2 :: v_2^+ : P_2}
\LabelRule{\rulename{\otimes}}
\BinRule{\vdash\Gamma_1, \Gamma_2 :: (v_1^+, v_2^+) : P_1\otimes P_2}
\DisplayProof\)

\(\AxRule{\vdash \Gamma :: v^+ : P_1}
\LabelRule{\rulename{\oplus_1}}
\UnaRule{\vdash\Gamma :: inl(v^+) : P_1\oplus P_2}
\DisplayProof
\qquad
\AxRule{\vdash \Gamma :: v^+ : P_2}
\LabelRule{\rulename{\oplus_2}}
\UnaRule{\vdash\Gamma :: inr(v^+) : P_1\oplus P_2}
\DisplayProof\)

\(\AxRule{\vdash \Gamma \mid t^- : N}
\LabelRule{\shpos}
\UnaRule{\vdash\Gamma :: \shpos t^- : \shpos N}
\DisplayProof\)

\(\AxRule{c \vdash \wn\Gamma, x^+ : N}
\LabelRule{\oc}
\UnaRule{\vdash\wn\Gamma :: \mu(\wn x^+).c : \oc N}
\DisplayProof\)

Positive terms: sequents are of the form \(\vdash\Gamma\mid t^+:P\).

\(\AxRule{\vdash \Gamma :: v^+ : P}
\LabelRule{\rulename{foc}}
\UnaRule{\vdash\Gamma \mid v^+ : P}
\DisplayProof\)

\(\AxRule{c \vdash \Gamma, x^- : P}
\LabelRule{\rulename{\mu^-}}
\UnaRule{\vdash\Gamma \mid\mu x^-.c : P}
\DisplayProof\)

Negative terms: sequents are of the form \(\vdash\Gamma\mid t^-:N\).

\(\LabelRule{\rulename{ax}^-}
\NulRule{\vdash x^-:N\orth \mid x^-: N}
\DisplayProof\)

\(\AxRule{c\vdash \Gamma, x^+: N}
\LabelRule{\mu^+}
\UnaRule{\vdash\Gamma \mid \mu x^+.c : N}
\DisplayProof\)

\(\AxRule{c \vdash \Gamma}
\LabelRule{\bot}
\UnaRule{\vdash \Gamma \mid \mu().c : \bot}
\DisplayProof\)

\(\AxRule{c\vdash \Gamma, x^+: N_1, y^+: N_2}
\LabelRule{\rulename{\parr}}
\UnaRule{\vdash\Gamma \mid \mu(x^+, y^+).c : N_1 \parr N_2}
\DisplayProof\)

\(\LabelRule{\rulename{\top}}
\NulRule{\vdash \Gamma \mid \mu[\cdot] : \top}
\DisplayProof\)

\(\AxRule{c_1\vdash \Gamma, x^+:N_1}
\AxRule{c_2\vdash \Gamma, y^+:N_2}
\LabelRule{\rulename{\with}}
\BinRule{\vdash\Gamma \mid \mu[inl(x^+).c_1 \mid inr(y^+).c_2] : N_1 \with N_2}
\DisplayProof\)

\(\AxRule{c\vdash \Gamma, x^-: P}
\LabelRule{\shneg}
\UnaRule{\vdash\Gamma \mid \mu(\shpos x^-).c : \shneg P}
\DisplayProof\)

\(\AxRule{\vdash \Gamma :: v^+ : P}
\LabelRule{\wn}
\UnaRule{\vdash\Gamma \mid \wn v^+ : \wn P}
\DisplayProof\)

Commands:

\(\AxRule{\vdash \Gamma \mid t^+ : P}
\AxRule{\vdash \Delta \mid t^- : P\orth}
\LabelRule{\rulename{cut}}
\BinRule{\langle t^+ \mid t^-\rangle\vdash\Gamma, \Delta}
\DisplayProof\)

\(\AxRule{c \vdash \Gamma}
\LabelRule{\rulename{wkn}}
\UnaRule{c \vdash\Gamma, x^+: \wn P}
\DisplayProof\)

\(\AxRule{c \vdash \Gamma, x_1^+:\wn P, x_2^+:\wn P}
\LabelRule{\rulename{ctr}}
\UnaRule{c[x_1^+ := x^+, x_2^+ := x^+] \vdash\Gamma, x^+: \wn P}
\DisplayProof\)

\subsection{Reduction rules}\label{reduction-rules}

\(\langle v^+ \mid \mu x^+.c \rangle \rightarrow c[ x^+ := v^+]\)

\(\langle \mu x^-.c \mid t^- \rangle \rightarrow c[x^- := t^-]\)

\(\langle () \mid \mu().c \rangle \rightarrow c\)

\(\langle (v_1^+, v_2^+) \mid \mu(x^+, y^+).c \rangle \rightarrow c[x^+ := v_1^+, y^+ := v_2^+]\)

\(\langle inl(v^+) \mid \mu[inl(x^+).c_1 \mid inr(y^+).c_2] \rangle \rightarrow c_1[x^+ := v^+]\)

\(\langle inr(v^+) \mid \mu[inl(x^+).c_1 \mid inr(y^+).c_2] \rangle \rightarrow c_2[y^+ := v^+]\)

\(\langle \shpos t^- \mid \mu(\shpos x^-).c \rangle \rightarrow c[x^- := t^-]\)

\(\langle \mu(\wn x^+).c \mid \wn v^+ \rangle \rightarrow c[x^+ := v^+]\)

\subsection{References}\label{references}

\begin{itemize}
\item
\end{itemize}



\allowdisplaybreaks

\chapter{Translations of intuitionistic logic}\label{translations-of-intuitionistic-logic}

The genesis of linear logic comes with a decomposition of the
intuitionistic implication. Once linear logic properly defined, it
corresponds to a translation of intuitionistic logic into linear logic,
often called \emph{Girard's translation}. In fact Jean-Yves Girard has
defined two translations in his linear logic paper~\cite{linearlogic}. We call them the \wantedpage{call-by-name} translation and the \wantedpage{call-by-value} translation.

These translations can be extended to \hyperref[translations-of-classical-logic]{translations of classical logic} into linear logic.

\section{\texorpdfstring{Call-by-name Girard's translation \(A\imp B \mapsto \oc{A}\limp B\)}{Call-by-name Girard's translation A\textbackslash{}imp B \textbackslash{}mapsto \textbackslash{}oc\{A\}\textbackslash{}limp B}}\label{call-by-name-girards-translation-aimp-b-mapsto-ocalimp-b}

Formulas are translated as:
\begin{align*}
X^n &  =  X \\
(A\imp B)^n &  = \oc{A^n}\limp B^n \\
(A\wedge B)^n &  = A^n \with B^n \\
T^n & = \top \\
(A\vee B)^n & = \oc{A^n}\plus\oc{B^n} \\
F^n & = \zero \\
(\forall\xi A)^n & = \forall\xi A^n \\
(\exists\xi A)^n & = \exists\xi \oc{A^n}
\end{align*}

This is extended to sequents by
\((\Gamma\vdash A)^n = \oc{\Gamma^n}\vdash A^n\).

This allows one to translate the rules of intuitionistic logic into linear logic:
\begin{gather*}
\LabelRule{\rulename{ax}}
\NulRule{A\vdash A}
\DisplayProof
\qquad\mapsto\qquad
\LabelRule{\rulename{ax}}
\NulRule{A^n\vdash A^n}
\LabelRule{\oc d L}
\UnaRule{\oc{A^n}\vdash A^n}
\DisplayProof
\\[2ex]
\AxRule{\Gamma\vdash A}
\AxRule{\Delta,A\vdash B}
\LabelRule{\rulename{cut}}
\BinRule{\Gamma,\Delta\vdash B}
\DisplayProof
\qquad\mapsto\qquad
\AxRule{\oc{\Gamma^n}\vdash A^n}
\LabelRule{\oc R}
\UnaRule{\oc{\Gamma^n}\vdash \oc{A^n}}
\AxRule{\oc{\Delta^n},\oc{A^n}\vdash B^n}
\LabelRule{\rulename{cut}}
\BinRule{\oc{\Gamma^n},\oc{\Delta^n}\vdash B^n}
\DisplayProof
\\[2ex]
\AxRule{\Gamma,A,A\vdash C}
\LabelRule{c L}
\UnaRule{\Gamma,A\vdash C}
\DisplayProof
\qquad\mapsto\qquad
\AxRule{\oc{\Gamma^n},\oc{A^n},\oc{A^n}\vdash C^n}
\LabelRule{\oc c L}
\UnaRule{\oc{\Gamma^n},\oc{A^n}\vdash C^n}
\DisplayProof
\\[2ex]
\AxRule{\Gamma\vdash C}
\LabelRule{w L}
\UnaRule{\Gamma,A\vdash C}
\DisplayProof
\qquad\mapsto\qquad
\AxRule{\oc{\Gamma^n}\vdash C^n}
\LabelRule{\oc w L}
\UnaRule{\oc{\Gamma^n},\oc{A^n}\vdash C^n}
\DisplayProof
\\[2ex]
\AxRule{\Gamma,A\vdash B}
\LabelRule{\imp R}
\UnaRule{\Gamma\vdash A\imp B}
\DisplayProof
\qquad\mapsto\qquad
\AxRule{\oc{\Gamma^n},\oc{A^n}\vdash B^n}
\LabelRule{\limp R}
\UnaRule{\oc{\Gamma^n}\vdash \oc{A^n}\limp B^n}
\DisplayProof
\\[2ex]
\AxRule{\Gamma\vdash A}
\AxRule{\Delta,B\vdash C}
\LabelRule{\imp L}
\BinRule{\Gamma,\Delta,A\imp B\vdash C}
\DisplayProof
\qquad\mapsto\qquad
\AxRule{\oc{\Gamma^n}\vdash A^n}
\LabelRule{\oc R}
\UnaRule{\oc{\Gamma^n}\vdash \oc{A^n}}
\LabelRule{\rulename{ax}}
\NulRule{B^n\vdash B^n}
\LabelRule{\limp L}
\BinRule{\oc{\Gamma^n},\oc{A^n}\limp B^n\vdash B^n}
\LabelRule{\oc d L}
\UnaRule{\oc{\Gamma^n},\oc{(\oc{A^n}\limp B^n)}\vdash B^n}
\LabelRule{\oc R}
\UnaRule{\oc{\Gamma^n},\oc{(\oc{A^n}\limp B^n)}\vdash \oc{B^n}}
\AxRule{\oc{\Delta^n},\oc{B^n}\vdash C^n}
\LabelRule{\rulename{cut}}
\BinRule{\oc{\Gamma^n},\oc{\Delta^n},\oc{(\oc{A^n}\limp B^n)}\vdash C^n}
\DisplayProof
\\[2ex]
\AxRule{\Gamma\vdash A}
\AxRule{\Gamma\vdash B}
\LabelRule{\wedge R}
\BinRule{\Gamma\vdash A\wedge B}
\DisplayProof
\qquad\mapsto\qquad
\AxRule{\oc{\Gamma^n}\vdash A^n}
\AxRule{\oc{\Gamma^n}\vdash B^n}
\LabelRule{\with R}
\BinRule{\oc{\Gamma^n}\vdash A^n\with B^n}
\DisplayProof
\\[2ex]
\AxRule{\Gamma,A\vdash C}
\LabelRule{\wedge_1 L}
\UnaRule{\Gamma,A\wedge B\vdash C}
\DisplayProof
\qquad\mapsto\qquad
\LabelRule{\rulename{ax}}
\NulRule{A^n\vdash A^n}
\LabelRule{\with_1 L}
\UnaRule{A^n\with B^n\vdash A^n}
\LabelRule{\oc d L}
\UnaRule{\oc{(A^n\with B^n)}\vdash A^n}
\LabelRule{\oc R}
\UnaRule{\oc{(A^n\with B^n)}\vdash \oc{A^n}}
\AxRule{\oc{\Gamma^n},\oc{A^n}\vdash C^n}
\LabelRule{\rulename{cut}}
\BinRule{\oc{\Gamma^n},\oc{(A^n\with B^n)}\vdash C^n}
\DisplayProof
\\[2ex]
\LabelRule{T R}
\NulRule{\Gamma\vdash T}
\DisplayProof
\qquad\mapsto\qquad
\LabelRule{\top R}
\NulRule{\oc{\Gamma^n}\vdash \top}
\DisplayProof
\\[2ex]
\AxRule{\Gamma\vdash A}
\LabelRule{\vee_1 R}
\UnaRule{\Gamma\vdash A\vee B}
\DisplayProof
\qquad\mapsto\qquad
\AxRule{\oc{\Gamma^n}\vdash A^n}
\LabelRule{\oc R}
\UnaRule{\oc{\Gamma^n}\vdash \oc{A^n}}
\LabelRule{\plus_1 R}
\UnaRule{\oc{\Gamma^n}\vdash \oc{A^n}\plus\oc{B^n}}
\DisplayProof
\\[2ex]
\AxRule{\Gamma,A\vdash C}
\AxRule{\Gamma,B\vdash C}
\LabelRule{\vee L}
\BinRule{\Gamma,A\vee B\vdash C}
\DisplayProof
\qquad\mapsto\qquad
\AxRule{\oc{\Gamma^n},\oc{A^n}\vdash C^n}
\AxRule{\oc{\Gamma^n},\oc{B^n}\vdash C^n}
\LabelRule{\plus L}
\BinRule{\oc{\Gamma^n},\oc{A^n}\plus\oc{B^n}\vdash C^n}
\LabelRule{\oc d L}
\UnaRule{\oc{\Gamma^n},\oc{(\oc{A^n}\plus\oc{B^n})}\vdash C^n}
\DisplayProof
\\[2ex]
\LabelRule{F L}
\NulRule{\Gamma,F\vdash C}
\DisplayProof
\qquad\mapsto\qquad
\LabelRule{\zero L}
\NulRule{\oc{\Gamma^n},\zero\vdash C^n}
\LabelRule{\oc d L}
\UnaRule{\oc{\Gamma^n},\oc{\zero}\vdash C^n}
\DisplayProof
\\[2ex]
\AxRule{\Gamma\vdash A}
\LabelRule{\forall R}
\UnaRule{\Gamma\vdash \forall\xi A}
\DisplayProof
\qquad\mapsto\qquad
\AxRule{\oc{\Gamma^n}\vdash A^n}
\LabelRule{\forall R}
\UnaRule{\oc{\Gamma^n}\vdash \forall\xi A^n}
\DisplayProof
\\[2ex]
\AxRule{\Gamma,A[\tau/\xi]\vdash C}
\LabelRule{\forall L}
\UnaRule{\Gamma,\forall\xi A\vdash C}
\DisplayProof
\qquad\mapsto\qquad
\LabelRule{\rulename{ax}}
\NulRule{A^n[\tau^n/\xi]\vdash A^n[\tau^n/\xi]}
\LabelRule{\forall L}
\UnaRule{\forall\xi A^n\vdash A^n[\tau^n/\xi]}
\LabelRule{\oc d L}
\UnaRule{\oc{\forall\xi A^n}\vdash A^n[\tau^n/\xi]}
\LabelRule{\oc R}
\UnaRule{\oc{\forall\xi A^n}\vdash \oc{(A^n[\tau^n/\xi])}}
\AxRule{\oc{\Gamma^n},\oc{(A^n[\tau^n/\xi])}\vdash C^n}
\LabelRule{\rulename{cut}}
\BinRule{\oc{\Gamma^n},\oc{\forall\xi A^n}\vdash C^n}
\DisplayProof
\\[2ex]
\AxRule{\Gamma\vdash A[\tau/\xi]}
\LabelRule{\exists R}
\UnaRule{\Gamma\vdash \exists\xi A}
\DisplayProof
\qquad\mapsto\qquad
\AxRule{\oc{\Gamma^n}\vdash A^n[\tau^n/\xi]}
\LabelRule{\oc R}
\UnaRule{\oc{\Gamma^n}\vdash \oc{(A^n[\tau^n/\xi])}}
\LabelRule{\exists R}
\UnaRule{\oc{\Gamma^n}\vdash \exists\xi\oc{A^n}}
\DisplayProof
\\[2ex]
\AxRule{\Gamma,A\vdash C}
\LabelRule{\exists L}
\UnaRule{\Gamma,\exists\xi A\vdash C}
\DisplayProof
\qquad\mapsto\qquad
\AxRule{\oc{\Gamma^n},\oc{A^n}\vdash C^n}
\LabelRule{\exists L}
\UnaRule{\oc{\Gamma^n},\exists\xi\oc{A^n}\vdash C^n}
\LabelRule{\oc d L}
\UnaRule{\oc{\Gamma^n},\oc{\exists\xi\oc{A^n}}\vdash C^n}
\DisplayProof
\end{gather*}

\section{\texorpdfstring{Call-by-value translation \(A\imp B \mapsto \oc{(A\limp B)}\)}{Call-by-value translation A\textbackslash{}imp B \textbackslash{}mapsto \textbackslash{}oc\{(A\textbackslash{}limp B)\}}}\label{call-by-value-translation-aimp-b-mapsto-ocalimp-b}

Formulas are translated as:
\begin{align*}
X^v & = \oc{X} \\
(A\imp B)^v & = \oc{(A^v\limp B^v)} \\
(A\wedge B)^v & = \oc{(A^v \tens B^v)} \\
T^v & = \oc{\one} \\
(A\vee B)^v & = \oc{(A^v\plus B^v)} \\
F^v & = \oc{\zero} \\
(\forall\xi A)^v & = \oc{\forall\xi A^v} \\
(\exists\xi A)^v & = \oc{\exists\xi A^v}
\end{align*}

The translation of any formula starts with \(\oc\), we define
\(A^{\underline{v}}\) such that \(A^v=\oc{A^{\underline{v}}}\).

The translation of sequents is \((\Gamma\vdash A)^v = \Gamma^v\vdash A^v\).

This allows one to translate the rules of intuitionistic logic into linear logic:
\begin{gather*}
\LabelRule{\rulename{ax}}
\NulRule{A\vdash A}
\DisplayProof
\qquad\mapsto\qquad
\LabelRule{\rulename{ax}}
\NulRule{A^v\vdash A^v}
\DisplayProof
\\[2ex]
\AxRule{\Gamma\vdash A}
\AxRule{\Delta,A\vdash B}
\LabelRule{\rulename{cut}}
\BinRule{\Gamma,\Delta\vdash B}
\DisplayProof
\qquad\mapsto\qquad
\AxRule{\Gamma^v\vdash A^v}
\AxRule{\Delta^v,A^v\vdash B^v}
\LabelRule{\rulename{cut}}
\BinRule{\Gamma^v,\Delta^v\vdash B^v}
\DisplayProof
\\[2ex]
\AxRule{\Gamma,A,A\vdash C}
\LabelRule{c L}
\UnaRule{\Gamma,A\vdash C}
\DisplayProof
\qquad\mapsto\qquad
\AxRule{\Gamma^v,A^v,A^v\vdash C^v}
\LabelRule{\oc c L}
\UnaRule{\Gamma^v,A^v\vdash C^v}
\DisplayProof
\\[2ex]
\AxRule{\Gamma\vdash C}
\LabelRule{w L}
\UnaRule{\Gamma,A\vdash C}
\DisplayProof
\qquad\mapsto\qquad
\AxRule{\Gamma^v\vdash C^v}
\LabelRule{\oc w L}
\UnaRule{\Gamma^v,A^v\vdash C^v}
\DisplayProof
\\[2ex]
\AxRule{\Gamma,A\vdash B}
\LabelRule{\imp R}
\UnaRule{\Gamma\vdash A\imp B}
\DisplayProof
\qquad\mapsto\qquad
\AxRule{\Gamma^v,A^v\vdash B^v}
\LabelRule{\limp R}
\UnaRule{\Gamma^v\vdash A^v\limp B^v}
\LabelRule{\oc R}
\UnaRule{\Gamma^v\vdash \oc{(A^v\limp B^v)}}
\DisplayProof
\\[2ex]
\AxRule{\Gamma\vdash A}
\AxRule{\Delta,B\vdash C}
\LabelRule{\imp L}
\BinRule{\Gamma,\Delta,A\imp B\vdash C}
\DisplayProof
\qquad\mapsto\qquad
\AxRule{\Gamma^v\vdash A^v}
\AxRule{\Delta^v,B^v\vdash C^v}
\LabelRule{\limp L}
\BinRule{\Gamma^v,\Delta^v,A^v\limp B^v\vdash C^v}
\LabelRule{\oc d L}
\UnaRule{\Gamma^v,\Delta^v,\oc{(A^v\limp B^v)}\vdash C^v}
\DisplayProof
\\[2ex]
\AxRule{\Gamma\vdash A}
\AxRule{\Delta\vdash B}
\LabelRule{\wedge R}
\BinRule{\Gamma,\Delta\vdash A\wedge B}
\DisplayProof
\qquad\mapsto\qquad
\AxRule{\Gamma^v\vdash A^v}
\AxRule{\Delta^v\vdash B^v}
\LabelRule{\tens R}
\BinRule{\Gamma^v,\Delta^v\vdash A^v\tens B^v}
\LabelRule{\oc R}
\UnaRule{\Gamma^v,\Delta^v\vdash \oc{(A^v\tens B^v)}}
\DisplayProof
\\[2ex]
\AxRule{\Gamma,A,B\vdash C}
\LabelRule{\wedge L}
\UnaRule{\Gamma,A\wedge B\vdash C}
\DisplayProof
\qquad\mapsto\qquad
\AxRule{\Gamma^v,A^v,B^v\vdash C^v}
\LabelRule{\tens L}
\UnaRule{\Gamma^v,A^v\tens B^v\vdash C^v}
\LabelRule{\oc d L}
\UnaRule{\Gamma^v,\oc{(A^v\tens B^v)}\vdash C^v}
\DisplayProof
\\[2ex]
\LabelRule{T R}
\NulRule{{}\vdash T}
\DisplayProof
\qquad\mapsto\qquad
\LabelRule{\one R}
\NulRule{{}\vdash \one}
\LabelRule{\oc R}
\UnaRule{{}\vdash \oc{\one}}
\DisplayProof
\\[2ex]
\AxRule{\Gamma\vdash C}
\LabelRule{T L}
\UnaRule{\Gamma,T\vdash C}
\DisplayProof
\qquad\mapsto\qquad
\AxRule{\Gamma^v\vdash C^v}
\LabelRule{\one L}
\UnaRule{\Gamma^v,\one\vdash C^v}
\LabelRule{\oc d L}
\UnaRule{\Gamma^v,\oc{\one}\vdash C^v}
\DisplayProof
\\[2ex]
\AxRule{\Gamma\vdash A}
\LabelRule{\vee_1 R}
\UnaRule{\Gamma\vdash A\vee B}
\DisplayProof
\qquad\mapsto\qquad
\AxRule{\Gamma^v\vdash A^v}
\LabelRule{\plus_1 R}
\UnaRule{\Gamma^v\vdash A^v\plus B^v}
\LabelRule{\oc R}
\UnaRule{\Gamma^v\vdash \oc{(A^v\plus B^v)}}
\DisplayProof
\\[2ex]
\AxRule{\Gamma,A\vdash C}
\AxRule{\Gamma,B\vdash C}
\LabelRule{\vee L}
\BinRule{\Gamma,A\vee B\vdash C}
\DisplayProof
\qquad\mapsto\qquad
\AxRule{\Gamma^v,A^v\vdash C^v}
\AxRule{\Gamma^v,B^v\vdash C^v}
\LabelRule{\plus L}
\BinRule{\Gamma^v,A^v\plus B^v\vdash C^v}
\LabelRule{\oc d L}
\UnaRule{\Gamma^v,\oc{(A^v\plus B^v)}\vdash C^v}
\DisplayProof
\\[2ex]
\LabelRule{F L}
\NulRule{\Gamma,F\vdash C}
\DisplayProof
\qquad\mapsto\qquad
\LabelRule{\zero L}
\NulRule{\Gamma^v,\zero\vdash C^v}
\LabelRule{\oc d L}
\UnaRule{\Gamma^v,\oc{\zero}\vdash C^v}
\DisplayProof
\\[2ex]
\AxRule{\Gamma\vdash A}
\LabelRule{\forall R}
\UnaRule{\Gamma\vdash \forall\xi A}
\DisplayProof
\qquad\mapsto\qquad
\AxRule{\Gamma^v\vdash A^v}
\LabelRule{\forall R}
\UnaRule{\Gamma^v\vdash \forall\xi A^v}
\LabelRule{\oc R}
\UnaRule{\Gamma^v\vdash \oc{\forall\xi A^v}}
\DisplayProof
\\[2ex]
\AxRule{\Gamma,A[\tau/\xi]\vdash C}
\LabelRule{\forall L}
\UnaRule{\Gamma,\forall\xi A\vdash C}
\DisplayProof
\qquad\mapsto\qquad
\AxRule{\Gamma^v,A^v[\tau^{\underline{v}}/\xi]\vdash C^v}
\LabelRule{\forall L}
\UnaRule{\Gamma^v,\forall\xi A^v\vdash C^v}
\LabelRule{\oc d L}
\UnaRule{\Gamma^v,\oc{\forall\xi A^v}\vdash C^v}
\DisplayProof
\\[2ex]
\AxRule{\Gamma\vdash A[\tau/\xi]}
\LabelRule{\exists R}
\UnaRule{\Gamma\vdash \exists\xi A}
\DisplayProof
\qquad\mapsto\qquad
\AxRule{\Gamma^v\vdash A^v[\tau^{\underline{v}}/\xi]}
\LabelRule{\exists R}
\UnaRule{\Gamma^v\vdash \exists\xi A^v}
\LabelRule{\oc R}
\UnaRule{\Gamma^v\vdash \oc{\exists\xi A^v}}
\DisplayProof
\\[2ex]
\AxRule{\Gamma,A\vdash C}
\LabelRule{\exists L}
\UnaRule{\Gamma,\exists\xi A\vdash C}
\DisplayProof
\qquad\mapsto\qquad
\AxRule{\Gamma^v,A^v\vdash C^v}
\LabelRule{\exists L}
\UnaRule{\Gamma^v,\exists\xi A^v\vdash C^v}
\LabelRule{\oc d L}
\UnaRule{\Gamma^v,\oc{\exists\xi A^v}\vdash C^v}
\DisplayProof
\end{gather*}
We use \((A[\tau/\xi])^v=A^v[\tau^{\underline{v}}/\xi]\).

\subsection{Alternative presentation}\label{alternative-presentation-2}

It is also possible to define \(A^{\underline{v}}\) as the primitive construction.
\begin{align*}
X^{\underline{v}} & = X \\
(A\imp B)^{\underline{v}} & = \oc{A^{\underline{v}}}\limp\oc{B^{\underline{v}}} \\
(A\wedge B)^{\underline{v}} & = \oc{A^{\underline{v}}}\tens\oc{B^{\underline{v}}} \\
T^{\underline{v}} & = \one \\
(A\vee B)^{\underline{v}} & = \oc{A^{\underline{v}}}\plus\oc{B^{\underline{v}}} \\
F^{\underline{v}} & = \zero \\
(\forall\xi A)^{\underline{v}} & = \forall\xi \oc{A^{\underline{v}}} \\
(\exists\xi A)^{\underline{v}} & = \exists\xi \oc{A^{\underline{v}}}
\end{align*}

If we define
\((\Gamma\vdash A)^{\underline{v}} = \oc{\Gamma^{\underline{v}}}\vdash\oc{A^{\underline{v}}}\),
we have \((\Gamma\vdash A)^{\underline{v}} = (\Gamma\vdash A)^v\) and
thus we obtain the same translation of proofs.

\section{\texorpdfstring{Call-by-value Girard's translation \(A\imp B \mapsto \oc{(A\limp B)}\)}{Call-by-value Girard's translation A\textbackslash{}imp B \textbackslash{}mapsto \textbackslash{}oc\{(A\textbackslash{}limp B)\}}}\label{call-by-value-girards-translation-aimp-b-mapsto-ocalimp-b}

The original version of the call-by-value translation given by Jean-Yves
Girard~\cite{linearlogic} is an optimisation of the previous one using
properties of \hyperref[positive-formula]{positive formulas}.

Formulas are translated as:
\begin{align*}
X^w & = \oc{X} \\
(A\imp B)^w & = \oc{(A^w\limp B^w)} \\
(A\wedge B)^w & = A^w \tens B^w \\
T^w & = \one \\
(A\vee B)^w & = A^w\plus B^w \\
F^w & = \zero \\
(\forall\xi A)^w & = \oc{\forall\xi A^w} \\
(\exists\xi A)^w & = \exists\xi A^w
\end{align*}

The translation of any formula is a \hyperref[positive-formula]{positive formula}.

The translation of sequents is \((\Gamma\vdash A)^w = \Gamma^w\vdash A^w\).

This allows one to translate the rules of intuitionistic logic into linear logic:
\begin{gather*}
\LabelRule{\rulename{ax}}
\NulRule{A\vdash A}
\DisplayProof
\qquad\mapsto\qquad
\LabelRule{\rulename{ax}}
\NulRule{A^w\vdash A^w}
\DisplayProof
\\[2ex]
\AxRule{\Gamma\vdash A}
\AxRule{\Delta,A\vdash B}
\LabelRule{\rulename{cut}}
\BinRule{\Gamma,\Delta\vdash B}
\DisplayProof
\qquad\mapsto\qquad
\AxRule{\Gamma^w\vdash A^w}
\AxRule{\Delta^w,A^w\vdash B^w}
\LabelRule{\rulename{cut}}
\BinRule{\Gamma^w,\Delta^w\vdash B^w}
\DisplayProof
\\[2ex]
\AxRule{\Gamma,A,A\vdash C}
\LabelRule{c L}
\UnaRule{\Gamma,A\vdash C}
\DisplayProof
\qquad\mapsto\qquad
\AxRule{\Gamma^w,A^w,A^w\vdash C^w}
\LabelRule{+ c L}
\UnaRule{\Gamma^w,A^w\vdash C^w}
\DisplayProof
\\[2ex]
\AxRule{\Gamma\vdash C}
\LabelRule{w L}
\UnaRule{\Gamma,A\vdash C}
\DisplayProof
\qquad\mapsto\qquad
\AxRule{\Gamma^w\vdash C^w}
\LabelRule{+ w L}
\UnaRule{\Gamma^w,A^w\vdash C^w}
\DisplayProof
\\[2ex]
\AxRule{\Gamma,A\vdash B}
\LabelRule{\imp R}
\UnaRule{\Gamma\vdash A\imp B}
\DisplayProof
\qquad\mapsto\qquad
\AxRule{\Gamma^w,A^w\vdash B^w}
\LabelRule{\limp R}
\UnaRule{\Gamma^w\vdash A^w\limp B^w}
\LabelRule{+ \oc R}
\UnaRule{\Gamma^w\vdash \oc{(A^w\limp B^w)}}
\DisplayProof
\\[2ex]
\AxRule{\Gamma\vdash A}
\AxRule{\Delta,B\vdash C}
\LabelRule{\imp L}
\BinRule{\Gamma,\Delta,A\imp B\vdash C}
\DisplayProof
\qquad\mapsto\qquad
\AxRule{\Gamma^w\vdash A^w}
\AxRule{\Delta^w,B^w\vdash C^w}
\LabelRule{\limp L}
\BinRule{\Gamma^w,\Delta^w,A^w\limp B^w\vdash C^w}
\LabelRule{\oc d L}
\UnaRule{\Gamma^w,\Delta^w,\oc{(A^w\limp B^w)}\vdash C^w}
\DisplayProof
\\[2ex]
\AxRule{\Gamma\vdash A}
\AxRule{\Delta\vdash B}
\LabelRule{\wedge R}
\BinRule{\Gamma,\Delta\vdash A\wedge B}
\DisplayProof
\qquad\mapsto\qquad
\AxRule{\Gamma^w\vdash A^w}
\AxRule{\Delta^w\vdash B^w}
\LabelRule{\tens R}
\BinRule{\Gamma^w,\Delta^w\vdash A^w\tens B^w}
\DisplayProof
\\[2ex]
\AxRule{\Gamma,A,B\vdash C}
\LabelRule{\wedge L}
\UnaRule{\Gamma,A\wedge B\vdash C}
\DisplayProof
\qquad\mapsto\qquad
\AxRule{\Gamma^w,A^w,B^w\vdash C^w}
\LabelRule{\tens L}
\UnaRule{\Gamma^w,A^w\tens B^w\vdash C^w}
\DisplayProof
\\[2ex]
\LabelRule{T R}
\NulRule{{}\vdash T}
\DisplayProof
\qquad\mapsto\qquad
\LabelRule{\one R}
\NulRule{{}\vdash \one}
\DisplayProof
\\[2ex]
\AxRule{\Gamma\vdash A}
\LabelRule{\vee_1 R}
\UnaRule{\Gamma\vdash A\vee B}
\DisplayProof
\qquad\mapsto\qquad
\AxRule{\Gamma^w\vdash A^w}
\LabelRule{\plus_1 R}
\UnaRule{\Gamma^w\vdash A^w\plus B^w}
\DisplayProof
\\[2ex]
\AxRule{\Gamma,A\vdash C}
\AxRule{\Gamma,B\vdash C}
\LabelRule{\vee L}
\BinRule{\Gamma,A\vee B\vdash C}
\DisplayProof
\qquad\mapsto\qquad
\AxRule{\Gamma^w,A^w\vdash C^w}
\AxRule{\Gamma^w,B^w\vdash C^w}
\LabelRule{\plus L}
\BinRule{\Gamma^w,A^w\plus B^w\vdash C^w}
\DisplayProof
\\[2ex]
\LabelRule{F L}
\NulRule{\Gamma,F\vdash C}
\DisplayProof
\qquad\mapsto\qquad
\LabelRule{\zero L}
\NulRule{\Gamma^w,\zero\vdash C^w}
\DisplayProof
\\[2ex]
\AxRule{\Gamma\vdash A}
\LabelRule{\forall R}
\UnaRule{\Gamma\vdash \forall\xi A}
\DisplayProof
\qquad\mapsto\qquad
\AxRule{\Gamma^w\vdash A^w}
\LabelRule{\forall R}
\UnaRule{\Gamma^w\vdash \forall\xi A^w}
\LabelRule{+ \oc R}
\UnaRule{\Gamma^w\vdash \oc{\forall\xi A^w}}
\DisplayProof
\\[2ex]
\AxRule{\Gamma,A[\tau/\xi]\vdash C}
\LabelRule{\forall L}
\UnaRule{\Gamma,\forall\xi A\vdash C}
\DisplayProof
\qquad\mapsto\qquad
\AxRule{\Gamma^w,(A[\tau/\xi])^w\vdash C^w}
\UnaRule{\Gamma^w,A^w[\tau^w/\xi]\vdash C^w}
\LabelRule{\forall L}
\UnaRule{\Gamma^w,\forall\xi A^w\vdash C^w}
\LabelRule{\oc d L}
\UnaRule{\Gamma^w,\oc{\forall\xi A^w}\vdash C^w}
\DisplayProof
\\[2ex]
\AxRule{\Gamma\vdash A[\tau/\xi]}
\LabelRule{\exists R}
\UnaRule{\Gamma\vdash \exists\xi A}
\DisplayProof
\qquad\mapsto\qquad
\AxRule{\Gamma^w\vdash (A[\tau/\xi])^w}
\UnaRule{\Gamma^w\vdash A^w[\tau^w/\xi]}
\LabelRule{\exists R}
\UnaRule{\Gamma^w\vdash \exists\xi A^w}
\DisplayProof
\\[2ex]
\AxRule{\Gamma,A\vdash C}
\LabelRule{\exists L}
\UnaRule{\Gamma,\exists\xi A\vdash C}
\DisplayProof
\qquad\mapsto\qquad
\AxRule{\Gamma^w,A^w\vdash C^w}
\LabelRule{\exists L}
\UnaRule{\Gamma^w,\exists\xi A^w\vdash C^w}
\DisplayProof
\end{gather*}
We use \((A[\tau/\xi])^w\linequiv A^w[\tau^w/\xi]\).

%%% Local Variables:
%%% mode: latex
%%% TeX-master: "main"
%%% End:


\section{Translations of classical logic}\label{translations-of-classical-logic}

\subsection{\texorpdfstring{T-translation \(A\imp B \mapsto \oc{\wn{A}}\limp\wn{B}\)}{T-translation A\textbackslash{}imp B \textbackslash{}mapsto \textbackslash{}oc\{\textbackslash{}wn\{A\}\}\textbackslash{}limp\textbackslash{}wn\{B\}}}\label{t-translation-aimp-b-mapsto-ocwnalimpwnb}

Formulas are translated as:

\(\begin{array}{rcl}
X^T &amp; = &amp; X \\
(A\imp B)^T &amp; = &amp; \oc{\wn{A^T}}\limp\wn{B^T} \\
(A\wedge B)^T &amp; = &amp; \wn{A^T} \with \wn{B^T} \\
T^T &amp; = &amp; \top \\
(A\vee B)^T &amp; = &amp; \wn{A^T}\parr\wn{B^T} \\
F^T &amp; = &amp; \bot \\
(\neg A)^T &amp; = &amp; \wn{\oc{(A^T)\orth}} \\
(\forall\xi A)^T &amp; = &amp; \forall\xi \wn{A^T} \\
(\exists\xi A)^T &amp; = &amp; \exists\xi \oc{\wn{A^T}}
\end{array}\)

This is extended to sequents by
\((\Gamma\vdash\Delta)^T = \oc{\wn{\Gamma^T}}\vdash\wn{\Delta^T}\).

This allows one to translate the rules of classical logic into linear
logic:

\(\LabelRule{\rulename{ax}}
\NulRule{A\vdash A}
\DisplayProof
\qquad\mapsto\qquad
\LabelRule{\rulename{ax}}
\NulRule{\wn{A^T}\vdash\wn{A^T}}
\LabelRule{\oc L}
\UnaRule{\oc{\wn{A^T}}\vdash\wn{A^T}}
\DisplayProof\)

\(\AxRule{\Gamma\vdash A,\Delta}
\AxRule{\Gamma',A\vdash\Delta'}
\LabelRule{\rulename{cut}}
\BinRule{\Gamma,\Gamma'\vdash\Delta,\Delta'}
\DisplayProof
\qquad\mapsto\qquad
\AxRule{\oc{\wn{\Gamma^T}}\vdash\wn{A^T},\wn{\Delta^T}}
\LabelRule{\oc R}
\UnaRule{\oc{\wn{\Gamma^T}}\vdash\oc{\wn{A^T}},\wn{\Delta^T}}
\AxRule{\oc{\wn{\Gamma'^T}},\oc{\wn{A^T}}\vdash\wn{\Delta'^T}}
\LabelRule{\rulename{cut}}
\BinRule{\oc{\wn{\Gamma^T}},\oc{\wn{\Gamma'^T}}\vdash\wn{\Delta^T},\wn{\Delta'^T}}
\DisplayProof\)

\(\AxRule{\Gamma,A,A\vdash\Delta}
\LabelRule{c L}
\UnaRule{\Gamma,A\vdash\Delta}
\DisplayProof
\qquad\mapsto\qquad
\AxRule{\oc{\wn{\Gamma^T}},\oc{\wn{A^T}},\oc{\wn{A^T}}\vdash\wn{\Delta^T}}
\LabelRule{\oc c L}
\UnaRule{\oc{\wn{\Gamma^T}},\oc{\wn{A^T}}\vdash\wn{\Delta^T}}
\DisplayProof\)

\(\AxRule{\Gamma\vdash A,A,\Delta}
\LabelRule{c R}
\UnaRule{\Gamma\vdash A,\Delta}
\DisplayProof
\qquad\mapsto\qquad
\AxRule{\oc{\wn{\Gamma^T}}\vdash\wn{A^T},\wn{A^T},\wn{\Delta^T}}
\LabelRule{\wn c R}
\UnaRule{\oc{\wn{\Gamma^T}}\vdash\wn{A^T},\wn{\Delta^T}}
\DisplayProof\)

\(\AxRule{\Gamma\vdash\Delta}
\LabelRule{w L}
\UnaRule{\Gamma,A\vdash\Delta}
\DisplayProof
\qquad\mapsto\qquad
\AxRule{\oc{\wn{\Gamma^T}}\vdash\wn{\Delta^T}}
\LabelRule{\oc w L}
\UnaRule{\oc{\wn{\Gamma^T}},\oc{\wn{A^T}}\vdash\wn{\Delta^T}}
\DisplayProof\)

\(\AxRule{\Gamma\vdash\Delta}
\LabelRule{w R}
\UnaRule{\Gamma\vdash A,\Delta}
\DisplayProof
\qquad\mapsto\qquad
\AxRule{\oc{\wn{\Gamma^T}}\vdash\wn{\Delta^T}}
\LabelRule{\wn w R}
\UnaRule{\oc{\wn{\Gamma^T}}\vdash\wn{A^T},\wn{\Delta^T}}
\DisplayProof\)

\(\AxRule{\Gamma,A\vdash B,\Delta}
\LabelRule{\imp R}
\UnaRule{\Gamma\vdash A\imp B,\Delta}
\DisplayProof
\qquad\mapsto\qquad
\AxRule{\oc{\wn{\Gamma^T}},\oc{\wn{A^T}}\vdash\wn{B^T},\wn{\Delta^T}}
\LabelRule{\limp R}
\UnaRule{\oc{\wn{\Gamma^T}}\vdash \oc{\wn{A^T}}\limp\wn{B^T},\wn{\Delta^T}}
\LabelRule{\wn d R}
\UnaRule{\oc{\wn{\Gamma^T}}\vdash \wn{(\oc{\wn{A^T}}\limp\wn{B^T})},\wn{\Delta^T}}
\DisplayProof\)

\(\AxRule{\Gamma\vdash A,\Delta}
\AxRule{\Gamma',B\vdash\Delta'}
\LabelRule{\imp L}
\BinRule{\Gamma,\Gamma',A\imp B\vdash\Delta,\Delta'}
\DisplayProof
\qquad\mapsto\qquad
\AxRule{\oc{\wn{\Gamma^T}}\vdash\wn{A^T},\wn{\Delta^T}}
\LabelRule{\oc R}
\UnaRule{\oc{\wn{\Gamma^T}}\vdash\oc{\wn{A^T}},\wn{\Delta^T}}
\LabelRule{\rulename{ax}}
\NulRule{\wn{B^T}\vdash\wn{B^T}}
\LabelRule{\limp L}
\BinRule{\oc{\wn{\Gamma^T}},\oc{\wn{A^T}}\limp\wn{B^T}\vdash\wn{B^T},\wn{\Delta^T}}
\LabelRule{\wn L}
\UnaRule{\oc{\wn{\Gamma^T}},\wn{(\oc{\wn{A^T}}\limp\wn{B^T})}\vdash\wn{B^T},\wn{\Delta^T}}
\LabelRule{\oc d L}
\UnaRule{\oc{\wn{\Gamma^T}},\oc{\wn{(\oc{\wn{A^T}}\limp\wn{B^T})}}\vdash\wn{B^T},\wn{\Delta^T}}
\LabelRule{\oc R}
\UnaRule{\oc{\wn{\Gamma^T}},\oc{\wn{(\oc{\wn{A^T}}\limp\wn{B^T})}}\vdash\oc{\wn{B^T}},\wn{\Delta^T}}
\AxRule{\oc{\wn{\Gamma'^T}},\oc{\wn{B^T}}\vdash\wn{\Delta'^T}}
\LabelRule{\rulename{cut}}
\BinRule{\oc{\wn{\Gamma^T}},\oc{\wn{\Gamma'^T}},\oc{\wn{(\oc{\wn{A^T}}\limp\wn{B^T})}}\vdash\wn{\Delta^T},\wn{\Delta'^T}}
\DisplayProof\)

\(\AxRule{\Gamma\vdash A,\Delta}
\AxRule{\Gamma\vdash B,\Delta}
\LabelRule{\wedge R}
\BinRule{\Gamma\vdash A\wedge B,\Delta}
\DisplayProof
\qquad\mapsto\qquad
\AxRule{\oc{\wn{\Gamma^T}}\vdash \wn{A^T},\wn{\Delta^T}}
\AxRule{\oc{\wn{\Gamma^T}}\vdash \wn{B^T},\wn{\Delta^T}}
\LabelRule{\with R}
\BinRule{\oc{\wn{\Gamma^T}}\vdash \wn{A^T}\with \wn{B^T},\wn{\Delta^T}}
\LabelRule{\wn d R}
\UnaRule{\oc{\wn{\Gamma^T}}\vdash \wn{(\wn{A^T}\with \wn{B^T})},\wn{\Delta^T}}
\DisplayProof\)

\(\AxRule{\Gamma,A\vdash \Delta}
\LabelRule{\wedge_1 L}
\UnaRule{\Gamma,A\wedge B\vdash \Delta}
\DisplayProof
\qquad\mapsto\qquad
\LabelRule{\rulename{ax}}
\NulRule{\wn{A^T}\vdash \wn{A^T}}
\LabelRule{\with_1 L}
\UnaRule{\wn{A^T}\with \wn{B^T}\vdash \wn{A^T}}
\LabelRule{\wn L}
\UnaRule{\wn{(\wn{A^T}\with \wn{B^T})}\vdash \wn{A^T}}
\LabelRule{\oc d L}
\UnaRule{\oc{\wn{(\wn{A^T}\with \wn{B^T})}}\vdash \wn{A^T}}
\LabelRule{\oc R}
\UnaRule{\oc{\wn{(\wn{A^T}\with \wn{B^T})}}\vdash \oc{\wn{A^T}}}
\AxRule{\oc{\wn{\Gamma^T}},\oc{\wn{A^T}}\vdash \wn{\Delta^T}}
\LabelRule{\rulename{cut}}
\BinRule{\oc{\wn{\Gamma^T}},\oc{\wn{(\wn{A^T}\with \wn{B^T})}}\vdash \wn{\Delta^T}}
\DisplayProof\)

\(\LabelRule{T R}
\NulRule{\Gamma\vdash T,\Delta}
\DisplayProof
\qquad\mapsto\qquad
\LabelRule{\top R}
\NulRule{\oc{\wn{\Gamma^T}}\vdash \top,\wn{\Delta^T}}
\LabelRule{\wn d R}
\UnaRule{\oc{\wn{\Gamma^T}}\vdash \wn{\top},\wn{\Delta^T}}
\DisplayProof\)

\(\AxRule{\Gamma\vdash A,B,\Delta}
\LabelRule{\vee R}
\UnaRule{\Gamma\vdash A\vee B,\Delta}
\DisplayProof
\qquad\mapsto\qquad
\AxRule{\oc{\wn{\Gamma^T}}\vdash \wn{A^T},\wn{B^T},\wn{\Delta^T}}
\LabelRule{\parr R}
\UnaRule{\oc{\wn{\Gamma^T}}\vdash \wn{A^T}\parr\wn{B^T},\wn{\Delta^T}}
\LabelRule{\wn d R}
\UnaRule{\oc{\wn{\Gamma^T}}\vdash \wn{(\wn{A^T}\parr\wn{B^T})},\wn{\Delta^T}}
\DisplayProof\)

\(\AxRule{\Gamma,A\vdash \Delta}
\AxRule{\Gamma',B\vdash \Delta'}
\LabelRule{\vee L}
\BinRule{\Gamma,\Gamma',A\vee B\vdash \Delta,\Delta'}
\DisplayProof
\qquad\mapsto\qquad
\LabelRule{\rulename{ax}}
\NulRule{\wn{A^T}\vdash \wn{A^T}}
\LabelRule{\rulename{ax}}
\NulRule{\wn{B^T}\vdash \wn{B^T}}
\LabelRule{\parr L}
\BinRule{\wn{A^T}\parr \wn{B^T}\vdash \wn{A^T},\wn{B^T}}
\LabelRule{\wn L}
\UnaRule{\wn{(\wn{A^T}\parr \wn{B^T})}\vdash \wn{A^T},\wn{B^T}}
\LabelRule{\oc d L}
\UnaRule{\oc{\wn{(\wn{A^T}\parr \wn{B^T})}}\vdash \wn{A^T},\wn{B^T}}
\LabelRule{\oc R}
\UnaRule{\oc{\wn{(\wn{A^T}\parr \wn{B^T})}}\vdash \wn{A^T},\oc{\wn{B^T}}}
\AxRule{\oc{\wn{\Gamma'^T}},\oc{\wn{B^T}}\vdash \wn{\Delta'^T}}
\LabelRule{\rulename{cut}}
\BinRule{\oc{\wn{\Gamma'^T}},\oc{\wn{(\wn{A^T}\parr \wn{B^T})}}\vdash \wn{A^T},\wn{\Delta'^T}}
\LabelRule{\oc R}
\UnaRule{\oc{\wn{\Gamma'^T}},\oc{\wn{(\wn{A^T}\parr \wn{B^T})}}\vdash \oc{\wn{A^T}},\wn{\Delta'^T}}
\AxRule{\oc{\wn{\Gamma^T}},\oc{\wn{A^T}}\vdash \wn{\Delta^T}}
\LabelRule{\rulename{cut}}
\BinRule{\oc{\wn{\Gamma^T}},\oc{\wn{\Gamma'^T}},\oc{\wn{(\wn{A^T}\parr \wn{B^T})}}\vdash \wn{\Delta^T},\wn{\Delta'^T}}
\DisplayProof\)

\(\AxRule{\Gamma\vdash \Delta}
\LabelRule{F R}
\UnaRule{\Gamma\vdash F,\Delta}
\DisplayProof
\qquad\mapsto\qquad
\AxRule{\oc{\wn{\Gamma^T}}\vdash \wn{\Delta^T}}
\LabelRule{\bot R}
\UnaRule{\oc{\wn{\Gamma^T}}\vdash \bot,\wn{\Delta^T}}
\LabelRule{\wn R}
\UnaRule{\oc{\wn{\Gamma^T}}\vdash \wn{\bot},\wn{\Delta^T}}
\DisplayProof\)

\(\LabelRule{F L}
\NulRule{F\vdash {}}
\DisplayProof
\qquad\mapsto\qquad
\LabelRule{\bot L}
\NulRule{\bot\vdash {}}
\LabelRule{\wn L}
\UnaRule{\wn{\bot}\vdash {}}
\LabelRule{\oc d L}
\UnaRule{\oc{\wn{\bot}}\vdash {}}
\DisplayProof\)

\(\AxRule{\Gamma,A\vdash \Delta}
\LabelRule{\neg R}
\UnaRule{\Gamma\vdash \neg A,\Delta}
\DisplayProof
\qquad\mapsto\qquad
\AxRule{\oc{\wn{\Gamma^T}},\oc{\wn{A^T}}\vdash \wn{\Delta^T}}
\LabelRule{(.)\orth R}
\UnaRule{\oc{\wn{\Gamma^T}}\vdash \wn{\oc{(A^T)\orth}},\wn{\Delta^T}}
\LabelRule{\wn d R}
\UnaRule{\oc{\wn{\Gamma^T}}\vdash \wn{\wn{\oc{(A^T)\orth}}},\wn{\Delta^T}}
\DisplayProof\)

\(\AxRule{\Gamma\vdash A,\Delta}
\LabelRule{\neg L}
\UnaRule{\Gamma,\neg A\vdash \Delta}
\DisplayProof
\qquad\mapsto\qquad
\AxRule{\oc{\wn{\Gamma^T}}\vdash \wn{A^T},\wn{\Delta^T}}
\LabelRule{(.)\orth L}
\UnaRule{\oc{\wn{\Gamma^T}},\oc{(A^T)\orth}\vdash \wn{\Delta^T}}
\LabelRule{\wn L}
\UnaRule{\oc{\wn{\Gamma^T}},\wn{\oc{(A^T)\orth}}\vdash \wn{\Delta^T}}
\LabelRule{\wn L}
\UnaRule{\oc{\wn{\Gamma^T}},\wn{\wn{\oc{(A^T)\orth}}}\vdash \wn{\Delta^T}}
\LabelRule{\oc d L}
\UnaRule{\oc{\wn{\Gamma^T}},\oc{\wn{\wn{\oc{(A^T)\orth}}}}\vdash \wn{\Delta^T}}
\DisplayProof\)

\(\AxRule{\Gamma\vdash A,\Delta}
\LabelRule{\forall R}
\UnaRule{\Gamma\vdash \forall\xi A,\Delta}
\DisplayProof
\qquad\mapsto\qquad
\AxRule{\oc{\wn{\Gamma^T}}\vdash \wn{A^T},\wn{\Delta^T}}
\LabelRule{\forall R}
\UnaRule{\oc{\wn{\Gamma^T}}\vdash \forall\xi \wn{A^T},\wn{\Delta^T}}
\LabelRule{\wn d R}
\UnaRule{\oc{\wn{\Gamma^T}}\vdash \wn{\forall\xi \wn{A^T}},\wn{\Delta^T}}
\DisplayProof\)

\(\AxRule{\Gamma,A[\tau/\xi]\vdash \Delta}
\LabelRule{\forall L}
\UnaRule{\Gamma,\forall\xi A\vdash \Delta}
\DisplayProof
\qquad\mapsto\qquad
\LabelRule{\rulename{ax}}
\NulRule{\wn{A^T}[\tau^T/\xi]\vdash \wn{A^T}[\tau^T/\xi]}
\LabelRule{\forall L}
\UnaRule{\forall\xi \wn{A^T}\vdash \wn{A^T}[\tau^T/\xi]}
\LabelRule{\wn L}
\UnaRule{\wn{\forall\xi \wn{A^T}}\vdash \wn{A^T}[\tau^T/\xi]}
\LabelRule{\oc d L}
\UnaRule{\oc{\wn{\forall\xi \wn{A^T}}}\vdash \wn{A^T}[\tau^T/\xi]}
\LabelRule{\oc R}
\UnaRule{\oc{\wn{\forall\xi \wn{A^T}}}\vdash \oc{\wn{A^T}}[\tau^T/\xi]}
\AxRule{\oc{\wn{\Gamma^T}},\oc{\wn{(A^T[\tau^T/\xi])}}\vdash \wn{\Delta^T}}
\LabelRule{\rulename{cut}}
\BinRule{\oc{\wn{\Gamma^T}},\oc{\wn{\forall\xi \wn{A^T}}}\vdash \wn{\Delta^T}}
\DisplayProof\)

\(\AxRule{\Gamma\vdash A[\tau/\xi],\Delta}
\LabelRule{\exists R}
\UnaRule{\Gamma\vdash \exists\xi A,\Delta}
\DisplayProof
\qquad\mapsto\qquad
\AxRule{\oc{\wn{\Gamma^T}}\vdash \wn{A^T[\tau^T/\xi]},\wn{\Delta^T}}
\LabelRule{\oc R}
\UnaRule{\oc{\wn{\Gamma^T}}\vdash \oc{\wn{A^T[\tau^T/\xi]}},\wn{\Delta^T}}
\LabelRule{\exists R}
\UnaRule{\oc{\wn{\Gamma^T}}\vdash \exists\xi \oc{\wn{A^T}},\wn{\Delta^T}}
\LabelRule{\wn d R}
\UnaRule{\oc{\wn{\Gamma^T}}\vdash \wn{\exists\xi \oc{\wn{A^T}}},\wn{\Delta^T}}
\DisplayProof\)

\(\AxRule{\Gamma,A\vdash \Delta}
\LabelRule{\exists L}
\UnaRule{\Gamma,\exists\xi A\vdash \Delta}
\DisplayProof
\qquad\mapsto\qquad
\AxRule{\oc{\wn{\Gamma^T}},\oc{\wn{A^T}}\vdash \wn{\Delta^T}}
\LabelRule{\exists L}
\UnaRule{\oc{\wn{\Gamma^T}},\exists\xi\oc{\wn{A^T}}\vdash \wn{\Delta^T}}
\LabelRule{\wn L}
\UnaRule{\oc{\wn{\Gamma^T}},\wn{\exists\xi\oc{\wn{A^T}}}\vdash \wn{\Delta^T}}
\LabelRule{\oc d L}
\UnaRule{\oc{\wn{\Gamma^T}},\oc{\wn{\exists\xi\oc{\wn{A^T}}}}\vdash \wn{\Delta^T}}
\DisplayProof\)

\subsubsection{Alternative presentation}\label{alternative-presentation}

It is also possible to define \(A^{\overline{T}}\) by:

\(\begin{array}{rcl}
X^{\overline{T}} &amp; = &amp; \wn{X} \\
(A\imp B)^{\overline{T}} &amp; = &amp; \wn{(\oc{A^{\overline{T}}}\limp B^{\overline{T}})} \\
(A\wedge B)^{\overline{T}} &amp; = &amp; \wn{(A^{\overline{T}} \with B^{\overline{T}})} \\
T^{\overline{T}} &amp; = &amp; \wn{\top} \\
(A\vee B)^{\overline{T}} &amp; = &amp; \wn{(A^{\overline{T}}\parr B^{\overline{T}})} \\
F^{\overline{T}} &amp; = &amp; \wn{\bot} \\
(\neg A)^{\overline{T}} &amp; = &amp; \wn{\wn{(A^{\overline{T}})\orth}} \\
(\forall\xi A)^{\overline{T}} &amp; = &amp; \wn{\forall\xi A^{\overline{T}}} \\
(\exists\xi A)^{\overline{T}} &amp; = &amp; \wn{\exists\xi \oc{A^{\overline{T}}}}
\end{array}\)

If we define
\((\Gamma\vdash\Delta)^{\overline{T}} = \oc{\Gamma^{\overline{T}}}\vdash \Delta^{\overline{T}}\),
we have \((\Gamma\vdash\Delta)^{\overline{T}} = (\Gamma\vdash\Delta)^T\)
and thus we obtain the same translation of proofs.

\subsection{\texorpdfstring{Q-translation \(A\imp B \mapsto \oc{(A\limp\wn{B})}\)}{Q-translation A\textbackslash{}imp B \textbackslash{}mapsto \textbackslash{}oc\{(A\textbackslash{}limp\textbackslash{}wn\{B\})\}}}\label{q-translation-aimp-b-mapsto-ocalimpwnb}

Formulas are translated as:

\(\begin{array}{rcl}
X^Q &amp; = &amp; \oc{X} \\
(A\imp B)^Q &amp; = &amp; \oc{(A^Q\limp\wn{B^Q})} \\
(A\wedge B)^Q &amp; = &amp; \oc{(A^Q \tens B^Q)} \\
T^Q &amp; = &amp; \oc{\one} \\
(A\vee B)^Q &amp; = &amp; \oc{(A^Q\plus B^Q)} \\
F^Q &amp; = &amp; \oc{\zero} \\
(\neg A)^Q &amp; = &amp; \oc{(A^Q)\orth} \\
(\forall\xi A)^Q &amp; = &amp; \oc{\forall\xi \wn{A^Q}} \\
(\exists\xi A)^Q &amp; = &amp; \oc{\exists\xi A^Q}
\end{array}\)

The translation of any formula starts with \(\oc\), we define
\(A^{\underline{Q}}\) such that \(A^Q=\oc{A^{\underline{Q}}}\).

The translation of sequents is
\((\Gamma\vdash\Delta)^Q = \Gamma^Q\vdash\wn{\Delta^Q}\).

This allows one to translate the rules of classical logic into linear
logic:

\(\LabelRule{\rulename{ax}}
\NulRule{A\vdash A}
\DisplayProof
\qquad\mapsto\qquad
\LabelRule{\rulename{ax}}
\NulRule{A^Q\vdash A^Q}
\LabelRule{\wn d R}
\UnaRule{A^Q\vdash \wn{A^Q}}
\DisplayProof\)

\(\AxRule{\Gamma\vdash A,\Delta}
\AxRule{\Gamma',A\vdash \Delta'}
\LabelRule{\rulename{cut}}
\BinRule{\Gamma,\Gamma'\vdash \Delta,\Delta'}
\DisplayProof
\qquad\mapsto\qquad
\AxRule{\Gamma^Q\vdash \wn{A^Q},\wn{\Delta^Q}}
\AxRule{\Gamma'^Q,A^Q\vdash \wn{\Delta'^Q}}
\LabelRule{\wn L}
\UnaRule{\Gamma'^Q,\wn{A^Q}\vdash \wn{\Delta'^Q}}
\LabelRule{\rulename{cut}}
\BinRule{\Gamma^Q,\Gamma'^Q\vdash \wn{\Delta^Q},\wn{\Delta'^Q}}
\DisplayProof\)

\(\AxRule{\Gamma,A,A\vdash \Delta}
\LabelRule{c L}
\UnaRule{\Gamma,A\vdash \Delta}
\DisplayProof
\qquad\mapsto\qquad
\AxRule{\Gamma^Q,A^Q,A^Q\vdash \wn{\Delta^Q}}
\LabelRule{\oc c L}
\UnaRule{\Gamma^Q,A^Q\vdash \wn{\Delta^Q}}
\DisplayProof\)

\(\AxRule{\Gamma\vdash A,A,\Delta}
\LabelRule{c R}
\UnaRule{\Gamma\vdash A,\Delta}
\DisplayProof
\qquad\mapsto\qquad
\AxRule{\Gamma^Q\vdash \wn{A^Q},\wn{A^Q},\wn{\Delta^Q}}
\LabelRule{\wn c R}
\UnaRule{\Gamma^Q\vdash \wn{A^Q},\wn{\Delta^Q}}
\DisplayProof\)

\(\AxRule{\Gamma\vdash \Delta}
\LabelRule{w L}
\UnaRule{\Gamma,A\vdash \Delta}
\DisplayProof
\qquad\mapsto\qquad
\AxRule{\Gamma^Q\vdash \wn{\Delta^Q}}
\LabelRule{\oc w L}
\UnaRule{\Gamma^Q,A^Q\vdash \wn{\Delta^Q}}
\DisplayProof\)

\(\AxRule{\Gamma\vdash \Delta}
\LabelRule{w R}
\UnaRule{\Gamma\vdash A,\Delta}
\DisplayProof
\qquad\mapsto\qquad
\AxRule{\Gamma^Q\vdash \wn{\Delta^Q}}
\LabelRule{\wn w R}
\UnaRule{\Gamma^Q\vdash \wn{A^Q},\wn{\Delta^Q}}
\DisplayProof\)

\(\AxRule{\Gamma,A\vdash B,\Delta}
\LabelRule{\imp R}
\UnaRule{\Gamma\vdash A\imp B,\Delta}
\DisplayProof
\qquad\mapsto\qquad
\AxRule{\Gamma^Q,A^Q\vdash \wn{B^Q},\wn{\Delta^Q}}
\LabelRule{\limp R}
\UnaRule{\Gamma^Q\vdash A^Q\limp \wn{B^Q},\wn{\Delta^Q}}
\LabelRule{\oc R}
\UnaRule{\Gamma^Q\vdash \oc{(A^Q\limp \wn{B^Q})},\wn{\Delta^Q}}
\LabelRule{\wn d R}
\UnaRule{\Gamma^Q\vdash \wn{\oc{(A^Q\limp \wn{B^Q})}},\wn{\Delta^Q}}
\DisplayProof\)

\(\AxRule{\Gamma\vdash A,\Delta}
\AxRule{\Gamma',B\vdash \Delta'}
\LabelRule{\imp L}
\BinRule{\Gamma,\Gamma',A\imp B\vdash \Delta,\Delta'}
\DisplayProof
\qquad\mapsto\qquad
\AxRule{\Gamma^Q\vdash \wn{A^Q},\wn{\Delta^Q}}
\LabelRule{\rulename{ax}}
\NulRule{A^Q\vdash A^Q}
\AxRule{\Gamma'^Q,B^Q\vdash \wn{\Delta'^Q}}
\LabelRule{\wn L}
\UnaRule{\Gamma'^Q,\wn{B^Q}\vdash \wn{\Delta'^Q}}
\LabelRule{\limp L}
\BinRule{\Gamma'^Q,A^Q\limp \wn{B^Q},A^Q\vdash \wn{\Delta'^Q}}
\LabelRule{\oc d L}
\UnaRule{\Gamma'^Q,\oc{(A^Q\limp \wn{B^Q})},A^Q\vdash \wn{\Delta'^Q}}
\LabelRule{\wn L}
\UnaRule{\Gamma'^Q,\oc{(A^Q\limp \wn{B^Q})},\wn{A^Q}\vdash \wn{\Delta'^Q}}
\LabelRule{\rulename{cut}}
\BinRule{\Gamma^Q,\Gamma'^Q,\oc{(A^Q\limp \wn{B^Q})}\vdash \wn{\Delta^Q},\wn{\Delta'^Q}}
\DisplayProof\)

\(\AxRule{\Gamma\vdash A,\Delta}
\AxRule{\Gamma'\vdash B,\Delta'}
\LabelRule{\wedge R}
\BinRule{\Gamma,\Gamma'\vdash A\wedge B,\Delta,\Delta'}
\DisplayProof
\qquad\mapsto\qquad
\AxRule{\Gamma^Q\vdash \wn{A^Q},\wn{\Delta^Q}}
\AxRule{\Gamma'^Q\vdash \wn{B^Q},\wn{\Delta'^Q}}
\LabelRule{\rulename{ax}}
\NulRule{A^Q\vdash A^Q}
\LabelRule{\rulename{ax}}
\NulRule{B^Q\vdash B^Q}
\LabelRule{\tens R}
\BinRule{A^Q,B^Q\vdash A^Q\tens B^Q}
\LabelRule{\oc R}
\UnaRule{A^Q,B^Q\vdash \oc{(A^Q\tens B^Q)}}
\LabelRule{\wn d R}
\UnaRule{A^Q,B^Q\vdash \wn{\oc{(A^Q\tens B^Q)}}}
\LabelRule{\wn L}
\UnaRule{A^Q,\wn{B^Q}\vdash \wn{\oc{(A^Q\tens B^Q)}}}
\LabelRule{\rulename{cut}}
\BinRule{\Gamma'^Q,A^Q\vdash \wn{\oc{(A^Q\tens B^Q)}},\wn{\Delta'^Q}}
\LabelRule{\wn L}
\UnaRule{\Gamma'^Q,\wn{A^Q}\vdash \wn{\oc{(A^Q\tens B^Q)}},\wn{\Delta'^Q}}
\LabelRule{\rulename{cut}}
\BinRule{\Gamma^Q,\Gamma'^Q\vdash \wn{\oc{(A^Q\tens B^Q)}},\wn{\Delta^Q},\wn{\Delta'^Q}}
\DisplayProof\)

\(\AxRule{\Gamma,A,B\vdash \Delta}
\LabelRule{\wedge L}
\UnaRule{\Gamma,A\wedge B\vdash \Delta}
\DisplayProof
\qquad\mapsto\qquad
\AxRule{\Gamma^Q,A^Q,B^Q\vdash \wn{\Delta^Q}}
\LabelRule{\tens L}
\UnaRule{\Gamma^Q,A^Q\tens B^Q\vdash \wn{\Delta^Q}}
\LabelRule{\oc d L}
\UnaRule{\Gamma^Q,\oc{(A^Q\tens B^Q)}\vdash \wn{\Delta^Q}}
\DisplayProof\)

\(\LabelRule{T R}
\NulRule{{}\vdash T}
\DisplayProof
\qquad\mapsto\qquad
\LabelRule{\one R}
\NulRule{{}\vdash \one}
\LabelRule{\oc R}
\UnaRule{{}\vdash \oc{\one}}
\LabelRule{\wn d R}
\UnaRule{{}\vdash \wn{\oc{\one}}}
\DisplayProof\)

\(\AxRule{\Gamma\vdash \Delta}
\LabelRule{T L}
\UnaRule{\Gamma,T\vdash \Delta}
\DisplayProof
\qquad\mapsto\qquad
\AxRule{\Gamma^Q\vdash \wn{\Delta^Q}}
\LabelRule{\one L}
\UnaRule{\Gamma^Q,\one\vdash \wn{\Delta^Q}}
\LabelRule{\oc d L}
\UnaRule{\Gamma^Q,\oc{\one}\vdash \wn{\Delta^Q}}
\DisplayProof\)

\(\AxRule{\Gamma\vdash A,\Delta}
\LabelRule{\vee_1 R}
\UnaRule{\Gamma\vdash A\vee B,\Delta}
\DisplayProof
\qquad\mapsto\qquad
\AxRule{\Gamma^Q\vdash \wn{A^Q},\wn{\Delta^Q}}
\LabelRule{\rulename{ax}}
\NulRule{A^Q\vdash A^Q}
\LabelRule{\plus_1 R}
\UnaRule{A^Q\vdash A^Q\plus B^Q}
\LabelRule{\oc R}
\UnaRule{A^Q\vdash \oc{(A^Q\plus B^Q)}}
\LabelRule{\wn d R}
\UnaRule{A^Q\vdash \wn{\oc{(A^Q\plus B^Q)}}}
\LabelRule{\wn L}
\UnaRule{\wn{A^Q}\vdash \wn{\oc{(A^Q\plus B^Q)}}}
\LabelRule{\rulename{cut}}
\BinRule{\Gamma^Q\vdash \wn{\oc{(A^Q\plus B^Q)}},\wn{\Delta^Q}}
\DisplayProof\)

\(\AxRule{\Gamma,A\vdash \Delta}
\AxRule{\Gamma,B\vdash \Delta}
\LabelRule{\vee L}
\BinRule{\Gamma,A\vee B\vdash \Delta}
\DisplayProof
\qquad\mapsto\qquad
\AxRule{\Gamma^Q,A^Q\vdash \wn{\Delta^Q}}
\AxRule{\Gamma^Q,B^Q\vdash \wn{\Delta^Q}}
\LabelRule{\plus L}
\BinRule{\Gamma^Q,A^Q\plus B^Q\vdash \wn{\Delta^Q}}
\LabelRule{\oc d L}
\UnaRule{\Gamma^Q,\oc{(A^Q\plus B^Q)}\vdash \wn{\Delta^Q}}
\DisplayProof\)

\(\LabelRule{F L}
\NulRule{\Gamma,F\vdash \Delta}
\DisplayProof
\qquad\mapsto\qquad
\LabelRule{\zero L}
\NulRule{\Gamma^Q,\zero\vdash \wn{\Delta^Q}}
\LabelRule{\oc d L}
\UnaRule{\Gamma^Q,\oc{\zero}\vdash \wn{\Delta^Q}}
\DisplayProof\)

\(\AxRule{\Gamma,A\vdash \Delta}
\LabelRule{\neg R}
\UnaRule{\Gamma\vdash \neg A,\Delta}
\DisplayProof
\qquad\mapsto\qquad
\AxRule{\Gamma^Q,A^Q\vdash \wn{\Delta^Q}}
\LabelRule{(.)\orth R}
\UnaRule{\Gamma^Q\vdash (A^Q)\orth,\wn{\Delta^Q}}
\LabelRule{\oc R}
\UnaRule{\Gamma^Q\vdash \oc{(A^Q)\orth},\wn{\Delta^Q}}
\LabelRule{\wn d R}
\UnaRule{\Gamma^Q\vdash \wn{\oc{(A^Q)\orth}},\wn{\Delta^Q}}
\DisplayProof\)

\(\AxRule{\Gamma\vdash A,\Delta}
\LabelRule{\neg L}
\UnaRule{\Gamma,\neg A\vdash \Delta}
\DisplayProof
\qquad\mapsto\qquad
\AxRule{\Gamma^Q\vdash \wn{A^Q},\wn{\Delta^Q}}
\LabelRule{(.)\orth L}
\UnaRule{\Gamma^Q,\oc{(A^Q)\orth}\vdash \wn{\Delta^Q}}
\DisplayProof\)

\(\AxRule{\Gamma\vdash A,\Delta}
\LabelRule{\forall R}
\UnaRule{\Gamma\vdash \forall\xi A,\Delta}
\DisplayProof
\qquad\mapsto\qquad
\AxRule{\Gamma^Q\vdash \wn{A^Q},\wn{\Delta^Q}}
\LabelRule{\forall R}
\UnaRule{\Gamma^Q\vdash \forall\xi \wn{A^Q},\wn{\Delta^Q}}
\LabelRule{\oc R}
\UnaRule{\Gamma^Q\vdash \oc{\forall\xi \wn{A^Q}},\wn{\Delta^Q}}
\DisplayProof\)

We use \((A[\tau/\xi])^Q=A^Q[\tau^{\underline{Q}}/\xi]\).

\(\AxRule{\Gamma,A[\tau/\xi]\vdash \Delta}
\LabelRule{\forall L}
\UnaRule{\Gamma,\forall\xi A\vdash \Delta}
\DisplayProof
\qquad\mapsto\qquad
\AxRule{\Gamma^Q,A^Q[\tau^{\underline{Q}}/\xi]\vdash \wn{\Delta^Q}}
\LabelRule{\wn L}
\UnaRule{\Gamma^Q,\wn{A^Q[\tau^{\underline{Q}}/\xi]}\vdash \wn{\Delta^Q}}
\LabelRule{\forall L}
\UnaRule{\Gamma^Q,\forall\xi \wn{A^Q}\vdash \wn{\Delta^Q}}
\LabelRule{\oc d L}
\UnaRule{\Gamma^Q,\oc{\forall\xi \wn{A^Q}}\vdash \wn{\Delta^Q}}
\DisplayProof\)

\(\AxRule{\Gamma\vdash A[\tau/\xi],\Delta}
\LabelRule{\exists R}
\UnaRule{\Gamma\vdash \exists\xi A,\Delta}
\DisplayProof
\qquad\mapsto\qquad
\AxRule{\Gamma^Q\vdash \wn{A^Q[\tau^{\underline{Q}}/\xi]},\wn{\Delta^Q}}
\LabelRule{\rulename{ax}}
\NulRule{A^Q[\tau^{\underline{Q}}/\xi]\vdash A^Q[\tau^{\underline{Q}}/\xi]}
\LabelRule{\exists R}
\UnaRule{A^Q[\tau^{\underline{Q}}/\xi]\vdash \exists\xi A^Q}
\LabelRule{\oc R}
\UnaRule{A^Q[\tau^{\underline{Q}}/\xi]\vdash \oc{\exists\xi A^Q}}
\LabelRule{\wn d R}
\UnaRule{A^Q[\tau^{\underline{Q}}/\xi]\vdash \wn{\oc{\exists\xi A^Q}}}
\LabelRule{\wn L}
\UnaRule{\wn{A^Q[\tau^{\underline{Q}}/\xi]}\vdash \wn{\oc{\exists\xi A^Q}}}
\LabelRule{\rulename{cut}}
\BinRule{\Gamma^Q\vdash \wn{\oc{\exists\xi A^Q}},\wn{\Delta^Q}}
\DisplayProof\)

\(\AxRule{\Gamma,A\vdash \Delta}
\LabelRule{\exists L}
\UnaRule{\Gamma,\exists\xi A\vdash \Delta}
\DisplayProof
\qquad\mapsto\qquad
\AxRule{\Gamma^Q,A^Q\vdash \wn{\Delta^Q}}
\LabelRule{\exists L}
\UnaRule{\Gamma^Q,\exists\xi A^Q\vdash \wn{\Delta^Q}}
\LabelRule{\oc d L}
\UnaRule{\Gamma^Q,\oc{\exists\xi A^Q}\vdash \wn{\Delta^Q}}
\DisplayProof\)

\subsubsection{Alternative presentation}\label{alternative-presentation-1}

It is also possible to define \(A^{\underline{Q}}\) as the primitive
construction.

\(\begin{array}{rcl}
X^{\underline{Q}} &amp; = &amp; X \\
(A\imp B)^{\underline{Q}} &amp; = &amp; \oc{A^{\underline{Q}}}\limp\wn{\oc{B^{\underline{Q}}}} \\
(A\wedge B)^{\underline{Q}} &amp; = &amp; \oc{A^{\underline{Q}}}\tens\oc{B^{\underline{Q}}} \\
T^{\underline{Q}} &amp; = &amp; \one \\
(A\vee B)^{\underline{Q}} &amp; = &amp; \oc{A^{\underline{Q}}}\plus\oc{B^{\underline{Q}}} \\
F^{\underline{Q}} &amp; = &amp; \zero \\
(\neg A)^{\underline{Q}} &amp; = &amp; \wn{(A^{\underline{Q}})\orth} \\
(\forall\xi A)^{\underline{Q}} &amp; = &amp; \forall\xi \wn{\oc{A^{\underline{Q}}}} \\
(\exists\xi A)^{\underline{Q}} &amp; = &amp; \exists\xi \oc{A^{\underline{Q}}}
\end{array}\)

If we define
\((\Gamma\vdash\Delta)^{\underline{Q}} = \oc{\Gamma^{\underline{Q}}}\vdash\wn{\oc{\Delta^{\underline{Q}}}}\),
we have
\((\Gamma\vdash\Delta)^{\underline{Q}} = (\Gamma\vdash\Delta)^Q\) and
thus we obtain the same translation of proofs.



\chapter{Light linear logics}\label{light-linear-logics}

Light linear logics are variants of linear logic characterizing
complexity classes. They are designed by defining alternative
exponential connectives, which induce a complexity bound on the
cut-elimination procedure.

Light linear logics are one of the approaches used in \emph{implicit
computational complexity}, the area studying the computational
complexity of programs without referring to external measuring
conditions or particular machine models.

\section{Elementary linear logic}\label{elementary-linear-logic}

We present here the intuitionistic version of \emph{elementary linear
logic}, ELL. Moreover we restrict to the fragment without additive
connectives.

The language of formulas is the same one as that of (multiplicative)
\hyperref[intuitionistic-linear-logic]{ILL}:
\begin{equation*}
A ::= X \mid A\tens A \mid A\limp A  \mid \oc{A} \mid \forall X A
\end{equation*}
The sequent calculus rules are the same ones as for
\hyperref[intuitionistic-linear-logic]{ILL}, except for the rules dealing
with the exponential connectives:
\begin{equation*}
\AxRule{\Gamma\vdash A}
\LabelRule{\oc\rulename{mf} }
\UnaRule{\oc{\Gamma}\vdash\oc{A}}
\DisplayProof
\qquad
\AxRule{\Gamma,\oc{A},\oc{A}\vdash C}
\LabelRule{\oc c L}
\UnaRule{\Gamma,\oc{A}\vdash C}
\DisplayProof
\qquad
\AxRule{\Gamma\vdash C}
\LabelRule{\oc w L}
\UnaRule{\Gamma,\oc{A}\vdash C}
\DisplayProof
\end{equation*}

The \emph{depth} of a derivation \(\pi\) is the maximum number of
\((\oc\rulename{mf})\) rules in a branch of \(\pi\).

We consider the function \(K(.,.)\) defined by:
\begin{equation*}
\begin{cases}
K(0,n) &=n,\\
K(k+1,n) &=2^{K(k,n)}.
\end{cases}
\end{equation*}

\begin{theorem}
If $\pi$ is an ELL proof of depth d, and R is the corresponding ELL proof-net, then R can be reduced to its normal form by cut elimination in at most $ K(d+1,|\pi|)$ steps, where $|\pi|$ is the size of $\pi$.
\end{theorem}

A function $f$ on integers is \emph{elementary recursive} if there exists
an integer $h$ and a Turing machine which computes $f$ in time bounded by
\(K(h,n)\), where $n$ is the size of the input.

\begin{theorem}
The functions representable in  ELL are exactly the elementary recursive functions.
\end{theorem}

One also often considers the \emph{affine} variant of ELL, called
\emph{elementary affine logic} EAL, which is defined by adding
unrestricted weakening:
\begin{prooftree}
\AxRule{\Gamma\vdash C}
\LabelRule{ w L}
\UnaRule{\Gamma,A\vdash C}
\end{prooftree}
It enjoys the same properties as ELL.

Elementary linear logic was introduced together with light linear logic~\cite{lightlinearlogic}.

\section{Light linear logic}\label{light-linear-logic}

We present the intuitionistic version of \emph{light linear logic} LLL,
without additive connectives. The language of formulas is:
\begin{equation*}
A ::= X \mid A\tens A \mid A\limp A  \mid \oc{A} \mid \pg{A} \mid \forall X A
\end{equation*}
The sequent calculus rules are the same ones as for ILL, except for the
rules dealing with the exponential connectives:
\begin{equation*}
\AxRule{\Gamma\vdash A}
\LabelRule{\oc\rulename{f} }
\UnaRule{\oc{\Gamma}\vdash\oc{A}}
\DisplayProof
\qquad
\AxRule{\Gamma, \Delta\vdash A}
\LabelRule{\pg }
\UnaRule{\oc{\Gamma}, \pg \Delta\vdash\pg{A}}
\DisplayProof
\qquad
\AxRule{\Gamma,\oc{A},\oc{A}\vdash C}
\LabelRule{\oc c L}
\UnaRule{\Gamma,\oc{A}\vdash C}
\DisplayProof
\qquad
\AxRule{\Gamma\vdash C}
\LabelRule{\oc w L}
\UnaRule{\Gamma,\oc{A}\vdash C}
\DisplayProof
\end{equation*}
In the \((\oc\rulename{f})\) rule, \(\Gamma\) must contain \emph{at most one} formula.

The \emph{depth} of a derivation \(\pi\) is the maximum number of
\((\oc\rulename{f})\) and \((\pg)\) rules in a branch of \(\pi\).

\begin{theorem}
If $\pi$ is an LLL proof of depth d, and R is the corresponding LLL proof-net, then R can be reduced to its normal form by cut elimination in  $ O((d+1)|\pi|^{2^{d+1}})$ steps, where $|\pi|$ is the size of $\pi$.
\end{theorem}

The class FP is the class of functions on binary lists which are
computable in polynomial time on a Turing machine.

\begin{theorem}
The class of functions on binary lists representable in LLL is exactly FP.
\end{theorem}

In the literature one also often considers the \emph{affine} variant of
LLL, called \emph{light affine logic}, LAL.

\section{Soft linear logic}\label{soft-linear-logic}

We consider the intuitionistic version of \emph{soft linear logic}, SLL.
The language of formulas is the same one as that of ILL:
\begin{equation*}
A ::= X \mid A\tens A \mid A\limp A \mid A\with A \mid  A\plus A   \mid \oc{A} \mid \forall X A
\end{equation*}
The sequent calculus rules are the same ones as for ILL, except for the
rules dealing with the exponential connectives:
\begin{equation*}
\AxRule{\Gamma\vdash A}
\LabelRule{\oc\rulename{mf} }
\UnaRule{\oc{\Gamma}\vdash\oc{A}}
\DisplayProof
\qquad
\AxRule{\Gamma,A^{(n)}\vdash C}
\LabelRule{\rulename{mplex}}
\UnaRule{\Gamma,\oc{A}\vdash C}
\DisplayProof
\end{equation*}
The rule mplex is the \emph{multiplexing} rule. In its premiss,
\(A^{(n)}\) stands for n occurrences of formula \(A\). As particular
instances of mplex for \(n=0\) and 1 respectively, we get weakening and
dereliction:
\begin{equation*}
\AxRule{\Gamma \vdash C}
\UnaRule{\Gamma,\oc{A}\vdash C}
\DisplayProof
\qquad
\AxRule{\Gamma,A\vdash C}
\UnaRule{\Gamma,\oc{A}\vdash C}
\DisplayProof
\end{equation*}

The \emph{depth} of a derivation \(\pi\) is the maximum number of
\((\oc\rulename{mf})\) rules in a branch of \(\pi\).

\begin{theorem}
If $\pi$ is an SLL proof of depth d, and R is the corresponding SLL proof-net, then R can be reduced to its normal form by cut elimination in $O(|\pi|^d)$ steps, where $|\pi|$ is the size of $\pi$.
\end{theorem}

\begin{theorem}
The class of functions on binary lists representable in SLL is exactly FP.
\end{theorem}

Soft linear logic was introduced in~\cite{softlinearlogic}.


%%% Local Variables:
%%% mode: latex
%%% TeX-master: "main"
%%% End:



\part{Semantics}

\chapter{Semantics}\label{semantics}

Linear Logic has numerous semantics some of which are described in
details in the next chapters.

\begin{itemize}
\item \hyperref[coherent-semantics]{Coherent semantics}
\item \hyperref[phase-semantics]{Phase semantics}
\item \hyperref[categorical-semantics]{Categorical semantics}
\item \hyperref[relational-semantics]{Relational semantics}
\item \hyperref[finiteness-semantics]{Finiteness semantics}
\item \hyperref[geometry-of-interaction]{Geometry of interaction}
\item \hyperref[game-semantics]{Game semantics}
\end{itemize}

\hyperref[provable-formulas]{Common properties} may be found in most of
these models. We will denote by \(A\longrightarrow B\) the fact that
there is a canonical morphism from \(A\) to \(B\) and by \(A\cong B\)
the fact that there is a canonical \hyperref[isomorphism]{isomorphism} between \(A\) and \(B\). By ``canonical'' we mean that these (iso)morphisms are natural
transformations.


%%% Local Variables:
%%% mode: latex
%%% TeX-master: "main"
%%% End:

\section{Orthogonality relation}\label{orthogonality-relation}

\textbf{Orthogonality relations} are used pervasively throughout linear
logic models, being often used to define somehow the duality operator
\((-)\orth\).

\begin{definition}[Orthogonality relation]
Let $A$ and $B$ be two sets. An \textbf{orthogonality relation} on $A$ and $B$ is a binary relation $\mathcal{R}\subseteq A\times B$. We say that $a\in A$ and $b\in B$ are \textbf{orthogonal}, and we note $a\perp b$, whenever $(a, b)\in\mathcal{R}$.
\end{definition}

Let us now assume an orthogonality relation over \(A\) and \(B\).

\begin{definition}[Orthogonal sets]
Let $\alpha\subseteq A$. We define its orthogonal set $\alpha\orth$ as $\alpha\orth:=\{b\in B \mid \forall a\in \alpha, a\perp b\}$.

Symmetrically, for any $\beta\subseteq B$, we define $\beta\orth:=\{a\in A \mid \forall b\in \beta, a\perp b\}$.
\end{definition}

Orthogonal sets define Galois connections and share many common
properties.

\begin{proposition}
For any sets $\alpha, \alpha'\subseteq A$:
\begin{itemize}
\item $\alpha\subseteq \alpha\biorth$
\item If $\alpha\subseteq\alpha'$, then ${\alpha'}\orth\subseteq\alpha\orth$
\item $\alpha\triorth = \alpha\orth$
\end{itemize}
\end{proposition}

%%% Local Variables:
%%% mode: latex
%%% TeX-master: "main"
%%% End:


\chapter{Coherent semantics}\label{coherent-semantics}

\emph{Coherent semantics} was invented by Girard in the paper \emph{The
system F, 15 years later}~\cite{systemF15} with the objective of building a
denotationnal interpretation of second order intuitionnistic logic (aka
polymorphic lambda-calculus).

Coherent semantics is based on the notion of \emph{stable functions}
that was initially proposed by Gérard Berry. Stability is a condition on
Scott continuous functions that expresses the determinism of the
relation between the output and the input: the typical Scott continuous
but non stable function is the \emph{parallel or} because when the two
inputs are both set to \texttt{true}, only one of them is the reason why
the result is \texttt{true} but there is no way to determine which one.

A further achievement of coherent semantics was that it allowed to endow
the set of stable functions from \(X\) to \(Y\) with a structure of
domain, thus closing the category of coherent spaces and stable
functions. However the most interesting point was the discovery of a
special class of stable functions, \emph{linear functions}, which was
the first step leading to Linear Logic.

\section{The cartesian closed structure of coherent semantics}\label{the-cartesian-closed-structure-of-coherent-semantics}

There are three equivalent definitions of coherent spaces: the first
one, \emph{coherent spaces as domains}, is interesting from a historical
point of view as it emphazises the fact that coherent spaces are
particular cases of Scott domains. The second one, \emph{coherent spaces
as graphs}, is the most commonly used and will be our ``official''
definition in the sequel. The last one, \emph{cliqued spaces} is a
particular example of a more general scheme that one could call
``symmetric reducibility''; this scheme is underlying lots of
constructions in linear logic such as \hyperref[phase-semantics]{phase
semantics} or the proof of strong normalisation for proof-nets.

\subsection{Coherent spaces}\label{coherent-spaces}

A coherent space \(X\) is a collection of subsets of a set \(\web X\)
satisfying some conditions that will be detailed shortly. The elements
of \(X\) are called the \emph{cliques} of \(X\) (for reasons that will
be made clear in a few lines). The set \(\web X\) is called the
\emph{web} of \(X\) and its elements are called the \emph{points} of
\(X\); thus a clique is a set of points. Note that the terminology is a
bit ambiguous as the points of \(X\) are the elements of the web of
\(X\), not the elements of \(X\).

The definitions below give three equivalent conditions that have to be
satisfied by the cliques of a coherent space.

\subsubsection{As domains}\label{as-domains}

The cliques of \(X\) have to satisfy:

\begin{itemize}
\item subset closure: if \(x\subset y\in X\) then \(x\in X\),
\item singletons: \(\{a\}\in X\) for \(a\in\web X\).
\item binary compatibility: if \(A\) is a family of pairwise compatible
  cliques of \(X\), that is if \(x\cup y\in X\) for any \(x,y\in A\),
  then \(\bigcup A\in X\).
\end{itemize}

A coherent space is thus ordered by inclusion; one easily checks that it
is a domain. In particular finite cliques of \(X\) correspond to compact
elements.

\subsubsection{As graphs}\label{as-graphs}

There is a reflexive and symmetric relation \(\coh_X\) on \(\web X\) (the
\emph{coherence relation}) such that any subset \(x\) of \(\web X\) is a
clique of \(X\) iff \(\forall a,b\in x,\, a\coh_X b\). In other terms
\(X\) is the set of complete subgraphs of the simple unoriented graph of
the \(\coh_X\) relation; this is the reason why elements of \(X\) are
called \emph{cliques}.

The \emph{strict coherence relation} \(\scoh_X\) on \(X\) is defined by:
\(a\scoh_X b\) iff \(a\neq b\) and \(a\coh_X b\).

A coherent space in the domain sense is seen to be a coherent space in
the graph sense by setting \(a\coh_X b\) iff \(\{a,b\}\in X\);
conversely one can check that cliques in the graph sense are subset
closed and satisfy the binary compatibility condition.

A coherent space is completely determined by its web and its coherence
relation, or equivalently by its web and its strict coherence.

\subsubsection{As cliqued spaces}\label{as-cliqued-spaces}

\begin{definition}[Duality]
Let $x, y\subseteq \web{X}$ be two sets. We will say that they are dual, written $x\perp y$ if their intersection contains at most one element: $\mathrm{Card}(x\cap y)\leq 1$. As usual, it defines an \hyperref[orthogonality-relation]{orthogonality relation} over $\powerset{\web{X}}$.
\end{definition}

The last way to express the conditions on the cliques of a coherent
space \(X\) is simply to say that we must have \(X\biorth = X\).

\subsubsection{Equivalence of definitions}\label{equivalence-of-definitions}

Let \(X\) be a cliqued space and define a relation on \(\web X\) by
setting \(a\coh_X b\) iff there is \(x\in X\) such that \(a, b\in x\).
This relation is obviously symmetric; it is also reflexive because all
singletons belong to \(X\): if \(a\in \web X\) then \(\{a\}\) is dual to
any element of \(X\orth\) (actually \(\{a\}\) is dual to any subset of
\(\web X\)), thus \(\{a\}\) is in \(X\biorth\), thus in \(X\).

Let \(a\coh_X b\). Then \(\{a,b\}\in X\); indeed there is an \(x\in X\)
such that \(a, b\in x\). This \(x\) is dual to any \(y\in X\orth\), that
is meets any \(y\in X\orth\) in a most one point. Since
\(\{a,b\}\subset x\) this is also true of \(\{a,b\}\), so that
\(\{a,b\}\) is in \(X\biorth\) thus in \(X\).

Now let \(x\) be a clique for \(\coh_X\) and \(y\) be an element of
\(X\orth\). Suppose \(a, b\in x\cap y\), then since \(a\) and \(b\) are
coherent (by hypothesis on \(x\)) we have \(\{a,b\}\in X\) and since
\(y\in X\orth\) we must have that \(\{a,b\}\) and \(y\) meet in at most
one point. Thus \(a = b\) and we have shown that \(x\) and \(y\) are
dual. Since \(y\) was arbitrary this means that \(x\) is in
\(X\biorth\), thus in \(X\). Finally we get that any set of pairwise
coherent points of \(X\) is in \(X\). Conversely given \(x\in X\) its
points are obviously pairwise coherent so eventually we get that \(X\)
is a coherent space in the graph sense.

Conversely given a coherent space \(X\) in the graph sense, one can
check that it is a cliqued space. Call \emph{anticlique} a set
\(y\subset \web X\) of pairwise incoherent points: for all \(a, b\) in
\(y\), if \(a\coh_X b\) then \(a=b\). Any anticlique intersects any
clique in at most one point: let \(x\) be a clique and \(y\) be an
anticlique, then if \(a,b\in x\cap y\), since \(a, b\in x\) we have
\(a\coh_X b\) and since \(y\) is an anticlique we have \(a = b\). Thus
\(y\in X\orth\). Conversely given any \(y\in X\orth\) and \(a, b\in y\),
suppose \(a\coh_X b\). Then \(\{a,b\}\in X\), thus \(\{a,b\}\perp y\)
which entails that \(\{a, b\}\) has at most one point so that \(a = b\):
we have shown that any two elements of \(y\) are incoherent.

Thus the collection of anticliques of \(X\) is the dual \(X\orth\) of
\(X\). Note that the incoherence relation defined above is reflexive and
symmetric, so that \(X\orth\) is a coherent space in the graph sense.
Thus we can do for \(X\orth\) exactly what we've just done for \(X\) and
consider the anti-anticliques, that is the anticliques for the
incoherent relation which are the cliques for the in-incoherent
relation. It is not difficult to see that this in-incoherence relation
is just the coherence relation we started with; we thus obtain that
\(X\biorth = X\), so that \(X\) is a cliqued space.

\subsection{Stable functions}\label{stable-functions}

\begin{definition}[Stable function]
Let $X$ and $Y$ be two coherent spaces. A function $F:X\longrightarrow Y$ is \emph{stable} if it satisfies:
\begin{itemize}
\item it is non decreasing: for any $x,y\in X$ if $x\subset y$ then $F(x)\subset F(y)$;
\item it is continuous (in the Scott sense): if $A$ is a directed family of cliques of $X$, that is if for any $x,y\in A$ there is a $z\in A$ such that $x\cup y\subset z$, then $\bigcup_{x\in A}F(x) = F(\bigcup A)$;
\item it satisfies the stability condition: if $x,y\in X$ are compatible, that is if $x\cup y\in X$, then $F(x\cap y) = F(x)\cap F(y)$.
\end{itemize}
\end{definition}

This definition is admitedly not very tractable. An equivalent and most
useful caracterisation of stable functions is given by the following
theorem.

\begin{theorem}
Let $F:X\longrightarrow Y$ be a non-decreasing function from the coherent space $X$ to the coherent space $Y$. The function $F$ is stable iff it satisfies: for any $x\in X$, $b\in\web Y$, if $b\in F(x)$ then there is a finite clique $x_0\subset x$ such that:
\begin{itemize}
\item $b\in F(x_0)$,
\item for any $y\subset x$ if $b\in F(y)$ then $x_0\subset y$ ($x_0$ is ``the'' minimum sub-clique of $x$ such that $b\in F(x_0)$). 
\end{itemize}
\end{theorem}

Note that the stability condition doesn't depend on the coherent space
structure and can be expressed more generally for continuous functions
on domains. However, as mentionned in the introduction, the restriction
to coherent spaces allows to endow the set of stable functions from
\(X\) to \(Y\) with a structure of coherent space.

\begin{definition}[The space of stable functions]
Let $X$ and $Y$ be coherent spaces. We denote by $X_{\mathrm{fin}}$ the set of \emph{finite} cliques of $X$. The function space $X\imp Y$ is defined by:
\begin{itemize}
\item $\web{X\imp Y} = X_{\mathrm{fin}}\times \web Y$,
\item $(x_0, a)\coh_{X\imp Y}(y_0, b)$ iff $\begin{cases}\text{if } x_0\cup y_0\in X\text{ then } a\coh_Y b,\\
\text{if } x_0\cup y_0\in X\text{ and } a = b\text{ then } x_0 = y_0\end{cases}$.
\end{itemize}
\end{definition}

One could equivalently define the strict coherence relation on
\(X\imp Y\) by: \((x_0,a)\scoh_{X\imp Y}(y_0, b)\) iff when
\(x_0\cup y_0\in X\) then \(a\scoh_Y b\) (equivalently
\(x_0\cup y_0\not\in X\) or \(a\scoh_Y b\)).

\begin{definition}[Trace of a stable function]
Let $F:X\longrightarrow Y$ be a function. The \emph{trace} of $F$ is the set:
\begin{equation*}
\mathrm{Tr}(F) = \{(x_0, b), x_0\text{ minimal such that } b\in F(x_0)\}.
\end{equation*}
\end{definition}

\begin{theorem}
$F$ is stable iff $\mathrm{Tr}(F)$ is a clique of the function space $X\imp Y$.
\end{theorem}

In particular the continuity of \(F\) entails that if \(x_0\) is minimal
such that \(b\in F(x_0)\), then \(x_0\) is finite.

\begin{definition}[The evaluation function]
Let $f$ be a clique in $X\imp Y$. We define a function $\mathrm{Fun}\,f:X\longrightarrow Y$ by: $\mathrm{Fun}\,f(x) = \{b\in Y,\text{ there is }x_0\subset x\text{ such that }(x_0, b)\in f\}$.
\end{definition}

\begin{theorem}[Closure]
If $f$ is a clique of the function space $X\imp Y$ then we have $\mathrm{Tr}(\mathrm{Fun}\,f) = f$. Conversely if $F:X\longrightarrow Y$ is a stable function then we have $F = \mathrm{Fun}\,\mathrm{Tr}(F)$.
\end{theorem}

\subsection{Cartesian product}\label{cartesian-product}

\begin{definition}[Cartesian product]
Let $X_1$ and $X_2$ be two coherent spaces. We define the coherent space $X_1\with X_2$ (read $X_1$ ``with'' $X_2$):
\begin{itemize}
\item the web is the disjoint union of the webs: $\web{X_1\with X_2} = \{1\}\times\web{X_1}\cup \{2\}\times\web{X_2}$;
\item the coherence relation is the serie composition of the relations on $X_1$ and $X_2$: $(i, a)\coh_{X_1\with X_2}(j, b)$ iff either $i\neq j$ or $i=j$ and $a\coh_{X_i} b$.
\end{itemize}
\end{definition}

This definition is just the way to put a coherent space structure on the
cartesian product. Indeed one easily shows the

\begin{theorem}
Given cliques $x_1$ and $x_2$ in $X_1$ and $X_2$, we define the subset $\langle x_1, x_2\rangle$ of $\web{X_1\with X_2}$ by: $\langle x_1, x_2\rangle = \{1\}\times x_1\cup \{2\}\times x_2$. Then $\langle x_1, x_2\rangle$ is a clique in $X_1\with X_2$.

Conversely, given a clique $x\in X_1\with X_2$, for $i=1,2$ we define $\pi_i(x) = \{a\in X_i, (i, a)\in x\}$. Then $\pi_i(x)$ is a clique in $X_i$ and the function $\pi_i:X_1\with X_2\longrightarrow X_i$ is stable.

Furthemore these two operations are inverse of each other: $\pi_i(\langle x_1, x_2\rangle) = x_i$ and $\langle\pi_1(x), \pi_2(x)\rangle = x$. In particular any clique in $X_1\with X_2$ is of the form $\langle x_1, x_2\rangle$.
\end{theorem}

Altogether the results above (and a few other more that we shall leave
to the reader) allow to get:

\begin{theorem}
The category of coherent spaces and stable functions is cartesian closed.
\end{theorem}

In particular this means that if we define
\(\mathrm{Eval}:(X\imp Y)\with X\longrightarrow Y\) by:
\(\mathrm{Eval}(\langle f, x\rangle) = \mathrm{Fun}\,f(x)\) then
\(\mathrm{Eval}\) is stable.

\section{The monoidal structure of coherent semantics}\label{the-monoidal-structure-of-coherent-semantics}

\subsection{Linear functions}\label{linear-functions}

\begin{definition}[Linear function]
A function $F:X\longrightarrow Y$ is \emph{linear} if it is stable and furthemore satisfies: for any family $A$ of pairwise compatible cliques of $X$, that is such that for any $x, y\in A$, $x\cup y\in X$, we have $\bigcup_{x\in A}F(x) = F(\bigcup A)$.
\end{definition}

In particular if we take \(A\) to be the empty family, then we have
\(F(\emptyset) = \emptyset\).

The condition for linearity is quite similar to the condition for Scott
continuity, except that we dropped the constraint that \(A\) is
\emph{directed}. Linearity is therefore much stronger than stability:
most stable functions are not linear.

However most of the functions seen so far are linear. Typically the
function \(\pi_i:X_1\with X_2\longrightarrow X_i\) is linear from wich
one may deduce that the \emph{with} construction is also a cartesian
product in the category of coherent spaces and linear functions.

As with stable function we have an equivalent and much more tractable
caracterisation of linear function:

\begin{theorem}
Let $F:X\longrightarrow Y$ be a continuous function. Then $F$ is linear iff it satisfies: for any clique $x\in X$ and any $b\in F(x)$ there is a unique $a\in x$ such that $b\in F(\{a\})$.
\end{theorem}

Just as the caracterisation theorem for stable functions allowed us to
build the coherent space of stable functions, this theorem will help us
to endow the set of linear maps with a structure of coherent space.

\begin{definition}[The linear functions space]
Let $X$ and $Y$ be coherent spaces. The \emph{linear function space} $X\limp Y$ is defined by:
\begin{itemize}
\item $\web{X\limp Y} = \web X\times \web Y$,
\item $(a,b)\coh_{X\limp Y}(a', b')$ iff $\begin{cases}\text{if }a\coh_X a'\text{ then } b\coh_Y b'\\
 \text{if }a\coh_X a' \text{ and }b=b'\text{ then }a=a'\end{cases}$
\end{itemize}
\end{definition}

Equivalently one could define the strict coherence to be:
\((a,b)\scoh_{X\limp Y}(a',b')\) iff \(a\scoh_X a'\) entails \(b\scoh_Y b'\).

\begin{definition}[Linear trace]
Let $F:X\longrightarrow Y$ be a function. The \emph{linear trace} of $F$ denoted as $\mathrm{LinTr}(F)$ is the set:
  $\mathrm{LinTr}(F) = \{(a, b)\in\web X\times\web Y$ such that $b\in F(\{a\})\}$.
\end{definition}

\begin{theorem}
If $F$ is linear then $\mathrm{LinTr}(F)$ is a clique of $X\limp Y$.
\end{theorem}

\begin{definition}[Evaluation of linear function]
Let $f$ be a clique of $X\limp Y$. We define the function $\mathrm{LinFun}\,f:X\longrightarrow Y$ by: $\mathrm{LinFun}\,f(x) = \{b\in\web Y$ such that there is an $a\in x$ satisfying $(a,b)\in f\}$.
\end{definition}

\begin{theorem}[Linear closure]
Let $f$ be a clique in $X\limp Y$. Then we have $\mathrm{LinTr}(\mathrm{LinFun}\, f) = f$. Conversely if $F:X\longrightarrow Y$ is linear then we have $F = \mathrm{LinFun}\,\mathrm{LinTr}(F)$.
\end{theorem}

It remains to define a tensor product and we will get that the category
of coherent spaces with linear functions is monoidal symmetric (it is
actually *-autonomous).

\subsection{Tensor product}\label{tensor-product}

\begin{definition}[Tensor product]
Let $X$ and $Y$ be coherent spaces. Their tensor product $X\tens Y$ is defined by: $\web{X\tens Y} = \web X\times\web Y$ and $(a,b)\coh_{X\tens Y}(a',b')$ iff $a\coh_X a'$ and $b\coh_Y b'$.
\end{definition}

\begin{theorem}
The category of coherent spaces with linear maps and tensor product is \hyperref[modeling-imll]{monoidal symmetric closed}.
\end{theorem}

The closedness is a consequence of the existence of the linear
isomorphism:
\begin{equation*}
\varphi:X\tens Y\limp Z\ \stackrel{\sim}{\longrightarrow}\ X\limp(Y\limp Z)
\end{equation*}
that is defined by its linear trace:
\(\mathrm{LinTr}(\varphi) = \{(((a, b), c), (a, (b, c))),\, a\in\web X,\, b\in \web Y,\, c\in\web Z\}\).

\subsection{Linear negation}\label{linear-negation}

\begin{definition}[Linear negation]
Let $X$ be a coherent space. We define the \emph{incoherence relation} on $\web X$ by: $a\incoh_X b$ iff $a\coh_X b$ entails $a=b$. The incoherence relation is reflexive and symmetric; we call \emph{dual} or \emph{linear negation} of $X$ the associated coherent space denoted $X\orth$, thus defined by: $\web{X\orth} = \web X$ and $a\coh_{X\orth} b$ iff $a\incoh_X b$.
\end{definition}

The cliques of \(X\orth\) are called the \emph{anticliques} of \(X\). As
seen in the section on cliqued spaces we have \(X\biorth=X\).

\begin{theorem}
The category of coherent spaces with linear maps, tensor product and linear negation is *-autonomous.
\end{theorem}

This is in particular consequence of the existence of the isomorphism:
\begin{equation*}
\varphi:X\limp Y\ \stackrel{\sim}{\longrightarrow}\ Y\orth\limp X\orth
\end{equation*}
defined by its linear trace:
\(\mathrm{LinTr}(\varphi) = \{((a, b), (b, a)),\, a\in\web X,\, b\in\web Y\}\).

\section{Exponentials}\label{exponentials}

In linear algebra, bilinear maps may be factorized through the tensor
product. Similarly there is a coherent space \(\oc X\) that allows to
factorize stable functions through linear functions.

\begin{definition}[Of course]
Let $X$ be a coherent space; recall that $X_{\mathrm{fin}}$ denotes the set of finite cliques of $X$. We define the space $\oc X$ (read ``of course $X$'') by: $\web{\oc X} = X_{\mathrm{fin}}$ and $x_0\coh_{\oc X}y_0$ iff $x_0\cup y_0$ is a clique of $X$.
\end{definition}

Thus a clique of \(\oc X\) is a set of finite cliques of \(X\) the union
of wich is a clique of \(X\).

\begin{theorem}
Let $X$ be a coherent space. Denote by $\beta:X\longrightarrow \oc X$ the stable function whose trace is: $\mathrm{Tr}(\beta) = \{(x_0, x_0),\, x_0\in X_{\mathrm{fin}}\}$. Then for any coherent space $Y$ and any stable function $F: X\longrightarrow Y$ there is a unique \emph{linear} function $\bar F:\oc X\longrightarrow Y$ such that $F = \bar F\circ \beta$.

Furthermore we have $X\imp Y = \oc X\limp Y$.
\end{theorem}

\begin{theorem}[The exponential isomorphism]
Let $X$ and $Y$ be two coherent spaces. Then there is a linear isomorphism:
\begin{equation*}
\varphi:\oc(X\with Y)\quad\stackrel{\sim}{\longrightarrow}\quad \oc X\tens\oc Y.
\end{equation*}
\end{theorem}

The iso \(\varphi\) is defined by its trace:
\begin{equation*}
\mathrm{Tr}(\varphi) = \{(x_0, (\pi_1(x_0), \pi_2(x_0)), x_0\text{ finite clique of } X\with Y\}.
\end{equation*}
This isomorphism, that sends an additive structure (the web of a with is
obtained by disjoint union) onto a multiplicative one (the web of a
tensor is obtained by cartesian product) is the reason why the of course
is called an \emph{exponential}.

\section{Dual connectives and neutrals}\label{dual-connectives-and-neutrals}

By linear negation all the constructions defined so far
(\(\with, \tens, \oc\)) have a dual.

\subsection{The direct sum}\label{the-direct-sum}

The dual of \(\with\) is \(\plus\) defined by:
\(X\plus Y = (X\orth\with Y\orth)\orth\). An equivalent definition is
given by:
\(\web{X\plus Y} = \web{X\with Y} = \{1\}\times \web X \cup \{2\}\times\web Y\)
and
\((i, a)\coh_{X\plus Y} (j, b)\text{ iff } i = j = 1 \text{ and } a\coh_X b,\text{ or }i = j = 2\text{ and } a\coh_Y b\).

\begin{theorem}
Let $x'$ be a clique of $X\plus Y$; then $x'$ is of the form $\{i\}\times x$ where $i = 1\text{ and }x\in X$, or $i = 2\text{ and }x\in Y$.

Denote $\mathrm{inl}:X\longrightarrow X\plus Y$ the function defined by $\mathrm{inl}(x) = \{1\}\times x$ and by $\mathrm{inr}:Y\longrightarrow X\plus Y$ the function defined by $\mathrm{inr}(x) = \{2\}\times x$. Then $\mathrm{inl}$ and $\mathrm{inr}$ are linear.

If $F:X\longrightarrow Z$ and $G:Y\longrightarrow Z$ are ``linear'' functions then the function $H:X\plus Y \longrightarrow Z$ defined by $H(\mathrm{inl}(x)) = F(x)$ and $H(\mathrm{inr}(y)) = G(y)$ is linear.
\end{theorem}

In other terms \(X\plus Y\) is the direct sum of \(X\) and \(Y\). Note
that in the theorem all functions are \emph{linear}. Things doesn't work
so smoothly for stable functions. Historically it was after noting this
defect of coherent semantics w.r.t. the intuitionnistic implication that
Girard was leaded to discover linear functions.

\subsection{The par and the why not}\label{the-par-and-the-why-not}

We now come to the most mysterious constructions of coherent semantics:
the duals of the tensor and the of course.

The \emph{par} is the dual of the tensor, thus defined by:
\(X\parr Y = (X\orth\tens Y\orth)\orth\). From this one can deduce the
definition in graph terms:
\(\web{X\parr Y} = \web{X\tens Y} = \web X\times \web Y\) and
\((a,b)\scoh_{X\parr Y} (a',b')\) iff \(a\scoh_X a'\) or
\(b\scoh_Y b'\). With this definition one sees that we have:
\begin{equation*}
X\limp Y = X\orth\parr Y
\end{equation*}
for any coherent spaces \(X\) and \(Y\). This equation can be seen as an
alternative definition of the par: \(X\parr Y = X\orth\limp Y\).

Similarly the dual of the of course is called \emph{why not} defined by:
\(\wn X = (\oc X\orth)\orth\). From this we deduce the definition in the
graph sense which is a bit tricky: \(\web{\wn X}\) is the set of finite
anticliques of \(X\), and given two finite anticliques \(x\) and \(y\)
of \(X\) we have \(x\scoh_{\wn X} y\) iff there is \(a\in x\) and
\(b\in y\) such that \(a\scoh_X b\).

Note that both for the par and the why not it is much more convenient to
define the strict coherence than the coherence.

With these two last constructions, the equation between the stable
function space, the of course and the linear function space may be
written:
\begin{equation*}
X\imp Y = \wn X\orth\parr Y.
\end{equation*}

\subsection{One and bottom}\label{one-and-bottom}

Depending on the context we denote by \(\one\) or \(\bot\) the coherent
space whose web is a singleton and whose coherence relation is the
trivial reflexive relation.

\begin{theorem}
$\one$ is neutral for tensor, that is, there is a linear isomorphism $\varphi:X\tens\one\ \stackrel{\sim}{\longrightarrow}\ X$.

Similarly $\bot$ is neutral for par.
\end{theorem}

\subsection{Zero and top}\label{zero-and-top}

Depending on the context we denote by \(\zero\) or \(\top\) the coherent
space with empty web.

\begin{theorem}
$\zero$ is neutral for the direct sum $\plus$, $\top$ is neutral for the cartesian product $\with$.
\end{theorem}

\begin{remark}
It is one of the main defect of coherent semantics w.r.t. linear logic that it identifies the neutrals: in coherent semantics $\zero = \top$ and $\one = \bot$. However there is no known semantics of LL that solves this problem in a satisfactory way.
\end{remark}

\section{After coherent semantics}\label{after-coherent-semantics}

Coherent semantics was an important milestone in the modern theory of
logic of programs, in particular because it leaded to the invention of
Linear Logic, and more generally because it establishes a strong link
between logic and linear algebra; this link is nowadays aknowledged by
the customary use of \hyperref[categorical-semantics]{monoidal categories}
in logic. In some sense coherent semantics is a precursor of many
forthcoming works that explore the linear nature of logic as for example
\hyperref[geometry-of-interaction]{geometry of interaction} which interprets
proofs by operators or \hyperref[finiteness-semantics]{finiteness semantics}
which interprets formulas as vector spaces and resulted in
\wantedpage{differential linear logic}...

Lots of this work have been motivated by the fact that coherent
semantics is not complete as a semantics of programs (technically one
says that it is not \emph{fully abstract}). In order to see this, let us
firts come back on the origin of the central concept of \emph{stability}
which as pointed above originated in the study of the sequentiality in
programs.

\subsection{Sequentiality}\label{sequentiality}

Sequentiality is a property that we will not define here (it would
diserve its own chapter). We rely on the intuition that a function of
\(n\) arguments is sequential if one can determine which of these
argument is examined first during the computation. Obviously any
function implemented in a functionnal language is sequential; for
example the function \emph{or} defined à la CAML by:
\begin{quotation}
\texttt{or\ =\ fun\ (x,\ y)\ -\textgreater{}\ if\ x\ then\ true\ else\ y}
\end{quotation}
examines its argument x first. Note that this may be expressed more
abstractly by the property: \(\mathrm{or}(\bot, x) = \bot\) for any
boolean \(x\): the function \emph{or} needs its first argument in order
to compute anything. On the other hand we have
\(\mathrm{or}(\mathrm{true}, \bot) = \mathrm{true}\): in some case (when
the first argument is true), the function doesn't need its second
argument at all.

The typical non sequential function is the \emph{parallel or} (that one
cannot define in a CAML like language).

For a while one may have believed that the stability condition on which
coherent semantics is built was enough to capture the notion of
\emph{sequentiality} of programs. A hint was the already mentionned fact
that the \emph{parallel or} is not stable. This diserves a bit of
explanation.

\subsubsection{The parallel or is not stable}\label{the-parallel-or-is-not-stable}

Let \(B\) be the coherent space of booleans, also know as the flat
domain of booleans: \(\web B = \{tt, ff\}\) where \(tt\) and \(ff\) are
two arbitrary distinct objects (for example one may take \(tt = 0\) and
\(ff = 1\)) and for any \(b_1, b_2\in \web B\), define \(b_1\coh_B b_2\)
iff \(b_1 = b_2\). Then \(B\) has exactly three cliques: the empty
clique that we shall denote \(\bot\), the singleton \(\{tt\}\) that we
shall denote \(T\) and the singleton \(\{ff\}\) that we shall denote
\(F\). These three cliques are ordered by inclusion: \(\bot \leq T, F\)
(we use \(\leq\) for \(\subset\) to enforce the idea that coherent
spaces are domains).

Recall the \hyperref[cartesian-product]{definition of the
with}, and in particular that any clique of \(B\with B\) has the form
\(\langle x, y\rangle\) where \(x\) and \(y\) are cliques of \(B\). Thus
\(B\with B\) has 9 cliques:
\(\langle\bot,\bot\rangle,\ \langle\bot, T\rangle,\ \langle\bot, F\rangle,\ \langle T,\bot\rangle,\ \dots\)
that are ordered by the product order:
\(\langle x,y\rangle\leq \langle x,y\rangle\) iff \(x\leq x'\) and
\(y\leq y'\).

With these notations in mind one may define the parallel or by:
\begin{align*}
\mathrm{Por} : B\with B & \longrightarrow  B\\
  \langle T,\bot\rangle & \longrightarrow  T\\
  \langle \bot,T\rangle & \longrightarrow  T\\
    \langle F, F\rangle & \longrightarrow  F
\end{align*}
The function is completely determined if we add the assumption that it
is non decreasing; for example one must have
\(\mathrm{Por}\langle\bot,\bot\rangle = \bot\) because the lhs has to be
less than both \(T\) and \(F\) (because
\(\langle\bot,\bot\rangle \leq \langle T,\bot\rangle\) and
\(\langle\bot,\bot\rangle \leq \langle F,F\rangle\)).
The function is not stable because
\(\langle T,\bot\rangle \cap \langle \bot, T\rangle = \langle\bot, \bot\rangle\),
thus
\(\mathrm{Por}(\langle T,\bot\rangle \cap \langle \bot, T\rangle) = \bot\)
whereas
\(\mathrm{Por}\langle T,\bot\rangle \cap \mathrm{Por}\langle \bot, T\rangle = T\cap T = T\).

Another way to see this is: suppose \(x\) and \(y\) are two cliques of
\(B\) such that \(tt\in \mathrm{Por}\langle x, y\rangle\), which means
that \(\mathrm{Por}\langle x, y\rangle = T\); according to the
\hyperref[stable-functions]{caracterisation theorem of stable
functions}, if \(\mathrm{Por}\) were stable then there would be a unique
minimum \(x_0\) included in \(x\), and a unique minimum \(y_0\) included
in \(y\) such that \(\mathrm{Por}\langle x_0, y_0\rangle = T\). This is
not the case because both \(\langle T,\bot\rangle\) and
\(\langle T,\bot\rangle\) are minimal such that their value is \(T\).

In other terms, knowing that \(\mathrm{Por}\langle x, y\rangle = T\)
doesn't tell which of \(x\) of \(y\) is responsible for that, although
we know by the definition of \(\mathrm{Por}\) that only one of them is.
Indeed the \(\mathrm{Por}\) function is not representable in sequential
programming languages such as (typed) lambda-calculus.

So the first genuine idea would be that stability caracterises
sequentiality; but...

\subsubsection{The Gustave function is stable}\label{the-gustave-function-is-stable}

The Gustave function, so-called after an old joke, was found by Gérard
Berry as an example of a function that is stable but non sequential. It
is defined by:
\begin{align*}
  B\with B\with B           & \longrightarrow B\\
  \langle T, F, \bot\rangle & \longrightarrow T\\
  \langle \bot, T, F\rangle & \longrightarrow T\\
  \langle F, \bot, T\rangle & \longrightarrow T\\
  \langle x, y, z\rangle    & \longrightarrow F
\end{align*}
The last clause is for all cliques \(x\), \(y\) and \(z\) such that
\(\langle x, y ,z\rangle\) is incompatible with the three cliques
\(\langle T, F, \bot\rangle\), \(\langle \bot, T, F\rangle\) and
\(\langle F, \bot, T\rangle\), that is such that the union with any of
these three cliques is not a clique in \(B\with B\with B\). We shall
denote \(x_1\), \(x_2\) and \(x_3\) these three cliques.
We furthemore assume that the Gustave function is non decreasing, so
that we get \(G\langle\bot,\bot,\bot\rangle = \bot\).

We note that \(x_1\), \(x_2\) and \(x_3\) are pairwise incompatible.
From this we can deduce that the Gustave function is stable: typically
if \(G\langle x,y,z\rangle = T\) then exactly one of the \(x_i\)s is
contained in \(\langle x, y, z\rangle\).

However it is not sequential because there is no way to determine which
of its three arguments is examined first: it is not the first one
otherwise we would have \(G\langle\bot, T, F\rangle = \bot\) and
similarly it is not the second one nor the third one.

In other terms there is no way to implement the Gustave function by a
lambda-term (or in any sequential programming language). Thus coherent
semantics is not complete w.r.t. lambda-calculus.

The research for a right model for sequentiality was the motivation for
lot of work, \emph{e.g.}, \emph{sequential algorithms} by Gérard Bérry
and Pierre-Louis Currien in the early eighties, that were more recently
reformulated as a kind of \hyperref[game-semantics]{game model}, and the
theory of \emph{hypercoherent spaces} by Antonio Bucciarelli and Thomas
Ehrhard.

\subsection{Multiplicative neutrals and the mix rule}\label{multiplicative-neutrals-and-the-mix-rule}

Coherent semantics is slightly degenerated w.r.t. linear logic because
it identifies multiplicative neutrals (it also identifies additive
neutrals but that's yet another problem): the coherent spaces \(\one\)
and \(\bot\) are equal.

The first consequence of the identity \(\one = \bot\) is that the
formula \(\one\limp\bot\) becomes provable, and so does the formula
\(\bot\). Note that this doesn't entail (as in classical logic or
intuitionnistic logic) that linear logic is incoherent because the
principle \(\bot\limp A\) for any formula \(A\) is still not provable.

The equality \(\one = \bot\) has also as consequence the fact that
\(\bot\limp\one\) (or equivalently the formula \(\one\parr\one\)) is
provable. This principle is also known as the \hyperref[mix]{mix rule}
\begin{prooftree}
\AxRule{\vdash \Gamma}
\AxRule{\vdash \Delta}
\LabelRule{\rulename{mix}}
\BinRule{\vdash \Gamma,\Delta}
\end{prooftree}
as it can be used to show that this rule is admissible:
\begin{prooftree}
\AxRule{\vdash\Gamma}
\LabelRule{\bot}
\UnaRule{\vdash\Gamma, \bot}
\AxRule{\vdash\Delta}
\LabelRule{\bot}
\UnaRule{\vdash\Delta, \bot}
\LabelRule{\tens}
\BinRule{\vdash \Gamma, \Delta, \bot\tens\bot}
\NulRule{\vdash \one\parr\one}
\LabelRule{\rulename{cut}}
\BinRule{\vdash\Gamma,\Delta}
\end{prooftree}

None of the two principles \(1\limp\bot\) and \(\bot\limp\one\) are
valid in linear logic. To correct this one could extend the syntax of
linear logic by adding the mix-rule. This is not very satisfactory as
the mix rule violates some principles of
\hyperref[polarized-linear-logic]{Polarized linear logic}, typically the
fact that as sequent of the form \(\vdash P_1, P_2\) where \(P_1\) and
\(P_2\) are positive, is never provable.

On the other hand the mix-rule is valid in coherent semantics so one
could try to find some other model that invalidates the mix-rule. For
example Girard's Coherent Banach spaces were an attempt to address this
issue.


%%% Local Variables:
%%% mode: latex
%%% TeX-master: "main"
%%% End:


\chapter{Phase semantics}\label{phase-semantics}

\section{Introduction}\label{introduction}

The semantics given by phase spaces is a kind of ``formula and
provability semantics'', and is thus quite different in spirit from the
more usual denotational semantics of linear logic. (Those are rather
some ``formulas and \emph{proofs} semantics''.)

\begin{quotation}
\texttt{- - - probably a whole lot more of blabla to put here... - - -}
\end{quotation}

\section{Preliminaries: relation and closure operators}\label{preliminaries-relation-and-closure-operators}

Part of the structure obtained from phase semantics works in a very
general framework and relies solely on the notion of relation between
two sets.

\subsection{Relations and operators on subsets}\label{relations-and-operators-on-subsets}

The starting point of phase semantics is the notion of \emph{duality}.
The structure needed to talk about duality is very simple: one just
needs a relation \(R\) between two sets \(X\) and \(Y\). Using standard
mathematical practice, we can write either \((a,b) \in R\) or
\(a\mathrel{R} b\) to say that \(a\in X\) and \(b\in Y\) are related.

\begin{definition}
If $R\subseteq X\times Y$ is a relation, we write $R^\sim\subseteq Y\times X$ for the converse relation: $(b,a)\in R^\sim$ iff $(a,b)\in R$.
\end{definition}

Such a relation yields three interesting operators sending subsets of
\(X\) to subsets of \(Y\):

\begin{definition}
Let $R\subseteq X\times Y$ be a relation, define the operators $\langle R\rangle$, $[R]$ and $\_^R$ taking subsets of $X$ to subsets of $Y$ as follows:
\begin{enumerate}
\item $b\in\langle R\rangle(x)$ iff $\exists a\in x,\ (a,b)\in R$
\item $b\in[R](x)$ iff $\forall a\in X,\ (a,b)\in R \implies a\in x$
\item $b\in x^R$ iff $\forall a\in x, (a,b)\in R$
\end{enumerate}
\end{definition}

The operator \(\langle R\rangle\) is usually called the \emph{direct
image} of the relation, \([R]\) is sometimes called the \emph{universal
image} of the relation.

It is trivial to check that \(\langle R\rangle\) and \([R]\) are
covariant (increasing for the \(\subseteq\) relation) while \(\_^R\) is
contravariant (decreasing for the \(\subseteq\) relation). More
interesting:

\begin{lemma}[Galois Connections]\
\begin{enumerate}
\item $\langle R\rangle$ is right-adjoint to $[R^\sim]$: for any $x\subseteq X$ and $y\subseteq Y$, we have $[R^\sim]y \subseteq x$ iff $y\subseteq \langle R\rangle(x)$
\item we have $y\subseteq x^R$ iff $x\subseteq y^{R^\sim}$
\end{enumerate}
\end{lemma}

This implies directly that \(\langle R\rangle\) commutes with arbitrary
unions and \([R]\) commutes with arbitrary intersections. (And in fact,
any operator commuting with arbitrary unions (resp. intersections) is of
the form \(\langle R\rangle\) (resp. \([R]\)).

\begin{remark}
The operator $\_^R$ sends unions to intersections because $\_^R : \mathcal{P}(X) \to \mathcal{P}(Y)^\mathrm{op}$ is right adjoint to $\_^{R^\sim} : \mathcal{P}(Y)^{\mathrm{op}} \to \mathcal{P}(X)$...
\end{remark}


\subsection{Closure operators}\label{closure-operators}

\begin{definition}
A closure operator on $\mathcal{P}(X)$ is a monotonic operator $P$ on the subsets of $X$ which satisfies:
\begin{enumerate}
\item for all $x\subseteq X$, we have $x\subseteq P(x)$
\item for all $x\subseteq X$, we have $P(P(x))\subseteq P(x)$
\end{enumerate}
\end{definition}

Closure operators are quite common in mathematics and computer science.
They correspond exactly to the notion of \emph{monad} on a preorder...

It follows directly from the definition that for any closure operator
\(P\), the image \(P(x)\) is a fixed point of \(P\). Moreover:

\begin{lemma}
$P(x)$ is the smallest fixed point of $P$ containing $x$.
\end{lemma}

One other important property is the following:

\begin{lemma}
Write $\mathcal{F}(P) = \{x\ |\ P(x)\subseteq x\}$ for the collection of fixed points of a closure operator $P$. We have that $\left(\mathcal{F}(P),\bigcap\right)$ is a complete inf-lattice.
\end{lemma}

\begin{remark}
A closure operator is in fact determined by its set of fixed points: we have $P(x) = \bigcup \{ y\ |\ y\in\mathcal{F}(P),\,y\subseteq x\}$
\end{remark}

Since any complete inf-lattice is automatically a complete sup-lattice,
\(\mathcal{F}(P)\) is also a complete sup-lattice. However, the sup
operation isn't given by plain union:

\begin{lemma}
If $P$ is a closure operator on $\mathcal{P}(X)$, and if $(x_i)_{i\in I}$ is a (possibly infinite) family of subsets of $X$, we write $\bigvee_{i\in I} x_i = P\left(\bigcup_{i\in I} x_i\right)$.

We have $\left(\mathcal{F}(P),\bigcap,\bigvee\right)$ is a complete lattice.
\end{lemma}

\begin{proof}
easy.
\end{proof}

A rather direct consequence of the Galois connections of the previous
section is:

\begin{lemma}
The operator and $\langle R\rangle \circ [R^\sim]$ and the operator $x\mapsto {x^R}^{R^\sim}$ are closures.
\end{lemma}

A last trivial lemma:

\begin{lemma}
We have $x^R = {{x^R}^{R^\sim}}^{R}$.

As a consequence, a subset $x\subseteq X$ is in $\mathcal{F}({\_^R}^{R^\sim})$ iff it is of the form $y^{R^\sim}$.
\end{lemma}

\begin{remark}
Everything gets a little simpler when $R$ is a symmetric relation on $X$.
\end{remark}

\section{Phase Semantics}\label{phase-semantics-1}

\subsection{Phase spaces}\label{phase-spaces}

\begin{definition}[monoid]
A \emph{monoid} is simply a set $X$ equipped with a binary operation $\_\cdot\_$ s.t.:
\begin{enumerate}
\item the operation is associative
\item there is a neutral element $1\in X$
\end{enumerate}
The monoid is \emph{commutative} when the binary operation is commutative.
\end{definition}

\begin{definition}[Phase space]
A phase space is given by:
\begin{enumerate}
\item a commutative monoid $(X,1,\cdot)$,
\item together with a subset $\Bot\subseteq X$.
\end{enumerate}
The elements of $X$ are called \emph{phases}.

We write $\bot$ for the relation $\{(a,b)\ |\ a\cdot b \in \Bot\}$. This relation is symmetric.

A \emph{fact} in a phase space is simply a fixed point for the \hyperref[orthogonality-relation]{closure operator $x\mapsto x\biorth$}.
\end{definition}

Thanks to the preliminary work, we have:

\begin{corollary}
The set of facts of a phase space is a complete lattice where:
\begin{enumerate}
\item $\bigwedge_{i\in I} x_i$ is simply $\bigcap_{i\in I} x_i$,
\item $\bigvee_{i\in I} x_i$ is $\left(\bigcup_{i\in I} x_i\right)\biorth$.
\end{enumerate}
\end{corollary}

\subsection{Additive connectives}\label{additive-connectives}

The previous corollary makes the following definition correct:

\begin{definition}[additive connectives]
If $(X,1,\cdot,\Bot)$ is a phase space, we define the following facts and operations on facts:
\begin{enumerate}
\item $\top = X = \emptyset\orth$
\item $\zero = \emptyset\biorth = X\orth$
\item $x\with y = x\cap y$
\item $x\plus y = (x\cup y)\biorth$
\end{enumerate}
\end{definition}

Once again, the next lemma follows from previous observations:

\begin{lemma}[additive de Morgan laws]
We have
\begin{enumerate}
\item $\zero\orth = \top$
\item $\top\orth = \zero$
\item $(x\with y)\orth = x\orth \plus y\orth$
\item $(x\plus y)\orth = x\orth \with y\orth$
\end{enumerate}
\end{lemma}

\subsection{Multiplicative connectives}\label{multiplicative-connectives}

In order to define the multiplicative connectives, we actually need to
use the monoid structure of our phase space. One interpretation that is
reminiscent in phase semantics is that our spaces are collections of
\emph{tests} / programs / proofs / \emph{strategies} that can interact
with each other. The result of the interaction between \(a\) and \(b\)
is simply \(a\cdot b\).

The set \(\Bot\) can be thought of as the set of ``good'' things, and we
thus have \(a\in x\orth\) iff ``\(a\) interacts correctly with all the
elements of \(x\)''.

\begin{definition}
If $x$ and $y$ are two subsets of a phase space, we write $x\cdot y$ for the set $\{a\cdot b\ |\ a\in x, b\in y\}$.
\end{definition}

Thus \(x\cdot y\) contains all the possible interactions between one
element of \(x\) and one element of \(y\).

The tensor connective of linear logic is now defined as:

\begin{definition}[multiplicative connectives]
If $x$ and $y$ are facts in a phase space, we define
\begin{itemize}
  \item $\one = \{1\}\biorth$;
  \item $\bot = \one\orth$;
  \item the tensor $x\tens y$ to be the fact $(x\cdot y)\biorth$;
  \item the par connective is the de Morgan dual of the tensor: $x\parr y = (x\orth \tens y\orth)\orth$;
  \item the linear arrow is just $x\limp y = x\orth\parr y = (x\tens y\orth)\orth$.
  \end{itemize}
\end{definition}

Note that by unfolding the definition of \(\limp\), we have the
following, ``intuitive'' definition of \(x\limp y\):

\begin{lemma}
If $x$ and $y$ are facts, we have $a\in x\limp y$ iff $\forall b\in x,\,a\cdot b\in y$.
\end{lemma}

\begin{proof}
easy exercise.
\end{proof}

Readers familiar with realisability will appreciate...

\begin{remark}
Some people say that this idea of orthogonality was implicitly present in Tait's proof of strong normalisation. More recently, Jean-Louis Krivine and Alexandre Miquel have used the idea explicitly to do realisability...
\end{remark}

\subsection{Properties}\label{properties}

All the expected properties hold:

\begin{lemma}\
\begin{itemize}
\item The operations $\tens$, $\parr$, $\plus$ and $\with$ are commutative and associative,
\item They have respectively $\one$, $\bot$, $\zero$ and $\top$ as neutral element,
\item $\zero$ is absorbing for $\tens$,
\item $\top$ is absorbing for $\parr$,
\item $\tens$ distributes over $\plus$,
\item $\parr$ distributes over $\with$.
\end{itemize}
\end{lemma}

\subsection{Exponentials}\label{exponentials-3}

\begin{definition}[Exponentials]
Write $I$ for the set of idempotents of a phase space: $I=\{a\ |\ a\cdot a=a\}$. We put:
\begin{enumerate}
\item $\oc x = (x\cap I\cap \one)\biorth$,
\item $\wn x = (x\orth\cap I\cap\one)\orth$.
\end{enumerate}
\end{definition}

This definition captures precisely the intuition behind the exponentials:
\begin{itemize}
\item we need to have contraction, hence we restrict to indempotents in \(x\),
\item and weakening, hence we restrict to \(\one\).
\end{itemize}
Since \(I\) isn't necessarily a fact, we then take the biorthogonal to get a fact...

\section{Soundness}\label{soundness}

\begin{definition}
Let $(X, 1, \cdot)$ be a commutative monoid.

Given a formula $A$ of linear logic and an assignation $\rho$ that associate a fact to any variable, we can inductively define the interpretation $\sem{A}_\rho$ of $A$ in $X$ as one would expect. Interpretation is lifted to sequents as $\sem{A_1, \dots, A_n}_\rho = \sem{A_1}_\rho \parr \cdots \parr \sem{A_n}_\rho$.
\end{definition}

\begin{theorem}
Let $\Gamma$ be a provable sequent in linear logic. Then $1_X \in \sem{\Gamma}$.
\end{theorem}

\begin{proof}
By induction on $\vdash\Gamma$.
\end{proof}

\section{Completeness}\label{completeness}

Phase semantics is complete w.r.t. linear logic. In order to prove this,
we need to build a particular commutative monoid.

\begin{definition}
We define the \emph{syntactic monoid} as follows:
\begin{itemize}
\item Its elements are sequents $\Gamma$ quotiented by the equivalence relation $\cong$ generated by the rules:
\begin{enumerate}
\item $\Gamma \cong \Delta$ if $\Gamma$ is a permutation of $\Delta$
\item $\wn{A}, \wn{A}, \Gamma \cong \wn{A}, \Gamma$
\end{enumerate}
\item Product is concatenation: $\Gamma \cdot \Delta := \Gamma, \Delta$
\item Neutral element is the empty sequent: $1 := \emptyset$.
\end{itemize}
\end{definition}

The equivalence relation intuitively means that we do not care about the
multiplicity of \(\wn\)-formulas.

\begin{lemma}
The syntactic monoid is indeed a commutative monoid.
\end{lemma}

\begin{definition}
The \emph{syntactic assignation} is the assignation that sends any variable $\alpha$ to the fact $\{\alpha\}\orth$.
\end{definition}

We instantiate the pole as \(\Bot := \{\Gamma \mid \vdash\Gamma\}\).

\begin{theorem}
If $\Gamma\in\sem{\Gamma}\orth$, then $\vdash\Gamma$.
\end{theorem}

\begin{proof}
By induction on $\Gamma$.
\end{proof}

\section{Cut elimination}\label{cut-elimination}

Actually, the completeness result is stronger, as the proof does not use
the cut-rule in the reconstruction of \(\vdash\Gamma\). By refining the
pole as the set of \emph{cut-free} provable formulas, we get:

\begin{theorem}
If $\Gamma\in\sem{\Gamma}\orth$, then $\Gamma$ is cut-free provable.
\end{theorem}

From soundness, one can retrieve the cut-elimination theorem.

\begin{corollary}
Linear logic enjoys the cut-elimination property.
\end{corollary}

% \section{The Rest}\label{the-rest}


%%% Local Variables:
%%% mode: latex
%%% TeX-master: "main"
%%% End:


\chapter{Categorical semantics}\label{categorical-semantics}

Constructing denotational models of linear logic can be a tedious work.
Categorical semantics are useful to identify the fundamental structure
of these models, and thus simplify and make more abstract the
elaboration of those models.

\begin{quotation}
\texttt{TODO: why categories? how to extract categorical models? etc.}
\end{quotation}

See~\cite{categoriesworkmath} for a more detailed introduction to category theory. See~\cite{catsemll} for a detailed treatment of categorical semantics of linear logic.

\section{Basic category theory recalled}\label{basic-category-theory-recalled}

\begin{definition}[Category]
\end{definition}

\begin{definition}[Functor]
\end{definition}

\begin{definition}[Natural transformation]
\end{definition}

\begin{definition}[Adjunction]
\end{definition}


\begin{definition}[Monad]
\end{definition}



\section{Overview}\label{overview}

In order to interpret the various \hyperref[fragment]{fragments} of linear
logic, we define incrementally what structure we need in a categorical
setting.

\begin{itemize}
\item The most basic underlying structure are \emph{symmetric monoidal
  categories} which model the symmetric tensor \(\otimes\) and its unit
  \(1\).
\item The \(\otimes, \multimap\) fragment (\wantedpage{IMLL}) is captured by
  so-called \emph{symmetric monoidal closed categories}.
\item Upgrading to \hyperref[intuitionistic-linear-logic]{ILL}, that is, adding the exponential \(\oc\)
  modality to IMLL requires modelling it categorically. There are
  various ways to do so: using rich enough \emph{adjunctions}, or with
  an ad-hoc definition of a well-behaved comonad which leads to
  \emph{linear categories} and close relatives.
\item Dealing with the additives \(\with, \oplus\) is quite easy, as they
  are plain \emph{cartesian product} and \emph{coproduct}, usually
  defined through universal properties in category theory.
\item Retrieving \(\parr\), \(\bot\) and \(\wn\) is just a matter of
  dualizing \(\otimes\), \(1\) and \(\oc\), thus requiring the model to
  be a \emph{*-autonomous category} for that purpose.
\end{itemize}

\section{\texorpdfstring{Modeling \wantedpage{IMLL}}{Modeling IMLL}}\label{modeling-imll}

A model of \wantedpage{IMLL} is a \emph{closed symmetric monoidal category}. We
recall the definition of these categories below.

\begin{definition}[Monoidal category]\label{monoidalcategory}
A \emph{monoidal category} $(\mathcal{C},\otimes,I,\alpha,\lambda,\rho)$ is a category $\mathcal{C}$ equipped with
\begin{itemize}
\item a functor $\otimes:\mathcal{C}\times\mathcal{C}\to\mathcal{C}$ called \emph{tensor product},
\item an object $I$ called \emph{unit object},
\item three natural isomorphisms $\alpha$, $\lambda$ and $\rho$, called respectively \emph{associator}, \emph{left unitor} and \emph{right unitor}, whose components are
\begin{equation*}
\alpha_{A,B,C}:(A\otimes B)\otimes C\to A\otimes (B\otimes C)
\qquad
\lambda_A:I\otimes A\to A
\qquad
\rho_A:A\otimes I\to A
\end{equation*}
\end{itemize}
such that
\begin{itemize}
\item for every objects $A,B,C,D$ in $\mathcal{C}$, the diagram
\begin{equation*}
\xymatrix{
    ((A\otimes B)\otimes C)\otimes D\ar[d]_{\alpha_{A\otimes B,C,D}}\ar[r]^{\alpha_{A,B,C}\otimes D}& (A\otimes(B\otimes C))\otimes D\ar[r]^{\alpha_{A,B\otimes C,D}}& A\otimes((B\otimes C)\otimes D)\ar[d]^{A\otimes\alpha_{B,C,D}}\\
    (A\otimes B)\otimes(C\otimes D)\ar[rr]_{\alpha_{A,B,C\otimes D}}& & A\otimes(B\otimes (C\otimes D))
}
\end{equation*}
commutes,
\item for every objects $A$ and $B$ in $\mathcal{C}$, the diagram
\begin{equation*}
\xymatrix{
    (A\otimes I)\otimes B\ar[dr]_{\rho_A\otimes B}\ar[rr]^{\alpha_{A,I,B}}& & \ar[dl]^{A\otimes\lambda_B}A\otimes(I\otimes B)\\
    & A\otimes B& 
}
\end{equation*}
commutes.
\end{itemize}
\end{definition}

\begin{definition}[Braided, symmetric monoidal category]
A \emph{braided} monoidal category is a category together with a natural isomorphism of components
\begin{equation*}
\gamma_{A,B}:A\otimes B\to B\otimes A
\end{equation*}
called \emph{braiding}, such that the two diagrams
\begin{equation*}
\xymatrix{
& A\otimes(B\otimes C)\ar[r]^{\gamma_{A,B\otimes C}}& (B\otimes C)\otimes A\ar[dr]^{\alpha_{B,C,A}}\\
(A\otimes B)\otimes C\ar[ur]^{\alpha_{A,B,C}}\ar[dr]_{\gamma_{A,B}\otimes C}& & & B\otimes (C\otimes A)\\
& (B\otimes A)\otimes C\ar[r]_{\alpha_{B,A,C}}& B\otimes(A\otimes C)\ar[ur]_{B\otimes\gamma_{A,C}}\\
}
\end{equation*}
and
\begin{equation*}
\xymatrix{
& (A\otimes B)\otimes C\ar[r]^{\gamma_{A\otimes B,C}}& C\otimes (A\otimes B)\ar[dr]^{\alpha^{-1}_{C,A,B}}& \\
A\otimes (B\otimes C)\ar[ur]^{\alpha^{-1}_{A,B,C}}\ar[dr]_{A\otimes\gamma_{B,C}}& & & (C\otimes A)\otimes B\\
& A\otimes(C\otimes B)\ar[r]_{\alpha^{-1}_{A,C,B}}& (A\otimes C)\otimes B\ar[ur]_{\gamma_{A,C}\otimes B}& \\
}
\end{equation*}
commute for every objects $A$, $B$ and $C$.

A \emph{symmetric} monoidal category is a braided monoidal category in which the braiding satisfies
\begin{equation*}
\gamma_{B,A}\circ\gamma_{A,B}=A\otimes B
\end{equation*}
for every objects $A$ and $B$.
\end{definition}

\begin{definition}[Closed monoidal category]\label{smcc}
A monoidal category $(\mathcal{C},\tens,I)$ is \emph{left closed} when for every object $A$, the functor $B\mapsto A\otimes B$ has a right adjoint, written $B\mapsto(A\limp B)$.
This means that there exists a bijection
\begin{equation*}
\mathcal{C}(A\tens B, C) \cong \mathcal{C}(B,A\limp C)
\end{equation*}
which is natural in $B$ and $C$.
Equivalently, a monoidal category is left closed when it is equipped with a \emph{left closed structure}, which consists of
\begin{itemize}
\item an object $A\limp B$,
\item a morphism $\mathrm{eval}_{A,B}:A\tens (A\limp B)\to B$, called \emph{left evaluation},
\end{itemize}
for every objects $A$ and $B$, such that for every morphism $f:A\otimes X\to B$ there exists a unique morphism $h:X\to A\limp B$ making the diagram
\begin{equation*}
\xymatrix{
A\tens X\ar@{.>}[d]_{A\tens h}\ar[dr]^{f}\\
A\tens(A\limp B)\ar[r]_-{\mathrm{eval}_{A,B}}& B
}
\end{equation*}
commute.

Dually, the monoidal category $\mathcal{C}$ is \emph{right closed} when the functor $B\mapsto B\otimes A$ admits a right adjoint. The notion of \emph{right closed structure} can be defined similarly.
\end{definition}

In a symmetric monoidal category, a left closed structure induces a
right closed structure and conversely, allowing us to simply speak of a
\emph{closed symmetric monoidal category}.

\section{Modeling the additives}\label{modeling-the-additives}

\begin{definition}[Product]\label{cartesianproduct}
A \emph{product} $(X,\pi_1,\pi_2)$ of two coinitial morphisms $f:A\to B$ and $g:A\to C$ in a category $\mathcal{C}$ is an object $X$ of $\mathcal{C}$ together with two morphisms $\pi_1:X\to B$ and $\pi_2:X\to C$ such that there exists a unique morphism $h:A\to X$ making the diagram
\begin{equation*}
\xymatrix{
& \ar[ddl]_fA\ar@{.>}[d]_h\ar[ddr]^g& \\
& \ar[dl]^{\pi_1}X\ar[dr]_{\pi_2}& \\
B& & C
}
\end{equation*}
commute.
\end{definition}

A category has \emph{finite products} when it has products and a terminal object.

\begin{definition}[Monoid]
A \emph{monoid} $(M,\mu,\eta)$ in a monoidal category $(\mathcal{C},\tens,I)$ is an object $M$ together with two morphisms $\mu:M\tens M \to M$ and $\eta:I\to M$
such that the diagrams
\begin{equation*}
\xymatrix{
& (M\tens M)\tens M\ar[dl]_{\alpha_{M,M,M}}\ar[r]^-{\mu\tens M}& M\tens M\ar[dd]^{\mu}\\
M\tens(M\tens M)\ar[d]_{M\tens\mu}& & \\
M\tens M\ar[rr]_{\mu}& & M\\
}
\end{equation*}
and
\begin{equation*}
\xymatrix{
I\tens M\ar[r]^{\eta\tens M}\ar[dr]_{\lambda_M}& M\tens M\ar[d]_\mu& \ar[l]_{M\tens\eta}\ar[dl]^{\rho_M}M\tens I\\
& M& 
}
\end{equation*}
commute.
\end{definition}

\begin{property}
Categories with products vs monoidal categories.
\end{property}


\section{\texorpdfstring{Modeling \hyperref[intuitionistic-linear-logic]{ILL}}{Modeling ILL}}\label{modeling-ill}

\begin{definition}[Linear-non linear (LNL) adjunction~\cite{mixedlnll}]
A \emph{linear-non linear adjunction} is a symmetric monoidal adjunction between lax monoidal functors
\begin{equation*}
\xymatrix{
(\mathcal{M},\times,\top)\ar@/^/[rr]^{(L,l)}& \bot& \ar@/^/[ll]^{(M,m)}(\mathcal{L},\otimes,I)
}
\end{equation*}
in which the category $\mathcal{M}$ has finite products.
\end{definition}

\[\oc=L\circ M\]

This section is devoted to defining the concepts necessary to define
these adjunctions.

\begin{definition}[Monoidal functor]
A \emph{lax monoidal functor} $(F,f)$ between two monoidal categories $(\mathcal{C},\tens,I)$ and $(\mathcal{D},\bullet,J)$ consists of
\begin{itemize}
\item a functor $F:\mathcal{C}\to\mathcal{D}$ between the underlying categories,
\item a natural transformation $f$ of components $f_{A,B}:FA\bullet FB\to F(A\tens B)$,
\item a morphism $f:J\to FI$
\end{itemize}
such that the diagrams
\begin{equation*}
\xymatrix{
    (FA\bullet FB)\bullet FC\ar[d]_{\phi_{A,B}\bullet FC}\ar[r]^{\alpha_{FA,FB,FC}}& FA\bullet(FB\bullet FC)\ar[dr]^{FA\bullet\phi_{B,C}}\\
    F(A\otimes B)\bullet FC\ar[dr]_{\phi_{A\otimes B,C}}& & FA\bullet F(B\otimes C)\ar[d]^{\phi_{A,B\otimes C}}\\
    & F((A\otimes B)\otimes C)\ar[r]_{F\alpha_{A,B,C}}& F(A\otimes(B\otimes C))
}
\end{equation*}
and
\begin{equation*}
\vcenter{\vbox{\xymatrix{
    FA\bullet J\ar[d]_{\rho_{FA}}\ar[r]^{FA\bullet\phi}& FA\bullet FI\ar[d]^{\phi_{A,I}}\\
    FA& \ar[l]^{F\rho_A}F(A\otimes I)
}}}
\qquad\text{and}\qquad
\vcenter{\vbox{\xymatrix{
    J\bullet FB\ar[d]_{\lambda_{FB}}\ar[r]^{\phi\bullet FB}& FI\bullet FB\ar[d]^{\phi_{I,B}}\\
    FB& \ar[l]^{F\lambda_B}F(I\otimes B)
}}}
\end{equation*}
commute for every objects $A$, $B$ and $C$ of $\mathcal{C}$. The morphisms $f_{A,B}$ and $f$ are called \emph{coherence maps}.

A lax monoidal functor is \emph{strong} when the coherence maps are invertible and \emph{strict} when they are identities.
\end{definition}

\begin{definition}[Monoidal natural transformation]
Suppose that $(\mathcal{C},\tens,I)$ and $(\mathcal{D},\bullet,J)$ are two monoidal categories and
$(F,f):(\mathcal{C},\tens,I)\Rightarrow(\mathcal{D},\bullet,J)$
and
$(G,g):(\mathcal{C},\tens,I)\Rightarrow(\mathcal{D},\bullet,J)$
are two monoidal functors between these categories. A \emph{monoidal natural transformation} $\theta:(F,f)\to (G,g)$ between these monoidal functors is a natural transformation $\theta:F\Rightarrow G$ between the underlying functors such that the diagrams
\begin{equation*}
\vcenter{\vbox{\xymatrix{
    FA\bullet FB\ar[d]_{f_{A,B}}\ar[r]^{\theta_A\bullet\theta_B}& \ar[d]^{g_{A,B}}GA\bullet GB\\
    F(A\tens B)\ar[r]_{\theta_{A\tens B}}& G(A\tens B)
}}}
\qquad\text{and}\qquad
\vcenter{\vbox{\xymatrix{
  & \ar[dl]_{f}J\ar[dr]^{g}& \\
  FI\ar[rr]_{\theta_I}& & GI
}}}
\end{equation*}
commute for every objects $A$ and $B$ of $\mathcal{D}$.
\end{definition}

\begin{definition}[Monoidal adjunction]
A \emph{monoidal adjunction} between two monoidal functors
$(F,f):(\mathcal{C},\tens,I)\Rightarrow(\mathcal{D},\bullet,J)$
and
$(G,g):(\mathcal{D},\bullet,J)\Rightarrow(\mathcal{C},\tens,I)$
is an adjunction between the underlying functors $F$ and $G$ such that the unit and the counit
$\eta:\mathcal{C}\Rightarrow G\circ F$ and $\varepsilon:F\circ G\Rightarrow\mathcal{D}$
induce monoidal natural transformations between the corresponding monoidal functors.
\end{definition}


\section{Modeling negation}\label{modeling-negation}

\subsection{*-autonomous categories}\label{autonomous-categories}

\begin{definition}[*-autonomous category]\label{staraut}
Suppose that we are given a symmetric monoidal closed category $(\mathcal{C},\tens,I)$ and an object $R$ of $\mathcal{C}$. For every object $A$, we define a morphism $\partial_{A}:A\to(A\limp R)\limp R$ as follows. By applying the bijection of the adjunction defining (left) closed monoidal categories to the identity morphism $\mathrm{id}_{A\limp R}:A\limp R \to A\limp R$, we get a morphism $A\tens (A\limp R)\to R$, and thus a morphism $(A\limp R)\tens A\to R$ by precomposing with the symmetry $\gamma_{A\limp R,A}$. The morphism $\partial_A$ is finally obtained by applying the bijection of the adjunction defining (left) closed monoidal categories to this morphism. The object $R$ is called \emph{dualizing} when the morphism $\partial_A$ is a bijection for every object $A$ of $\mathcal{C}$. A symmetric monoidal closed category is \emph{*-autonomous} when it admits such a dualizing object.
\end{definition}

\subsection{Compact closed categories}\label{compact-closed-categories}

\begin{definition}[Dual objects]
A \emph{dual object} structure $(A,B,\eta,\varepsilon)$ in a monoidal category $(\mathcal{C},\tens,I)$ is a pair of objects $A$ and $B$ together with two morphisms
$\eta:I\to B\otimes A$ and $\varepsilon:A\otimes B\to I$
such that the diagrams
\begin{equation*}
\xymatrix{
& A\tens(B\tens A)\ar[r]^{\alpha_{A,B,A}^{-1}}& (A\tens B)\tens A\ar[dr]^{\varepsilon\tens A}\\
A\tens I\ar[ur]^{A\tens\eta}& & & I\tens A\ar[d]^{\lambda_A}\\
A\ar[u]^{\rho_A^{-1}}\ar@{=}[rrr]& & & A\\
}
\end{equation*}
and
\begin{equation*}
\xymatrix{
& (B\tens A)\tens B\ar[r]^{\alpha_{B,A,B}}& B\tens(A\tens B)\ar[dr]^{B\tens\varepsilon}\\
I\tens B\ar[ur]^{\eta\tens B}& & & B\tens I\ar[d]^{\rho_B}\\
B\ar[u]^{\lambda_B^{-1}}\ar@{=}[rrr]& & & B\\
}
\end{equation*}
commute. The object $A$ is called a left dual of $B$ (and conversely $B$ is a right dual of $A$).
\end{definition}

\begin{lemma}
Two left (resp.\ right) duals of a same object $B$ are necessarily isomorphic.
\end{lemma}

\begin{definition}[Compact closed category]
A symmetric monoidal category $(\mathcal{C},\tens,I)$ is \emph{compact closed} when every object $A$ has a right dual $A^*$. We write
$\eta_A:I\to A^*\tens A$ and $\varepsilon_A:A\tens A^*\to I$
for the corresponding duality morphisms.
\end{definition}

\begin{lemma}
In a compact closed category the left and right duals of an object $A$ are isomorphic.
\end{lemma}

\begin{property}
A compact closed category $\mathcal{C}$ is monoidal closed, with closure defined by
$\mathcal{C}(A\tens B,C)\cong\mathcal{C}(B,A^*\tens C)$.
\end{property}

\begin{proof}
To every morphism $f:A\tens B\to C$, we associate a morphism $\ulcorner f\urcorner:B\to A^*\tens C$ defined as
\begin{equation*}
\xymatrix{
B\ar[r]^-{\lambda_B^{-1}}& I\tens B\ar[r]^-{\eta_A\tens B}& (A^*\tens A)\tens B\ar[r]^-{\alpha_{A^*,A,B}}& A^*\tens(A\tens B)\ar[r]^-{A^*\tens f}& A\tens C\\
}
\end{equation*}
and to every morphism $g:B\to A^*\tens C$, we associate a morphism $\llcorner g\lrcorner:A\tens B\to C$ defined as
\begin{equation*}
\xymatrix{
A\tens B\ar[r]^-{A\tens g}& A\tens(A^*\tens C)\ar[r]^-{\alpha_{A,A^*,C}^{-1}}& (A\tens A^*)\tens C\ar[r]^-{\varepsilon_A\tens C}& I\tens C\ar[r]^-{\lambda_C}& C
}
\end{equation*}
It is easy to show that $\llcorner \ulcorner f\urcorner\lrcorner=f$ and $\ulcorner\llcorner g\lrcorner\urcorner=g$ from which we deduce the required bijection.
\end{proof}

\begin{property}
A compact closed category is a (degenerated) *-autonomous category, with the obvious duality structure. In particular, $(A \otimes B)^* \cong A^*\otimes B^*$.
\end{property}

\begin{remark}
The above isomorphism does not hold in *-autonomous categories in general. This means that models which are compact closed categories identify $\otimes$ and $\parr$ as well as $1$ and $\bot$.
\end{remark}

\begin{proof}
The dualizing object $R$ is simply $I^*$.

For any $A$, the reverse isomorphism $\delta_A : (A \multimap R)\multimap R \rightarrow A$ is constructed as follows:
\begin{equation*}
\mathcal{C}((A \multimap R)\multimap R, A) := \mathcal{C}((A \otimes I^{**})\otimes I^{**}, A) \cong \mathcal{C}((A \otimes I)\otimes I, A) \cong \mathcal{C}(A, A)
\end{equation*}

Identity on $A$ is taken as the canonical morphism required.
\end{proof}


\section{Other categorical models}\label{other-categorical-models}

\subsection{Lafont categories}\label{lafont-categories}

\subsection{Seely categories}\label{seely-categories}

\subsection{Linear categories}\label{linear-categories}

\section{Properties of categorical models}\label{properties-of-categorical-models}

\subsection{The Kleisli category}\label{the-kleisli-category}

%%% Local Variables:
%%% mode: latex
%%% TeX-master: "main"
%%% End:


\chapter{Relational semantics}\label{relational-semantics}

This is the simplest denotational semantics of linear logic. It consists
in interpreting a formula \(A\) as a set \(A^*\) and a proof \(\pi\) of
\(A\) as a subset \(\pi^*\) of \(A^*\).

\section{The category of sets and relations}\label{the-category-of-sets-and-relations}

It is the category \(\mathbf{Rel}\) whose objects are sets, and such
that \(\mathbf{Rel}(X,Y)=\powerset{X\times Y}\). Composition is the
ordinary composition of relations: given \(s\in\mathbf{Rel}(X,Y)\) and
\(t\in\mathbf{Rel}(Y,Z)\), one sets
\(t\circ s=\set{(a,c)\in X\times Z}{\exists b\in Y\ (a,b)\in s\ \text{and}\ (b,c)\in t}\)
and the identity morphism is the diagonal relation
\(\mathsf{Id}_X=\set{(a,a)}{a\in X}\).

An isomorphism in the category \(\mathbf{Rel}\) is a relation which is a
bijection, as easily checked.

\subsection{Monoidal structure}\label{monoidal-structure}

The tensor product is the usual cartesian product of sets
\(X\tens Y=X\times Y\) (which is not a cartesian product in the category
\(\mathbf{Rel}\) in the categorical sense). It is a bifunctor: given
\(s_i\in\mathbf{Rel}(X_i,Y_i)\) (for \(i=1,2\)), one sets
\(s_1\tens s_2=\set{((a_1,a_2),(b_1,b_2))}{(a_i,b_i)\in s_i\ \text{for}\ i=1,2}\).
The unit of this tensor product is \(\one=\{*\}\) where \(*\) is an
arbitrary element.

For defining a monoidal category, it is not sufficient to provide the
definition of the tensor product functor \(\tens\) and its unit
\(\one\), one has also to provide natural isomorphisms
\(\lambda_X\in\mathbf{Rel}(\one\tens X,X)\),
\(\rho_X\in\mathbf{Rel}(X\tens\one,X)\) (left and right neutrality of
\(\one\) for \(\tens\)) and
\(\alpha_{X,Y,Z}\in\mathbf{Rel}((X\tens Y)\tens Z,X\tens(Y\tens Z))\)
(associativity of \(\tens\)). All these isomorphisms have to satisfy a
number of commutations. In the present case, they are defined in the
obvious way.

This monoidal category \((\mathbf{Rel},\tens,\one,\lambda,\rho)\) is
symmetric, meaning that it is endowed with an additional natural
isomorphism \(\sigma_{X,Y}\in\mathbf{Rel}(X\tens Y,Y\tens X)\), also
subject to some commutations. Here, again, this isomorphism is defined
in the obvious way (symmetry of the cartesian product). So, to be
precise, the SMCC (symmetric monoidal closed category) \(\mathbf{Rel}\)
is the tuple \((\mathbf{Rel},\tens,\one,\lambda,\rho,\alpha,\sigma)\),
but we shall simply denote it as \(\mathbf{Rel}\).

The SMCC \(\mathbf{Rel}\) is closed. This means that, given any object
\(X\) of \(\mathbf{Rel}\) (a set), the functor \(Z\mapsto Z\tens X\)
(from \(\mathbf{Rel}\) to \(\mathbf{Rel}\)) admits a right adjoint
\(Y\mapsto (X\limp Y)\) (from \(\mathbf{Rel}\) to \(\mathbf{Rel}\)). In
other words, for any objects \(X\) and \(Y\), we are given an object
\(X\limp Y\) and a morphism
\(\mathsf{ev}_{X,Y}\in\mathbf{Rel}((X\limp Y)\tens X,Y)\) with the
following universal property: for any morphism
\(s\in\mathbf{Rel}(Z\tens X,Y)\), there is a unique morphism
\(\mathsf{fun}(s)\in\mathbf{Rel}(Z,X\limp Y)\) such that
\(\mathsf{ev}_{X,Y}\circ(\mathsf{fun}(s)\tens\mathsf{Id}_X)=s\).

The definition of all these data is quite simple in \(\mathbf{Rel}\):
\(X\limp Y=X\times Y\),
\(\mathsf{ev}_{X,Y}=\set{(((a,b),a),b)}{(a,b)\in X\times Y}\) and
\(\mathsf{fun}(s)=\set{(c,(a,b))}{((c,a),b)\in s}\).

Let \(\bot=\one=\{*\}\). Then we have
\(\mathsf{ev}\circ\sigma:\mathbf{Rel}(X\tens(X\limp\bot),\bot)\) and
hence
\(\eta_X=\mathsf{fun}(\mathsf{ev}\circ\sigma)\in\mathbf{Rel}(X,(X\limp\bot)\limp\bot)\).
It is clear that \(\eta=\set{(a,((a,*),*))}{a\in X}\) and hence \(\eta\)
is a natural isomorphism: one says that the SMCC \(\mathbf{Rel}\) is a
*-autonomous category, with \(\bot\) as dualizing object.

\subsection{Additives}\label{additives-2}

Given a family \((X_i)_{i\in I}\), let
\(\with_{i\in I}X_i=\cup_{i\in I}\{i\}\times X_i\). Let
\(\pi_j\in\mathbf{Rel}(\with_{i\in I}X_i,X_j)\) given by
\(\pi_j=\set{((j,a),a)}{a\in X_j}\). Then
\((\with_{i\in I}X_i,(\pi_i)_{i\in I})\) is the cartesian product of the
\(X_i\)s in the category \(\mathbf{Rel}\).

\subsection{Exponentials}\label{exponentials-4}

One defines \(\oc X\) as the set of all finite multisets of elements of
\(X\). Given \(s\in\mathbf{Rel}(X,Y)\), one defines
\(\oc s=\set{([a_1,\dots,a_n],[b_1,\dots,b_n])}{n\in\mathbb N\ \text{and}\ \forall i\ (a_i,b_i)\in s}\)
where \([a_1,\dots,a_n]\) is the multiset containing \(a_1,\dots,a_n\),
taking multiplicities into account. This defines a functor
\(\mathbf{Rel}\to\mathbf{Rel}\), that we endow with a comonad structure
as follows:

\begin{itemize}
\tightlist
\item
  the counit, called dereliction, is the natural transformation
  \(\mathsf{der}_X\in\mathbf{Rel}(\oc X,X)\), given by
  \(\mathsf{der}_X=\set{([a],a)}{a\in X}\)
\item
  the comultiplication, called digging, is the natural transformation
  \(\mathsf{digg}_X\in\mathbf{Rel}(\oc X,\oc\oc X)\), given by
  \(\mathsf{digg}_X=\set{(m_1+\cdots+m_n,[m_1,\dots,m_n])}{n\in\mathbb N\ \text{and}\ m_1,\dots,m_n\in\oc X}\)
\end{itemize}

\section{\texorpdfstring{Interpretation of propositional linear logic (\(LL_0\))}{Interpretation of propositional linear logic (LL\_0)}}\label{interpretation-of-propositional-linear-logic-ll_0}

The structure described above gives rise to the following interpretation
of formulas and proofs of linear logic.

For all propositional variable \(X\), fix a set \(\web X\). Then with
each formula \(A\), we associate a set \(\web A\) as follows:

\begin{itemize}
\tightlist
\item
  \(\web{A\orth}=\web A\);
\item
  \(\web{A\tens B}=\web{A\parr B}=\web A\times\web B\);
\item
  \(\web{A\with B}=\web{A\plus B}=(\{1\}\times\web A)\cup(\{2\}\times\web B)\);
\item
  \(\web{\oc A}=\web{\wn A}=\finmulset{\web A}\).
\end{itemize}

We then interpret the proofs of \(LL_0\) as follows: with each proof
\(\pi\) of sequent \(\vdash A_1,\ldots,A_n\), we associate a subset
\(\sem\pi\subseteq\web{A_1}\times\cdots\times\web{A_n}\).

\begin{itemize}
\tightlist
\item
  Identity group:

  \begin{description}
  \tightlist
  \item[]
  \textbackslash{}sem\{
  \end{description}
\end{itemize}

\textbackslash{}LabelRule\{ \textbackslash{}rulename\{axiom\} \}
\textbackslash{}NulRule\{ \textbackslash{}vdash A\textbackslash{}orth, A
\}
\textbackslash{}DisplayProof\}=\textbackslash{}set\{(a,a)\}\{a\textbackslash{}in\textbackslash{}web
A\}

*: \(\sem{
\AxRule{}
\VdotsRule{ \pi }{ \vdash \Gamma, A }
\AxRule{}
\VdotsRule{ \rho }{ \vdash \Delta, A\orth }
\LabelRule{ \rulename{cut} }
\BinRule{ \vdash \Gamma, \Delta }
\DisplayProof} = \set{(\gamma,\delta)}{\exists a\in\web A,\ (\gamma,a)\in\sem\pi\land(\delta,a)\in\sem\rho}\)

\begin{itemize}
\tightlist
\item
  Multiplicative group:

  \begin{description}
  \tightlist
  \item[]
  \end{description}
\end{itemize}

\textbackslash{}sem\{ \textbackslash{}AxRule\{\}
\textbackslash{}VdotsRule\{ \textbackslash{}pi \}\{
\textbackslash{}vdash \textbackslash{}Gamma, A \}
\textbackslash{}AxRule\{\} \textbackslash{}VdotsRule\{
\textbackslash{}rho \}\{ \textbackslash{}vdash \textbackslash{}Delta, B
\} \textbackslash{}LabelRule\{ \textbackslash{}tens \}
\textbackslash{}BinRule\{ \textbackslash{}vdash \textbackslash{}Gamma,
\textbackslash{}Delta, A \textbackslash{}tens B \}
\textbackslash{}DisplayProof\} =
\textbackslash{}set\{(\textbackslash{}gamma,\textbackslash{}delta,a,b)\}\{(\textbackslash{}gamma,a)\textbackslash{}in\textbackslash{}sem\textbackslash{}pi\textbackslash{}land(\textbackslash{}delta,b)\textbackslash{}in\textbackslash{}sem\textbackslash{}rho\}

*: \(\sem{
\AxRule{ }
\VdotsRule{ \pi }{ \vdash \Gamma, A, B }
\LabelRule{ \parr }
\UnaRule{ \vdash \Gamma, A \parr B }
\DisplayProof} =  \set{(\gamma,(a,b))}{(\gamma,a,b)\in\sem\pi}\)

*: \(\sem{
\LabelRule{ \one }
\NulRule{ \vdash \one }
\DisplayProof} = \{ * \}\)

*: \(\sem{
\AxRule{}
\VdotsRule{ \pi }{ \vdash \Gamma }
\LabelRule{ \bot }
\UnaRule{ \vdash \Gamma, \bot }
\DisplayProof} = \set{(\gamma,*)}{\gamma\in\sem\pi}\)

\begin{itemize}
\tightlist
\item
  Additive group:

  \begin{description}
  \tightlist
  \item[]
  \end{description}
\end{itemize}

\textbackslash{}sem\{ \textbackslash{}AxRule\{\}
\textbackslash{}VdotsRule\{ \textbackslash{}pi \}\{
\textbackslash{}vdash \textbackslash{}Gamma, A \}
\textbackslash{}LabelRule\{ \textbackslash{}plus\_1 \}
\textbackslash{}UnaRule\{ \textbackslash{}vdash \textbackslash{}Gamma, A
\textbackslash{}plus B \} \textbackslash{}DisplayProof\} =
\textbackslash{}set\{(\textbackslash{}gamma,(1,a))\}\{(\textbackslash{}gamma,a)\textbackslash{}in\textbackslash{}sem\textbackslash{}pi\}

*: \(\sem{
\AxRule{}
\VdotsRule{ \pi }{ \vdash \Gamma, B }
\LabelRule{ \plus_2 }
\UnaRule{ \vdash \Gamma, A \plus B }
\DisplayProof} = \set{(\gamma,(2,b))}{(\gamma,b)\in\sem\pi}\)

*: \(\sem{
\AxRule{}
\VdotsRule{ \pi }{ \vdash \Gamma, A }
\AxRule{}
\VdotsRule{ \rho }{ \vdash \Gamma, B }
\LabelRule{ \with }
\BinRule{ \vdash \Gamma, A \with B }
\DisplayProof} = \set{(\gamma,(1,a))}{(\gamma,a)\in\sem\pi} \cup \set{(\gamma,(2,b))}{(\gamma,b)\in\sem\rho}\)

*: \(\sem{
\LabelRule{ \top }
\NulRule{ \vdash \Gamma, \top }
\DisplayProof} = \emptyset\)

\begin{itemize}
\tightlist
\item
  Exponential group:

  \begin{description}
  \tightlist
  \item[]
  \end{description}
\end{itemize}

\textbackslash{}sem\{ \textbackslash{}AxRule\{\}
\textbackslash{}VdotsRule\{ \textbackslash{}pi \}\{
\textbackslash{}vdash \textbackslash{}Gamma, A \}
\textbackslash{}LabelRule\{ d \} \textbackslash{}UnaRule\{
\textbackslash{}vdash \textbackslash{}Gamma, \textbackslash{}wn A \}
\textbackslash{}DisplayProof\} =
\textbackslash{}set\{(\textbackslash{}gamma,{[}a{]})\}\{(\textbackslash{}gamma,a)\textbackslash{}in\textbackslash{}sem\textbackslash{}pi\}

*: \(\sem{
\AxRule{}
\VdotsRule{ \pi }{ \vdash \Gamma }
\LabelRule{ w }
\UnaRule{ \vdash \Gamma, \wn A }
\DisplayProof} =  \set{(\gamma,[])}{\gamma\in\sem\pi}\)

*: \(\sem{
\AxRule{}
\VdotsRule{ \pi }{ \vdash \Gamma, \wn A, \wn A }
\LabelRule{ c }
\UnaRule{ \vdash \Gamma, \wn A }
\DisplayProof} =  \set{(\gamma,m+n)}{(\gamma,m,n)\in\sem\pi}\)

*: \(\sem{
\AxRule{}
\VdotsRule{ \pi }{ \vdash \wn A_1,\ldots,\wn A_n,B }
\LabelRule{ \oc }
\UnaRule{ \vdash \wn A_1,\ldots,\wn A_n,\oc B }
\DisplayProof} = \set{
\left(\sum_{i=1}^k m_1^i,\ldots,\sum_{i=1}^k m_n^i,[b_1,\ldots,b_k]\right)}
{k \in \mathbb{N} \ \text{and} \ \forall 1 \leq i \leq k,\ (m_1^i,\ldots,m_n^i,b_i)\in\sem\pi}\)


%%% Local Variables:
%%% mode: latex
%%% TeX-master: "main"
%%% End:


\chapter{Finiteness semantics}\label{finiteness-semantics}

The category \(\mathbf{Fin}\) of finiteness spaces and finitary
relations was introduced by Ehrhard, refining the
\hyperref[relational-semantics]{purely relational model of linear logic}. A
finiteness space is a set equipped with a finiteness structure, i.e. a
particular set of subsets which are said to be finitary; and the model
is such that the usual relational denotation of a proof in linear logic
is always a finitary subset of its conclusion. By the usual co-Kleisli
construction, this also provides a model of the simply typed
lambda-calculus: the cartesian closed category \(\mathbf{Fin}_\oc\).

The main property of finiteness spaces is that the intersection of two
finitary subsets of dual types is always finite. This feature allows to
reformulate Girard's quantitative semantics in a standard algebraic
setting, where morphisms interpreting typed \(\lambda\)-terms are
analytic functions between the topological vector spaces generated by
vectors with finitary supports. This provided the semantical foundations
of Ehrhard-Regnier's differential \(\lambda\)-calculus and motivated the
general study of a differential extension of linear logic.

It is worth noticing that finiteness spaces can accomodate typed
\(\lambda\)-calculi only: for instance, the relational semantics of
fixpoint combinators is never finitary. The whole point of the
finiteness construction is actually to reject infinite computations.
Indeed, from a logical point of view, computation is cut elimination:
the finiteness structure ensures the intermediate sets involved in the
relational interpretation of a cut are all finite. In that sense, the
finitary semantics is intrinsically typed.

\section{Finiteness spaces}\label{finiteness-spaces}

The construction of finiteness spaces follows a well known pattern. It
is given by the following notion of orthogonality: \(a\mathrel \bot a'\)
iff \(a\cap a'\) is finite. Then one unrolls \hyperref[orthogonality-relation]{familiar definitions}, as we do in the following paragraphs.

Let \(A\) be a set. Denote by \(\powerset A\) the powerset of \(A\) and
by \(\finpowerset A\) the set of all finite subsets of \(A\). Let
\({\mathfrak F} \subseteq \powerset A\) any set of subsets of \(A\). We
define the pre-dual of \({\mathfrak F}\) in \(A\) as
\({\mathfrak F}^{\bot_{A}}=\left\{a'\subseteq A;\ \forall a\in{\mathfrak F},\ a\cap a'\in\finpowerset A\right\}\).
In general we will omit the subscript in the pre-dual notation and just
write \({\mathfrak F}\orth\). For all
\({\mathfrak F}\subseteq\powerset A\), we have the following immediate
properties: \(\finpowerset A\subseteq {\mathfrak F}\orth\);
\({\mathfrak F}\subseteq {\mathfrak F}\biorth\); if
\({\mathfrak G}\subseteq{\mathfrak F}\),
\({\mathfrak F}\orth\subseteq {\mathfrak G}\orth\). By the last two, we
get \({\mathfrak F}\orth = {\mathfrak F}\triorth\). A finiteness
structure on \(A\) is then a set \({\mathfrak F}\) of subsets of \(A\)
such that \({\mathfrak F}\biorth = {\mathfrak F}\).

A finiteness space is a dependant pair
\({\mathcal A}=\left(\web{\mathcal A},\mathfrak F\left(\mathcal A\right)\right)\)
where \(\web {\mathcal A}\) is the underlying set (the web of
\({\mathcal A}\)) and \(\mathfrak F\left(\mathcal A\right)\) is a
finiteness structure on \(\web {\mathcal A}\). We then write
\({\mathcal A}\orth\) for the dual finiteness space:
\(\web {{\mathcal A}\orth} = \web {\mathcal A}\) and
\(\mathfrak F\left({\mathcal A}\orth\right)=\mathfrak F\left({\mathcal A}\right)^{\bot}\).
The elements of \(\mathfrak F\left(\mathcal A\right)\) are called the
finitary subsets of \({\mathcal A}\).

\begin{example}\label{example.}
For all set \(A\), \((A,\finpowerset A)\) is a finiteness space and \((A,\finpowerset A)\orth = (A,\powerset A)\). In particular, each finite set \(A\) is the web of exactly one finiteness space: \((A,\finpowerset A)=(A,\powerset A)\). We introduce the following two: \(\zero = \zero\orth = \left(\emptyset, \{\emptyset\}\right)\) and \(\one = \one\orth = \left(\{\emptyset\}, \{\emptyset, \{\emptyset\}\}\right)\). We also introduce the finiteness space of natural numbers \({\mathcal N}\) by: \(|{\mathcal N}|={\mathbf N}\) and \(a\in\mathfrak F\left(\mathcal N\right)\) iff \(a\) is finite. We write \(\mathcal O=\{0\}\in\mathfrak F\left({\mathcal N}\right)\).

Notice that \({\mathfrak F}\) is a finiteness structure iff it is of the
form \({\mathfrak G}\orth\). It follows that any finiteness structure
\({\mathfrak F}\) is downwards closed for inclusion, and closed under
finite unions and arbitrary intersections. Notice however that
\({\mathfrak F}\) is not closed under directed unions in general: for
all \(k\in{\mathbf N}\), write
\(k{\downarrow}=\left\{j;\  j\le k\right\}\in\mathfrak F\left({\mathcal N}\right)\);
then \(k{\downarrow}\subseteq k'{\downarrow}\) as soon as \(k\le k'\),
but
\(\bigcup_{k\ge0} k{\downarrow}={\mathbf N}\not\in\mathfrak F\left({\mathcal N}\right)\).
\end{example}

\subsection{Multiplicatives}\label{multiplicatives}

For all finiteness spaces \({\mathcal A}\) and \({\mathcal B}\), we
define \({\mathcal A} \tens {\mathcal B}\) by
\(\web {{\mathcal A} \tens {\mathcal B}} = \web{\mathcal A} \times \web{\mathcal B}\)
and
\(\mathfrak F\left({\mathcal A} \tens {\mathcal B}\right) = \left\{a\times b;\ a\in \mathfrak F\left(\mathcal A\right),\ b\in\mathfrak F\left(\mathcal B\right)\right\}\biorth\).

It can be shown that
\(\mathfrak F\left({\mathcal A} \tens {\mathcal B}\right) = \left\{ c \subseteq \web{\mathcal A}\times\web{\mathcal B};\  \left.c\right|_l\in \mathfrak F\left(\mathcal A\right),\ \left.c\right|_r\in\mathfrak F\left(\mathcal B\right)\right\}\),
where \(\left.c\right|_l\) and \(\left.c\right|_r\) are the obvious
projections.

Let \(f\subseteq A \times B\) be a relation from \(A\) to \(B\), we
write \(f\orth=\left\{(\beta,\alpha);\  (\alpha,\beta)\in f\right\}\).
For all \(a\subseteq A\), we set
\(f\cdot a = \left\{\beta\in B;\  \exists \alpha\in a,\ (\alpha,\beta)\in f\right\}\).
If moreover \(g\subseteq B \times C\), we define
\(g \bullet f = \left\{(\alpha,\gamma)\in A\times C;\  \exists \beta\in B,\ (\alpha,\beta)\in f\wedge(\beta,\gamma)\in g\right\}\).
Then, setting
\({\mathcal A}\limp{\mathcal B} = \left({\mathcal A}\otimes {\mathcal B}\orth\right)\orth\),
\(\mathfrak F\left({\mathcal A}\limp{\mathcal B}\right)\subseteq {\web{\mathcal A}\times\web{\mathcal B}}\)
is characterized as follows:
\begin{align*}
        f\in \mathfrak F\left({\mathcal A}\limp{\mathcal B}\right) &\iff \forall a\in \mathfrak F\left({\mathcal A}\right), f\cdot a \in\mathfrak F\left({\mathcal B}\right) \text{ and } \forall b\in \mathfrak F\left({\mathcal B}\orth\right), f\orth\cdot b \in\mathfrak F\left({\mathcal A}\orth\right)
        \\
        &\iff \forall a\in \mathfrak F\left({\mathcal A}\right), f\cdot a \in\mathfrak F\left({\mathcal B}\right) \text{ and } \forall \beta\in \web{{\mathcal B}}, f\orth\cdot \left\{\beta\right\} \in\mathfrak F\left({\mathcal A}\orth\right)
        \\
        &\iff \forall \alpha\in \web{{\mathcal A}}, f\cdot \left\{\alpha\right\} \in\mathfrak F\left({\mathcal B}\right) \text{ and } \forall b\in \mathfrak F\left({\mathcal B}\orth\right), f\orth\cdot b \in\mathfrak F\left({\mathcal A}\orth\right)
\end{align*}
The elements of
\(\mathfrak F\left({\mathcal A}\limp{\mathcal B}\right)\) are called
finitary relations from \({\mathcal A}\) to \({\mathcal B}\). By the
previous characterization, the identity relation
\(\mathsf{id}_{{\mathcal A}} = \left\{(\alpha,\alpha);\  \alpha\in\web{{\mathcal A}}\right\}\)
is finitary, and the composition of two finitary relations is also
finitary. One can thus define the category \(\mathbf{Fin}\) of
finiteness spaces and finitary relations: the objects of
\(\mathbf{Fin}\) are all finiteness spaces, and
\(\mathbf{Fin}({\mathcal A},{\mathcal B})=\mathfrak F\left({\mathcal A}\limp{\mathcal B}\right)\).
Equipped with the tensor product \(\tens\), \(\mathbf{Fin}\) is
symmetric monoidal, with unit \(\one\); it is monoidal closed by the
definition of \(\limp\); it is \(*\)-autonomous by the obvious
isomorphism between \({\mathcal A}\orth\) and \({\mathcal A}\limp\one\).

\begin{example}\label{example.-1}
Setting \(\mathcal{S}=\left\{(k,k+1);\  k\in{\mathbf N}\right\}\) and \(\mathcal{P}=\left\{(k+1,k);\  k\in{\mathbf N}\right\}\), we have \(\mathcal{S},\mathcal{P}\in\mathbf{Fin}({\mathcal N},{\mathcal N})\) and \(\mathcal{P}\bullet\mathcal{S}=\mathsf{id}_{{\mathcal N}}\).
\end{example}

\subsection{Additives}\label{additives}

We now introduce the cartesian structure of \(\mathbf{Fin}\). We define
\({\mathcal A} \oplus {\mathcal B}\) by
\(\web {{\mathcal A} \oplus {\mathcal B}} = \web{\mathcal A} \uplus \web{\mathcal B}\)
and
\(\mathfrak F\left({\mathcal A} \oplus {\mathcal B}\right) = \left\{ a\uplus b;\  a\in \mathfrak F\left(\mathcal A\right),\ b\in\mathfrak F\left(\mathcal B\right)\right\}\)
where \(\uplus\) denotes the disjoint union of sets:
\(x\uplus y=(\{1\}\times x)\cup(\{2\}\times y)\). We have
\(\left({\mathcal A}\oplus {\mathcal B}\right)\orth = {\mathcal A}\orth\oplus{\mathcal B}\orth\).\footnote{The
  fact that the additive connectives are identified, \ie\ that we obtain
  a biproduct, is to be related with the enrichment of \(\mathbf{Fin}\)
  over the monoid structure of set union: see~\cite{differentialstructurebiadditive}. This identification can
  also be shown to be a \hyperref[isomorphism]{isomorphism} of LL with sums of proofs.}
The category \(\mathbf{Fin}\) is both cartesian and co-cartesian, with
\(\oplus\) being the product and co-product, and \(\zero\) the initial
and terminal object. Projections are given by:
\begin{align*}
\lambda_{{\mathcal A},{\mathcal B}}&=\left\{\left((1,\alpha),\alpha\right);\ \alpha\in\web{\mathcal A}\right\}
\in\mathbf{Fin}({\mathcal A}\oplus{\mathcal B},{\mathcal A}) \\
\rho_{{\mathcal A},{\mathcal B}}&=\left\{\left((2,\beta),\beta\right);\ \beta\in\web{\mathcal B}\right\}
\in\mathbf{Fin}({\mathcal A}\oplus{\mathcal B},{\mathcal B}) 
\end{align*}
and if \(f\in\mathbf{Fin}({\mathcal C},{\mathcal A})\) and
\(g\in\mathbf{Fin}({\mathcal C},{\mathcal B})\), pairing is given by:
\begin{equation*}
\left\langle f,g\right\rangle = \left\{\left(\gamma,(1,\alpha)\right);\ (\gamma,\alpha)\in f\right\} \cup \left\{\left(\gamma,(2,\beta)\right);\ (\gamma,\beta)\in g\right\} \in\mathbf{Fin}({\mathcal C},{\mathcal A}\oplus{\mathcal B}).
\end{equation*}

The unique morphism from \({\mathcal A}\) to \(\zero\) is the empty
relation. The co-cartesian structure is obtained symmetrically.

\begin{example}\label{example.-2}
Write \({\mathcal O}\orth=\left\{(0,\emptyset)\right\}\in\mathbf{Fin}({\mathcal N},\one)\). Then \(\left\langle{{\mathcal O}\orth},{\mathcal{P}}\right\rangle =\{ (0,(1,\emptyset)) \}\cup \{ (k+1,(2,k)) ;\  k\in{\mathbf N} \} \in\mathbf{Fin}\left({\mathcal N},\one\oplus{\mathcal N}\right)\) is an isomorphism.
\end{example}

\subsection{Exponentials}\label{exponentials-1}

If \(A\) is a set, we denote by \(\finmulset A\) the set of all finite
multisets of elements of \(A\), and if \(a\subseteq A\), we write
\(a^{\oc}=\finmulset a\subseteq\finmulset A\). If
\(\overline\alpha\in\finmulset A\), we denote its support by
\(\mathrm{Support}\left(\overline \alpha\right)\in\finpowerset A\). For
all finiteness space \({\mathcal A}\), we define \(\oc {\mathcal A}\)
by: \(\web{\oc {\mathcal A}}= \finmulset{\web{{\mathcal A}}}\) and
\(\mathfrak F\left(\oc{\mathcal A}\right)=\left\{a^{\oc};\  a\in\mathfrak F\left({\mathcal A}\right)\right\}\biorth\).
It can be shown that
\(\mathfrak F\left(\oc{\mathcal A}\right) = \left\{\overline a\subseteq\finmulset{\web{{\mathcal A}}};\ \bigcup_{\overline\alpha\in \overline a}\mathrm{Support}\left(\overline \alpha\right)\in\mathfrak F\left(\mathcal A\right)\right\}\).
Then, for all \(f\in\mathbf{Fin}({\mathcal A},{\mathcal B})\), we set
\begin{equation*}
\oc f =\left\{\left(\left[\alpha_1,\ldots,\alpha_n\right],\left[\beta_1,\ldots,\beta_n\right]\right);\  \forall i,\ (\alpha_i,\beta_i)\in f\right\} \in \mathbf{Fin}(\oc {\mathcal A}, \oc {\mathcal B}),
\end{equation*}
which defines a functor. Natural transformations
\(\mathsf{der}_{{\mathcal A}}=\left\{([\alpha],\alpha);\  \alpha\in \web{{\mathcal A}}\right\}\in\mathbf{Fin}(\oc{\mathcal A},{\mathcal A})\)
and
\(\mathsf{digg}_{{\mathcal A}}=\left\{\left(\sum_{i=1}^n\overline\alpha_i,\left[\overline\alpha_1,\ldots,\overline\alpha_n\right]\right);\ \forall i,\ \overline\alpha_i\in\web{\oc {\mathcal A}}\right\}\)
make this functor a comonad.

\begin{example}\label{example.-3}
We have isomorphisms:
\begin{align*}
\left\{([],\emptyset)\right\}&\in\mathbf{Fin}(\oc\zero,\one)
\\
\left\{ \left(\overline\alpha_l+\overline\beta_r,\left(\overline\alpha,\overline\beta\right)\right);\ (\overline\alpha_l,\overline\alpha)\in\oc\lambda_{{\mathcal A},{\mathcal B}}\wedge(\overline\beta_r,\overline\beta)\in\oc\rho_{{\mathcal A},{\mathcal B}}\right\} &\in\mathbf{Fin}(\oc({\mathcal A}\oplus{\mathcal B}),\oc{\mathcal A}\tens\oc{\mathcal B}).
\end{align*}
More generally, we have
\(\oc\left({\mathcal A}_1\oplus\cdots\oplus{\mathcal A}_n\right)\cong\oc{\mathcal A}_1\tens\cdots\tens\oc{\mathcal A}_n\).
\end{example}


%%% Local Variables:
%%% mode: latex
%%% TeX-master: "main"
%%% End:


\chapter{Geometry of interaction}\label{geometry-of-interaction}

\section{Introduction}

The \emph{geometry of interaction}, GoI in short, was defined in the
early nineties by Girard as an interpretation of linear logic into
operators algebra: formulas were interpreted by Hilbert spaces and
proofs by partial isometries.

This was a striking novelty as it was the first time that a mathematical
model of logic (lambda-calculus) didn't interpret a proof of
\(A\limp B\) as a morphism \emph{from} \(A\) \emph{to} \(B\) and proof
composition (cut rule) as the composition of morphisms. Rather the proof
was interpreted as an operator acting \emph{on} \(A\limp B\), that is a
morphism from \(A\limp B\) to \(A\limp B\). For proof composition the
problem was then, given an operator on \(A\limp B\) and another one on
\(B\limp C\) to construct a new operator on \(A\limp C\). This problem
was solved by the \emph{execution formula} that bares some formal
analogies with Kleene's formula for recursive functions. For this reason
GoI was claimed to be an \emph{operational semantics}, as opposed to
traditionnal \hyperref[semantics]{denotational semantics}.

The first instance of the GoI was restricted to the {\(MELL\) fragment of
linear logic (\hyperref[exponential-fragments]{Multiplicative and Exponential fragment}) which is enough
to encode lambda-calculus. Since then Girard proposed several
improvements: firstly the extension to the additive connectives known as
\emph{Geometry of Interaction 3} and more recently a complete
reformulation using Von Neumann algebras that allows to deal with some
aspects of \hyperref[light-linear-logics]{implicit complexity}

The GoI has been a source of inspiration for various authors. Danos and
Regnier have reformulated the original model exhibiting its
combinatorial nature using a theory of reduction of paths in proof-nets
and showing the link with abstract machines; the execution formula
appears as the composition of two automata interacting through a common
interface. Also the execution formula has rapidly been understood as
expressing the composition of strategies in game semantics. It has been
used in the theory of sharing reduction for lambda-calculus in the
Abadi-Gonthier-Lévy reformulation and simplification of Lamping's
representation of sharing. Finally the original GoI for the \(MELL\)
fragment has been reformulated in the framework of traced monoidal
categories following an idea originally proposed by Joyal.


\subsection{The Geometry of Interaction as operators}\label{the-geometry-of-interaction-as-operators}

The original construction of GoI by Girard follows a general pattern
already mentionned in the section on \hyperref[coherent-semantics]{coherent
semantics} under the name \emph{symmetric reducibility} and that was
first put to use in \hyperref[phase-semantics]{phase semantics}. First set a
general space \(P\) called the \emph{proof space} because this is where
the interpretations of proofs will live. Make sure that \(P\) is a (not
necessarily commutative) monoid. In the case of GoI, the proof space is
a subset of the space of bounded operators on \(\ell^2\).

Second define a particular subset of \(P\) that will be denoted by
\(\bot\); then derive a duality on \(P\): for \(u,v\in P\), \(u\) and
\(v\) are dual\footnote{In modern terms one says that \(u\) and \(v\)
  are \emph{polar}.} iff \(uv\in\bot\).

Such a duality defines an \hyperref[orthogonality-relation]{orthogonality relation}, with the usual derived definitions and properties.

For the GoI, two dualities have proved to work; we will consider the
first one: nilpotency, \emph{ie}, \(\bot\) is the set of nilpotent
operators in \(P\). Let us explicit this: two operators \(u\) and \(v\)
are dual if there is a nonnegative integer \(n\) such that
\((uv)^n = 0\). This duality is symmetric: if \(uv\) is nilpotent then
\(vu\) is nilpotent also.

Last define a \emph{type} as a subset \(T\) of the proof space that is
equal to its bidual: \(T = T\biorth\). This means that \(u\in T\) iff
for all operator \(v\in T\orth\), that is such that \(u'v\in\bot\) for
all \(u'\in T\), we have \(uv\in\bot\).

The real work\footnote{The difficulty is to find the right duality that
  will make logical operations interpretable. General conditions that
  allow to achieve this have been formulated by Hyland and Schalk
  thanks to their theory of \emph{\wantedpage{double glueing}}.}
is now to interpret logical operations, that is to
associate a type to each formula, an object to each proof and show the
\emph{adequacy lemma}: if \(u\) is the interpretation of a proof of the
formula \(A\) then \(u\) belongs to the type associated to \(A\).

\subsubsection{\texorpdfstring{\hyperref[goi-for-mell-partial-isometries]{Partial isometries}}{Partial isometries}}\label{partial-isometries}

The first step is to build the proof space. This is constructed as a
special set of partial isometries on a separable Hilbert space \(H\)
which turns out to be generated by partial permutations on the canonical
basis of \(H\).

These so-called \emph{\(p\)-isometries} enjoy some nice properties, the
most important one being that a \(p\)-isometry is a sum of
\(p\)-isometries iff all the terms of the sum have disjoint domains and
disjoint codomains. As a consequence we get that a sum of
\(p\)-isometries is null iff each term of the sum is null.

A second important property is that operators on \(H\) can be
\emph{externalized} using \(p\)-isometries into operators acting on
\(H\oplus H\), and conversely operators on \(H\oplus H\) may be
\emph{internalized} into operators on \(H\). This is widely used in the
sequel.

\subsubsection{\texorpdfstring{\hyperref[goi-for-mell-the--autonomous-structure]{The *-autonomous structure}}{The *-autonomous structure}}\label{the--autonomous-structure}

The second step is to interpret the linear logic multiplicative
operations, most importantly the cut rule.

Internalization/externalization is the key for this: typically the type
\(A\tens B\) is interpreted by a set of \(p\)-isometries which are
internalizations of operators acting on \(H\oplus H\).

The (interpretation of) the cut-rule is defined in two steps: firstly we
use nilpotency to define an operation corresponding to lambda-calculus
application which given two \(p\)-isometries in respectively
\(A\limp B\) and \(A\) produces an operator in \(B\). From this we
deduce the composition and finally obtain a structure of *-autonomous
category (\cref{staraut}), that is a model of multiplicative linear logic.

\subsubsection{\texorpdfstring{\hyperref[goi-for-mell-exponentials]{The exponentials}}{The exponentials}}\label{the-exponentials}

Finally we turn to define exponentials, that is connectives managing
duplication. To do this we introduce an isomorphism (induced by a
\(p\)-isometry) between \(H\) and \(H\tens H\): the first component of
the tensor is intended to hold the address of the the copy whereas the
second component contains the content of the copy.

We eventually get a quasi-model of full MELL; quasi in the sense that if
we can construct \(p\)-isometries for usual structural operations in
MELL (contraction, dereliction, digging), the interpretation of linear
logic proofs is not invariant w.r.t. cut elimination in general. It is
however invariant in some good cases, which are enough to get a
correction theorem for the interpretation.


% \subsection{The Geometry of Interaction as an abstract machine}\label{the-geometry-of-interaction-as-an-abstract-machine}


%%% Local Variables:
%%% mode: latex
%%% TeX-master: "main"
%%% End:

\section{GoI for MELL: partial isometries}\label{goi-for-mell-partial-isometries}


\subsection{Operators, partial isometries}\label{operators-partial-isometries}

We will denote by \(H\) the Hilbert space \(\ell^2(\mathbb{N})\) of
sequences \((x_n)_{n\in\mathbb{N}}\) of complex numbers such that the
series \(\sum_{n\in\mathbb{N}}|x_n|^2\) converges. If
\(x = (x_n)_{n\in\mathbb{N}}\) and \(y = (y_n)_{n\in\mathbb{N}}\) are
two vectors of \(H\) their \emph{scalar product} is:
\begin{equation*}
\langle x, y\rangle = \sum_{n\in\mathbb{N}} x_n\bar y_n.
\end{equation*}

Two vectors of \(H\) are \emph{othogonal} if their scalar product is
nul. We will say that two subspaces are \emph{disjoint} when any two
vectors taken in each subspace are orthorgonal. Note that this notion is
different from the set theoretic one, in particular two disjoint
subspaces always have exactly one vector in common: \(0\).

The \emph{norm} of a vector is the square root of the scalar product
with itself:
\begin{equation*}
\|x\| = \sqrt{\langle x, x\rangle}.
\end{equation*}

Let us denote by \((e_k)_{k\in\mathbb{N}}\) the canonical
\emph{hilbertian basis} of \(H\):
\(e_k = (\delta_{kn})_{n\in\mathbb{N}}\) where \(\delta_{kn}\) is the
Kroenecker symbol: \(\delta_{kn}=1\) if \(k=n\), \(0\) otherwise. Thus
if \(x=(x_n)_{n\in\mathbb{N}}\) is a sequence in \(H\) we have:
\begin{equation*}
x = \sum_{n\in\mathbb{N}} x_ne_n.
\end{equation*}

An \emph{operator} on \(H\) is a \emph{continuous} linear map from \(H\)
to \(H\).\footnote{Continuity is equivalent to the fact that operators are
\emph{bounded}, which means that one may define the \emph{norm} of an
operator \(u\) as the sup on the unit ball of the norms of its values:
\begin{equation*}
\|u\| = \sup_{\{x\in H,\, \|x\| = 1\}}\|u(x)\|.
\end{equation*}}
The set of (bounded) operators is denoted by \(\mathcal{B}(H)\).

The \emph{range} or \emph{codomain} of the operator \(u\) is the set of
images of vectors; the \emph{kernel} of \(u\) is the set of vectors that
are anihilated by \(u\); the \emph{domain} of \(u\) is the set of
vectors orthogonal to the kernel, \emph{ie}, the maximal subspace
disjoint with the kernel:
\begin{itemize}
\item \(\mathrm{Codom}(u) = \{u(x),\, x\in H\}\);
\item \(\mathrm{Ker}(u) = \{x\in H,\, u(x) = 0\}\);
\item \(\mathrm{Dom}(u) = \{x\in H,\, \forall y\in\mathrm{Ker}(u), \langle x, y\rangle = 0\}\).
\end{itemize}

These three sets are closed subspaces of \(H\).

The \emph{adjoint} of an operator \(u\) is the operator \(u^*\) defined
by \(\langle u(x), y\rangle = \langle x, u^*(y)\rangle\) for any
\(x,y\in H\). Adjointness is well behaved w.r.t. composition of
operators:
\begin{equation*}
(uv)^* = v^*u^*.
\end{equation*}

A \emph{projector} is an idempotent operator of norm \(0\) (the
projector on the null subspace) or \(1\), that is an operator \(p\) such
that \(p^2 = p\) and \(\|p\| = 0\) or \(1\). A projector is auto-adjoint
and its domain is equal to its codomain.

A \emph{partial isometry} is an operator \(u\) satisfying \(uu^* u =
u\); this condition entails that we also have \(u^*uu^* =
u^*\). As a consequence \(uu^*\) and \(uu^*\) are both projectors,
called respectively the \emph{initial} and the \emph{final} projector of
\(u\) because their (co)domains are respectively the domain and the
codomain of \(u\):
\begin{itemize}
\item \(\mathrm{Dom}(u^*u) = \mathrm{Codom}(u^*u) = \mathrm{Dom}(u)\);
\item \(\mathrm{Dom}(uu^*) = \mathrm{Codom}(uu^*) = \mathrm{Codom}(u)\).
\end{itemize}

The restriction of \(u\) to its domain is an isometry. Projectors are
particular examples of partial isometries.

If \(u\) is a partial isometry then \(u^*\) is also a partial isometry
the domain of which is the codomain of \(u\) and the codomain of which
is the domain of \(u\).

If the domain of \(u\) is \(H\) that is if \(u^* u = 1\) we say that
\(u\) has \emph{full domain}, and similarly for codomain. If \(u\) and
\(v\) are two partial isometries then we have:
\begin{itemize}
\item \(uv^* = 0\) iff \(u^*uv^*v = 0\) iff the domains of \(u\) and \(v\)
  are disjoint;
\item \(u^*v = 0\) iff \(uu^*vv^* = 0\) iff the codomains of \(u\) and \(v\)
  are disjoint;
\item \(uu^* + vv^* = 1\) iff the codomains of \(u\) and \(v\) are disjoint
  and their their direct sum is \(H\).
\end{itemize}


\subsection{Partial permutations}\label{partial-permutations}

We will now define our proof space which turns out to be the set of
partial isometries acting as permutations on the canonical basis
\((e_n)_{n\in\mathbb{N}}\).

More precisely a \emph{partial permutation} \(\varphi\) on
\(\mathbb{N}\) is a one-to-one map defined on a subset \(D_\varphi\) of
\(\mathbb{N}\) onto a subset \(C_\varphi\) of \(\mathbb{N}\).
\(D_\varphi\) is called the \emph{domain} of \(\varphi\) and
\(C_\varphi\) its \emph{codomain}. Partial permutations may be composed:
if \(\psi\) is another partial permutation on \(\mathbb{N}\) then
\(\varphi\circ\psi\) is defined by:
\begin{itemize}
\item \(n\in D_{\varphi\circ\psi}\) iff \(n\in D_\psi\) and
  \(\psi(n)\in D_\varphi\);
\item if \(n\in D_{\varphi\circ\psi}\) then
  \(\varphi\circ\psi(n) = \varphi(\psi(n))\);
\item the codomain of \(\varphi\circ\psi\) is the image of the domain:
  \(C_{\varphi\circ\psi} = \{\varphi(\psi(n)), n\in D_{\varphi\circ\psi}\}\).
\end{itemize}

Partial permutations are well known to form a structure of \emph{inverse
monoid} that we detail now.

Given a a subset \(D\) of \(\mathbb{N}\), the \emph{partial identity} on
\(D\) is the partial permutation \(\varphi\) defined by:
\begin{itemize}
\item \(D_\varphi = D\);
\item \(\varphi(n) = n\) for any \(n\in D_\varphi\).
\end{itemize}

Thus the codomain of \(\varphi\) is \(D\).

The partial identity on \(D\) will be denoted by \(1_D\). Partial
identities are idempotent for composition.

Among partial identities one finds the identity on the empty subset,
that is the empty map, that we will denote by \(0\) and the identity on
\(\mathbb{N}\) that we will denote by \(1\). This latter permutation is
the neutral for composition.

If \(\varphi\) is a partial permutation there is an inverse partial
permutation \(\varphi^{-1}\) whose domain is
\(D_{\varphi^{-1}} = C_{\varphi}\) and who satisfies:
\begin{align*}
\varphi^{-1}\circ\varphi &= 1_{D_\varphi} \\
\varphi\circ\varphi^{-1} &= 1_{C_\varphi}
\end{align*}


\subsection{The proof space}\label{the-proof-space}

Given a partial permutation \(\varphi\) one defines a partial isometry
\(u_\varphi\) by:
\begin{equation*}
u_\varphi(e_n) =
\begin{cases}
e_{\varphi(n)} & \text{if $n\in D_\varphi$} \\
0 & \text{otherwise}
\end{cases}
\end{equation*}

In other terms if \(x=(x_n)_{n\in\mathbb{N}}\) is a sequence in
\(\ell^2\) then \(u_\varphi(x)\) is the sequence
\((y_n)_{n\in\mathbb{N}}\) defined by:
\begin{equation*}
y_n = x_{\varphi^{-1}(n)} \text{ if $n\in C_\varphi$, $0$ otherwise.}
\end{equation*}

We will (not so abusively) write \(e_{\varphi(n)} = 0\) when
\(\varphi(n)\) is undefined so that the definition of \(u_\varphi\)
reads:
\begin{equation*}
u_\varphi(e_n) = e_{\varphi(n)}.
\end{equation*}

The domain of \(u_\varphi\) is the subspace spanned by the family
\((e_n)_{n\in D_\varphi}\) and the codomain of \(u_\varphi\) is the
subspace spanned by \((e_n)_{n\in C_\varphi}\). In particular if
\(\varphi\) is \(1_D\) then \(u_\varphi\) is the projector on the
subspace spanned by \((e_n)_{n\in D}\).

\begin{definition}
We call \emph{$p$-isometry} a partial isometry of the form $u_\varphi$ where $\varphi$ is a partial permutation on $\mathbb{N}$. The \emph{proof space} $\mathcal{P}$ is the set of all $p$-isometries.
\end{definition}

\begin{proposition}
Let $\varphi$ and $\psi$ be two partial permutations. We have:
\begin{equation*}
u_\varphi u_\psi = u_{\varphi\circ\psi}.
\end{equation*}

The adjoint of $u_\varphi$ is:
\begin{equation*}
u_\varphi^* = u_{\varphi^{-1}}.
\end{equation*}

In particular the initial projector of $u_{\varphi}$ is given by:
\begin{equation*}
u_\varphi u^*_\varphi = u_{1_{D_\varphi}}.
\end{equation*}
and the final projector of $u_\varphi$ is:
\begin{equation*}
u^*_\varphi u_\varphi = u_{1_{C_\varphi}}.
\end{equation*}

If $p$ is a projector in $\mathcal{P}$ then there is a partial identity $1_D$ such that $p= u_{1_D}$.

Projectors commute, in particular we have:
\begin{equation*}
u_\varphi u_\varphi^*u_\psi u_\psi^* = u_\psi u_\psi^*u_\varphi u_\varphi^*.
\end{equation*}
\end{proposition}

Note that this entails all the other commutations of projectors:
\(u^*_\varphi u_\varphi u_\psi u^*_\psi = u_\psi u^*_\psi u^*_\varphi u_\varphi\)
and
\(u^*_\varphi u_\varphi u^*_\psi u\psi = u^*_\psi u_\psi u^*_\varphi u_\varphi\).

In particular note that \(0\) is a \(p\)-isometry. The set
\(\mathcal{P}\) is a submonoid of \(\mathcal{B}(H)\) but it is not a
subalgebra.\footnote{\(\mathcal{P}\) is the normalizing groupoid of the
  maximal commutative subalgebra of \(\mathcal{B}(H)\) consisiting of
  all operators \emph{diagonalizable} in the canonical basis.}
In general given \(u,v\in\mathcal{P}\) we don't necessarily have
\(u+v\in\mathcal{P}\). However we have:

\begin{proposition}
Let $u, v\in\mathcal{P}$. Then $u+v\in\mathcal{P}$ iff $u$ and $v$ have disjoint domains and disjoint codomains, that is:
\begin{equation*}
u+v\in\mathcal{P}\text{ iff }uu^*vv^* = u^*uv^*v = 0.
\end{equation*}
\end{proposition}

\begin{proof}
Suppose for contradiction that $e_n$ is in the domains of $u$ and $v$. There are integers $p$ and $q$ such that $u(e_n) = e_p$ and $v(e_n) = e_q$ thus $(u+v)(e_n) = e_p + e_q$ which is not a basis vector; therefore $u+v$ is not a $p$-permutation.
\end{proof}

As a corollary note that if \(u+v=0\) then \(u=v=0\).


\subsection{From operators to matrices: internalization/externalization}\label{from-operators-to-matrices-internalizationexternalization}

It will be convenient to view operators on \(H\) as acting on
\(H\oplus H\), and conversely. For this purpose we define an isomorphism
\(H\oplus H \cong H\) by \(x\oplus y\rightsquigarrow p(x)+q(y)\) where
\(p:H\mapsto H\) and \(q:H\mapsto H\) are partial isometries given by:
\begin{align*}
p(e_n) &= e_{2n} \\
q(e_n) &= e_{2n+1}.
\end{align*}

From the definition \(p\) and \(q\) have full domain, that is satisfy
\(p^* p = q^* q = 1\). On the other hand their codomains are disjoint,
thus we have \(p^*q = q^*p = 0\). As the sum of their codomains is the
full space \(H\) we also have \(pp^* + qq^* = 1\).

Note that we have choosen \(p\) and \(q\) in \(\mathcal{P}\). However
the choice is arbitrary: any two \(p\)-isometries with full domain and
disjoint codomains would do the job.

Given an operator \(u\) on \(H\) we may \emph{externalize} it obtaining
an operator \(U\) on \(H\oplus H\) defined by the matrix:
\begin{equation*}
U = \begin{pmatrix}
  u_{11} & u_{12}\\
  u_{21} & u_{22}
  \end{pmatrix}
\end{equation*}
where the \(u_{ij}\)'s are given by:
\begin{align*}
u_{11} &= p^*up \\
u_{12} &= p^*uq \\
u_{21} &= q^*up \\
u_{22} &= q^*uq.
\end{align*}

The \(u_{ij}\)'s are called the \emph{external components} of \(u\). The
externalization is functorial in the sense that if \(v\) is another
operator externalized as:
\begin{equation*}
V = \begin{pmatrix}
  v_{11} & v_{12}\\
  v_{21} & v_{22}
  \end{pmatrix} 
= \begin{pmatrix}
  p^*vp & p^*vq\\
  q^*vp & q^*vq
  \end{pmatrix}
\end{equation*}
then the externalization of \(uv\) is the matrix product \(UV\).

As \(pp^* + qq^* = 1\) we have:
\begin{equation*}
u = (pp^*+qq^*)u(pp^*+qq^*) = pu_{11}p^* + pu_{12}q^* + qu_{21}p^* + qu_{22}q^*
\end{equation*}
which entails that externalization is reversible, its converse being
called \emph{internalization}.

If we suppose that \(u\) is a \(p\)-isometry then so are the components
\(u_{ij}\)'s. Thus the formula above entails that the four terms of the
sum have pairwise disjoint domains and pairwise disjoint codomains from
which we deduce:

\begin{proposition}
If $u$ is a $p$-isometry and $u_{ij}$ are its external components then:
\begin{itemize}
\item $u_{1j}$ and $u_{2j}$ have disjoint domains, that is $u_{1j}^*u_{1j}u_{2j}^*u_{2j} = 0$ for $j=1,2$;
\item $u_{i1}$ and $u_{i2}$ have disjoint codomains, that is $u_{i1}u_{i1}^*u_{i2}u_{i2}^* = 0$ for $i=1,2$.
\end{itemize}
\end{proposition}

As an example of computation in \(\mathcal{P}\) let us check that the
product of the final projectors of \(pu_{11}p^*\) and \(pu_{12}q^*\) is
null:
\begin{align*}
    (pu_{11}p^*)(pu^*_{11}p^*)(pu_{12}q^*)(qu_{12}^*p^*)
    &= pu_{11}u_{11}^*u_{12}u_{12}^*p^*\\
    &= pp^*upp^*u^*pp^*uqq^*u^*pp^*\\
    &= pp^*u(pp^*)(u^*pp^*u)qq^*u^*pp^*\\
    &= pp^*u(u^*pp^*u)(pp^*)qq^*u^*pp^*\\
    &= pp^*uu^*pp^*u(pp^*)(qq^*)u^*pp^*\\
    &= 0
\end{align*}
where we used the fact that all projectors in \(\mathcal{P}\) commute,
which is in particular the case of \(pp^*\) and \(u^*pp^*u\).


%%% Local Variables:
%%% mode: latex
%%% TeX-master: "main"
%%% End:

\section{GoI for MELL: the *-autonomous structure}\label{goi-for-mell-the--autonomous-structure}

Recall that when \(u\) and \(v\) are \(p\)-isometries we say they are
dual when \(uv\) is nilpotent, and that \(\bot\) denotes the set of
nilpotent operators. A \emph{type} is a subset of \(\mathcal{P}\) that
is equal to its bidual. In particular \(X\orth\) is a type for any
\(X\subset\mathcal{P}\). We say that \(X\) \emph{generates} the type
\(X\biorth\).

\subsection{The tensor and the linear application}\label{the-tensor-and-the-linear-application}

If \(u\) and \(v\) are two \(p\)-isometries summing them doesn't in
general produces a \(p\)-isometry. However as \(pup^*\) and \(qvq^*\)
have disjoint domains and disjoint codomains it is true that
\(pup^* + qvq^*\) is a \(p\)-isometry. Given two types \(A\) and \(B\),
we thus define their \emph{tensor} by:
\begin{equation*}
A\tens B = \{pup^* + qvq^*, u\in A, v\in B\}\biorth
\end{equation*}

Note the closure by bidual to make sure that we obtain a type.

From what precedes we see that \(A\tens B\) is generated by the
internalizations of operators on \(H\oplus H\) of the form:
\begin{equation*}
\begin{pmatrix}
  u & 0 \\
  0 & v
\end{pmatrix}
\end{equation*}

\begin{remark}
This so-called tensor resembles a sum rather than a product. We will stick to this terminology though because it defines the interpretation of the tensor connective of linear logic.
\end{remark}

The linear implication is derived from the tensor by duality: given two
types \(A\) and \(B\) the type \(A\limp B\) is defined by:
\begin{equation*}
A\limp B = (A\tens B\orth)\orth.
\end{equation*}

Unfolding this definition we get:
\begin{equation*}
A\limp B = \{u\in\mathcal{P}\text{ s.t. } \forall v\in A, \forall w\in B\orth,\, u.(pvp^* + qwq^*) \in\bot\}.
\end{equation*}

\subsection{The identity}\label{the-identity}

Given a type \(A\) we are to find an operator \(\iota\) in type
\(A\limp A\), thus satisfying:
\begin{equation*}
\forall u\in A, v\in A\orth,\, \iota(pup^* + qvq^*)\in\bot.
\end{equation*}

An easy solution is to take \(\iota = pq^* + qp^*\). In this way we get
\(\iota(pup^* + qvq^*) = qup^* + pvq^*\). Therefore
\((\iota(pup^* + qvq^*))^2 = quvq^* + pvup^*\), from which one deduces
that this operator is nilpotent iff \(uv\) is nilpotent. It is the case
since \(u\) is in \(A\) and \(v\) in \(A\orth\).

It is interesting to note that the \(\iota\) thus defined is actually
the internalization of the operator on \(H\oplus H\) given by the
matrix:
\begin{equation*}
  \begin{pmatrix}0 & 1\\1 & 0\end{pmatrix}
\end{equation*}

We will see once the composition is defined that the \(\iota\) operator
is the interpretation of the identity proof, as expected.

\subsection{The execution formula, version 1: application}\label{the-execution-formula-version-1-application}

\begin{definition}
Let $u$ and $v$ be two operators; as above denote by $u_{ij}$ the external components of $u$. If $u_{11}v$ is nilpotent we define the \emph{application of $u$ to $v$} by:
\begin{equation*}
\mathrm{App}(u,v) = u_{22} + u_{21}v\sum_k(u_{11}v)^ku_{12}.
\end{equation*}
\end{definition}

Note that the hypothesis that \(u_{11}v\) is nilpotent entails that the
sum \(\sum_k(u_{11}v)^k\) is actually finite. It would be enough to
assume that this sum converges. For simplicity we stick to the
nilpotency condition, but we should mention that weak nilpotency would
do as well.

\begin{theorem}
If $u$ and $v$ are $p$-isometries such that $u_{11}v$ is nilpotent, then $\mathrm{App}(u,v)$ is also a $p$-isometry.
\end{theorem}

\begin{proof}
Let us note $E_k = u_{21}v(u_{11}v)^ku_{12}$. Recall that $u_{22}$ and $u_{12}$ being external components of the $p$-isometry $u$, they have disjoint domains. Thus it is also the case of $u_{22}$ and $E_k$. Similarly $u_{22}$ and $E_k$ have disjoint codomains because $u_{22}$ and $u_{21}$ have disjoint codomains.

Let now $k$ and $l$ be two integers such that $k>l$ and let us compute for example the intersection of the codomains of $E_k$ and $E_l$:
\begin{equation*}
    E_kE^*_kE_lE^*_l = (u_{21}v(u_{11}v)^ku_{12})(u^*_{12}(v^*u^*_{11})^kv^*u^*_{21})(u_{21}v(u_{11}v)^lu_{12})(u^*_{12}(v^*u^*_{11})^lv^*u_{21}^*)
\end{equation*}
  As $k>l$ we may write $(v^*u_{11}^*)^l = (v^*u^*_{11})^{k-l-1}v^*u^*_{11}(v^*u^*_{11})^l$. Let us note $E = u^*_{11}(v^*u^*_{11})^lv^*u_{21}^*u_{21}v(u_{11}v)^lu_{12}$ so that $E_kE^*_kE_lE^*_l = u_{21}v(u_{11}v)^ku_{12}u^*_{12}(v^*u^*_{11})^{k-l-1}v^*Eu^*_{12}(v^*u^*_{11})^lv^*u_{21}^*$. We have:
\begin{align*}
     E & = u^*_{11}(v^*u^*_{11})^lv^*u_{21}^*u_{21}v(u_{11}v)^lu_{12}\\
       & = (u^*_{11}u_{11}u^*_{11})(v^*u^*_{11})^lv^*u_{21}^*u_{21}v(u_{11}v)^lu_{12}\\
       & = u^*_{11}(u_{11}u^*_{11})\bigl((v^*u^*_{11})^lv^*u_{21}^*u_{21}v(u_{11}v)^l\bigr)u_{12}\\
       & = u^*_{11}\bigl((v^*u^*_{11})^lv^*u_{21}^*u_{21}v(u_{11}v)^l\bigr)(u_{11}u^*_{11})u_{12}\\
       & = u^*_{11}(v^*u^*_{11})^lv^*u_{21}^*u_{21}v(u_{11}v)^lu_{11}u^*_{11}u_{12}\\
       & = 0
\end{align*}
because $u_{11}$ and $u_{12}$ have disjoint codomains, thus $u^*_{11}u_{12} = 0$. 

Similarly we can show that $E_k$ and $E_l$ have disjoint domains. Therefore we have proved that all terms of the sum $\mathrm{App}(u,v)$ have disjoint domains and disjoint codomains. Consequently $\mathrm{App}(u,v)$ is a $p$-isometry.
\end{proof}

\begin{theorem}
Let $A$ and $B$ be two types and $u$ a $p$-isometry. Then the two following conditions are equivalent:
\begin{enumerate}
\item $u\in A\limp B$;
\item for any $v\in A$ we have:
  \begin{itemize}
  \item $u_{11}v$ is nilpotent and
  \item $\mathrm{App}(u, v)\in B$.
  \end{itemize}
\end{enumerate}
\end{theorem}

\begin{proof}
Let $v$ and $w$ be two $p$-isometries. If we compute
\begin{equation*}
(u.(pvp^* + qwq^*))^n = \bigl((pu_{11}p^* + pu_{12}q^* + qu_{21}p^* + qu_{22}q^*)(pvp^* + qwq^*)\bigr)^n
\end{equation*}
we get a finite sum of monomial operators of the form:
\begin{enumerate}
\item $p(u_{11}v)^{i_0}u_{12}w(u_{22}w)^{i_1}\dots u_{21}v(u_{11}v)^{i_m}p^*$
\item $p(u_{11}v)^{i_0}u_{12}w(u_{22}w)^{i_1}\dots u_{12}w(u_{22}w)^{i_m}q^*$,
\item $q(u_{22}w)^{i_0}u_{21}v(u_{11}v)^{i_1}\dots u_{21}v(u_{11}v)^{i_m}p^*$ or
\item $q(u_{22}w)^{i_0}u_{21}v(u_{11}v)^{i_1}\dots u_{12}w(u_{22}w)^{i_m}q^*$,
\end{enumerate}
for all tuples of (nonnegative) integers $(i_1,\dots, i_m)$ such that $i_0+\cdots+i_m+m = n$.

Each of these monomial is a $p$-isometry. Furthermore they have disjoint domains and disjoint codomains because their sum is the $p$-isometry $(u.(pvp^* + qwq^*))^n$. This entails that $(u.(pvp^* + qwq^*))^n = 0$ iff all these monomials are null.

Suppose $u_{11}v$ is nilpotent and consider:
\begin{equation*}
\bigl(\mathrm{App}(u,v)w\bigr)^n = \biggl(\bigl(u_{22} + u_{21}v\sum_k(u_{11}v)^k u_{12}\bigr)w\biggr)^n.
\end{equation*}
Developping we get a finite sum of monomials of the form:
\begin{enumerate}\setcounter{enumi}{4}
\item $(u_{22}w)^{l_0}u_{21}v(u_{11}v)^{k_1}u_{12}w(u_{22}w)^{l_1}\dots u_{21}v(u_{11}v)^{k_m}u_{12}w(u_{22}w)^{l_m}$
\end{enumerate}
for all tuples $(l_0, k_1, l_1,\dots, k_m, l_m)$ such that $l_0\cdots l_m + m = n$ and $k_i$ is less than the degree of nilpotency of $u_{11}v$ for all $i$.

Again as these monomials are $p$-isometries and their sum is the $p$-isometry $(\mathrm{App}(u,v)w)^n$, they have pairwise disjoint domains and pairwise disjoint codomains. Note that each of these monomial is equal to $q^*Mq$ where $M$ is a monomial of type 4 above.

As before we thus have that $\bigl(\mathrm{App}(u,v)w\bigr)^n = 0$ iff all monomials of type 5 are null.

Suppose now that $u\in A\limp B$ and $v\in A$. Then, since $0\in B\orth$ ($0$ belongs to any type) $u.(pvp^*) = pu_{11}vp^*$ is nilpotent, thus $u_{11}v$ is nilpotent.

Suppose further that $w\in B\orth$. Then $u.(pvp^*+qwq^*)$ is nilpotent, thus there is a $N$ such that $(u.(pvp^* + qwq^*))^n=0$ for any $n\geq N$. This entails that all monomials of type 1 to 4 are null. Therefore all monomials appearing in the developpment of $(\mathrm{App}(u,v)w)^N$ are null which proves that $\mathrm{App}(u,v)w$ is nilpotent. Thus $\mathrm{App}(u,v)\in B$.

Conversely suppose for any $v\in A$ and $w\in B\orth$, $u_{11}v$ and $\mathrm{App}(u,v)w$ are nilpotent. Let $P$ and $N$ be their respective degrees of nilpotency and put $n=N(P+1)+N$. Then we claim that all monomials of type 1 to 4 appearing in the development of $(u.(pvp^*+qwq^*))^n$ are null.

Consider for example a monomial of type 1:
\begin{equation*}
p(u_{11}v)^{i_0}u_{12}w(u_{22}w)^{i_1}\dots u_{21}v(u_{11}v)^{i_m}p^*
\end{equation*}
with $i_0+\cdots+i_m + m = n$. Note that $m$ must be even.

If $i_{2k}\geq P$ for some $0\leq k\leq m/2$ then $(u_{11}v)^{i_{2k}}=0$ thus our monomial is null. Otherwise if $i_{2k}<P$ for all $k$ we have:
\begin{equation*}
i_1+i_3+\cdots +i_{m-1} + m/2 = n - m/2 - (i_0+i_2+\cdots +i_m)
\end{equation*}
thus:
\begin{equation*}
i_1+i_3+\cdots +i_{m-1} + m/2\geq n - m/2 - (1+m/2)P.
\end{equation*}
Now if $m/2\geq N$ then $i_1+\cdots+i_{m-1}+m/2 \geq N$. Otherwise $1+m/2\leq N$ thus
\begin{equation*}
i_1+i_3+\cdots +i_{m-1} + m/2\geq n - N - NP = N.
\end{equation*}
Since $N$ is the degree of nilpotency of $\mathrm{App}(u,v)w$ we have that the monomial:
\begin{equation*}
(u_{22}w)^{i_1}u_{21}v(u_{11}v)^{i_2}u_{12}w\dots(u_{11}v)^{i_{m-2}}u_{12}w(u_{22}w)^{i_{m-1}}
\end{equation*}
is null, thus also the monomial of type 1 we started with.
\end{proof}

\begin{corollary}
If $A$ and $B$ are types then we have:
\begin{equation*}
A\limp B = \{u\in\mathcal{P} \text{ such that }\forall v\in A: u_{11}v\in\bot\text{ and } \mathrm{App}(u, v)\in B\}.
\end{equation*}
\end{corollary}

As an example if we compute the application of the interpretation of the
identity \(\iota\) in type \(A\limp A\) to the operator \(v\in A\) then
we have:
\begin{equation*}
\mathrm{App}(\iota, v) = \iota_{22} + \iota_{21}v\sum(\iota_{11}v)^k\iota_{12}.
\end{equation*}

Now recall that \(\iota = pq^* + qp^*\) so that
\(\iota_{11} = \iota_{22} = 0\) and \(\iota_{12} = \iota_{21} = 1\) and
we thus get:
\begin{equation*}
\mathrm{App}(\iota, v) = v
\end{equation*}
as expected.

\subsection{The tensor rule}\label{the-tensor-rule}

Let now \(A, A', B\) and \(B'\) be types and consider two operators
\(u\) and \(u'\) respectively in \(A\limp B\) and \(A\limp B'\). We
define an operator \(u\tens u'\) by:
\begin{align*}
    u\tens u' &= ppp^*upp^*p^* + qpq^*upp^*p^* + ppp^*uqp^*q^* + qpq^*uqp^*q^*\\
              &+ pqp^*u'pq^*p^* + qqq^*u'pq^*p^* + pqp^*u'qq^*q^* + qqq^*u'qq^*q^*
  \end{align*}

Once again the notation is motivated by linear logic syntax and is
contradictory with linear algebra practice since what we denote by
\(u\tens u'\) actually is the internalization of the direct sum
\(u\oplus u'\).

Indeed if we think of \(u\) and \(u'\) as the internalizations of the
matrices:
\begin{equation*}
  \begin{pmatrix}u_{11}   &  u_{12}\\
                   u_{21}   &  u_{22}
    \end{pmatrix}
  \qquad\text{and}\qquad
    \begin{pmatrix}u'_{11} &  u'_{12}\\
                   u'_{21} &  u'_{22}
    \end{pmatrix}
\end{equation*}
then we may write:
\begin{align*}
    u\tens u' & = ppu_{11}p^*p^* + qpu_{21}p^*p^* + ppu_{12}p^*q^* + qpu_{22}p^*q^*\\
              & + pqu'_{11}q^*p^* + qqu'_{21}q^*p^* + pqu'_{12}q^*q^* + qqu'_{22}q^*q^*
  \end{align*}
Thus the components of \(u\tens u'\) are given by:
\begin{equation*}
(u\tens u')_{ij} = pu_{ij}p^* + qu'_{ij}q^*.
\end{equation*}
and we see that \(u\tens u'\) is actually the internalization of the
matrix:
\begin{equation*}
    \begin{pmatrix}
      u_{11} &  0       &  u_{12}  &  0       \\
      0      &  u'_{11} &  0       &  u'_{12} \\
      u_{21} &  0       &  u_{22}  &  0       \\
      0      &  u'_{21} &  0       &  u'_{22} \\
    \end{pmatrix}
\end{equation*}

We are now to show that if we suppose \(u\)and \(u'\) are in types
\(A\limp B\) and \(A'\limp B'\), then \(u\tens u'\) is in
\(A\tens A'\limp B\tens B'\). For this we consider \(v\) and \(v'\)
respectively in \(A\) and \(A'\), so that \(pvp^* + qv'q^*\) is in
\(A\tens A'\), and we show that
\(\mathrm{App}(u\tens u', pvp^* + qv'q^*)\in B\tens B'\).

Since \(u\) and \(u'\) are in \(A\limp B\) and \(A'\limp B'\) we have
that \(u_{11}v\) and \(u'_{11}v'\) are nilpotent and that
\(\mathrm{App}(u, v)\) and \(\mathrm{App}(u', v')\) are respectively in
\(B\) and \(B'\), thus:
\begin{equation*}
p\mathrm{App}(u, v)p^* + q\mathrm{App}(u', v')q^* \in B\tens B'.
\end{equation*}
But we have:
\begin{align*}
    \bigl((u\tens u')_{11}(pvp^* + qv'q^*)\bigr)^n
      & = \bigl((pu_{11}p^* + qu'_{11}q^*)(pvp^* + qv'q^*)\bigr)^n\\
      & = (pu_{11}vp^* + qu'_{11}v'q^*)^n\\
      & = p(u_{11}v)^np^* + q(u'_{11}v')^nq^*
\end{align*}
Therefore \((u\tens u')_{11}(pvp^* + qv'q^*)\) is nilpotent. So we can
compute \(\mathrm{App}(u\tens u', pvp^* + qv'q^*)\):
\begin{align*}
    & \mathrm{App}(u\tens u', pvp^* + qv'q^*)\\
      & = (u\tens u')_{22} + (u\tens u')_{21}(pvp^* + qv'q^*)\sum\bigl((u\tens u')_{11}(pvp^* + qv'q^*)\bigr)^k(u\tens u')_{12}\\
      & = pu_{22}p^* + qu'_{22}q^* + (pu_{21}p^* + qu'_{21}q^*)(pvp^* + qv'q^*)\sum\bigl((pu_{11}p^* + qu'_{11}q^*)(pvp^* + qv'q^*)\bigr)^k(pu_{12}p^* + qu'_{12}q^*)\\
      & = p\bigl(u_{22} + u_{21}v\sum(u_{11}v)^ku_{12}\bigr)p^* + q\bigl(u'_{22} + u'_{21}v'\sum(u'_{11}v')^ku'_{12}\bigr)q^*\\
      & = p\mathrm{App}(u, v)p^* + q\mathrm{App}(u', v')q^*
\end{align*}
thus lives in \(B\tens B'\).

\subsection{Other monoidal constructions}\label{other-monoidal-constructions}

\subsubsection{Contraposition}\label{contraposition}

Let \(A\) and \(B\) be some types; we have:
\begin{equation*}
A\limp B = A\orth\limpinv B\orth
\end{equation*}
Indeed, \(u\in A\limp B\) means that for any \(v\) and \(w\) in
respectively \(A\) and \(B\orth\) we have \(u.(pvp^* + qwq^*)\in\bot\)
which is exactly the definition of \(A\orth\limpinv B\orth\).

We will denote \(u\orth\) the operator:
\begin{equation*}
u\orth = pu_{22}p^* + pu_{12}q^* + qu_{12}p^* + qu_{11}q^*
\end{equation*}
where \(u_{ij}\) is given by externalization. Therefore the
externalization of \(u\orth\) is:
\begin{equation*}
(u\orth)_{ij} = u_{\bar i\,\bar j}\text{ where $\bar .$ is defined by $\bar1 = 2, \bar2 = 1$}.
\end{equation*}
From this we deduce that \(u\orth\in B\orth\limp A\orth\) and that
\((u\orth)\orth = u\).

\subsubsection{Commutativity}\label{commutativity}

Let \(\sigma\) be the operator:
\begin{equation*}
\sigma = ppq^*q^* +pqp^*q^* + qpq^*p^* + qqp^*p^*.
\end{equation*}

One can check that \(\sigma\) is the internalization of the operator
\(S\) on \(H\oplus H\oplus H\oplus H\) defined by:
\(S(x_1\oplus x_2\oplus x_3\oplus x_4) = x_4\oplus x_3\oplus x_2\oplus x_1\).
In particular the components of \(\sigma\) are:
\begin{align*}
\sigma_{11} &= \sigma_{22} = 0 \\
\sigma_{12} &= \sigma_{21} = pq^* + qp^*.
\end{align*}

Let \(A\) and \(B\) be types and \(u\) and \(v\) be operators in \(A\)
and \(B\). Then \(pup^* + qvq^*\) is in \(A\tens B\) and as
\(\sigma_{11}.(pup^* + qvq^*) = 0\) we may compute:
\begin{align*}
    \mathrm{App}(\sigma, pup^* + qvq^*) 
      & = \sigma_{22} + \sigma_{21}(pup^* + qvq^*)\sum(\sigma_{11}(pup^* + qvq^*))^k\sigma_{12}\\
      & = (pq^* + qp^*)(pup^* + qvq^*)(pq^* + qp^*)\\
      & = pvp^* + quq^*
\end{align*}
But \(pvp^* + quq^*\in B\tens A\), thus we have shown that:
\begin{equation*}
\sigma\in (A\tens B) \limp (B\tens A).
\end{equation*}

\subsubsection{Distributivity}\label{distributivity}

We get distributivity by considering the operator:
\begin{equation*}
\delta = ppp^*p^*q^* + pqpq^*p^*q^* + pqqq^*q^* + qppp^*p^* + qpqp^*q^*p^* + qqq^*q^*p^*
\end{equation*}
that is similarly shown to be in type
\(A\tens(B\tens C)\limp(A\tens B)\tens C\) for any types \(A\), \(B\)
and \(C\).

\subsubsection{Weak distributivity}\label{weak-distributivity}

Similarly we get weak distributivity thanks to the operators:
\begin{align*}
\delta_1 & = pppp^*q^* + ppqp^*q^*q^* + pqq^*q^*q^* + qpp^*p^*p^* + qqp q^*p^*p^* + qqq q^*p^* \\
\delta_2 &= ppp^*p^*q^* + pqpq^*p^*q^* + pqqq^*q^* + qppp^*p^* + qpqp^*q^*p^* + qqq^*q^*p^*.
\end{align*}
Given three types \(A\), \(B\) and \(C\) then one can show that:
\begin{itemize}
\item \(\delta_1\) has type \(((A\limp B)\tens C)\limp A\limp (B\tens C)\) and
\item \(\delta_2\) has type \((A\tens(B\limp C))\limp (A\limp B)\limp C\).
\end{itemize}

\subsection{Execution formula, version 2: composition}\label{execution-formula-version-2-composition}

Let \(A\), \(B\) and \(C\) be types and \(u\) and \(v\) be operators
respectively in types \(A\limp B\) and \(B\limp C\).

As usual we will denote \(u_{ij}\) and \(v_{ij}\) the operators obtained
by externalization of \(u\) and \(v\), eg, \(u_{11} = p^*up\), ...

As \(u\) is in \(A\limp B\) we have that
\(\mathrm{App}(u, 0)=u_{22}\in B\); similarly as \(v\in B\limp C\), thus
\(v\orth\in C\orth\limp B\orth\), we have
\(\mathrm{App}(v\orth, 0) = v_{11}\in B\orth\). Thus \(u_{22}v_{11}\) is
nilpotent.

We define the operator \(\mathrm{Comp}(u, v)\) by:
\begin{align*}
    \mathrm{Comp}(u, v) & = p(u_{11} + u_{12}\sum(v_{11}u_{22})^k\,v_{11}u_{21})p^*\\
                        & + p(u_{12}\sum(v_{11}u_{22})^k\,v_{12})q^*\\
                        & + q(v_{21}\sum(u_{22}v_{11})^k\,u_{21})p^*\\
			& + q(v_{22} + v_{21}\sum(u_{22}v_{11})^k\,u_{22}v_{12})q^*
\end{align*}
This is well defined since \(u_{11}v_{22}\) is nilpotent. As an example
let us compute the composition of \(u\) and \(\iota\) in type
\(B\limp B\); recall that \(\iota_{ij} = \delta_{ij}\), so we get:
\begin{equation*}
\mathrm{Comp}(u, \iota) = pu_{11}p^* + pu_{12}q^* + qu_{21}p^* + qu_{22}q^*  = u
\end{equation*}
Similar computation would show that \(\mathrm{Comp}(\iota, v) = v\) (we
use \(pp^* + qq^* = 1\) here).

Coming back to the general case we claim that \(\mathrm{Comp}(u, v)\) is
in \(A\limp C\): let \(a\) be an operator in \(A\). By computation we
can check that:
\begin{equation*}
\mathrm{App}(\mathrm{Comp}(u, v), a) = \mathrm{App}(v, \mathrm{App}(u, a)).
\end{equation*}

Now since \(u\) is in \(A\limp B\), \(\mathrm{App}(u, a)\) is in \(B\)
and since \(v\) is in \(B\limp C\),
\(\mathrm{App}(v, \mathrm{App}(u, a))\) is in \(C\).

If we now consider a type \(D\) and an operator \(w\) in \(C\limp D\)
then we have:
\begin{equation*}
\mathrm{Comp}(\mathrm{Comp}(u, v), w) = \mathrm{Comp}(u, \mathrm{Comp}(v, w))
\end{equation*}

Putting together the results of this section we finally have:

\begin{theorem}
Let GoI(H) be defined by:
\begin{itemize}
\item objects are types, \ie\ sets $A$ of $p$-isometries satisfying: $A\biorth = A$;
\item morphisms from $A$ to $B$ are $p$-isometries in type $A\limp B$;
\item composition is given by the formula above.
\end{itemize}
Then GoI(H) is a star-autonomous category.
\end{theorem}


%%% Local Variables:
%%% mode: latex
%%% TeX-master: "main"
%%% End:

\section{GoI for MELL: exponentials}\label{goi-for-mell-exponentials}

\subsection{The tensor product of Hilbert spaces}\label{the-tensor-product-of-hilbert-spaces}

Recall that we work in the Hilbert space \(H=\ell^2(\mathbb{N})\)
endowed with its canonical hilbertian basis denoted by
\((e_k)_{k\in\mathbb{N}}\).

The space \(H\tens H\) is the collection of sequences
\((x_{np})_{n,p\in\mathbb{N}}\) of complex numbers such that
\(\sum_{n,p}|x_{np}|^2\) converges. The scalar product is defined just
as before:

\begin{description}
\tightlist
\item[]
\(\langle (x_{np}), (y_{np})\rangle = \sum_{n,p} x_{np}\bar y_{np}\).
\end{description}

If \(x = (x_n)_{n\in\mathbb{N}}\) and \(y = (y_p)_{p\in\mathbb{N}}\) are
vectors in \(H\) then their tensor is the sequence:

\begin{description}
\tightlist
\item[]
\(x\tens y = (x_ny_p)_{n,p\in\mathbb{N}}\).
\end{description}

We define: \(e_{np} = e_n\tens e_p\) so that \(e_{np}\) is the sequence
\((e_{npij})_{i,j\in\mathbb{N}}\) of complex numbers given by
\(e_{npij} = \delta_{ni}\delta_{pj}\). By bilinearity of tensor we have:

\begin{description}
\tightlist
\item[]
x\textbackslash{}tens y = \textbackslash{}left(\textbackslash{}sum\_n
x\_ne\_n\textbackslash{}right)\textbackslash{}tens\textbackslash{}left(\textbackslash{}sum\_p
y\_pe\_p\textbackslash{}right) =
\end{description}

\texttt{~\textbackslash{}sum\_\{n,p\}~x\_ny\_p\textbackslash{},~e\_n\textbackslash{}tens~e\_p~=~\textbackslash{}sum\_\{n,p\}~x\_ny\_p\textbackslash{},e\_\{np\}}

Furthermore the system of vectors \((e_{np})\) is a hilbertian basis of
\(H\tens H\): the sequence \(x=(x_{np})_{n,p\in\mathbb{N}}\) may be
written:

\begin{description}
\tightlist
\item[]
x =
\textbackslash{}sum\_\{n,p\textbackslash{}in\textbackslash{}mathbb\{N\}\}x\_\{np\}\textbackslash{},e\_\{np\}
\end{description}

\texttt{~~~~~~~~~=~\textbackslash{}sum\_\{n,p\textbackslash{}in\textbackslash{}mathbb\{N\}\}x\_\{np\}\textbackslash{},e\_n\textbackslash{}tens~e\_p}\texttt{.}

\subsubsection{An algebra isomorphism}\label{an-algebra-isomorphism}

Being both separable Hilbert spaces, \(H\) and \(H\tens H\) are
isomorphic. We will now define explicitely an iso based on partial
permutations.

We fix, once for all, a bijection from couples of natural numbers to
natural numbers that we will denote by
\((n,p)\mapsto\langle n,p\rangle\). For example set
\(\langle n,p\rangle = 2^n(2p+1) - 1\). Conversely, given
\(n\in\mathbb{N}\) we denote by \(n_{(1)}\) and \(n_{(2)}\) the unique
integers such that \(\langle n_{(1)},
n_{(2)}\rangle = n\).

This bijection can be extended into a Hilbert space isomorphism
\(\Phi:H\tens H\rightarrow H\) by defining:

\begin{description}
\tightlist
\item[]
\(e_n\tens e_p = e_{np} \mapsto e_{\langle n,p\rangle}\).
\end{description}

Now given an operator \(u\) on \(H\) we define the operator \(!u\) on
\(H\) by:

\begin{description}
\tightlist
\item[]
\(!u(e_{\langle n,p\rangle}) = \Phi(e_n\tens u(e_p))\).
\end{description}

One can check that given two operators \(u\) and \(v\) we have:

\begin{itemize}
\tightlist
\item
  \(!u!v = {!(uv)}\);
\item
  \(!(u^*) = (!u)^*\).
\end{itemize}

Due to the fact that \(\Phi\) is an isomorphism \emph{onto} we also have
\(!1=1\); this however will not be used.

We therefore have that \(!\) is a morphism on \(\mathcal{B}(H)\); it is
easily seen to be an iso (not \emph{onto} though). As this is the
crucial ingredient for interpreting the structural rules of linear
logic, we will call it the \emph{copying iso}.

\subsubsection{Interpretation of
exponentials}\label{interpretation-of-exponentials}

If we suppose that \(u = u_\varphi\) is a \(p\)-isometry generated by
the partial permutation \(\varphi\) then we have:

\begin{description}
\tightlist
\item[]
\(!u(e_{\langle n,p\rangle}) = \Phi(e_n\tens u(e_p)) = \Phi(e_n\tens e_{\varphi(p)}) = e_{\langle n,\varphi(p)\rangle}\).
\end{description}

Thus \(!u_\varphi\) is itself a \(p\)-isometry generated by the partial
permutation
\(!\varphi:n\mapsto \langle n_{(1)}, \varphi(n_{(2)})\rangle\), which
shows that the proof space is stable under the copying iso.

Given a type \(A\) we define the type \(!A\) by:

\begin{description}
\tightlist
\item[]
\(!A = \{!u, u\in A\}\biorth\)
\end{description}



\section{Game semantics}\label{game-semantics}

This article presents the game-theoretic
\href{fully_complete_model}{fully complete model} of \(MLL\). Formulas
are interpreted by games between two players, Player and Opponent, and
proofs are interpreted by strategies for Player.

\subsection{Preliminary definitions and
notations}\label{preliminary-definitions-and-notations}

\subsubsection{Sequences, Polarities}\label{sequences-polarities}

We introduce some convenient notations on sequences.

\begin{itemize}
\tightlist
\item
  If \(s\in M^*\), \(|s|\) will denote the \emph{length} of \(s\);
\item
  If \(1\leq i\leq |s|\), \(s_i\) will denote the i-th move of \(s\);
\item
  We denote by \(\sqsubseteq\) the prefix partial order on \(M^*\);
\item
  If \(s_1\) is an even-length prefix of \(s_2\), we denote it by
  \(s_1\sqsubseteq^P s_2\);
\item
  The empty sequence will be denoted by \(\epsilon\).
\end{itemize}

All moves will be equipped with a \textbf{polarity}, which will be
either Player (\(P\)) or Opponent (\(O\)).

\begin{itemize}
\tightlist
\item
  We define \(\overline{(\_)}:\{O,P\}\to \{O,P\}\) with
  \(\overline{O} = P\) and \(\overline{P} = O\).
\item
  This operation extends in a pointwise way to functions onto
  \(\{O,P\}\).
\end{itemize}

\subsubsection{Sequences on Components}\label{sequences-on-components}

We will often need to speak of sequences over (the disjoint sum of)
multiple sets of moves, along with a restriction operation.

\begin{itemize}
\tightlist
\item
  If \(M_1\) and \(M_2\) are two sets, \(M_1 + M_2\) will denote their
  disjoint sum, implemented as
  \(M_1 + M_2 = \{1\}\times M_1 \cup \{2\}\times M_2\);
\item
  In this case, if we have two functions \(\lambda_1:M_1 \to R\) and
  \(\lambda_2:M_2\to R\), we denote by
  \([\lambda_1,\lambda_2]:M_1 + M_2 \to R\) their \emph{co-pairing};
\item
  If \(s\in (M_A + M_B)^*\), the \textbf{restriction} of \(s\) to
  \(M_A\) (resp. \(M_B\)) is denoted by \(s\upharpoonright M_A\)
  (resp.\(s \upharpoonright M_B\)). Later, if \(A\) and \(B\) are games,
  this will be abbreviated \(s\upharpoonright A\) and
  \(s\upharpoonright B\).
\end{itemize}

\subsection{Games and Strategies}\label{games-and-strategies}

\subsubsection{Game constructions}\label{game-constructions}

We first give the definition for a game, then all the constructions used
to interpret the connectives and operations of \(MLL\)

The \emph{par} connective can be defined either as
\(A\parr B = (A^\bot \tens B^\bot)^\bot\), or similarly to the
\emph{tensor} except that the switching convention is in favor of
Player. We will refer to this as the \emph{switching convention for par
game}. Likewise, we define \(A\limp B = A
^\bot\parr B\).

\subsubsection{Strategies}\label{strategies}

Composition is defined by parallel interaction plus hiding. We take all
valid sequences on \(A, B\) and \(C\) which behave accordingly to
\(\sigma\) (resp. \(\tau\)) on \(A, B\) (resp. \(B,C\)). Then, we hide
all the communication in \(B\).

We also define the identities, which are simple copycat strategies :
they immediately copy on the left (resp. right) component the last
Opponent's move on the right (resp.left) component. In the following
definition, let \(L\) (resp. \(R\)) denote the left (resp. right)
occurrence of \(A\) in \(A\limp A\).

With these definitions, we get the following theorem:



\appendix
%\part{Appendices}

\chapter{Notations}\label{notations}

\section{Logical systems}\label{logical-systems}

For a given logical system such as $MLL$ (for
multiplicative linear logic), we consider the following variations:
\begin{center}
\begin{tabular}{lll}
\hline
Notation & Meaning & Connectives\\
\hline
$MLL$ &
propositional without units &
$X,{\tens},{\parr}$\\
$MLL_u$ &
propositional with units only &
$\one,\bot,{\tens},{\parr}$\\
$MLL_0$ &
propositional with units and variables &
$X,\one,\bot,{\tens},{\parr}$\\
$MLL_1$ &
first-order without units &
$X\vec{t},{\tens},{\parr},\forall
x A,\exists x
A$\\
$MLL_{01}$ &
first-order with units &
$X\vec{t},\one,\bot,{\tens},{\parr},\forall
x A,\exists x
A$\\
$MLL_2$ &
second-order propositional without units &
$X,{\tens},{\parr},\forall
X A,\exists X
A$\\
$MLL_{02}$ &
second-order propositional with units &
$X,\one,\bot,{\tens},{\parr},\forall
X A,\exists X
A$\\
$MLL_{12}$ &
first-order and second-order without units &
$X\vec{t},{\tens},{\parr},\forall
x A,\exists x A,\forall X
A,\exists X
A$\\
$MLL_{012}$
& first-order and second-order with units &
$X\vec{t},\one,\bot,{\tens},{\parr},\forall x A,\exists x A,\forall X A,\exists X A$\\
\hline
\end{tabular}
\end{center}

\section{Formulas and proof trees}\label{formulas-and-proof-trees}

\subsection{Formulas}

\begin{itemize}
\item
  First order quantification:
  $\forall x
  A$ with substitution
  $A[t/x]$
\item
  Second order quantification:
  $\forall X
  A$ with substitution
  $A[B/X]$
\item
  Quantification of arbitrary order (mainly first or second):
  $\forall\xi
  A$ with substitution
  $A[\tau/\xi]$
\end{itemize}

\subsection{Rule names}\label{rule-names}

Name of the connective, followed by some additional information if
required, followed by ``L'' for a left rule or ``R'' for a right rule. This
is for a two-sided system, ``R'' is implicit for one-sided systems. For
example: $\wedge_1
\text{add} L$.

\section{Semantics}

\subsection{\texorpdfstring{\hyperref[coherent-semantics]{Coherent spaces}}{Coherent spaces}}\label{notations-coherent-spaces}

\begin{itemize}
\item Web of the space $X$: $\web X$
\item Coherence relation of the space $X$: large $\coh_X$ and strict $\scoh_X$
\end{itemize}

\subsection{\texorpdfstring{\hyperref[finiteness-semantics]{Finiteness spaces}}{Finiteness spaces}}\label{notations-finiteness-spaces}

\begin{itemize}
\item
  Web of the finiteness space
  $\mathcal
  A$:
  $\web{\mathcal
  A}$
\item
  Finiteness structure of the space
  $\mathcal
  A$:
  $\mathfrak
  F(\mathcal A)$ (we use
  \texttt{\\mathfrak},
  which is consistent with the fact that
  $\finpowerset{\web{\mathcal
  A}}\subseteq \mathfrak
  F(\mathcal A)
  \subseteq\powerset{\web{\mathcal
  A}}$).
\end{itemize}

\section{\texorpdfstring{\hyperref[nets]{Nets}}{Nets}}

\begin{itemize}
\item The free ports of a net $R$:~$\mathrm{fp}(R)$.
\item The result of the connection of two nets
  $R$ and
  $R'$, given
  the partial bijection
  $f:\mathrm{fp}(R)\pinj
  \mathrm{fp}(R')$:
  $R\bowtie_f
  R'$.
\item The number of loops in the resulting net:
  $\Inner{R}{R'}_f$
  (includes the loops already present in
  $R$ and
  $R'$).
\end{itemize}

\section{Miscellaneous}\label{notations-miscellaneous}

\begin{itemize}
\item \hyperref[isomorphism]{Isomorphism}: $A\cong B$
\item injection: $A\hookrightarrow B$
\item partial injection: $A\pinj B$
\end{itemize}

%%% Local Variables:
%%% mode: latex
%%% TeX-master: "main"
%%% End:


\bibliographystyle{alpha}
\bibliography{ll}

\end{document}
