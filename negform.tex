\section{Negative formula}\label{negative-formula}

A \emph{negative formula} is a formula \(N\) such that \(\wn N\limp N\)
(thus a \href{https://en.wikipedia.org/wiki/F-algebra}{algebra} for the
\href{https://en.wikipedia.org/wiki/Monad_(category_theory)}{monad} \(\wn\)). As a consequence \(N\) and \(\wn N\) are \hyperref[equivalences]{equivalent}.

A formula \(N\) is negative if and only if \(N\orth\) is \hyperref[positive-formula]{positive}.

\subsection{Negative connectives}\label{negative-connectives}

A connective \(c\) of arity \(n\) is \emph{negative} if for any negative
formulas \(N_1\),...,\(N_n\), \(c(N_1,\dots,N_n)\) is negative.

\begin{proposition}[Negative connectives]
$\parr$, $\bot$, $\with$, $\top$, $\wn$ and $\forall$ are negative connectives.
\end{proposition}

\begin{proof}
This is equivalent to the fact that $\tens$, $\one$, $\plus$, $\zero$, $\oc$ and $\exists$ are \hyperref[positive-formula]{positive connectives}.
\end{proof}

More generally, \(\wn A\) is negative for any formula \(A\).

The notion of negative connective is related with but different from the
notion of \hyperref[asynchrony]{asynchronous connective}.

\subsection{Generalized structural rules}\label{generalized-structural-rules}

Negative formulas admit generalized right structural rules corresponding
to a structure of
\href{https://en.wikipedia.org/wiki/Monoid_(category_theory)}{\(\parr\)-monoid}:
\(N\parr N\limp N\) and \(\bot\limp N\).
The following rules are derivable:
\begin{equation*}
\AxRule{\Gamma\vdash N,N,\Delta}
\LabelRule{- c R}
\UnaRule{\Gamma\vdash N,\Delta}
\DisplayProof
\qquad\qquad
\AxRule{\Gamma\vdash\Delta}
\LabelRule{- w R}
\UnaRule{\Gamma\vdash N,\Delta}
\DisplayProof
\end{equation*}

\begin{proof}
This is equivalent to the \hyperref[generalized-structural-rules-pos]{generalized left structural rules} for positive formulas.
\end{proof}

Negative formulas are also acceptable in the context of the promotion
rule. The following rule is derivable:
\begin{prooftree}
\AxRule{\vdash A,N_1,\dots,N_n}
\LabelRule{- \oc R}
\UnaRule{\vdash \oc{A},N_1,\dots,N_n}
\end{prooftree}

\begin{proof}
This is equivalent to the possibility of having positive formulas in the \hyperref[generalized-structural-rules-pos]{left-hand side context of the promotion rule}.
\end{proof}


%%% Local Variables:
%%% mode: latex
%%% TeX-master: "main"
%%% End:
