\section{Negative formula}\label{negative-formula}

A \emph{negative formula} is a formula \(N\) such that \(\wn N\limp N\)
(thus a \href{Wikipedia:F-algebra}{algebra} for the
\href{Wikipedia:Monad_(category_theory)}{monad} \(\wn\)). As a
consequence \(N\) and \(\wn N\) are
\href{Sequent_calculus\#Equivalences}{equivalent}.

A formula \(N\) is negative if and only if \(N\orth\) is
\href{Positive_formula}{positive}.

\subsection{Negative connectives}\label{negative-connectives}

A connective \(c\) of arity \(n\) is \emph{negative} if for any negative
formulas \(N_1\),...,\(N_n\), \(c(N_1,\dots,N_n)\) is negative.

More generally, \(\wn A\) is negative for any formula \(A\).

The notion of negative connective is related with but different from the
notion of \href{synchronous_connective}{synchronous connective}.

\subsection{Generalized structural
rules}\label{generalized-structural-rules}

Negative formulas admit generalized right structural rules corresponding
to a structure of
\href{Wikipedia:Monoid_(category_theory)}{\(\parr\)-monoid}:
\(N\parr N\limp N\) and \(\bot\limp N\). The following rule is
derivable:

\(\AxRule{\Gamma\vdash N,N,\Delta}
\LabelRule{- c R}
\UnaRule{\Gamma\vdash N,\Delta}
\DisplayProof
\qquad
\AxRule{\Gamma\vdash\Delta}
\LabelRule{- w R}
\UnaRule{\Gamma\vdash N,\Delta}
\DisplayProof\)

Negative formulas are also acceptable in the context of the promotion
rule. The following rule is derivable:

\(\AxRule{\vdash A,N_1,\dots,N_n}
\LabelRule{- \oc R}
\UnaRule{\vdash \oc{A},N_1,\dots,N_n}
\DisplayProof\)


%%% Local Variables:
%%% mode: latex
%%% TeX-master: "main"
%%% End:
