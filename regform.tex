\section{Regular formula}\label{regular-formula}

A \emph{regular formula} is a formula \(R\) such that
\(R\linequiv\wn\oc R\).

A formula \(L\) is \emph{co-regular} if its dual \(L\orth\) is regular,
that is if \(L\linequiv\oc\wn L\).

\subsection{Alternative
characterization}\label{alternative-characterization}

\(R\) is regular if and only if it is
\href{Sequent_calculus\#Equivalences}{equivalent} to a formula of the
shape \(\wn P\) for some \href{positive_formula}{positive formula}
\(P\).

\subsection{Regular connectives}\label{regular-connectives}

A connective \(c\) of arity \(n\) is \emph{regular} if for any regular
formulas \(R_1\),...,\(R_n\), \(c(R_1,\dots,R_n)\) is regular.

More generally, \(\wn\oc A\) is regular for any formula \(A\).

