\section{Regular formula}\label{regular-formula}

A \emph{regular formula} is a formula \(R\) such that
\(R\linequiv\wn\oc R\).

A formula \(L\) is \emph{co-regular} if its dual \(L\orth\) is regular,
that is if \(L\linequiv\oc\wn L\).

\subsection{Alternative characterization}\label{alternative-characterization}

\(R\) is regular if and only if it is
\hyperref[equivalences]{equivalent} to a formula of the
shape \(\wn P\) for some \hyperref[positive-formula]{positive formula}
\(P\).

\begin{proof}
If $R$ is regular then $R\linequiv\wn\oc R$ with $\oc R$ positive. If $R\linequiv\wn P$ with $P$ positive then $R$ is regular since $P\linequiv\oc P$.
\end{proof}

\subsection{Regular connectives}\label{regular-connectives}

A connective \(c\) of arity \(n\) is \emph{regular} if for any regular
formulas \(R_1\),...,\(R_n\), \(c(R_1,\dots,R_n)\) is regular.

\begin{proposition}[Regular connectives]
$\parr$, $\bot$ and $\wn\oc$ define regular connectives.
\end{proposition}

\begin{proof}
If $R$ and $S$ are regular, $R\parr S \linequiv \wn\oc R \parr \wn\oc S \linequiv \wn{(\oc R\plus\oc S)}$ thus it is regular since $\oc R\plus\oc S$ is positive.

$\bot\linequiv\wn\zero$ thus it is regular since $\zero$ is positive.

If $R$ is regular then $\wn\oc R$ is regular, since $\wn\oc\wn\oc R\linequiv \wn\oc R$.
\end{proof}

More generally, \(\wn\oc A\) is regular for any formula \(A\).


%%% Local Variables:
%%% mode: latex
%%% TeX-master: "main"
%%% End:
