\section{Additive cut rule}\label{additive-cut-rule}

The additive cut rule is: \(\AxRule{\Gamma\vdash A,\Delta}
\AxRule{\Gamma,A\vdash\Delta}
\LabelRule{\rulename{cut\;add}}
\BinRule{\Gamma\vdash\Delta}
\DisplayProof\)

In contrary to what happens in classical logic, this rule is
\emph{not} admissible in linear logic.

The formula \(\alpha\plus\alpha\orth\) is not provable in linear logic,
while it is derivable with the additive cut rule:
\begin{prooftree}
\NulRule{\alpha\vdash\alpha}
\UnaRule{\vdash\alpha,\alpha\orth}
\LabelRule{\plus_{R2}}
\UnaRule{\vdash\alpha,\alpha\plus\alpha\orth}
\NulRule{\alpha\vdash\alpha}
\LabelRule{\plus_{R1}}
\UnaRule{\alpha\vdash\alpha\plus\alpha\orth}
\LabelRule{\rulename{cut\;add}}
\BinRule{\vdash\alpha\plus\alpha\orth}
\end{prooftree}

%%% Local Variables:
%%% mode: latex
%%% TeX-master: "main"
%%% End:
