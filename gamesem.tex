\chapter{Game semantics}\label{game-semantics}

This article presents the game-theoretic
\href{fully_complete_model}{fully complete model} of \(MLL\). Formulas
are interpreted by games between two players, Player and Opponent, and
proofs are interpreted by strategies for Player.

\subsection{Preliminary definitions and
notations}\label{preliminary-definitions-and-notations}

\subsubsection{Sequences, Polarities}\label{sequences-polarities}

We introduce some convenient notations on sequences.

\begin{itemize}
\tightlist
\item
  If \(s\in M^*\), \(|s|\) will denote the \emph{length} of \(s\);
\item
  If \(1\leq i\leq |s|\), \(s_i\) will denote the i-th move of \(s\);
\item
  We denote by \(\sqsubseteq\) the prefix partial order on \(M^*\);
\item
  If \(s_1\) is an even-length prefix of \(s_2\), we denote it by
  \(s_1\sqsubseteq^P s_2\);
\item
  The empty sequence will be denoted by \(\epsilon\).
\end{itemize}

All moves will be equipped with a \textbf{polarity}, which will be
either Player (\(P\)) or Opponent (\(O\)).

\begin{itemize}
\tightlist
\item
  We define \(\overline{(\_)}:\{O,P\}\to \{O,P\}\) with
  \(\overline{O} = P\) and \(\overline{P} = O\).
\item
  This operation extends in a pointwise way to functions onto
  \(\{O,P\}\).
\end{itemize}

\subsubsection{Sequences on Components}\label{sequences-on-components}

We will often need to speak of sequences over (the disjoint sum of)
multiple sets of moves, along with a restriction operation.

\begin{itemize}
\tightlist
\item
  If \(M_1\) and \(M_2\) are two sets, \(M_1 + M_2\) will denote their
  disjoint sum, implemented as
  \(M_1 + M_2 = \{1\}\times M_1 \cup \{2\}\times M_2\);
\item
  In this case, if we have two functions \(\lambda_1:M_1 \to R\) and
  \(\lambda_2:M_2\to R\), we denote by
  \([\lambda_1,\lambda_2]:M_1 + M_2 \to R\) their \emph{co-pairing};
\item
  If \(s\in (M_A + M_B)^*\), the \textbf{restriction} of \(s\) to
  \(M_A\) (resp. \(M_B\)) is denoted by \(s\upharpoonright M_A\)
  (resp.\(s \upharpoonright M_B\)). Later, if \(A\) and \(B\) are games,
  this will be abbreviated \(s\upharpoonright A\) and
  \(s\upharpoonright B\).
\end{itemize}

\subsection{Games and Strategies}\label{games-and-strategies}

\subsubsection{Game constructions}\label{game-constructions}

We first give the definition for a game, then all the constructions used
to interpret the connectives and operations of \(MLL\)

The \emph{par} connective can be defined either as
\(A\parr B = (A^\bot \tens B^\bot)^\bot\), or similarly to the
\emph{tensor} except that the switching convention is in favor of
Player. We will refer to this as the \emph{switching convention for par
game}. Likewise, we define \(A\limp B = A
^\bot\parr B\).

\subsubsection{Strategies}\label{strategies}

Composition is defined by parallel interaction plus hiding. We take all
valid sequences on \(A, B\) and \(C\) which behave accordingly to
\(\sigma\) (resp. \(\tau\)) on \(A, B\) (resp. \(B,C\)). Then, we hide
all the communication in \(B\).

We also define the identities, which are simple copycat strategies :
they immediately copy on the left (resp. right) component the last
Opponent's move on the right (resp.left) component. In the following
definition, let \(L\) (resp. \(R\)) denote the left (resp. right)
occurrence of \(A\) in \(A\limp A\).

With these definitions, we get the following theorem:


%%% Local Variables:
%%% mode: latex
%%% TeX-master: "main"
%%% End:
