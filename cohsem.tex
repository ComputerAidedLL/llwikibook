\chapter{Coherent semantics}\label{coherent-semantics}

\emph{Coherent semantics} was invented by Girard in the paper \emph{The
system F, 15 years later}~\cite{systemF15} with the objective of building a
denotationnal interpretation of second order intuitionnistic logic (aka
polymorphic lambda-calculus).

Coherent semantics is based on the notion of \emph{stable functions}
that was initially proposed by Gérard Berry. Stability is a condition on
Scott continuous functions that expresses the determinism of the
relation between the output and the input: the typical Scott continuous
but non stable function is the \emph{parallel or} because when the two
inputs are both set to \textbf{true}, only one of them is the reason why
the result is \textbf{true} but there is no way to determine which one.

A further achievement of coherent semantics was that it allowed to endow
the set of stable functions from \(X\) to \(Y\) with a structure of
domain, thus closing the category of coherent spaces and stable
functions. However the most interesting point was the discovery of a
special class of stable functions, \emph{linear functions}, which was
the first step leading to Linear Logic.

\section{The cartesian closed structure of coherent semantics}\label{the-cartesian-closed-structure-of-coherent-semantics}

There are three equivalent definitions of coherent spaces: the first
one, \emph{coherent spaces as domains}, is interesting from a historical
point of view as it emphazises the fact that coherent spaces are
particular cases of Scott domains. The second one, \emph{coherent spaces
as graphs}, is the most commonly used and will be our ``official''
definition in the sequel. The last one, \emph{cliqued spaces} is a
particular example of a more general scheme that one could call
``symmetric reducibility''; this scheme is underlying lots of
constructions in linear logic such as \hyperref[phase-semantics]{phase
semantics} or the proof of strong normalisation for proof-nets.

\subsection{Coherent spaces}\label{coherent-spaces}

A coherent space \(X\) is a collection of subsets of a set \(\web X\)
satisfying some conditions that will be detailed shortly. The elements
of \(X\) are called the \emph{cliques} of \(X\) (for reasons that will
be made clear in a few lines). The set \(\web X\) is called the
\emph{web} of \(X\) and its elements are called the \emph{points} of
\(X\); thus a clique is a set of points. Note that the terminology is a
bit ambiguous as the points of \(X\) are the elements of the web of
\(X\), not the elements of \(X\).

The definitions below give three equivalent conditions that have to be
satisfied by the cliques of a coherent space.

\subsubsection{As domains}\label{as-domains}

The cliques of \(X\) have to satisfy:

\begin{itemize}
\item subset closure: if \(x\subset y\in X\) then \(x\in X\),
\item singletons: \(\{a\}\in X\) for \(a\in\web X\).
\item binary compatibility: if \(A\) is a family of pairwise compatible
  cliques of \(X\), that is if \(x\cup y\in X\) for any \(x,y\in A\),
  then \(\bigcup A\in X\).
\end{itemize}

A coherent space is thus ordered by inclusion; one easily checks that it
is a domain. In particular finite cliques of \(X\) correspond to compact
elements.

\subsubsection{As graphs}\label{as-graphs}

There is a reflexive and symetric relation \(\coh_X\) on \(\web X\) (the
\emph{coherence relation}) such that any subset \(x\) of \(\web X\) is a
clique of \(X\) iff \(\forall a,b\in x,\, a\coh_X b\). In other terms
\(X\) is the set of complete subgraphs of the simple unoriented graph of
the \(\coh_X\) relation; this is the reason why elements of \(X\) are
called \emph{cliques}.

The \emph{strict coherence relation} \(\scoh_X\) on \(X\) is defined by:
\(a\scoh_X b\) iff \(a\neq b\) and \(a\coh_X b\).

A coherent space in the domain sense is seen to be a coherent space in
the graph sense by setting \(a\coh_X b\) iff \(\{a,b\}\in X\);
conversely one can check that cliques in the graph sense are subset
closed and satisfy the binary compatibility condition.

A coherent space is completely determined by its web and its coherence
relation, or equivalently by its web and its strict coherence.

\subsubsection{As cliqued spaces}\label{as-cliqued-spaces}

\begin{definition}[Duality]
Let $x, y\subseteq \web{X}$ be two sets. We will say that they are dual, written $x\perp y$ if their intersection contains at most one element: $\mathrm{Card}(x\cap y)\leq 1$. As usual, it defines an \hyperref[orthogonality-relation]{orthogonality relation} over $\powerset{\web{X}}$.
\end{definition}

The last way to express the conditions on the cliques of a coherent
space \(X\) is simply to say that we must have \(X\biorth = X\).

\subsubsection{Equivalence of definitions}\label{equivalence-of-definitions}

Let \(X\) be a cliqued space and define a relation on \(\web X\) by
setting \(a\coh_X b\) iff there is \(x\in X\) such that \(a, b\in x\).
This relation is obviously symetric; it is also reflexive because all
singletons belong to \(X\): if \(a\in \web X\) then \(\{a\}\) is dual to
any element of \(X\orth\) (actually \(\{a\}\) is dual to any subset of
\(\web X\)), thus \(\{a\}\) is in \(X\biorth\), thus in \(X\).

Let \(a\coh_X b\). Then \(\{a,b\}\in X\); indeed there is an \(x\in X\)
such that \(a, b\in x\). This \(x\) is dual to any \(y\in X\orth\), that
is meets any \(y\in X\orth\) in a most one point. Since
\(\{a,b\}\subset x\) this is also true of \(\{a,b\}\), so that
\(\{a,b\}\) is in \(X\biorth\) thus in \(X\).

Now let \(x\) be a clique for \(\coh_X\) and \(y\) be an element of
\(X\orth\). Suppose \(a, b\in x\cap y\), then since \(a\) and \(b\) are
coherent (by hypothesis on \(x\)) we have \(\{a,b\}\in X\) and since
\(y\in X\orth\) we must have that \(\{a,b\}\) and \(y\) meet in at most
one point. Thus \(a = b\) and we have shown that \(x\) and \(y\) are
dual. Since \(y\) was arbitrary this means that \(x\) is in
\(X\biorth\), thus in \(X\). Finally we get that any set of pairwise
coherent points of \(X\) is in \(X\). Conversely given \(x\in X\) its
points are obviously pairwise coherent so eventually we get that \(X\)
is a coherent space in the graph sense.

Conversely given a coherent space \(X\) in the graph sense, one can
check that it is a cliqued space. Call \emph{anticlique} a set
\(y\subset \web X\) of pairwise incoherent points: for all \(a, b\) in
\(y\), if \(a\coh_X b\) then \(a=b\). Any anticlique intersects any
clique in at most one point: let \(x\) be a clique and \(y\) be an
anticlique, then if \(a,b\in x\cap y\), since \(a, b\in x\) we have
\(a\coh_X b\) and since \(y\) is an anticlique we have \(a = b\). Thus
\(y\in X\orth\). Conversely given any \(y\in X\orth\) and \(a, b\in y\),
suppose \(a\coh_X b\). Then \(\{a,b\}\in X\), thus \(\{a,b\}\perp y\)
which entails that \(\{a, b\}\) has at most one point so that \(a = b\):
we have shown that any two elements of \(y\) are incoherent.

Thus the collection of anticliques of \(X\) is the dual \(X\orth\) of
\(X\). Note that the incoherence relation defined above is reflexive and
symetric, so that \(X\orth\) is a coherent space in the graph sense.
Thus we can do for \(X\orth\) exactly what we've just done for \(X\) and
consider the anti-anticliques, that is the anticliques for the
incoherent relation which are the cliques for the in-incoherent
relation. It is not difficult to see that this in-incoherence relation
is just the coherence relation we started with; we thus obtain that
\(X\biorth = X\), so that \(X\) is a cliqued space.

\subsection{Stable functions}\label{stable-functions}

\begin{definition}[Stable function]
Let $X$ and $Y$ be two coherent spaces. A function $F:X\longrightarrow Y$ is \emph{stable} if it satisfies:
\begin{itemize}
\item it is non decreasing: for any $x,y\in X$ if $x\subset y$ then $F(x)\subset F(y)$;
\item it is continuous (in the Scott sense): if $A$ is a directed family of cliques of $X$, that is if for any $x,y\in A$ there is a $z\in A$ such that $x\cup y\subset z$, then $\bigcup_{x\in A}F(x) = F(\bigcup A)$;
\item it satisfies the stability condition: if $x,y\in X$ are compatible, that is if $x\cup y\in X$, then $F(x\cap y) = F(x)\cap F(y)$.
\end{itemize}
\end{definition}

This definition is admitedly not very tractable. An equivalent and most
useful caracterisation of stable functions is given by the following
theorem.

\begin{theorem}
Let $F:X\longrightarrow Y$ be a non-decreasing function from the coherent space $X$ to the coherent space $Y$. The function $F$ is stable iff it satisfies: for any $x\in X$, $b\in\web Y$, if $b\in F(x)$ then there is a finite clique $x_0\subset x$ such that:
\begin{itemize}
\item $b\in F(x_0)$,
\item for any $y\subset x$ if $b\in F(y)$ then $x_0\subset y$ ($x_0$ is ``the'' minimum sub-clique of $x$ such that $b\in F(x_0)$). 
\end{itemize}
\end{theorem}

Note that the stability condition doesn't depend on the coherent space
structure and can be expressed more generally for continuous functions
on domains. However, as mentionned in the introduction, the restriction
to coherent spaces allows to endow the set of stable functions from
\(X\) to \(Y\) with a structure of coherent space.

\begin{definition}[The space of stable functions]
Let $X$ and $Y$ be coherent spaces. We denote by $X_{\mathrm{fin}}$ the set of \emph{finite} cliques of $X$. The function space $X\imp Y$ is defined by:
\begin{itemize}
\item $\web{X\imp Y} = X_{\mathrm{fin}}\times \web Y$,
\item $(x_0, a)\coh_{X\imp Y}(y_0, b)$ iff $\begin{cases}\text{if } x_0\cup y_0\in X\text{ then } a\coh_Y b,\\
\text{if } x_0\cup y_0\in X\text{ and } a = b\text{ then } x_0 = y_0\end{cases}$.
\end{itemize}
\end{definition}

One could equivalently define the strict coherence relation on
\(X\imp Y\) by: \((x_0,a)\scoh_{X\imp Y}(y_0, b)\) iff when
\(x_0\cup y_0\in X\) then \(a\scoh_Y b\) (equivalently
\(x_0\cup y_0\not\in X\) or \(a\scoh_Y b\)).

\begin{definition}[Trace of a stable function]
Let $F:X\longrightarrow Y$ be a function. The \emph{trace} of $F$ is the set:
\begin{equation*}
\mathrm{Tr}(F) = \{(x_0, b), x_0\text{ minimal such that } b\in F(x_0)\}.
\end{equation*}
\end{definition}

\begin{theorem}
$F$ is stable iff $\mathrm{Tr}(F)$ is a clique of the function space $X\imp Y$.
\end{theorem}

In particular the continuity of \(F\) entails that if \(x_0\) is minimal
such that \(b\in F(x_0)\), then \(x_0\) is finite.

\begin{definition}[The evaluation function]
Let $f$ be a clique in $X\imp Y$. We define a function $\mathrm{Fun}\,f:X\longrightarrow Y$ by: $\mathrm{Fun}\,f(x) = \{b\in Y,\text{ there is }x_0\subset x\text{ such that }(x_0, b)\in f\}$.
\end{definition}

\begin{theorem}[Closure]
If $f$ is a clique of the function space $X\imp Y$ then we have $\mathrm{Tr}(\mathrm{Fun}\,f) = f$. Conversely if $F:X\longrightarrow Y$ is a stable function then we have $F = \mathrm{Fun}\,\mathrm{Tr}(F)$.
\end{theorem}

\subsection{Cartesian product}\label{cartesian-product}

\begin{definition}[Cartesian product]
Let $X_1$ and $X_2$ be two coherent spaces. We define the coherent space $X_1\with X_2$ (read $X_1$ ``with'' $X_2$):
\begin{itemize}
\item the web is the disjoint union of the webs: $\web{X_1\with X_2} = \{1\}\times\web{X_1}\cup \{2\}\times\web{X_2}$;
\item the coherence relation is the serie composition of the relations on $X_1$ and $X_2$: $(i, a)\coh_{X_1\with X_2}(j, b)$ iff either $i\neq j$ or $i=j$ and $a\coh_{X_i} b$.
\end{itemize}
\end{definition}

This definition is just the way to put a coherent space structure on the
cartesian product. Indeed one easily shows the

\begin{theorem}
Given cliques $x_1$ and $x_2$ in $X_1$ and $X_2$, we define the subset $\langle x_1, x_2\rangle$ of $\web{X_1\with X_2}$ by: $\langle x_1, x_2\rangle = \{1\}\times x_1\cup \{2\}\times x_2$. Then $\langle x_1, x_2\rangle$ is a clique in $X_1\with X_2$.

Conversely, given a clique $x\in X_1\with X_2$, for $i=1,2$ we define $\pi_i(x) = \{a\in X_i, (i, a)\in x\}$. Then $\pi_i(x)$ is a clique in $X_i$ and the function $\pi_i:X_1\with X_2\longrightarrow X_i$ is stable.

Furthemore these two operations are inverse of each other: $\pi_i(\langle x_1, x_2\rangle) = x_i$ and $\langle\pi_1(x), \pi_2(x)\rangle = x$. In particular any clique in $X_1\with X_2$ is of the form $\langle x_1, x_2\rangle$.
\end{theorem}

Altogether the results above (and a few other more that we shall leave
to the reader) allow to get:

\begin{theorem}
The category of coherent spaces and stable functions is cartesian closed.
\end{theorem}

In particular this means that if we define
\(\mathrm{Eval}:(X\imp Y)\with X\longrightarrow Y\) by:
\(\mathrm{Eval}(\langle f, x\rangle) = \mathrm{Fun}\,f(x)\) then
\(\mathrm{Eval}\) is stable.

\section{The monoidal structure of coherent semantics}\label{the-monoidal-structure-of-coherent-semantics}

\subsection{Linear functions}\label{linear-functions}

\begin{definition}[Linear function]
A function $F:X\longrightarrow Y$ is \emph{linear} if it is stable and furthemore satisfies: for any family $A$ of pairwise compatible cliques of $X$, that is such that for any $x, y\in A$, $x\cup y\in X$, we have $\bigcup_{x\in A}F(x) = F(\bigcup A)$.
\end{definition}

In particular if we take \(A\) to be the empty family, then we have
\(F(\emptyset) = \emptyset\).

The condition for linearity is quite similar to the condition for Scott
continuity, except that we dropped the constraint that \(A\) is
\emph{directed}. Linearity is therefore much stronger than stability:
most stable functions are not linear.

However most of the functions seen so far are linear. Typically the
function \(\pi_i:X_1\with X_2\longrightarrow X_i\) is linear from wich
one may deduce that the \emph{with} construction is also a cartesian
product in the category of coherent spaces and linear functions.

As with stable function we have an equivalent and much more tractable
caracterisation of linear function:

\begin{theorem}
Let $F:X\longrightarrow Y$ be a continuous function. Then $F$ is linear iff it satisfies: for any clique $x\in X$ and any $b\in F(x)$ there is a unique $a\in x$ such that $b\in F(\{a\})$.
\end{theorem}

Just as the caracterisation theorem for stable functions allowed us to
build the coherent space of stable functions, this theorem will help us
to endow the set of linear maps with a structure of coherent space.

\begin{definition}[The linear functions space]
Let $X$ and $Y$ be coherent spaces. The \emph{linear function space} $X\limp Y$ is defined by:
\begin{itemize}
\item $\web{X\limp Y} = \web X\times \web Y$,
\item $(a,b)\coh_{X\limp Y}(a', b')$ iff $\begin{cases}\text{if }a\coh_X a'\text{ then } b\coh_Y b'\\
 \text{if }a\coh_X a' \text{ and }b=b'\text{ then }a=a'\end{cases}$
\end{itemize}
\end{definition}

Equivalently one could define the strict coherence to be:
\((a,b)\scoh_{X\limp Y}(a',b')\) iff \(a\scoh_X a'\) entails \(b\scoh_Y b'\).

\begin{definition}[Linear trace]
Let $F:X\longrightarrow Y$ be a function. The \emph{linear trace} of $F$ denoted as $\mathrm{LinTr}(F)$ is the set:
  $\mathrm{LinTr}(F) = \{(a, b)\in\web X\times\web Y$ such that $b\in F(\{a\})\}$.
\end{definition}

\begin{theorem}
If $F$ is linear then $\mathrm{LinTr}(F)$ is a clique of $X\limp Y$.
\end{theorem}

\begin{definition}[Evaluation of linear function]
Let $f$ be a clique of $X\limp Y$. We define the function $\mathrm{LinFun}\,f:X\longrightarrow Y$ by: $\mathrm{LinFun}\,f(x) = \{b\in\web Y$ such that there is an $a\in x$ satisfying $(a,b)\in f\}$.
\end{definition}

\begin{theorem}[Linear closure]
Let $f$ be a clique in $X\limp Y$. Then we have $\mathrm{LinTr}(\mathrm{LinFun}\, f) = f$. Conversely if $F:X\longrightarrow Y$ is linear then we have $F = \mathrm{LinFun}\,\mathrm{LinTr}(F)$.
\end{theorem}

It remains to define a tensor product and we will get that the category
of coherent spaces with linear functions is monoidal symetric (it is
actually *-autonomous).

\subsection{Tensor product}\label{tensor-product}

\begin{definition}[Tensor product]
Let $X$ and $Y$ be coherent spaces. Their tensor product $X\tens Y$ is defined by: $\web{X\tens Y} = \web X\times\web Y$ and $(a,b)\coh_{X\tens Y}(a',b')$ iff $a\coh_X a'$ and $b\coh_Y b'$.
\end{definition}

\begin{theorem}
The category of coherent spaces with linear maps and tensor product is \hyperref[modeling-imll]{monoidal symetric closed}.
\end{theorem}

The closedness is a consequence of the existence of the linear
isomorphism:
\begin{equation*}
\varphi:X\tens Y\limp Z\ \stackrel{\sim}{\longrightarrow}\ X\limp(Y\limp Z)
\end{equation*}
that is defined by its linear trace:
\(\mathrm{LinTr}(\varphi) = \{(((a, b), c), (a, (b, c))),\, a\in\web X,\, b\in \web Y,\, c\in\web Z\}\).

\subsection{Linear negation}\label{linear-negation}

\begin{definition}[Linear negation]
Let $X$ be a coherent space. We define the \emph{incoherence relation} on $\web X$ by: $a\incoh_X b$ iff $a\coh_X b$ entails $a=b$. The incoherence relation is reflexive and symetric; we call \emph{dual} or \emph{linear negation} of $X$ the associated coherent space denoted $X\orth$, thus defined by: $\web{X\orth} = \web X$ and $a\coh_{X\orth} b$ iff $a\incoh_X b$.
\end{definition}

The cliques of \(X\orth\) are called the \emph{anticliques} of \(X\). As
seen in the section on cliqued spaces we have \(X\biorth=X\).

\begin{theorem}
The category of coherent spaces with linear maps, tensor product and linear negation is *-autonomous.
\end{theorem}

This is in particular consequence of the existence of the isomorphism:
\begin{equation*}
\varphi:X\limp Y\ \stackrel{\sim}{\longrightarrow}\ Y\orth\limp X\orth
\end{equation*}
defined by its linear trace:
\(\mathrm{LinTr}(\varphi) = \{((a, b), (b, a)),\, a\in\web X,\, b\in\web Y\}\).

\section{Exponentials}\label{exponentials}

In linear algebra, bilinear maps may be factorized through the tensor
product. Similarly there is a coherent space \(\oc X\) that allows to
factorize stable functions through linear functions.

\begin{definition}[Of course]
Let $X$ be a coherent space; recall that $X_{\mathrm{fin}}$ denotes the set of finite cliques of $X$. We define the space $\oc X$ (read ``of course $X$'') by: $\web{\oc X} = X_{\mathrm{fin}}$ and $x_0\coh_{\oc X}y_0$ iff $x_0\cup y_0$ is a clique of $X$.
\end{definition}

Thus a clique of \(\oc X\) is a set of finite cliques of \(X\) the union
of wich is a clique of \(X\).

\begin{theorem}
Let $X$ be a coherent space. Denote by $\beta:X\longrightarrow \oc X$ the stable function whose trace is: $\mathrm{Tr}(\beta) = \{(x_0, x_0),\, x_0\in X_{\mathrm{fin}}\}$. Then for any coherent space $Y$ and any stable function $F: X\longrightarrow Y$ there is a unique \emph{linear} function $\bar F:\oc X\longrightarrow Y$ such that $F = \bar F\circ \beta$.

Furthermore we have $X\imp Y = \oc X\limp Y$.
\end{theorem}

\begin{theorem}[The exponential isomorphism]
Let $X$ and $Y$ be two coherent spaces. Then there is a linear isomorphism:
\begin{equation*}
\varphi:\oc(X\with Y)\quad\stackrel{\sim}{\longrightarrow}\quad \oc X\tens\oc Y.
\end{equation*}
\end{theorem}

The iso \(\varphi\) is defined by its trace:
\begin{equation*}
\mathrm{Tr}(\varphi) = \{(x_0, (\pi_1(x_0), \pi_2(x_0)), x_0\text{ finite clique of } X\with Y\}.
\end{equation*}
This isomorphism, that sends an additive structure (the web of a with is
obtained by disjoint union) onto a multiplicative one (the web of a
tensor is obtained by cartesian product) is the reason why the of course
is called an \emph{exponential}.

\section{Dual connectives and neutrals}\label{dual-connectives-and-neutrals}

By linear negation all the constructions defined so far
(\(\with, \tens, \oc\)) have a dual.

\subsection{The direct sum}\label{the-direct-sum}

The dual of \(\with\) is \(\plus\) defined by:
\(X\plus Y = (X\orth\with Y\orth)\orth\). An equivalent definition is
given by:
\(\web{X\plus Y} = \web{X\with Y} = \{1\}\times \web X \cup \{2\}\times\web Y\)
and
\((i, a)\coh_{X\plus Y} (j, b)\text{ iff } i = j = 1 \text{ and } a\coh_X b,\text{ or }i = j = 2\text{ and } a\coh_Y b\).

\begin{theorem}
Let $x'$ be a clique of $X\plus Y$; then $x'$ is of the form $\{i\}\times x$ where $i = 1\text{ and }x\in X$, or $i = 2\text{ and }x\in Y$.

Denote $\mathrm{inl}:X\longrightarrow X\plus Y$ the function defined by $\mathrm{inl}(x) = \{1\}\times x$ and by $\mathrm{inr}:Y\longrightarrow X\plus Y$ the function defined by $\mathrm{inr}(x) = \{2\}\times x$. Then $\mathrm{inl}$ and $\mathrm{inr}$ are linear.

If $F:X\longrightarrow Z$ and $G:Y\longrightarrow Z$ are ''linear'' functions then the function $H:X\plus Y \longrightarrow Z$ defined by $H(\mathrm{inl}(x)) = F(x)$ and $H(\mathrm{inr}(y)) = G(y)$ is linear.
\end{theorem}

In other terms \(X\plus Y\) is the direct sum of \(X\) and \(Y\). Note
that in the theorem all functions are \emph{linear}. Things doesn't work
so smoothly for stable functions. Historically it was after noting this
defect of coherent semantics w.r.t. the intuitionnistic implication that
Girard was leaded to discover linear functions.

\subsection{The par and the why not}\label{the-par-and-the-why-not}

We now come to the most mysterious constructions of coherent semantics:
the duals of the tensor and the of course.

The \emph{par} is the dual of the tensor, thus defined by:
\(X\parr Y = (X\orth\tens Y\orth)\orth\). From this one can deduce the
definition in graph terms:
\(\web{X\parr Y} = \web{X\tens Y} = \web X\times \web Y\) and
\((a,b)\scoh_{X\parr Y} (a',b')\) iff \(a\scoh_X a'\) or
\(b\scoh_Y b'\). With this definition one sees that we have:
\begin{equation*}
X\limp Y = X\orth\parr Y
\end{equation*}
for any coherent spaces \(X\) and \(Y\). This equation can be seen as an
alternative definition of the par: \(X\parr Y = X\orth\limp Y\).

Similarly the dual of the of course is called \emph{why not} defined by:
\(\wn X = (\oc X\orth)\orth\). From this we deduce the definition in the
graph sense which is a bit tricky: \(\web{\wn X}\) is the set of finite
anticliques of \(X\), and given two finite anticliques \(x\) and \(y\)
of \(X\) we have \(x\scoh_{\wn X} y\) iff there is \(a\in x\) and
\(b\in y\) such that \(a\scoh_X b\).

Note that both for the par and the why not it is much more convenient to
define the strict coherence than the coherence.

With these two last constructions, the equation between the stable
function space, the of course and the linear function space may be
written:
\begin{equation*}
X\imp Y = \wn X\orth\parr Y.
\end{equation*}

\subsection{One and bottom}\label{one-and-bottom}

Depending on the context we denote by \(\one\) or \(\bot\) the coherent
space whose web is a singleton and whose coherence relation is the
trivial reflexive relation.

\begin{theorem}
$\one$ is neutral for tensor, that is, there is a linear isomorphism $\varphi:X\tens\one\ \stackrel{\sim}{\longrightarrow}\ X$.

Similarly $\bot$ is neutral for par.
\end{theorem}

\subsection{Zero and top}\label{zero-and-top}

Depending on the context we denote by \(\zero\) or \(\top\) the coherent
space with empty web.

\begin{theorem}
$\zero$ is neutral for the direct sum $\plus$, $\top$ is neutral for the cartesian product $\with$.
\end{theorem}

\begin{remark}
It is one of the main defect of coherent semantics w.r.t. linear logic that it identifies the neutrals: in coherent semantics $\zero = \top$ and $\one = \bot$. However there is no known semantics of LL that solves this problem in a satisfactory way.
\end{remark}

\section{After coherent semantics}\label{after-coherent-semantics}

Coherent semantics was an important milestone in the modern theory of
logic of programs, in particular because it leaded to the invention of
Linear Logic, and more generally because it establishes a strong link
between logic and linear algebra; this link is nowadays aknowledged by
the customary use of \hyperref[categorical-semantics]{monoidal categories}
in logic. In some sense coherent semantics is a precursor of many
forthcoming works that explore the linear nature of logic as for example
\hyperref[geometry-of-interaction]{geometry of interaction} which interprets
proofs by operators or \hyperref[finiteness-semantics]{finiteness semantics}
which interprets formulas as vector spaces and resulted in
\wantedpage{differential linear logic}...

Lots of this work have been motivated by the fact that coherent
semantics is not complete as a semantics of programs (technically one
says that it is not \emph{fully abstract}). In order to see this, let us
firts come back on the origin of the central concept of \emph{stability}
which as pointed above originated in the study of the sequentiality in
programs.

\subsection{Sequentiality}\label{sequentiality}

Sequentiality is a property that we will not define here (it would
diserve its own article). We rely on the intuition that a function of
\(n\) arguments is sequential if one can determine which of these
argument is examined first during the computation. Obviously any
function implemented in a functionnal language is sequential; for
example the function \emph{or} defined à la CAML by:
\begin{quotation}
\texttt{or\ =\ fun\ (x,\ y)\ -\textgreater{}\ if\ x\ then\ true\ else\ y}
\end{quotation}
examines its argument x first. Note that this may be expressed more
abstractly by the property: \(\mathrm{or}(\bot, x) = \bot\) for any
boolean \(x\): the function \emph{or} needs its first argument in order
to compute anything. On the other hand we have
\(\mathrm{or}(\mathrm{true}, \bot) = \mathrm{true}\): in some case (when
the first argument is true), the function doesn't need its second
argument at all.

The typical non sequential function is the \emph{parallel or} (that one
cannot define in a CAML like language).

For a while one may have believed that the stability condition on which
coherent semantics is built was enough to capture the notion of
\emph{sequentiality} of programs. A hint was the already mentionned fact
that the \emph{parallel or} is not stable. This diserves a bit of
explanation.

\subsubsection{The parallel or is not stable}\label{the-parallel-or-is-not-stable}

Let \(B\) be the coherent space of booleans, also know as the flat
domain of booleans: \(\web B = \{tt, ff\}\) where \(tt\) and \(ff\) are
two arbitrary distinct objects (for example one may take \(tt = 0\) and
\(ff = 1\)) and for any \(b_1, b_2\in \web B\), define \(b_1\coh_B b_2\)
iff \(b_1 = b_2\). Then \(B\) has exactly three cliques: the empty
clique that we shall denote \(\bot\), the singleton \(\{tt\}\) that we
shall denote \(T\) and the singleton \(\{ff\}\) that we shall denote
\(F\). These three cliques are ordered by inclusion: \(\bot \leq T, F\)
(we use \(\leq\) for \(\subset\) to enforce the idea that coherent
spaces are domains).

Recall the \hyperref[cartesian-product]{definition of the
with}, and in particular that any clique of \(B\with B\) has the form
\(\langle x, y\rangle\) where \(x\) and \(y\) are cliques of \(B\). Thus
\(B\with B\) has 9 cliques:
\(\langle\bot,\bot\rangle,\ \langle\bot, T\rangle,\ \langle\bot, F\rangle,\ \langle T,\bot\rangle,\ \dots\)
that are ordered by the product order:
\(\langle x,y\rangle\leq \langle x,y\rangle\) iff \(x\leq x'\) and
\(y\leq y'\).

With these notations in mind one may define the parallel or by:
\(\begin{array}{rcl}
\mathrm{Por} : B\with B & \longrightarrow&  B\\
  \langle T,\bot\rangle & \longrightarrow&  T\\
  \langle \bot,T\rangle & \longrightarrow&  T\\
    \langle F, F\rangle & \longrightarrow&  F
\end{array}\)

The function is completely determined if we add the assumption that it
is non decreasing; for example one must have
\(\mathrm{Por}\langle\bot,\bot\rangle = \bot\) because the lhs has to be
less than both \(T\) and \(F\) (because
\(\langle\bot,\bot\rangle \leq \langle T,\bot\rangle\) and
\(\langle\bot,\bot\rangle \leq \langle F,F\rangle\)).

The function is not stable because
\(\langle T,\bot\rangle \cap \langle \bot, T\rangle = \langle\bot, \bot\rangle\),
thus
\(\mathrm{Por}(\langle T,\bot\rangle \cap \langle \bot, T\rangle) = \bot\)
whereas
\(\mathrm{Por}\langle T,\bot\rangle \cap \mathrm{Por}\langle \bot, T\rangle = T\cap T = T\).

Another way to see this is: suppose \(x\) and \(y\) are two cliques of
\(B\) such that \(tt\in \mathrm{Por}\langle x, y\rangle\), which means
that \(\mathrm{Por}\langle x, y\rangle = T\); according to the
\hyperref[stable-functions]{caracterisation theorem of stable
functions}, if \(\mathrm{Por}\) were stable then there would be a unique
minimum \(x_0\) included in \(x\), and a unique minimum \(y_0\) included
in \(y\) such that \(\mathrm{Por}\langle x_0, y_0\rangle = T\). This is
not the case because both \(\langle T,\bot\rangle\) and
\(\langle T,\bot\rangle\) are minimal such that their value is \(T\).

In other terms, knowing that \(\mathrm{Por}\langle x, y\rangle = T\)
doesn't tell which of \(x\) of \(y\) is responsible for that, although
we know by the definition of \(\mathrm{Por}\) that only one of them is.
Indeed the \(\mathrm{Por}\) function is not representable in sequential
programming languages such as (typed) lambda-calculus.

So the first genuine idea would be that stability caracterises
sequentiality; but...

\subsubsection{The Gustave function is stable}\label{the-gustave-function-is-stable}

The Gustave function, so-called after an old joke, was found by Gérard
Berry as an example of a function that is stable but non sequential. It
is defined by:

\(\begin{array}{rcl}
  B\with B\with B           & \longrightarrow&  B\\
  \langle T, F, \bot\rangle & \longrightarrow&  T\\
  \langle \bot, T, F\rangle & \longrightarrow&  T\\
  \langle F, \bot, T\rangle & \longrightarrow&  T\\
  \langle x, y, z\rangle    & \longrightarrow&  F
\end{array}\)

The last clause is for all cliques \(x\), \(y\) and \(z\) such that
\(\langle x, y ,z\rangle\) is incompatible with the three cliques
\(\langle T, F, \bot\rangle\), \(\langle \bot, T, F\rangle\) and
\(\langle F, \bot, T\rangle\), that is such that the union with any of
these three cliques is not a clique in \(B\with B\with B\). We shall
denote \(x_1\), \(x_2\) and \(x_3\) these three cliques.

We furthemore assume that the Gustave function is non decreasing, so
that we get \(G\langle\bot,\bot,\bot\rangle = \bot\).

We note that \(x_1\), \(x_2\) and \(x_3\) are pairwise incompatible.
From this we can deduce that the Gustave function is stable: typically
if \(G\langle x,y,z\rangle = T\) then exactly one of the \(x_i\)s is
contained in \(\langle x, y, z\rangle\).

However it is not sequential because there is no way to determine which
of its three arguments is examined first: it is not the first one
otherwise we would have \(G\langle\bot, T, F\rangle = \bot\) and
similarly it is not the second one nor the third one.

In other terms there is no way to implement the Gustave function by a
lambda-term (or in any sequential programming language). Thus coherent
semantics is not complete w.r.t. lambda-calculus.

The research for a right model for sequentiality was the motivation for
lot of work, \emph{e.g.}, \emph{sequential algorithms} by Gérard Bérry
and Pierre-Louis Currien in the early eighties, that were more recently
reformulated as a kind of \hyperref[game-semantics]{game model}, and the
theory of \emph{hypercoherent spaces} by Antonio Bucciarelli and Thomas
Ehrhard.

\subsection{Multiplicative neutrals and the mix rule}\label{multiplicative-neutrals-and-the-mix-rule}

Coherent semantics is slightly degenerated w.r.t. linear logic because
it identifies multiplicative neutrals (it also identifies additive
neutrals but that's yet another problem): the coherent spaces \(\one\)
and \(\bot\) are equal.

The first consequence of the identity \(\one = \bot\) is that the
formula \(\one\limp\bot\) becomes provable, and so does the formula
\(\bot\). Note that this doesn't entail (as in classical logic or
intuitionnistic logic) that linear logic is incoherent because the
principle \(\bot\limp A\) for any formula \(A\) is still not provable.

The equality \(\one = \bot\) has also as consequence the fact that
\(\bot\limp\one\) (or equivalently the formula \(\one\parr\one\)) is
provable. This principle is also known as the \hyperref[mix]{mix rule}
\begin{prooftree}
\AxRule{\vdash \Gamma}
\AxRule{\vdash \Delta}
\LabelRule{\rulename{mix}}
\BinRule{\vdash \Gamma,\Delta}
\end{prooftree}
as it can be used to show that this rule is admissible:
\begin{prooftree}
\AxRule{\vdash\Gamma}
\LabelRule{\bot_R}
\UnaRule{\vdash\Gamma, \bot}
\AxRule{\vdash\Delta}
\LabelRule{\bot_R}
\UnaRule{\vdash\Delta, \bot}
\BinRule{\vdash \Gamma, \Delta, \bot\tens\bot}
\NulRule{\vdash \one\parr\one}
\LabelRule{\rulename{cut}}
\BinRule{\vdash\Gamma,\Delta}
\end{prooftree}

None of the two principles \(1\limp\bot\) and \(\bot\limp\one\) are
valid in linear logic. To correct this one could extend the syntax of
linear logic by adding the mix-rule. This is not very satisfactory as
the mix rule violates some principles of
\hyperref[polarized-linear-logic]{Polarized linear logic}, typically the
fact that as sequent of the form \(\vdash P_1, P_2\) where \(P_1\) and
\(P_2\) are positive, is never provable.

On the other hand the mix-rule is valid in coherent semantics so one
could try to find some other model that invalidates the mix-rule. For
example Girard's Coherent Banach spaces were an attempt to address this
issue.


%%% Local Variables:
%%% mode: latex
%%% TeX-master: "main"
%%% End:
