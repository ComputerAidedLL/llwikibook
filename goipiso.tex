\section{GoI for MELL: partial isometries}\label{goi-for-mell-partial-isometries}


\subsection{Operators, partial isometries}\label{operators-partial-isometries}

We will denote by \(H\) the Hilbert space \(\ell^2(\mathbb{N})\) of
sequences \((x_n)_{n\in\mathbb{N}}\) of complex numbers such that the
series \(\sum_{n\in\mathbb{N}}|x_n|^2\) converges. If
\(x = (x_n)_{n\in\mathbb{N}}\) and \(y = (y_n)_{n\in\mathbb{N}}\) are
two vectors of \(H\) their \emph{scalar product} is:
\begin{equation*}
\langle x, y\rangle = \sum_{n\in\mathbb{N}} x_n\bar y_n.
\end{equation*}

Two vectors of \(H\) are \emph{orthogonal} if their scalar product is null.
We will say that two subspaces are \emph{disjoint} when any two
vectors taken in each subspace are orthorgonal. Note that this notion is
different from the set theoretic one, in particular two disjoint
subspaces always have exactly one vector in common: \(0\).

The \emph{norm} of a vector is the square root of the scalar product
with itself:
\begin{equation*}
\|x\| = \sqrt{\langle x, x\rangle}.
\end{equation*}

Let us denote by \((e_k)_{k\in\mathbb{N}}\) the canonical
\emph{hilbertian basis} of \(H\):
\(e_k = (\delta_{kn})_{n\in\mathbb{N}}\) where \(\delta_{kn}\) is the
Kroenecker symbol: \(\delta_{kn}=1\) if \(k=n\), \(0\) otherwise. Thus
if \(x=(x_n)_{n\in\mathbb{N}}\) is a sequence in \(H\) we have:
\begin{equation*}
x = \sum_{n\in\mathbb{N}} x_ne_n.
\end{equation*}

An \emph{operator} on \(H\) is a \emph{continuous} linear map from \(H\)
to \(H\).\footnote{Continuity is equivalent to the fact that operators are
\emph{bounded}, which means that one may define the \emph{norm} of an
operator \(u\) as the sup on the unit ball of the norms of its values:
\begin{equation*}
\|u\| = \sup_{\{x\in H,\, \|x\| = 1\}}\|u(x)\|.
\end{equation*}}
The set of (bounded) operators is denoted by \(\mathcal{B}(H)\).

The \emph{range} or \emph{codomain} of the operator \(u\) is the set of
images of vectors; the \emph{kernel} of \(u\) is the set of vectors that
are annihilated by \(u\); the \emph{domain} of \(u\) is the set of
vectors orthogonal to the kernel, \emph{ie}, the maximal subspace
disjoint with the kernel:
\begin{itemize}
\item \(\mathrm{Codom}(u) = \{u(x),\, x\in H\}\);
\item \(\mathrm{Ker}(u) = \{x\in H,\, u(x) = 0\}\);
\item \(\mathrm{Dom}(u) = \{x\in H,\, \forall y\in\mathrm{Ker}(u), \langle x, y\rangle = 0\}\).
\end{itemize}

These three sets are closed subspaces of \(H\).

The \emph{adjoint} of an operator \(u\) is the operator \(u^*\) defined
by \(\langle u(x), y\rangle = \langle x, u^*(y)\rangle\) for any
\(x,y\in H\). Adjointness is well behaved w.r.t. composition of
operators:
\begin{equation*}
(uv)^* = v^*u^*.
\end{equation*}

A \emph{projector} is an idempotent operator of norm \(0\) (the
projector on the null subspace) or \(1\), that is an operator \(p\) such
that \(p^2 = p\) and \(\|p\| = 0\) or \(1\). A projector is auto-adjoint
and its domain is equal to its codomain.

A \emph{partial isometry} is an operator \(u\) satisfying \(uu^* u =
u\); this condition entails that we also have \(u^*uu^* =
u^*\). As a consequence \(uu^*\) and \(uu^*\) are both projectors,
called respectively the \emph{initial} and the \emph{final} projector of
\(u\) because their (co)domains are respectively the domain and the
codomain of \(u\):
\begin{itemize}
\item \(\mathrm{Dom}(u^*u) = \mathrm{Codom}(u^*u) = \mathrm{Dom}(u)\);
\item \(\mathrm{Dom}(uu^*) = \mathrm{Codom}(uu^*) = \mathrm{Codom}(u)\).
\end{itemize}

The restriction of \(u\) to its domain is an isometry. Projectors are
particular examples of partial isometries.

If \(u\) is a partial isometry then \(u^*\) is also a partial isometry
the domain of which is the codomain of \(u\) and the codomain of which
is the domain of \(u\).

If the domain of \(u\) is \(H\) that is if \(u^* u = 1\) we say that
\(u\) has \emph{full domain}, and similarly for codomain. If \(u\) and
\(v\) are two partial isometries then we have:
\begin{itemize}
\item \(uv^* = 0\) iff \(u^*uv^*v = 0\) iff the domains of \(u\) and \(v\)
  are disjoint;
\item \(u^*v = 0\) iff \(uu^*vv^* = 0\) iff the codomains of \(u\) and \(v\)
  are disjoint;
\item \(uu^* + vv^* = 1\) iff the codomains of \(u\) and \(v\) are disjoint
  and their direct sum is \(H\).
\end{itemize}


\subsection{Partial permutations}\label{partial-permutations}

We will now define our proof space which turns out to be the set of
partial isometries acting as permutations on the canonical basis
\((e_n)_{n\in\mathbb{N}}\).

More precisely a \emph{partial permutation} \(\varphi\) on
\(\mathbb{N}\) is a one-to-one map defined on a subset \(D_\varphi\) of
\(\mathbb{N}\) onto a subset \(C_\varphi\) of \(\mathbb{N}\).
\(D_\varphi\) is called the \emph{domain} of \(\varphi\) and
\(C_\varphi\) its \emph{codomain}. Partial permutations may be composed:
if \(\psi\) is another partial permutation on \(\mathbb{N}\) then
\(\varphi\circ\psi\) is defined by:
\begin{itemize}
\item \(n\in D_{\varphi\circ\psi}\) iff \(n\in D_\psi\) and
  \(\psi(n)\in D_\varphi\);
\item if \(n\in D_{\varphi\circ\psi}\) then
  \(\varphi\circ\psi(n) = \varphi(\psi(n))\);
\item the codomain of \(\varphi\circ\psi\) is the image of the domain:
  \(C_{\varphi\circ\psi} = \{\varphi(\psi(n)), n\in D_{\varphi\circ\psi}\}\).
\end{itemize}

Partial permutations are well known to form a structure of \emph{inverse
monoid} that we detail now.

Given a a subset \(D\) of \(\mathbb{N}\), the \emph{partial identity} on
\(D\) is the partial permutation \(\varphi\) defined by:
\begin{itemize}
\item \(D_\varphi = D\);
\item \(\varphi(n) = n\) for any \(n\in D_\varphi\).
\end{itemize}

Thus the codomain of \(\varphi\) is \(D\).

The partial identity on \(D\) will be denoted by \(1_D\). Partial
identities are idempotent for composition.

Among partial identities one finds the identity on the empty subset,
that is the empty map, that we will denote by \(0\) and the identity on
\(\mathbb{N}\) that we will denote by \(1\). This latter permutation is
the neutral for composition.

If \(\varphi\) is a partial permutation there is an inverse partial
permutation \(\varphi^{-1}\) whose domain is
\(D_{\varphi^{-1}} = C_{\varphi}\) and who satisfies:
\begin{align*}
\varphi^{-1}\circ\varphi &= 1_{D_\varphi} \\
\varphi\circ\varphi^{-1} &= 1_{C_\varphi}
\end{align*}


\subsection{The proof space}\label{the-proof-space}

Given a partial permutation \(\varphi\) one defines a partial isometry
\(u_\varphi\) by:
\begin{equation*}
u_\varphi(e_n) =
\begin{cases}
e_{\varphi(n)} & \text{if $n\in D_\varphi$} \\
0 & \text{otherwise}
\end{cases}
\end{equation*}

In other terms if \(x=(x_n)_{n\in\mathbb{N}}\) is a sequence in
\(\ell^2\) then \(u_\varphi(x)\) is the sequence
\((y_n)_{n\in\mathbb{N}}\) defined by:
\begin{equation*}
y_n = x_{\varphi^{-1}(n)} \text{ if $n\in C_\varphi$, $0$ otherwise.}
\end{equation*}

We will (not so abusively) write \(e_{\varphi(n)} = 0\) when
\(\varphi(n)\) is undefined so that the definition of \(u_\varphi\)
reads:
\begin{equation*}
u_\varphi(e_n) = e_{\varphi(n)}.
\end{equation*}

The domain of \(u_\varphi\) is the subspace spanned by the family
\((e_n)_{n\in D_\varphi}\) and the codomain of \(u_\varphi\) is the
subspace spanned by \((e_n)_{n\in C_\varphi}\). In particular if
\(\varphi\) is \(1_D\) then \(u_\varphi\) is the projector on the
subspace spanned by \((e_n)_{n\in D}\).

\begin{definition}
We call \emph{$p$-isometry} a partial isometry of the form $u_\varphi$ where $\varphi$ is a partial permutation on $\mathbb{N}$. The \emph{proof space} $\mathcal{P}$ is the set of all $p$-isometries.
\end{definition}

\begin{proposition}
Let $\varphi$ and $\psi$ be two partial permutations. We have:
\begin{equation*}
u_\varphi u_\psi = u_{\varphi\circ\psi}.
\end{equation*}

The adjoint of $u_\varphi$ is:
\begin{equation*}
u_\varphi^* = u_{\varphi^{-1}}.
\end{equation*}

In particular the initial projector of $u_{\varphi}$ is given by:
\begin{equation*}
u_\varphi u^*_\varphi = u_{1_{D_\varphi}}.
\end{equation*}
and the final projector of $u_\varphi$ is:
\begin{equation*}
u^*_\varphi u_\varphi = u_{1_{C_\varphi}}.
\end{equation*}

If $p$ is a projector in $\mathcal{P}$ then there is a partial identity $1_D$ such that $p= u_{1_D}$.

Projectors commute, in particular we have:
\begin{equation*}
u_\varphi u_\varphi^*u_\psi u_\psi^* = u_\psi u_\psi^*u_\varphi u_\varphi^*.
\end{equation*}
\end{proposition}

Note that this entails all the other commutations of projectors:
\(u^*_\varphi u_\varphi u_\psi u^*_\psi = u_\psi u^*_\psi u^*_\varphi u_\varphi\)
and
\(u^*_\varphi u_\varphi u^*_\psi u_\psi = u^*_\psi u_\psi u^*_\varphi u_\varphi\).

In particular note that \(0\) is a \(p\)-isometry. The set
\(\mathcal{P}\) is a submonoid of \(\mathcal{B}(H)\) but it is not a
subalgebra.\footnote{\(\mathcal{P}\) is the normalizing groupoid of the
  maximal commutative subalgebra of \(\mathcal{B}(H)\) consisiting of
  all operators \emph{diagonalizable} in the canonical basis.}
In general given \(u,v\in\mathcal{P}\) we don't necessarily have
\(u+v\in\mathcal{P}\). However we have:

\begin{proposition}
Let $u, v\in\mathcal{P}$. Then $u+v\in\mathcal{P}$ iff $u$ and $v$ have disjoint domains and disjoint codomains, that is:
\begin{equation*}
u+v\in\mathcal{P}\text{ iff }uu^*vv^* = u^*uv^*v = 0.
\end{equation*}
\end{proposition}

\begin{proof}
Suppose for contradiction that $e_n$ is in the domains of $u$ and $v$. There are integers $p$ and $q$ such that $u(e_n) = e_p$ and $v(e_n) = e_q$ thus $(u+v)(e_n) = e_p + e_q$ which is not a basis vector; therefore $u+v$ is not a $p$-permutation.
\end{proof}

As a corollary note that if \(u+v=0\) then \(u=v=0\).


\subsection{From operators to matrices: internalization/externalization}\label{from-operators-to-matrices-internalizationexternalization}

It will be convenient to view operators on \(H\) as acting on
\(H\oplus H\), and conversely. For this purpose we define an isomorphism
\(H\oplus H \cong H\) by \(x\oplus y\rightsquigarrow p(x)+q(y)\) where
\(p:H\mapsto H\) and \(q:H\mapsto H\) are partial isometries given by:
\begin{align*}
p(e_n) &= e_{2n} \\
q(e_n) &= e_{2n+1}.
\end{align*}

From the definition \(p\) and \(q\) have full domain, that is satisfy
\(p^* p = q^* q = 1\). On the other hand their codomains are disjoint,
thus we have \(p^*q = q^*p = 0\). As the sum of their codomains is the
full space \(H\) we also have \(pp^* + qq^* = 1\).

Note that we have choosen \(p\) and \(q\) in \(\mathcal{P}\). However
the choice is arbitrary: any two \(p\)-isometries with full domain and
disjoint codomains would do the job.

Given an operator \(u\) on \(H\) we may \emph{externalize} it obtaining
an operator \(U\) on \(H\oplus H\) defined by the matrix:
\begin{equation*}
U = \begin{pmatrix}
  u_{11} & u_{12}\\
  u_{21} & u_{22}
  \end{pmatrix}
\end{equation*}
where the \(u_{ij}\)'s are given by:
\begin{align*}
u_{11} &= p^*up \\
u_{12} &= p^*uq \\
u_{21} &= q^*up \\
u_{22} &= q^*uq.
\end{align*}

The \(u_{ij}\)'s are called the \emph{external components} of \(u\). The
externalization is functorial in the sense that if \(v\) is another
operator externalized as:
\begin{equation*}
V = \begin{pmatrix}
  v_{11} & v_{12}\\
  v_{21} & v_{22}
  \end{pmatrix} 
= \begin{pmatrix}
  p^*vp & p^*vq\\
  q^*vp & q^*vq
  \end{pmatrix}
\end{equation*}
then the externalization of \(uv\) is the matrix product \(UV\).

As \(pp^* + qq^* = 1\) we have:
\begin{equation*}
u = (pp^*+qq^*)u(pp^*+qq^*) = pu_{11}p^* + pu_{12}q^* + qu_{21}p^* + qu_{22}q^*
\end{equation*}
which entails that externalization is reversible, its converse being
called \emph{internalization}.

If we suppose that \(u\) is a \(p\)-isometry then so are the components
\(u_{ij}\)'s. Thus the formula above entails that the four terms of the
sum have pairwise disjoint domains and pairwise disjoint codomains from
which we deduce:

\begin{proposition}
If $u$ is a $p$-isometry and $u_{ij}$ are its external components then:
\begin{itemize}
\item $u_{1j}$ and $u_{2j}$ have disjoint domains, that is $u_{1j}^*u_{1j}u_{2j}^*u_{2j} = 0$ for $j=1,2$;
\item $u_{i1}$ and $u_{i2}$ have disjoint codomains, that is $u_{i1}u_{i1}^*u_{i2}u_{i2}^* = 0$ for $i=1,2$.
\end{itemize}
\end{proposition}

As an example of computation in \(\mathcal{P}\) let us check that the
product of the final projectors of \(pu_{11}p^*\) and \(pu_{12}q^*\) is
null:
\begin{align*}
    (pu_{11}p^*)(pu^*_{11}p^*)(pu_{12}q^*)(qu_{12}^*p^*)
    &= pu_{11}u_{11}^*u_{12}u_{12}^*p^*\\
    &= pp^*upp^*u^*pp^*uqq^*u^*pp^*\\
    &= pp^*u(pp^*)(u^*pp^*u)qq^*u^*pp^*\\
    &= pp^*u(u^*pp^*u)(pp^*)qq^*u^*pp^*\\
    &= pp^*uu^*pp^*u(pp^*)(qq^*)u^*pp^*\\
    &= 0
\end{align*}
where we used the fact that all projectors in \(\mathcal{P}\) commute,
which is in particular the case of \(pp^*\) and \(u^*pp^*u\).


%%% Local Variables:
%%% mode: latex
%%% TeX-master: "main"
%%% End:
