\chapter{Phase semantics}\label{phase-semantics}

\section{Introduction}\label{introduction}

The semantics given by phase spaces is a kind of "formula and
provability semantics", and is thus quite different in spirit from the
more usual denotational semantics of linear logic. (Those are rather
some "formulas and \emph{proofs} semantics".)

\begin{quotation}
\texttt{- - - probably a whole lot more of blabla to put here... - - -}
\end{quotation}

\section{Preliminaries: relation and closure operators}\label{preliminaries-relation-and-closure-operators}

Part of the structure obtained from phase semantics works in a very
general framework and relies solely on the notion of relation between
two sets.

\subsection{Relations and operators on subsets}\label{relations-and-operators-on-subsets}

The starting point of phase semantics is the notion of \emph{duality}.
The structure needed to talk about duality is very simple: one just
needs a relation \(R\) between two sets \(X\) and \(Y\). Using standard
mathematical practice, we can write either \((a,b) \in R\) or
\(a\mathrel{R} b\) to say that \(a\in X\) and \(b\in Y\) are related.

\begin{definition}
If $R\subseteq X\times Y$ is a relation, we write $R^\sim\subseteq Y\times X$ for the converse relation: $(b,a)\in R^\sim$ iff $(a,b)\in R$.
\end{definition}

Such a relation yields three interesting operators sending subsets of
\(X\) to subsets of \(Y\):

\begin{definition}
Let $R\subseteq X\times Y$ be a relation, define the operators $\langle R\rangle$, $[R]$ and $\_^R$ taking subsets of $X$ to subsets of $Y$ as follows:
\begin{enumerate}
\item $b\in\langle R\rangle(x)$ iff $\exists a\in x,\ (a,b)\in R$
\item $b\in[R](x)$ iff $\forall a\in X,\ (a,b)\in R \implies a\in x$
\item $b\in x^R$ iff $\forall a\in x, (a,b)\in R$
\end{enumerate}
\end{definition}

The operator \(\langle R\rangle\) is usually called the \emph{direct
image} of the relation, \([R]\) is sometimes called the \emph{universal
image} of the relation.

It is trivial to check that \(\langle R\rangle\) and \([R]\) are
covariant (increasing for the \(\subseteq\) relation) while \(\_^R\) is
contravariant (decreasing for the \(\subseteq\) relation). More
interesting:

\begin{lemma}[Galois Connections]\
\begin{enumerate}
\item $\langle R\rangle$ is right-adjoint to $[R^\sim]$: for any $x\subseteq X$ and $y\subseteq Y$, we have $[R^\sim]y \subseteq x$ iff $y\subseteq \langle R\rangle(x)$
\item we have $y\subseteq x^R$ iff $x\subseteq y^{R^\sim}$
\end{enumerate}
\end{lemma}

This implies directly that \(\langle R\rangle\) commutes with arbitrary
unions and \([R]\) commutes with arbitrary intersections. (And in fact,
any operator commuting with arbitrary unions (resp. intersections) is of
the form \(\langle R\rangle\) (resp. \([R]\)).

\begin{remark}
The operator $\_^R$ sends unions to intersections because $\_^R : \mathcal{P}(X) \to \mathcal{P}(Y)^\mathrm{op}$ is right adjoint to $\_^{R^\sim} : \mathcal{P}(Y)^{\mathrm{op}} \to \mathcal{P}(X)$...
\end{remark}


\subsection{Closure operators}\label{closure-operators}

\begin{definition}
A closure operator on $\mathcal{P}(X)$ is a monotonic operator $P$ on the subsets of $X$ which satisfies:
\begin{enumerate}
\item for all $x\subseteq X$, we have $x\subseteq P(x)$
\item for all $x\subseteq X$, we have $P(P(x))\subseteq P(x)$
\end{enumerate}
\end{definition}

Closure operators are quite common in mathematics and computer science.
They correspond exactly to the notion of \emph{monad} on a preorder...

It follows directly from the definition that for any closure operator
\(P\), the image \(P(x)\) is a fixed point of \(P\). Moreover:

\begin{lemma}
$P(x)$ is the smallest fixed point of $P$ containing $x$.
\end{lemma}

One other important property is the following:

\begin{lemma}
Write $\mathcal{F}(P) = \{x\ |\ P(x)\subseteq x\}$ for the collection of fixed points of a closure operator $P$. We have that $\left(\mathcal{F}(P),\bigcap\right)$ is a complete inf-lattice.
\end{lemma}

\begin{remark}
A closure operator is in fact determined by its set of fixed points: we have $P(x) = \bigcup \{ y\ |\ y\in\mathcal{F}(P),\,y\subseteq x\}$
\end{remark}

Since any complete inf-lattice is automatically a complete sup-lattice,
\(\mathcal{F}(P)\) is also a complete sup-lattice. However, the sup
operation isn't given by plain union:

\begin{lemma}
If $P$ is a closure operator on $\mathcal{P}(X)$, and if $(x_i)_{i\in I}$ is a (possibly infinite) family of subsets of $X$, we write $\bigvee_{i\in I} x_i = P\left(\bigcup_{i\in I} x_i\right)$.

We have $\left(\mathcal{F}(P),\bigcap,\bigvee\right)$ is a complete lattice.
\end{lemma}

\begin{proof}
easy.
\end{proof}

A rather direct consequence of the Galois connections of the previous
section is:

\begin{lemma}
The operator and $\langle R\rangle \circ [R^\sim]$ and the operator $x\mapsto {x^R}^{R^\sim}$ are closures.
\end{lemma}

A last trivial lemma:

\begin{lemma}
We have $x^R = {{x^R}^{R^\sim}}^{R}$.

As a consequence, a subset $x\subseteq X$ is in $\mathcal{F}({\_^R}^{R^\sim})$ iff it is of the form $y^{R^\sim}$.
\end{lemma}

\begin{remark}
Everything gets a little simpler when $R$ is a symmetric relation on $X$.
\end{remark}

\section{Phase Semantics}\label{phase-semantics-1}

\subsection{Phase spaces}\label{phase-spaces}

\begin{definition}[monoid]
A \emph{monoid} is simply a set $X$ equipped with a binary operation $\_\cdot\_$ s.t.:
\begin{enumerate}
\item the operation is associative
\item there is a neutral element $1\in X$
\end{enumerate}
The monoid is \emph{commutative} when the binary operation is commutative.
\end{definition}

\begin{definition}[Phase space]
A phase space is given by:
\begin{enumerate}
\item a commutative monoid $(X,1,\cdot)$,
\item together with a subset $\Bot\subseteq X$.
\end{enumerate}
The elements of $X$ are called \emph{phases}.

We write $\bot$ for the relation $\{(a,b)\ |\ a\cdot b \in \Bot\}$. This relation is symmetric.

A \emph{fact} in a phase space is simply a fixed point for the closure operator $x\mapsto x\biorth$.
\end{definition}

Thanks to the preliminary work, we have:

\begin{corollary}
The set of facts of a phase space is a complete lattice where:
\begin{enumerate}
\item $\bigwedge_{i\in I} x_i$ is simply $\bigcap_{i\in I} x_i$,
\item $\bigvee_{i\in I} x_i$ is $\left(\bigcup_{i\in I} x_i\right)\biorth$.
\end{enumerate}
\end{corollary}

\subsection{Additive connectives}\label{additive-connectives}

The previous corollary makes the following definition correct:

\begin{definition}[additive connectives]
If $(X,1,\cdot,\Bot)$ is a phase space, we define the following facts and operations on facts:
\begin{enumerate}
\item $\top = X = \emptyset\orth$
\item $\zero = \emptyset\biorth = X\orth$
\item $x\with y = x\cap y$
\item $x\plus y = (x\cup y)\biorth$
\end{enumerate}
\end{definition}

Once again, the next lemma follows from previous observations:

\begin{lemma}[additive de Morgan laws]
We have
\begin{enumerate}
\item $\zero\orth = \top$
\item $\top\orth = \zero$
\item $(x\with y)\orth = x\orth \plus y\orth$
\item $(x\plus y)\orth = x\orth \with y\orth$
\end{enumerate}
\end{lemma}

\subsection{Multiplicative connectives}\label{multiplicative-connectives}

In order to define the multiplicative connectives, we actually need to
use the monoid structure of our phase space. One interpretation that is
reminiscent in phase semantics is that our spaces are collections of
\emph{tests} / programs / proofs / \emph{strategies} that can interact
with each other. The result of the interaction between \(a\) and \(b\)
is simply \(a\cdot b\).

The set \(\Bot\) can be thought of as the set of "good" things, and we
thus have \(a\in x\orth\) iff "\(a\) interacts correctly with all the
elements of \(x\)".

\begin{definition}
If $x$ and $y$ are two subsets of a phase space, we write $x\cdot y$ for the set $\{a\cdot b\ |\ a\in x, b\in y\}$.
\end{definition}

Thus \(x\cdot y\) contains all the possible interactions between one
element of \(x\) and one element of \(y\).

The tensor connective of linear logic is now defined as:

\begin{definition}[multiplicative connectives]
If $x$ and $y$ are facts in a phase space, we define
\begin{itemize}
  \item $\one = \{1\}\biorth$;
  \item $\bot = \one\orth$;
  \item the tensor $x\tens y$ to be the fact $(x\cdot y)\biorth$;
  \item the par connective is the de Morgan dual of the tensor: $x\parr y = (x\orth \tens y\orth)\orth$;
  \item the linear arrow is just $x\limp y = x\orth\parr y = (x\tens y\orth)\orth$.
  \end{itemize}
\end{definition}

Note that by unfolding the definition of \(\limp\), we have the
following, "intuitive" definition of \(x\limp y\):

\begin{lemma}
If $x$ and $y$ are facts, we have $a\in x\limp y$ iff $\forall b\in x,\,a\cdot b\in y$.
\end{lemma}

\begin{proof}
easy exercise.
\end{proof}

Readers familiar with realisability will appreciate...

\begin{remark}
Some people say that this idea of orthogonality was implicitly present in Tait's proof of strong normalisation. More recently, Jean-Louis Krivine and Alexandre Miquel have used the idea explicitly to do realisability...
\end{remark}

\subsection{Properties}\label{properties}

All the expected properties hold:

\begin{lemma}\
\begin{itemize}
\item The operations $\tens$, $\parr$, $\plus$ and $\with$ are commutative and associative,
\item They have respectively $\one$, $\bot$, $\zero$ and $\top$ as neutral element,
\item $\zero$ is absorbing for $\tens$,
\item $\top$ is absorbing for $\parr$,
\item $\tens$ distributes over $\plus$,
\item $\parr$ distributes over $\with$.
\end{itemize}
\end{lemma}

\subsection{Exponentials}\label{exponentials-3}

\begin{definition}[Exponentials]
Write $I$ for the set of idempotents of a phase space: $I=\{a\ |\ a\cdot a=a\}$. We put:
\begin{enumerate}
\item $\oc x = (x\cap I\cap \one)\biorth$,
\item $\wn x = (x\orth\cap I\cap\one)\orth$.
\end{enumerate}
\end{definition}

This definition captures precisely the intuition behind the exponentials:
\begin{itemize}
\item we need to have contraction, hence we restrict to indempotents in \(x\),
\item and weakening, hence we restrict to \(\one\).
\end{itemize}
Since \(I\) isn't necessarily a fact, we then take the biorthogonal to get a fact...

\section{Soundness}\label{soundness}

\begin{definition}
Let $(X, 1, \cdot)$ be a commutative monoid.

Given a formula $A$ of linear logic and an assignation $\rho$ that associate a fact to any variable, we can inductively define the interpretation $\sem{A}_\rho$ of $A$ in $X$ as one would expect. Interpretation is lifted to sequents as $\sem{A_1, \dots, A_n}_\rho = \sem{A_1}_\rho \parr \cdots \parr \sem{A_n}_\rho$.
\end{definition}

\begin{theorem}
Let $\Gamma$ be a provable sequent in linear logic. Then $1_X \in \sem{\Gamma}$.
\end{theorem}

\begin{proof}
By induction on $\vdash\Gamma$.
\end{proof}

\section{Completeness}\label{completeness}

Phase semantics is complete w.r.t. linear logic. In order to prove this,
we need to build a particular commutative monoid.

\begin{definition}
We define the \emph{syntactic monoid} as follows:
\begin{itemize}
\item Its elements are sequents $\Gamma$ quotiented by the equivalence relation $\cong$ generated by the rules:
\begin{enumerate}
\item $\Gamma \cong \Delta$ if $\Gamma$ is a permutation of $\Delta$
\item $\wn{A}, \wn{A}, \Gamma \cong \wn{A}, \Gamma$
\end{enumerate}
\item Product is concatenation: $\Gamma \cdot \Delta := \Gamma, \Delta$
\item Neutral element is the empty sequent: $1 := \emptyset$.
\end{itemize}
\end{definition}

The equivalence relation intuitively means that we do not care about the
multiplicity of \(\wn\)-formulae.

\begin{lemma}
The syntactic monoid is indeed a commutative monoid.
\end{lemma}

\begin{definition}
The \emph{syntactic assignation} is the assignation that sends any variable $\alpha$ to the fact $\{\alpha\}\orth$.
\end{definition}

We instantiate the pole as \(\Bot := \{\Gamma \mid \vdash\Gamma\}\).

\begin{theorem}
If $\Gamma\in\sem{\Gamma}\orth$, then $\vdash\Gamma$.
\end{theorem}

\begin{proof}
By induction on $\Gamma$.
\end{proof}

\section{Cut elimination}\label{cut-elimination}

Actually, the completeness result is stronger, as the proof does not use
the cut-rule in the reconstruction of \(\vdash\Gamma\). By refining the
pole as the set of \emph{cut-free} provable formulae, we get:

\begin{theorem}
If $\Gamma\in\sem{\Gamma}\orth$, then $\Gamma$ is cut-free provable.
\end{theorem}

From soundness, one can retrieve the cut-elimination theorem.

\begin{corollary}
Linear logic enjoys the cut-elimination property.
\end{corollary}

\section{The Rest}\label{the-rest}

\todo[missing in LLWiki]


%%% Local Variables:
%%% mode: latex
%%% TeX-master: "main"
%%% End:
