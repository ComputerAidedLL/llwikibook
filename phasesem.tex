\section{Phase semantics}\label{phase-semantics}

\subsection{Introduction}\label{introduction}

The semantics given by phase spaces is a kind of "formula and
provability semantics", and is thus quite different in spirit from the
more usual denotational semantics of linear logic. (Those are rather
some "formulas and \emph{proofs} semantics".)

\texttt{~-\/-\/-~probably~a~whole~lot~more~of~blabla~to~put~here...~-\/-\/-}

\subsection{Preliminaries: relation and closure
operators}\label{preliminaries-relation-and-closure-operators}

Part of the structure obtained from phase semantics works in a very
general framework and relies solely on the notion of relation between
two sets.

\subsubsection{Relations and operators on
subsets}\label{relations-and-operators-on-subsets}

The starting point of phase semantics is the notion of \emph{duality}.
The structure needed to talk about duality is very simple: one just
needs a relation \(R\) between two sets \(X\) and \(Y\). Using standard
mathematical practice, we can write either \((a,b) \in R\) or
\(a\mathrel{R} b\) to say that \(a\in X\) and \(b\in Y\) are related.

Such a relation yields three interesting operators sending subsets of
\(X\) to subsets of \(Y\):

The operator \(\langle R\rangle\) is usually called the \emph{direct
image} of the relation, \([R]\) is sometimes called the \emph{universal
image} of the relation.

It is trivial to check that \(\langle R\rangle\) and \([R]\) are
covariant (increasing for the \(\subseteq\) relation) while \(\_^R\) is
contravariant (decreasing for the \(\subseteq\) relation). More
interesting:

This implies directly that \(\langle R\rangle\) commutes with arbitrary
unions and \([R]\) commutes with arbitrary intersections. (And in fact,
any operator commuting with arbitrary unions (resp. intersections) is of
the form \(\langle R\rangle\) (resp. \([R]\)).

\textbackslash{}to \textbackslash{}mathcal\{P\}(X)...\}\}

\subsubsection{Closure operators}\label{closure-operators}

Closure operators are quite common in mathematics and computer science.
They correspond exactly to the notion of \emph{monad} on a preorder...

It follows directly from the definition that for any closure operator
\(P\), the image \(P(x)\) is a fixed point of \(P\). Moreover:

One other important property is the following:

Since any complete inf-lattice is automatically a complete sup-lattice,
\(\mathcal{F}(P)\) is also a complete sup-lattice. However, the sup
operation isn't given by plain union:

A rather direct consequence of the Galois connections of the previous
section is:

A last trivial lemma:

\subsection{Phase Semantics}\label{phase-semantics-1}

\subsubsection{Phase spaces}\label{phase-spaces}

Thanks to the preliminary work, we have:

\subsubsection{Additive connectives}\label{additive-connectives}

The previous corollary makes the following definition correct:

Once again, the next lemma follows from previous observations:

\subsubsection{Multiplicative
connectives}\label{multiplicative-connectives}

In order to define the multiplicative connectives, we actually need to
use the monoid structure of our phase space. One interpretation that is
reminiscent in phase semantics is that our spaces are collections of
\emph{tests} / programs / proofs / \emph{strategies} that can interact
with each other. The result of the interaction between \(a\) and \(b\)
is simply \(a\cdot b\).

The set \(\Bot\) can be thought of as the set of "good" things, and we
thus have \(a\in x\orth\) iff "\(a\) interacts correctly with all the
elements of \(x\)".

Thus \(x\cdot y\) contains all the possible interactions between one
element of \(x\) and one element of \(y\).

The tensor connective of linear logic is now defined as:

Note that by unfolding the definition of \(\limp\), we have the
following, "intuitive" definition of \(x\limp y\):

Readers familiar with realisability will appreciate...

\subsubsection{Properties}\label{properties}

All the expected properties hold:

\subsubsection{Exponentials}\label{exponentials-3}

This definition captures precisely the intuition behind the
exponentials:

\begin{itemize}
\tightlist
\item
  we need to have contraction, hence we restrict to indempotents in
  \(x\),
\item
  and weakening, hence we restrict to \(\one\).
\end{itemize}

Since \(I\) isn't necessarily a fact, we then take the biorthogonal to
get a fact...

\subsection{Soundness}\label{soundness}

\subsection{Completeness}\label{completeness}

Phase semantics is complete w.r.t. linear logic. In order to prove this,
we need to build a particular commutative monoid.

The equivalence relation intuitively means that we do not care about the
multiplicity of \(\wn\)-formulae.

We instantiate the pole as \(\Bot := \{\Gamma \mid \vdash\Gamma\}\).

\subsection{Cut elimination}\label{cut-elimination}

Actually, the completeness result is stronger, as the proof does not use
the cut-rule in the reconstruction of \(\vdash\Gamma\). By refining the
pole as the set of \emph{cut-free} provable formulae, we get:

From soundness, one can retrieve the cut-elimination theorem.

\subsection{The Rest}\label{the-rest}

